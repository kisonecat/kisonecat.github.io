\documentclass[12pt]{amsart}

\title{Research Statement}
\author{Jim Fowler}

\newcommand{\mr}[1]{\cite{MR#1}}
\newcommand{\nomr}[1]{\nocite{MR#1}}
\newcommand{\arxiv}[1]{\cite{arxiv#1}}

\usepackage[left=1.5in,right=1.5in,top=1.25in,bottom=1.25in]{geometry}

\usepackage{amsmath}
\usepackage{amsthm}
\newtheorem{theorem}{Theorem}
\newtheorem{corollary}[theorem]{Corollary}
\newtheorem*{corollary*}{Corollary}
\newtheorem*{theorem*}{Theorem}
\newtheorem{lemma}[theorem]{Lemma}
\newtheorem*{lemma*}{Lemma}

\theoremstyle{definition}
\newtheorem{remark}[theorem]{Remark}
\newtheorem*{remark*}{Remark}
\newtheorem{example}[theorem]{Example}
\newtheorem*{example*}{Example}
\newtheorem{definition}[theorem]{Definition}
\newtheorem*{definition*}{Definition}
\newtheorem{conjecture}[theorem]{Conjecture}
\newtheorem*{conjecture*}{Conjecture}
\newtheorem{question}[theorem]{Question}

\DeclareMathOperator{\Sp}{Sp}

\usepackage{amssymb}
\newcommand{\R}{\mathbb{R}}
\newcommand{\Q}{\mathbb{Q}}
\newcommand{\N}{\mathbb{N}}
\newcommand{\Z}{\mathbb{Z}}

\newcommand{\B}{\mathcal{B}}

\usepackage[style=numeric]{biblatex}
\addbibresource{references.bib}

\begin{document}
\maketitle

I am a geometer, and I am interested in the geometric topology of
manifolds and in geometric group theory.  Manifolds and their
symmetries are ubiquitous throughout in mathematics and its
applications.  Surfaces were studied already in the nineteenth century
by Riemann, and higher dimensional manifolds became central of objects
in the twentieth beginning with the work of Poincar\'e, and continuing
with the work of Thom, Milnor, Wall, Smale, Novikov---to name just a
few.

My work makes use of the techniques and ideas from algebraic and
combinatorial topology, surgery theory, number theory, and complexity
theory.  A major theme of much of my work is the extent to which an
object with some singularities or deficiencies can be ``resolved'' or
``improved'' to a nicer object.  For instance,
\begin{itemize}
\item in Section~\ref{subsection:rational-poincare-duality},
  there is the question of whether the classifying space for a
  $\Q$-PD group can be ``resolved'' to an aspherical closed
  $\mathbb{Q}$-homology manifold;
\item in
  Section~\ref{subsection:aspherical-manifold-and-triangulations},
  there is the question of whether an aspherical topological manifold
  can be ``resolved'' by a simplicial complex;
\item in Section~\ref{subsection:computing-the-l-polynomial}, there
  is the question of whether a PL $\mathbb{Q}$-homology manifold $M$
  having cohomology ring $H^\star(M;\Q) = \Q[x]/(x^3)$ can be resolved
  by a smooth manifold; and
\item in Section~\ref{subsection:polynomially-bounded-topology}, there 
  is the issue of whether, say, a cochain can be replaced by a
  ``polynomially bounded'' cochain.
\end{itemize}
To make progress on questions like these, I make significant use of
computational techniques, and often use the computer algebra system
\texttt{sage} to run experiments.  My research has three specific
directions:
\begin{itemize}
\item aspherical manifolds and Poincar\'e duality groups,
\item computations surrounding high-dimensional manifolds, and
\item undergraduate research.
\end{itemize}
%BADBAD include overview of the first two of the three directions
By ``undergraduate research,'' I mean research projects involving
undergraduates (some of which are listed on
Page~\pageref{section:undergraduate-research}) but also research
projects which occur as part of online courses and my NSF~TUES grant.
As of November 2013, there have been 147k enrollments in my online
courses, which have resulted in significant amounts of data.  This
project is discussed in Section~\ref{subsection:moocs}.

\newpage
\section{Aspherical manifolds and Poincar\'e duality groups}
\label{section:aspherical-manifolds}

\subsection{Rational Poincar\'e duality}
\label{subsection:rational-poincare-duality}
A group $G$ is a Poincar\'e duality group if its classifying space
$BG$ satisfies Poincar\'e duality; examples include fundamental groups
of aspherical manifolds.  C.~T.~C.~Wall asked whether a Poincar\'e
duality group is necessarily the fundamental group of an aspherical
manifold, and M.~Davis in \cite{MR1747535} broadened Wall's question to
$R$-homology manifolds:
\begin{question}
Is every finitely presented torsion-free group satisfying Poincar\'e
duality with $R$-coefficients (``$R$-PD'') the fundamental group of an
aspherical closed $R$-homology manifold?
\end{question}
\noindent
My thesis answers M.~Davis' question in the negative, using
Bestvina--Brady Morse theory to produce $\Q$-PD groups $G$ for which
$BG$ does not have the homotopy type of a finite complex.

I also addressed a generalization version of M.~Davis' question.
There are weaker geometric conditions that nonetheless imply a group
is $R$-PD: for $\pi$ to satisfy Poincar\'e duality with $R$
coefficients, it suffices that $\pi$ act freely on an $R$-acyclic
$R$-homology manifold.
\begin{question}
  Does every finitely presented group $\pi$ satisfying Poincar\'e
  duality with $R$ coefficients act freely on an $R$-acyclic
  $R$-homology manifold?
\end{question}
\noindent
This is interesting when $R = \Q$.  All finite groups are
$0$-dimensional $\Q$-PD, and extensions of $\Q$-PD groups by finite
groups are therefore $\Q$-PD.  The restriction to torsion-free groups
is not necessary, as groups with torsion may act freely on $\Q$-acyclic
(albeit not contractible) finite complexes.

A particular class of $\Q$-PD groups to consider are uniform lattices
with torsion (which, by Selberg, are extensions of manifold groups by
finite groups).
\begin{theorem*}[Fowler]
  Let $\Gamma$ be a uniform lattice containing an element of finite
  order ($\neq$ 2).  Although $\Gamma$ is $\Q$-PD, $\Gamma$ does not
  act freely on a $\Q$-acyclic $\Q$-homology manifold.
\end{theorem*}
\noindent
The proof involves two obstructions: a finiteness obstruction, and a
controlled symmetric signature.  With more work, I hope to address the
situation for $\Gamma$ containing $2$-torsion
\begin{proof}[Sketch of Proof:]
The group $\Gamma$ is given as an extension $0 \to \pi \to \Gamma \to G \to 0$, where
$\pi$ is torsion-free, and $G$ is a finite group.  Given the
Baum-Connes conjecture for $\Gamma$, the $\Gamma$-equivariant
signature operator on $E\pi$ (which is the universal space for proper
$\Gamma$ actions) cannot be lifted back to the signature operator on
$B\Gamma$.

To obstruct this lift, the space $E\pi$ is used to determine the
equivariant signature operator of $\Gamma$ acting on the universal
cover of the hypothetical homology manifold; geometric localization
arguments can then be combined with calculations near fixed points to
give a contradiction---although this last step fails in the absence of
odd torsion.
\end{proof}

%For the proof, I use the Novikov conjecture for $\Gamma$ as an input.
%The stratified space $B\pi/G$ has singularities corresponding to the
%fixed sets of the $G$-action; the map $p : B\Gamma \to B\pi/G$
%resolves these singularities.  The controlled symmetric signature
%$\sigma_\star(B\Gamma) \in H_\star(B\Gamma;\mathbb{L}\Q)$ is sent via
%the assembly map to $L_\star(\Q\Gamma)$, but that assembly map factors
%through $H_\star(B\pi/G;\mathbb{L}\Q\Gamma_x)$, the cohomology of
%$B\pi/G$ with values in a spectral cosheaf corresponding to the
%$L$-theory of the fibers of $p$; this corresponds to examining the
%$\Gamma$-equivariant signature operator on $E\pi$ in the Baum-Connes
%picture.

%By using a splitting calculation as in Yamasaki \cite{MR884801}, we
%restrict to a neighborhood of a singularity, reducing the calculation
%to a $\rho$ invariant.

%This does not produce a
%counterexample to the Baum-Connes conjecture or Novikov conjecture,
%however, as the above proof does not apply, since these examples
%contain two torsion.

In addition to controlled surgery considerations, there is a finiteness question:
\begin{question}
For a given group $G$, does there exist a free action of $G$
on an $R$-acyclic complex, with finite quotient?
\end{question}
\noindent
Bestvina and Brady \cite{MR1465330} call such groups FH over $R$, in
analogy with other homological finiteness conditions.

In many cases, a finiteness obstruction will prevent $\Gamma$ from
acting on a $\Q$-acyclic finite complex, let alone a $\Q$-homology
manifold.  Again, consider a uniform lattice $\Gamma$, and write $0
\to \pi \to \Gamma \to G \to 0$, with $\pi$ torsion-free and $G$
finite.  For convenience, $G = \Z/p\Z$.  If $G$ acts freely on a
finite complex $B\pi$, then the Lefschetz fixed point theorem implies
$\chi(B\pi^G) = 0$.

The equivariant finiteness obstruction fits into a categorical
framework defined by L\"uck \cite{MR1027600}; the condition
$\chi(B\pi^G) = 0$, while necessary for $\Gamma$ to be FH, is not
enough: $\chi(B\pi^G)$ can vanish while the Euler characteristics of
the connected components of $B\pi^G$ may not vanish.  If $\chi(C) = 0$
for each connected component $C$ of $B\pi^G$, then the equivariant
finiteness obstruction does indeed vanish.  The componentwise
vanishing of the Euler characteristic is easy to verify in some cases,
such as when the fixed sets are all odd-dimensional submanifolds.

\nomr{859767}
\nomr{868892}
\nomr{1076524}
\nomr{1381603}
\nomr{2222492}

\nocite{MR2036915}
\nocite{MR1943285}
\nocite{MR2177038}
\nocite{thesis}

\subsection{Aspherical manifolds and triangulations}
\label{subsection:aspherical-manifold-and-triangulations}

Kirby and Siebenmann showed that there are manifolds that do not admit
PL~structures \cite{MR242166}, and yet the possibility remained that
all manifolds could be triangulated, meaning that for every manifold
$M$, there is a simplicial complex $K$ so that the geometric
realization of $K$ is homeomorphic to $M$, but of course the
simplicial complex $K$ is not a PL~triangulation, meaning the links
are not spheres.

Freedman showed that there are 4-manifolds that cannot be triangulated
\cite{MR679066}.  Davis and Januszkiewicz applied a hyperbolization
procedure to Freedman's 4-manifolds to get closed aspherical
4-manifolds that cannot be triangulated \cite{MR1131435}.  What about
higher dimensions?

In the late 1970s Galewski and Stern \cite{MR558395} and
independently, Matumoto, showed that non-triangulable manifolds exist
in all dimensions $> 4$ if and only if homology $3$-spheres with
certain properties do not exist. In \cite{manolescu2013conley},
Manolescu showed that there were no such homology $3$-spheres, and
hence non-triangulable manifolds exist in every dimension $>4$.  

Applying a hyperbolization technique to the Galewski-Stern manifolds shows the following.
\begin{theorem}[Davis--Lafont--Fowler \cite{aspherical}]
  Let $n > 5$.  There exists a closed aspherical $n$-manifold which
  cannot be triangulated.
\end{theorem}
However, the following question remains open.
\begin{question}
  Do there exist closed aspherical $5$-manifolds that cannot be
  triangulated?
\end{question}

\subsection*{Computer search for Poincar\'e duality groups}

There are well-known algorithms for computing the group cohomology of
finite groups \cite{MR603653}, but there are also algorithms for
computing the group cohomology of automatic groups \cite{MR2093885},
which I could improve to compute the cup product structure.  The
examples of $\Q$-PD groups in
Section~\ref{subsection:rational-poincare-duality} have a common
structure as extensions of finite (0-dimensional) groups.
\begin{question}
Can one find new examples of $\Q$-PD groups among automatic groups?
\end{question}
Considering that it is not yet clear what sorts of restrictions
automaticity might impose on a Poincar\'e duality group, such examples
are even interesting independent of their usefulness for understanding
$\Q$-homology manifolds.

With the goal of generalizing the theory of automatic groups, Ranicki
posed the problem of extending automatic structures on groups to a
similar structure on group rings and forms over them \cite{MR1388296};
developing a theory of ``automatic algebra'' is a great goal in its
own right, and, practically speaking, such a theory would permit
experimentation on algebraic Poincar\'e complexes using computer
algebra systems.

A computer search could also proceed for new hyperbolic $\Z$-PD
groups; for such $\Gamma$, it is now known that $B\Gamma$ is homotopy
equivalent to a $\Z$-homology manifold (if $\dim > 4$, by combining
\cite{MR1394965} with work of Bartels and L\"uck).  Upon finding some
examples, it might then be possible to discover a group $\Gamma$ so
that $B\Gamma$ has the homotopy type of a $\Z$-homology manifold, but
not the homotopy type of a manifold.

\section{Computations surrounding high-dimensional manifolds}

\subsection{Computing the $L$-polynomial}
\label{subsection:computing-the-l-polynomial}

The Hirzebruch $L$-polynomial is one place where number theory very
strongly interacts with high-dimensional topology \cite{MR339202}.
Recall that the Hirzebruch signature theorem relates the signature of
a smooth closed manifold $M^{4k}$ to $\sum_I L_I
p_I(M)$. Unfortunately, the na\"ive method to compute coefficients
$L_I$ of the Hirzebruch $L$-polynomial is much too slow for
applications; Zhixu~Su and I have discovered a recursive method which
is fast enough to compute many coefficients.

Solutions to some Diophantine equations related to these
$L$-polynomials give rise to manifolds having a truncated polynomial
algebra as their rational cohomology ring; such manifolds may exist
even when the corresponding truncated polynomial algebra over $\Z$ is
not the cohomology ring of any space.  For instance, there is a
manifold having the rational cohomology that $\mathbb{O}P^4$ would be expected
to have, if $\mathbb{O}P^4$ existed.

On the other hand, nonexistence results are also possible.
A rational projective plane means a smooth manifold $M^{4n}$ with
$$
H^\star(M;\Q) = \Q[x]/(x^3) \mbox{ with $|x| = 2n$}.
$$
\begin{question}
  For which $n$ is there a rational projective plane $M^{4n}$?
\end{question}
There is a piecewise linear $\mathbb{Q}$-homology manifold with
$\Q[x]/(x^3)$ as its cohomology ring, so this question can be viewed
as a question about whether that $\mathbb{Q}$-homology manifold can be
``resolved'' by a smooth manifold.  Integrally, this is not always
possible by the celebrated work of Adams \cite{MR133837}, but in Zhixu
Su's thesis, it was shown that there \textit{is} a rational projective
plane in dimension~32.  But it is not always possible, even
rationally.
\begin{theorem}[Fowler--Su]
  There is no rational projective plane in dimension~64.
\end{theorem}
This boils down to some number theory.  Since a rational projective
plane has signature one, we would be seeking a solution to $s_{8,8}
x^2 \pm s_{16} y = \pm 1$ for integers $x$ and $y$.  Note that $37$
divides the numerator of $s_{16}$, because $37$ divides
$B_{32}$---perhaps not so surprising considering 37's status as the
smallest irregular prime.  So it is enough to show there is no
solution to $x^2 \not\equiv \pm 1/s_{8,8} \pmod{37}$.  Since $s_{16}
\equiv 0 \pmod{37}$,
\begin{align*}
  s_{8,8}
  &\equiv \frac{{s_k}^2 - s_{2k}}{2} \pmod{37} \\
  &\equiv \frac{{s_k}^2}{2} \pmod{37},
\end{align*}
but neither $2$ nor $-2$ is a quadratic residue modulo 37.

The same argument works to rule out a rational projective plane in
dimension $2^{k+3}$ provided one can find a prime $p$ so that
\begin{itemize}
\item $2$ and $-2$ are quadratic nonresidues modulo $p$,
\item $\nu_p(s_{2 \cdot 2^k}) > 0$, but
\item $\nu_p(s_{2^k}) = 0$.
\end{itemize}
To ensure $2$ is a quadratic nonresidue, it is enough that $p
\not\equiv \pm 1 \pmod 8$; to ensure that $-2$ is also a quadratic
nonresidue, we further want $p \equiv 5 \pmod 8$.

The fact that divisors of $2^{2^k - 1} - 1$ are rarely (never?) $5
\bmod 8$ tells us not to look for such primes among the divisors of
the Mersenne factor.  The above desiderata are satisfied by finding a
prime $p$ so that
\begin{itemize}
\item $p \equiv 5 \pmod 8$,
\item $p > 4 \cdot 2^k$,
\item $p$ divides the numerator of $B_{4 \cdot 2^k}$,
\item $p$ does not divide the numerator of $B_{2 \cdot 2^k}$.
\end{itemize}
The number theory becomes rather involved computationally.  For
example, the prime $p = 502261$ is $5 \bmod 8$ and divides the
numerator of $B_{4 \cdot 2^{11}}$ but not $B_{2 \cdot 2^{11}}$, which
rules out a rational projective plane in dimension $2^{14} = 4096$;
similarly, the prime $p = 69399493$ is $5 \bmod 8$, divides the
numerator of $B_{4 \cdot 2^{21}}$ but not $B_{2 \cdot 2^{21}}$, which
rules out a projective plane in dimension $2^{24}$.  These
calculations are possible due to tables of irregular primes produced
by Joe P.~Buhler and David Harvey \cite{MR2813369}.

Looking over these calculations, there are things that we can say in
general.
\begin{theorem}[Fowler--Su]
  If $M^n$ is a rational projective plane, then $n = 2^a + 2^b$ for
  $a, b \in \mathbb{N}$.
\end{theorem}
This result has some interesting consequences, such as the fact that
there is a topological manifold which is not $\Q$-homotopy equivalent
to a smooth manifold.
%\begin{theorem}[Fowler--Su]
%    The Milnor $E_8$ manifold $M^{504}$ is such.
%    A smooth manifold $\Q$-homotopy equivalent to $M^{504}$ has signature 8.
%   $504 = 256 + 128 + 64 + 32 + 16 + 8$,  so $\nu_2(s_{126}) = \nu_2(s_{63,63}) = 4$.
%    A smooth manifold $\Q$-homotopy equivalent to $M^{504}$ 
%    has signature divisible by 16.
%\end{theorem}
And even in dimensions $4n$ for which there is a rational projective plane, a refined question can be explored.
\begin{question}
  Suppose $M^{4n}$ is a rational projective plane.  How highly
  connected can $M^{4n}$ be, integrally?
\end{question}
When $4n = 32$, we have a particularly nice answer by doing
$\hat{A}$-genus calculations: there does not exist a simply-connected
closed Spin manifold $M^{32}$ which is a rational projective plane, so
a 32-dimensional example cannot even be integrally 2-connected.
Calculations involving the Steenrod algebra and Stiefel-Whitney
classes may provide other methods for determining how highly connected
a rational projective plane must be.

Finally, another interesting question is to study the asymptotic
running time for algorithms computing the $L$-polynomial.
\begin{question}
How quickly can the $L$-polynomial be computed?
\end{question}
Such questions tie into Bernoulli number calculations, for which there
are impressively fast analytic methods \cite{MR2684369}.  This would
be a nice historical story, since the computation of Bernoulli numbers
was the goal of a computer program from 1843 written by Ada Lovelace
\cite{MR550674}.

\subsection{Computing $L$-classes}

Beyond the $L$-polynomial, computation of the total $L$-class would
also permit certain manifolds to be recognized.  Here is an example of
such a problem and a possible approach.  Brehm and K\"uhnel in
\cite{MR1180457} exhibit a few different combinatorial 15-vertex
triangulations of an 8-manifold ``like'' the quaternionic projective
plane $\mathbb{H}P^8$.  One of these examples $X^8$ is especially
symmetric, and likely PL homeomorphic to $\mathbb{H}P^8$.
\begin{question}
Is there a PL homeomorphism between the 15-vertex complex of Brehm--K\"uhnel and $\mathbb{H}P^8$?
\end{question}
Despite more recent work (e.g., \cite{MR3038783}) which has placed
these examples in a nice context, this question remains open.  I
propose answering this question with a direct computation of the
rational $L$-class by implementing the procedure in \cite{MR440554}.
It is perhaps surprising that this can be done effectively.  The
relevant steps are to
\begin{itemize}
\item find a simplicial map $f : X^8 \times S^n \to S^{n+8}$ of nonzero degree,
\item consider the preimage $f^{-1}(x)$ of some point $x \in S^{n+8}$, and
\item compute the signature of $f^{-1}(x)$ by computing the cup product pairing on $H^\star(f^{-1}(x);\Q)$.
\end{itemize}
Of these, finding the map $f$ has proven to be more involved than I
would have hoped; the stabilized copy of $X^8$ needs to be subdivided
to get a map to a sphere, and this subdivision quickly increases the
number of simplexes that need to be stored, in spite of how small the
15-vertex triangulation is at first.  Nevertheless, I am optimistic
that this is possible.

\subsection{Computing Stiefel-Whitney classes}

Having considered experiments with the $L$-classes---which amount to
experiments with the Pontrjagin classes---there are also experiments
one can do with Stiefel-Whitney classes.  This project is joint work
in progress with Jean Lafont.

Let $G$ be a finite group.  The Vasquez invariant $n(G)$ is the
natural number so that, for any flat manifold $M$ with holonomy $G$,
then $M$ is a toral expansion of some flat manifold with dimension
$\leq n(G)$.  Computing $n(G)$ is Problem~4 of \cite{MR2252897}.

That such a number $n(G)$ exists is a result of Vasquez'
\cite{MR267487}, where Vasquez also points out that, in dimension $>
n(G)$, the characteristic algebra of $M$ vanishes.  In other words,
for a flat manifold $M$ with holonomy $G$, 
$$
w_{i_1}(M) w_{i_2}(M) \cdots w_{i_n}(M) = 0
$$
when $i_1 + i_2 + \cdots + i_n > n(G)$. For a given group $G$, by
explicitly producing flat manifolds $M$ with holonomy $G$, a nonzero
product of Stiefel-Whitney classes yields a lower bound on $n(G)$.  So
 how is one to produce a ``random'' flat manifold with holonomy $G$?
 Restrict to the case of $G = (\Z/2\Z)^k$.  Then a nice reasonable
 collection of flat manifolds are the real Bott manifolds, which are
 given by some combinatorial information \cite{MR2915482}, namely a
 matrix with entries in $\{0,1\}$.
\begin{question}
  Let $G = (\Z/2\Z)^k$ and let $M$ be a real Bott manifold with
   holonomy $G$.  What is the largest nonzero product of
   Stiefel-Whitney classes of $M$?
\end{question}
 Given a random matrix, I have already written code to compute the
 Stiefel-Whitney classes.  Probably more can be said by using the fact
 that the holonomy group we are dealing with is abelian.

 It would also be interesting to have some of explanation of the
 statistics I am seeing in the data.  For example, with 26831 random
 12-by-12 upper triangular matrices so that $w_2 w_3 w_4$ is
 nontrivial, only forty were found with holonomy $(\Z/2\Z)^6$ but 10999
 were found with holonomy $(\Z/2\Z)^9$.

%BADBAD
%bott towers
%iterated sphere bundles

\subsection{Polynomially bounded homotopy theory}
\label{subsection:polynomially-bounded-topology}

In controlled topology, notions like homotopy are refined to include a
condition on their size, measured via a reference map to a metric
space.  There are different versions of controlled topology in current
use, including bounded control and continuous control. But there are
other versions of control that are worth considering.

That is not so surprisingly: much of the success of geometric group
theory comes by thinking about asymptotic invariants of a group as a
metric space \cite{MR1253544}, but one can consider other asymptotic
invariants for spaces.
\begin{question}
When does a CW complex have the homotopy type of a complex having polynomially many cells in
dimension $n$?
\end{question}
This is a quantitative version of Wall's finiteness obstruction
\cite{MR171284}.  Interestingly, a space might have Poincar\'e duality but
not in a polynomially bounded sense, so there are analogies to be made
between this project and the homology manifold projects.

Crichton Ogle has developed polynomially bounded cohomology
\cite{MR2109110}.  I have collaborated with Ogle to work through the
foundations of polynomially bounded---and more generally,
$\B$-bounded---homotopy theory \cite{MR2962981}, which would be one
framework within which the above question could be addressed.
Specifically, given a bounding class $\B$, we construct a bounded
refinement ${\B K}(-)$ of Quillen's $K$-theory functor from rings to
spaces. As defined, ${\B K}(-)$ is a functor from weighted rings to
spaces, and is equipped with a comparison map $BK \to K$ induced by
``forgetting control.''  In contrast to the situation with
$\B$-bounded cohomology, there is a functorial splitting ${\B K}(-)
\simeq K(-) \times {\B K}^{rel}(-)$ where ${\B K}^{rel}(-)$ is the
homotopy fiber of the comparison map.

Finally, there is also an analogy to be made between ``weighted
rings'' and the geometric modules of Quinn \cite{MR802791}, along with
$\B$-bounded homotopy theory and controlled topology.  Much of our
work in \cite{MR2962981} is in building an appropriate $\B$-bounded
Waldhausen category.  At the point where one can define $\B$-bounded
Waldhausen categories, why not go all the way and consider a
$\B$-bounded model theory?

Christensen and Munkholm have placed continuous and bounded notions of
control into a common, categorical framework \cite{MR1983017};
Higson--Pedersen--Roe have introduced a different framework unifying
various kinds of coarse spaces, from an analytic perspective
\cite{MR1451755}.  Weiss--Williams formulate some examples of control
within Waldhausen categories with duality \cite{MR1644309}.  All of
these notions could be unified into a single theory of
\textit{controlled model categories}.  Anderson's homotopy theory for
boundedly controlled topology \cite{MR953961} is a starting point.

There is some precedent for considering a controlled model category,
namely the parametrized homotopy theory of May--Sigurdsson
\cite{MR2271789}.  This latter theory, however, is more topological
than geometric---the reference maps are not to metric spaces.
Nevertheless, elucidating the precise relationship between the
parametrized homotopy theory of May--Sigurdsson and the spectral
cosheaves used in stratified surgery (see \cite{MR1308714}) would be
very worthwhile.

% MR1949110
%http://www.ams.org/mathscinet/search/publdoc.html?arg3=&co4=AND&co5=AND&co6=AND&co7=AND&dr=all&pg4=AUCN&pg5=TI&pg6=PC&pg7=ALLF&pg8=ET&r=1&review_format=html&s4=&s5=&s6=&s7=MR1949110&s8=All&vfpref=html&yearRangeFirst=&yearRangeSecond=&yrop=eq
%http://ac.els-cdn.com/S016686419800056X/1-s2.0-S016686419800056X-main.pdf?_tid=38b5ebec-3b2d-11e3-8718-00000aab0f01&acdnat=1382455297_7f36baf326362a6e70c6dd0a0170e065

%%%%%%%%%%%%%%%%%%%%%%%%%%%%%%%%%%%%%%%%%%%%%%%%%%%%%%%%%%%%%%%%

\section{Undergraduate research}
\label{section:undergraduate-research}

There are two different ways to involve undergraduates in research.
One way is to do research projects \textit{with} undergraduates: I
have two successful projects that involved undergraduates in a
significant way, as well as some proposed projects that I hope to do
with future students.  The other way to involve undergraduates is to
do research \textit{on} them: I have an ongoing project called
MOOCulus, an adaptive learning platform that I built.  There have been
about 150k enrollments in this online course, so we have quite a bit
of data.

\subsection{No three in line}
\label{subsection:no-three-in-line}

% http://arxiv.org/abs/1203.6604
% The no-three-in-line problem on a torus

For a group $G$, let $T(G)$ denote the cardinality of the largest
subset $S \subset G$ so that no three elements of $S$ are in the same
coset of a cyclic subgroup.  Undergraduates Andrew Groot and Deven Pandya, advised
by myself and my colleague Bart Snapp, considered the case $G =
\mathbb{Z}/m\mathbb{Z} \times \mathbb{Z}/n\mathbb{Z}$, and showed that
\begin{align*}
  T(\Z_p \times \Z_{p^2}) &= 2p, \\
  T(\Z_p \times \Z_{pq})  &= p+1.
\end{align*}
This problem can also be formulated as a Gr\"obner basis question;
after doing so, we computed $T(\Z_m \times \Z_n)$ for $2 \leq m \leq
7$ and $2 \leq n \leq 19$.  These results are available in
\cite{2012arXiv1203.6604F}.

This problem presents some nice connections for undergraduates.
Thinking of a coset of a cyclic subgroup as a ``line,'' there is then
connection with the usual ``no three in line problem.''  Paul Erd\"os
proved that for a prime $p$, one can place $p$ points on the $p\times
p$ lattice in the plane \cite{MR41889}; the construction goes via a
parabola modulo $p$.  Other more complicated constructions manage to
place more points \cite{MR366817}.

Another nice connection is made by considering other groups.  Although
the no-three-in-line problem for $G = (\mathbb{Z}/p\mathbb{Z})^2$ can
be considered as the $k$-arc problem from projective geometry
\cite{MR554919}, the question is also interesting for, say, $G = S_n$ or
$G = A_n$ where, say, Bezout's theorem doesn't make sense anymore.

\subsection{Transversal of primes}
\label{subsection:transversal-of-primes}

Matej Penciak, then a student at the University of Rochester, among
other students at the Ross program, became interested in the following
question.
\begin{question}[Question 1.6 ``Transversal of Primes'' from \cite{MR2847943}]
  Let $p$ be the $n^{\mbox{\scriptsize th}}$ prime, and place the
  integers $1$ through $p^2$ into a $p \times p$ array in order.  Is
  it possible to choose a set of $p$ primes from the array so that no
  two appear in the same row or column?
\end{question}
During Summer 2012, we gathered some evidence on this
question.  Our approach used WalkSAT \cite{walksat}, a local search
algorithm for Boolean satisfiability problems.  We verified that it is
possible to arrange $p$ primes on a $p \times p$ board, no two in a
row or column, for primes $\leq 1291$.

\subsection{Proposed undergraduate research projects}
\label{subsection:proposed-projects}

Here are some proposed projects that could become good activities for
undergraduates.
\begin{description}
\item[Space of five-by-five natural images]
Carlsson--Ishkhanov--De Silva--Zomorodian found a Klein bottle in the
space of $3 \times 3$ pixel patches of natural images
\cite{natural-images}.  To be a bit more precise, each $3 \times 3$
pixel patch from a black-and-white image yields a point in $[0,1]^9
\subset \mathbb{R}^9$; one can normalize for brightness and contrast,
and then apply a ``thresholding'' procedure to throw away some
outliers.  Many of the remaining points are close to a Klein bottle
embedded in $\mathbb{R}^9$.  A nice project would be describe the
structure of, say, $5 \times 5$ pixel patches.  A first step would be
to compute the persistent homology \cite{MR2121296} of the (suitably
``thresholded'') point cloud, which could be done with available
software by an undergraduate having some background in computer
science.

\item[Roots of random quaternionic polynomials]  There are
other examples of point cloud data coming from ``pure mathematics''
for which it would be interesting to calculate the persistent
homology.  For example, there are well-known results on the
distribution of complex roots of random polynomials, some of which are
topologically interesting---for some distributions, the roots lie
close to a circle.  There are algorithms for finding roots of a
quaternionic polynomial \cite{MR6980} which can be performed in
practice \cite{MR1851239}, so what sort of topological structure might
one find in the roots of random quaternionic polynomials?  The same
skillset that would help an undergraduate analyze the space of $5
\times 5$ pixel patches could be applied to this question, as well.

\item[Representation theory and the genetic code] The genetic code describes how sequences of DNA are transformed into proteins; the code is redundant, but why?  Considering that there are a large number of genetic
codes, why should we see only a handful of possible codes in nature?
Hornos and Hornos investigated the symmetries of the genetic code via
representation theory \cite{originalhornos}, akin to the techniques
used in particle physics with Lie groups.  But the genetic code is
discrete, so instead of the representation theory of Lie groups,
\textit{modular} representation theory is arguably the natural
mathematical home for such study.  The ``wobble pair'' phenomena might
be suggestive that symplectic geometry is relevant, i.e., thinking of
a codon---three base pairs---not as a vector in $(\mathbb{F}_4)^3$ but
as a vector in $(\mathbb{F}_2)^6$ adorned with a symplectic form. I
found a 64-dimensional (i.e., $64 = 4^3$ codons) representation of 
$\Sp(6,\mathbb{F}_2)$ over
the field $\mathbb{F}_2$ in which there are 21 minimal submodules for a
smaller symmetry group (i.e., 20 amino acids and 1 stop code);
moreover, the dimensions of these minimal submodules are just large
enough to account for the redundancy in the genetic code.

It would be interesting to work with, say, an undergraduate student
interested in math biology, teach them a bit of representation theory,
and work out some of the consequences.  Of course, this project, like
most of the work done on representation theory and the genetic code,
is highly speculative.  One complaint is that such work is ``mere
numerology'' which does not even succeed in giving the correct answer
for the standard genetic code: for instance, there is no example in
\cite{forger-1999,forger-1998} of a Lie group having precisely the
correct representation theory to give rise to the standard genetic
code, though there is a near-miss differing only in the symmetry
breaking at the last stage.
\end{description}

\subsection{MOOCs and adaptive learning}
\label{subsection:moocs}

In January~2013, I launched the first massive open online course
(MOOC) at The Ohio State University. My course was designed to cover
the same content as the local, in-person sections of Math~1151, the
first semester calculus course at OSU.  The largest component of this
MOOC consisted of a home-built adaptive learning platform I designed,
called MOOCulus, which delivers randomly-generated interactive
problems to students \cite{evans}.  

Beyond my $\sim 200$ lecture videos---some of which make use of
augmented reality---the MOOCulus platform has generated significant
amounts of data on the learning of calculus.  As of November 2013, we
have had 147k students enroll in either the Coursera or iTunes~U
version of my courses.  These enrollments have led to millions of
attempts, and over two million correct answers, being submitted to
MOOCulus.

The Division of Undergraduate Education at the NSF has funded a
Transforming Undergraduate Education in STEM (TUES) Type~1 award for
my proposal DUE--1245433 (``Interactive Textbook'') with Bart Snapp
and Herb Clemens.  The proposed activity will produce an online
platform making it easier for other instructors to build engaging
online courses like MOOCulus.  The goal is to build one of the
``customizable, sustainable platforms'' that online education requires
to succeed \cite{bowen2013higher}; parts of MOOCulus have already been
used for English courses as well \cite{gates-foundation-grant}.

Much previous work has been done on using adaptive learning systems to
teach college mathematics.  Examples of such are ALEKS
\cite{hagerty2005using,albert1999knowledge} or Knewton at Arizona
State \cite{parry2012big}, though both of these emphasize
pre-calculus, developmental coursework.  Some of these platforms use
Bayesian networks to estimate a student's current understanding
\cite{romero2010educational}.  Hidden Markov models (HMMs) have also
been used in educational data mining for clustering
\cite{shihdiscovery}.  These techniques and related ``machine
learning'' techniques can predict student grades based on only a few
assignments \cite{predict-grades}, even with incomplete data
\cite{Zafra201115020}, which make them especially useful for an
adaptive learning platform where not too much student data may be
available on which to base the initial predictions.

For MOOCulus, I chose to use a hidden Markov model to estimate student
understanding; exactly which problems are assigned to the student
depend on the output of the model.  Hidden Markov models can provide
strong predictions of student behavior; in one example, a hidden
Markov model---trained, as I propose here, via the Baum--Welch
algorithm---did a better job of predicting student behavior in an
e-Learning context than did a neural net \cite{anari2012intelligent}.
Nevertheless, exactly how best to train the MOOCulus model remains a
question.
\begin{question}
  What are the ``correct'' parameters for the hidden Markov model in
  MOOCulus?
\end{question}
Parameter estimation is an extremely well-studied topic in the theory
of Markov processes, e.g., \cite{MR202264} and \cite{MR123419}.
Before having access to student data, instructors often ``have to
provide appropriate values for the parameters in advance in order to
obtain good results/model and therefore, the user must possess a
certain amount of expertise in order to find the right
settings'' \cite{romero2010educational}.  This is a serious
usability problem with many adaptive learning systems---MOOCulus
included.

Determining appropriate values for these parameters is a usability
issue, and also one of the major research questions my work addresses.
The plan is to use the Baum--Welch algorithm from the GHMM library
\cite{ghmm} so improvements to the parameters can be made
automatically.  The overarching goal is that MOOCulus will improve
student outcomes in math courses, as has been shown to happen with
other online assessment systems \cite{angus2009does}.

\printbibliography

\end{document}

%%% Local Variables: 
%%% mode: latex
%%% TeX-master: t
%%% End: 
