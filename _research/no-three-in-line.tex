---
title: The no-three-in-line problem on a torus
status: unpublished
year: 2012
arxiv: 1203.6604v1
joint:
  - Andrew Groot
  - Deven Pandya
  - Bart Snapp
---

For a group $G$, let $T(G)$ denote the cardinality of the largest subset $S \subset G$ so that no three elements of $S$ are in the same coset of a cyclic subgroup.  Undergraduates Andrew Groot and Deven Pandya, advised by myself and my colleague Bart Snapp, considered the case $G = \mathbb{Z}/m\mathbb{Z} \times \mathbb{Z}/n\mathbb{Z}$, and showed that 
  \begin{align*}
    T(\Z_p \times \Z_{p^2}) &= 2p, \\
    T(\Z_p \times \Z_{pq})  &= p+1.
  \end{align*}
  This problem can also be formulated as a Gr\"obner basis question; after doing so, we computed $T(\Z_m \times \Z_n)$ for $2 \leq m \leq 7$ and $2 \leq n \leq 19$.

Thinking of a coset of a cyclic subgroup as a ``line,'' there is then connection with the usual ``no three in line problem.''  Paul Erd\H{o}s proved that for a prime $p$, one can place $p$ points on the $p\times p$ lattice in the plane \cite{MR41889}; the construction goes via a parabola modulo $p$.  Other more complicated constructions manage to place more points \cite{MR366817}.

My interest lately has been considering the question for other groups.  Although the no-three-in-line problem for $G = (\mathbb{Z}/p\mathbb{Z})^2$ can be considered as the $k$-arc problem from projective geometry \cite{MR554919}, the question is interesting for, say, $G = S_n$ or $G = A_n$ where, say, Bezout's theorem doesn't make sense anymore.
