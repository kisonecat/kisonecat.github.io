---
title: Bounded homotopy theory
status: published
joint:
  - Crichton Ogle
year: 2011
summary: Given a bounding class ${\mathcal B}$, we construct a bounded refinement ${\mathcal{B}K}(-)$ of Quillen's $K$-theory functor from rings to spaces. As defined, ${\mathcal{B}K}(-)$ is a functor from weighted rings to spaces, and is equipped with a comparison map $BK \to K$ induced by &ldquo;forgetting control.&rdquo; In contrast to the situation with $\mathcal{B}$-bounded cohomology, there is a functorial splitting ${\mathcal{B}K}(-) \simeq K(-) \times {\mathcal{B}K}^{rel}(-)$ where ${\mathcal{B}K}^{rel}(-)$ is the homotopy fiber of the comparison map.
mr: MR2962981
---

In controlled topology, notions like homotopy are refined to include a condition on their size, measured via a reference map to a metric space.  There are different versions of controlled topology in current use, including bounded control and continuous control. But there are other versions of control that are worth considering.

That is not so surprisingly: much of the success of geometric group theory comes by thinking about asymptotic invariants of a group as a metric space \cite{MR1253544}, but one can consider other asymptotic invariants for spaces.
\begin{question}
When does a CW complex have the homotopy type of a complex having polynomially many cells in dimension $n$?
\end{question}
This is a quantitative version of Wall's finiteness obstruction \cite{MR171284}.  Interestingly, a space might have Poincar\'e duality but not in a polynomially bounded sense, so there are analogies to be made between this project and the homology manifold projects.

Crichton Ogle has developed polynomially bounded cohomology \cite{MR2109110}.  I have collaborated with Ogle to work through the foundations of polynomially bounded---and more generally, $\mathcal{B}$-bounded---homotopy theory \cite{MR2962981}, which would be one framework within which the above question could be addressed. Specifically, given a bounding class $\mathcal{B}$, we construct a bounded refinement ${\mathcal{B} K}(-)$ of Quillen's $K$-theory functor from rings to spaces. As defined, ${\mathcal{B} K}(-)$ is a functor from weighted rings to spaces, and is equipped with a comparison map $BK \to K$ induced by ``forgetting control.''  In contrast to the situation with $\mathcal{B}$-bounded cohomology, there is a functorial splitting ${\mathcal{B} K}(-) \simeq K(-) \times {\mathcal{B} K}^{rel}(-)$ where ${\mathcal{B} K}^{rel}(-)$ is the homotopy fiber of the comparison map.

Finally, there is also an analogy to be made between ``weighted rings'' and the geometric modules of Quinn \cite{MR802791}, along with $\mathcal{B}$-bounded homotopy theory and controlled topology.  Much of our work in \cite{MR2962981} is in building an appropriate $\mathcal{B}$-bounded Waldhausen category.  At the point where one can define $\mathcal{B}$-bounded Waldhausen categories, why not go all the way and consider a $\mathcal{B}$-bounded model theory?

Christensen and Munkholm have placed continuous and bounded notions of control into a common, categorical framework \cite{MR1983017}; Higson--Pedersen--Roe have introduced a different framework unifying various kinds of coarse spaces, from an analytic perspective \cite{MR1451755}.  Weiss--Williams formulate some examples of control within Waldhausen categories with duality \cite{MR1644309}.  All of these notions could be unified into a single theory of \textit{controlled model categories}.  Anderson's homotopy theory for boundedly controlled topology \cite{MR953961} is a starting point.

There is some precedent for considering a controlled model category, namely the parametrized homotopy theory of May--Sigurdsson \cite{MR2271789}.  This latter theory, however, is more topological than geometric---the reference maps are not to metric spaces. Nevertheless, elucidating the precise relationship between the parametrized homotopy theory of May--Sigurdsson and the spectral cosheaves used in stratified surgery (see \cite{MR1308714}) would be very worthwhile.
