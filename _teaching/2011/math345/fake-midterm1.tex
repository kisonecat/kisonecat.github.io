\documentclass[12pt]{midterm}

\title{Fake Midterm 1}
\course{Math 345}
\date{October 2010}

\excludeversion{solutiontext}

\newcommand{\abs}[1]{\left|#1\right|}
\DeclareMathOperator{\realpart}{Re}
\DeclareMathOperator{\imagpart}{Im}

\setlength{\introductionwidth}{0.6\paperwidth}
\introduction{%
Music is the pleasure the human soul experiences from counting without being aware that it is counting.
\null\hfill---Leibniz
}

\instructions{%
\begin{enumerate}
\item Do not write your name above.
\item Calculators are forbidden (and useless, anyhow).
\item Look inside the exam before you are instructed to do so.
\item Give yourself have \textbf{48 minutes} for five problems on this fake exam.
\item Justify your answers.
\item Show your work.
\item Write your answers down.
\item Answer all questions.
\item To prevent fire, do not divide by zero.
\vfill
\end{enumerate}
}

\usepackage{xcolor}

\begin{document}
\begin{exam}

%%%%%%%%%%%%%%%%%%%%%%%%%%%%%%%%%%%%%%%%%%%%%%%%%%%%%%%%%%%%%%%%
\begin{problem}[360]
  Write down a truth table for the proposition
  $$
  \left( P \wedge (Q \Rightarrow R) \right) \Rightarrow \left(P \wedge Q \wedge R\right)
  $$
\end{problem}

\begin{solution}\begin{solutiontext}
\end{solutiontext}\end{solution}

%%%%%%%%%%%%%%%%%%%%%%%%%%%%%%%%%%%%%%%%%%%%%%%%%%%%%%%%%%%%%%%%
\begin{problem}[360]
  Consider the proposition:
  $$
  \left( P \vee Q \right) \Rightarrow \left( \left( \left(P \Rightarrow R\right) \wedge \left(Q \Rightarrow R\right) \right) \Rightarrow R \right)
  $$
  Is the proposition a tautology?  If not, explain why not.  If it is
  a tautology, use the method of conditional proof to prove that it is
  a tautology.  Do not use cases, and be careful not to skip any
  steps.
\end{problem}

\begin{solution}\begin{solutiontext}
\end{solutiontext}\end{solution}

%%%%%%%%%%%%%%%%%%%%%%%%%%%%%%%%%%%%%%%%%%%%%%%%%%%%%%%%%%%%%%%%
\begin{problem}[360]
  Consider the proposition:
  $$
  \neg \left( P \Rightarrow \left(\neg P\right) \right)
  $$
  Is the proposition a tautology?  If not, explain why not.  If it is
  a tautology, prove it.  
\end{problem}

\begin{solution}\begin{solutiontext}
\end{solutiontext}\end{solution}

%%%%%%%%%%%%%%%%%%%%%%%%%%%%%%%%%%%%%%%%%%%%%%%%%%%%%%%%%%%%%%%%
\begin{problem}[360]
  Consider the proposition:
  $$
  (P \vee Q) \Rightarrow (P \Rightarrow Q)
  $$
  Is the proposition a tautology?  If not, explain why not.  If it is
  a tautology, use the method of conditional proof to prove that it is
  a tautology.  Do not use cases, and be careful not to skip any
  steps.
\end{problem}

\begin{solution}\begin{solutiontext}
\end{solutiontext}\end{solution}

%%%%%%%%%%%%%%%%%%%%%%%%%%%%%%%%%%%%%%%%%%%%%%%%%%%%%%%%%%%%%%%%
\begin{problem}[360]
  Consider the proposition:
  $$
  \left( P \Rightarrow (Q \Rightarrow P) \right) \Rightarrow Q
  $$
  Is the proposition a tautology?  If not, explain why not.  If it is
  a tautology, use the method of conditional proof to prove that it is
  a tautology.  Do not use cases, and be careful not to skip any
  steps.
\end{problem}

\begin{solution}\begin{solutiontext}
\end{solutiontext}\end{solution}

%%%%%%%%%%%%%%%%%%%%%%%%%%%%%%%%%%%%%%%%%%%%%%%%%%%%%%%%%%%%%%%%
\begin{problem}[360]
  Consider the proposition:
  $$
  \left(Q \Rightarrow R\right) \Rightarrow \left( (P \Rightarrow Q) \Rightarrow (P \Rightarrow R) \right)
  $$
  Is the proposition a tautology?  If not, explain why not.  If it is
  a tautology, use the method of conditional proof to prove that it is
  a tautology.  Do not use cases, and be careful not to skip any
  steps.
\end{problem}

\begin{solution}\begin{solutiontext}
\end{solutiontext}\end{solution}


%%%%%%%%%%%%%%%%%%%%%%%%%%%%%%%%%%%%%%%%%%%%%%%%%%%%%%%%%%%%%%%%
\begin{problem}[360]
  Consider the proposition:
  $$
  \left( (P \Rightarrow Q) \vee (P \Rightarrow R) \right) \Rightarrow \left(P \Rightarrow \left(Q \vee R\right) \right)
  $$
  Is the proposition a tautology?  If not, explain why not.  If it is
  a tautology, use the method of conditional proof to prove that it is
  a tautology.  Do not use cases, and be careful not to skip any
  steps.
\end{problem}

\begin{solution}\begin{solutiontext}
\end{solutiontext}\end{solution}

%%%%%%%%%%%%%%%%%%%%%%%%%%%%%%%%%%%%%%%%%%%%%%%%%%%%%%%%%%%%%%%%
\begin{problem}[360]
  Write down the contrapositive of the conditional sentence:
  \begin{quote}
    If $x > 2$, then $x^2 > 4$.
  \end{quote}
\end{problem}

\begin{solution}\begin{solutiontext}
\end{solutiontext}\end{solution}


%%%%%%%%%%%%%%%%%%%%%%%%%%%%%%%%%%%%%%%%%%%%%%%%%%%%%%%%%%%%%%%%
\begin{problem}[360]
  Write down the contrapositive of the conditional sentence:
  \begin{quote}
    If $x > 2$ and $x < 17$, then $x^2 > 4$ or $x^2 < 289$.
  \end{quote}
\end{problem}

\begin{solution}\begin{solutiontext}
\end{solutiontext}\end{solution}

%%%%%%%%%%%%%%%%%%%%%%%%%%%%%%%%%%%%%%%%%%%%%%%%%%%%%%%%%%%%%%%%
\begin{problem}[360]
  Write down the contrapositive of the conditional sentence:
  $$
  (P \vee Q) \Rightarrow (R \wedge S).
  $$
\end{problem}

\begin{solution}\begin{solutiontext}
\end{solutiontext}\end{solution}

%%%%%%%%%%%%%%%%%%%%%%%%%%%%%%%%%%%%%%%%%%%%%%%%%%%%%%%%%%%%%%%%
\begin{problem}[360]
  Consider the three propositions
  \begin{eqnarray}
    P \wedge (Q  \vee \neg (R \wedge S)) \label{prop1} \\
    (P \wedge Q) \vee (P \wedge ((\neg R) \vee (\neg S))) \label{prop2} \\
    P \wedge (Q  \vee (\neg R) \vee (\neg S)) \label{prop3} 
  \end{eqnarray}
  Which of these propositions are logically equivalent to which other
  propositions?  Provide justifications for any claims you make; in
  particular, if you claim that two propositions are logically
  equivalent, you must prove this, and if you claim that they are \textit{not} equivalent, you must explain why not.
\end{problem}

\begin{solution}\begin{solutiontext}
\end{solutiontext}\end{solution}

%%%%%%%%%%%%%%%%%%%%%%%%%%%%%%%%%%%%%%%%%%%%%%%%%%%%%%%%%%%%%%%%
\begin{problem}[360]
  Consider the four propositions
  \begin{eqnarray}
    P \wedge \neg (Q \Rightarrow R) \\
    P \wedge \left( (\neg Q) \vee R) \right) \\
    P \wedge Q \wedge \neg R \\
    \left( P \wedge \left(\neg Q\right) \right) \vee \left(P \wedge R\right) 
  \end{eqnarray}
  Which of these propositions are logically equivalent to which other
  propositions?  Provide justifications for any claims you make; in
  particular, if you claim that two propositions are logically
  equivalent, you must prove this, and if you claim that they are \textit{not} equivalent, you must explain why not.
\end{problem}

\begin{solution}\begin{solutiontext}
\end{solutiontext}\end{solution}

%%%%%%%%%%%%%%%%%%%%%%%%%%%%%%%%%%%%%%%%%%%%%%%%%%%%%%%%%%%%%%%%
\begin{problem}[360]
  Write down a proposition logically equivalent to
  $$
  \neg \left( \left( P \vee Q \right)  \Rightarrow \left(Q \wedge R\right) \right)
  $$
  without using the symbol ``$\Rightarrow$.''
\end{problem}

\begin{solution}\begin{solutiontext}
\end{solutiontext}\end{solution}

%%%%%%%%%%%%%%%%%%%%%%%%%%%%%%%%%%%%%%%%%%%%%%%%%%%%%%%%%%%%%%%%
\begin{problem}[360]
  Is the proposition
  $$
  \exists x \in \mathbb{R}\,\forall y \in \mathbb{R}\, (x + y = 0)
  $$
  true or false?  If it is false, give a counterexample.  If it is true, give a proof.
\end{problem}

\begin{solution}\begin{solutiontext}
\end{solutiontext}\end{solution}

%%%%%%%%%%%%%%%%%%%%%%%%%%%%%%%%%%%%%%%%%%%%%%%%%%%%%%%%%%%%%%%%
\begin{problem}[360]
  Is the proposition
  $$
  \exists x \in \mathbb{R}\,\forall y \in \mathbb{R}\, (xy = 1)
  $$
  true or false?  If it is false, give a counterexample.  If it is true, give a proof.
\end{problem}

\begin{solution}\begin{solutiontext}
\end{solutiontext}\end{solution}

%%%%%%%%%%%%%%%%%%%%%%%%%%%%%%%%%%%%%%%%%%%%%%%%%%%%%%%%%%%%%%%%
\begin{problem}[360]
  Is the proposition
  $$
  \forall x \in \mathbb{R}\,\exists y \in \mathbb{R}\, (xy = 1)
  $$
  true or false?  If it is false, give a counterexample.  If it is true, give a proof.
\end{problem}

\begin{solution}\begin{solutiontext}
\end{solutiontext}\end{solution}


%%%%%%%%%%%%%%%%%%%%%%%%%%%%%%%%%%%%%%%%%%%%%%%%%%%%%%%%%%%%%%%%
\begin{problem}[360]
  Is the proposition
  $$
  \exists x \in \mathbb{R}\, \forall y \in \mathbb{R}\, (x > y)
  $$
  true or false?  If it is false, give a counterexample.  If it is true, give a proof.
\end{problem}

\begin{solution}\begin{solutiontext}
\end{solutiontext}\end{solution}


%%%%%%%%%%%%%%%%%%%%%%%%%%%%%%%%%%%%%%%%%%%%%%%%%%%%%%%%%%%%%%%%
\begin{problem}[360]
  Is the proposition
  $$
  \forall x \in \mathbb{R}\, \forall y \in \mathbb{R}\, \exists z \in \mathbb{R}\, \left( \left( x < z \right) \wedge (z < y) \right)
  $$
  true or false?  If it is false, give a counterexample.  If it is true, give a proof.
\end{problem}

\begin{solution}\begin{solutiontext}
\end{solutiontext}\end{solution}

%%%%%%%%%%%%%%%%%%%%%%%%%%%%%%%%%%%%%%%%%%%%%%%%%%%%%%%%%%%%%%%%
\begin{problem}[360]
  Is the proposition
  $$
  \forall x \in \mathbb{R}\, \forall y \in \mathbb{R}\, \left( \left(x < y\right) \Rightarrow \exists z \in \mathbb{R}\, \left( \left( x < z \right) \wedge (z < y) \right) \right)
  $$
  true or false?  If it is false, give a counterexample.  If it is true, give a proof.
\end{problem}

\begin{solution}\begin{solutiontext}
\end{solutiontext}\end{solution}

%%%%%%%%%%%%%%%%%%%%%%%%%%%%%%%%%%%%%%%%%%%%%%%%%%%%%%%%%%%%%%%%
\begin{problem}[360]
  Is the proposition
  $$
  \forall x \in \mathbb{R}\, \exists y \in \mathbb{R}\, \forall z \in \mathbb{R}\, \left( x+y < z^2 \right)
  $$
  true or false?  If it is false, give a counterexample.  If it is
  true, give a proof.  You may use the fact that the square of a real
  number is nonnegative.
\end{problem}

\begin{solution}\begin{solutiontext}
\end{solutiontext}\end{solution}

%%%%%%%%%%%%%%%%%%%%%%%%%%%%%%%%%%%%%%%%%%%%%%%%%%%%%%%%%%%%%%%%
\begin{problem}[360]
  Let $P(x,y)$ be a proposition with free variables $x,y$.  Is the statement
  $$
  \left( \exists x\,\forall y\, P(x,y) \right) \Rightarrow
  \left( \forall y\,\exists x\, P(x,y) \right)
  $$
  true or false? If it is false, give a counterexample.  If it is true, give a proof.
\end{problem}

\begin{solution}\begin{solutiontext}
\end{solutiontext}\end{solution}

%%%%%%%%%%%%%%%%%%%%%%%%%%%%%%%%%%%%%%%%%%%%%%%%%%%%%%%%%%%%%%%%
\begin{problem}[360]
  Consider the following proposition:
  \begin{quote}
    If $x$ and $y$ are irrational, then $x + y$ is irrational.
  \end{quote}
  If the proposition is true, prove it; if not, give a counterexample.
\end{problem}

\begin{solution}\begin{solutiontext}
\end{solutiontext}\end{solution}

%%%%%%%%%%%%%%%%%%%%%%%%%%%%%%%%%%%%%%%%%%%%%%%%%%%%%%%%%%%%%%%%
\begin{problem}[360]
  Consider the following proposition:
  \begin{quote}
    If $x$ and $y$ are rational, then $x + 2y$ is rational.
  \end{quote}
  If the proposition is true, prove it; if not, give a counterexample.
\end{problem}

\begin{solution}\begin{solutiontext}
\end{solutiontext}\end{solution}

%%%%%%%%%%%%%%%%%%%%%%%%%%%%%%%%%%%%%%%%%%%%%%%%%%%%%%%%%%%%%%%%
\begin{problem}[360]
  Consider the following proposition:
  \begin{quote}
    Let $a,b,c \in \mathbb{Z}$.  If $a$ divides $b$ and $a$ divides $c$, then $a$ divides $17a + 13b$.
  \end{quote}
  If the proposition is true, prove it; if not, give a counterexample.
\end{problem}

\begin{solution}\begin{solutiontext}
\end{solutiontext}\end{solution}

%%%%%%%%%%%%%%%%%%%%%%%%%%%%%%%%%%%%%%%%%%%%%%%%%%%%%%%%%%%%%%%%
\begin{problem}[360]
  Consider the following proposition:
  \begin{quote}
    Let $a \in \mathbb{Z}$.  The integer $a^3+3 a^2+2 a$ is divisible by three.
  \end{quote}
  If the proposition is true, prove it; if not, give a counterexample.
\end{problem}

\begin{solution}\begin{solutiontext}
\end{solutiontext}\end{solution}

%%%%%%%%%%%%%%%%%%%%%%%%%%%%%%%%%%%%%%%%%%%%%%%%%%%%%%%%%%%%%%%%
\begin{problem}[360]
  Consider the following proposition:
  \begin{quote}
    Let $a,b \in \mathbb{Z}$.  If $a$ and $b$ are both even or both odd, then $a + b$ is odd.
  \end{quote}
  If the proposition is true, prove it; if not, give a counterexample.
\end{problem}

\begin{solution}\begin{solutiontext}
\end{solutiontext}\end{solution}

%%%%%%%%%%%%%%%%%%%%%%%%%%%%%%%%%%%%%%%%%%%%%%%%%%%%%%%%%%%%%%%%
\begin{problem}[360]
  Consider the following proposition:
  \begin{quote}
    Let $a,b \in \mathbb{Z}$.  If $a$ and $b$ are both odd, then $ab$ is odd.
  \end{quote}
  If the proposition is true, prove it; if not, give a counterexample.
\end{problem}

\begin{solution}\begin{solutiontext}
\end{solutiontext}\end{solution}

%%%%%%%%%%%%%%%%%%%%%%%%%%%%%%%%%%%%%%%%%%%%%%%%%%%%%%%%%%%%%%%%
\begin{problem}[360]
  Consider the following proposition:
  \begin{quote}
    Let $a \in \mathbb{Z}$.  If $a$ is even, then $a^{6}$ is even.
  \end{quote}
  If the proposition is true, prove it; if not, give a counterexample.
\end{problem}

\begin{solution}\begin{solutiontext}
\end{solutiontext}\end{solution}

%%%%%%%%%%%%%%%%%%%%%%%%%%%%%%%%%%%%%%%%%%%%%%%%%%%%%%%%%%%%%%%%
\begin{problem}[360]
  Consider the following proposition:
  \begin{quote}
    Let $a,b \in \mathbb{Z}$.  If $a+b$ is odd, then $a$ is even or $b$ is even.
  \end{quote}
  If the proposition is true, prove it; if not, give a counterexample.
\end{problem}

\begin{solution}\begin{solutiontext}
\end{solutiontext}\end{solution}


%%%%%%%%%%%%%%%%%%%%%%%%%%%%%%%%%%%%%%%%%%%%%%%%%%%%%%%%%%%%%%%%
% De Morgan's laws
% distributive laws
% Propositional calculus

%%%%%%%%%%%%%%%%%%%%%%%%%%%%%%%%%%%%%%%%%%%%%%%%%%%%%%%%%%%%%%%%
\begin{problem}[360]
  Let $P(x)$ and $Q(x)$ be propositions, and consider
  \begin{eqnarray}
    \left( \left( \forall x \, P(x) \right) \wedge  \left( \forall x \, Q(x) \right)  \right) & \Rightarrow & \left(\forall x \left( P(x) \wedge Q(x) \right) \right) \label{wrong-answer} \\
    \left( \left( \exists x \, P(x) \right) \wedge  \left( \exists x \, Q(x) \right)  \right) & \Rightarrow & \left(\exists x \left( P(x) \wedge Q(x) \right) \right) \label{right-answer}
  \end{eqnarray}
  Is (\ref{wrong-answer}) true or false?  Is (\ref{right-answer}) true or false?  Explain your answer.
\end{problem}

\begin{solution}\begin{solutiontext}
\end{solutiontext}\end{solution}


\end{exam}
\end{document}
