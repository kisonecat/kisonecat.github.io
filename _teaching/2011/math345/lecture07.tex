\documentclass[12pt]{handout}

\title{Lecture 7: Bound variables}
\author{Jim Fowler}
\course{Math 345}
\date{Monday, October  4, 2010}



\begin{document}
\maketitle

\section*{Textbook}

This lecture discusses section 3 of the textbook.

\section*{Homework} 

The homework is due Wednesday, October  6, 2010.

From Section 3 of the textbook, do exercises 2, 4, and 7.

\section*{Have some people come up and present solutions to homework from before}

\section*{Bound and free variables}

In the statement $P(x)$, the variable $x$ is free.

In the statement $\forall x \, P(x)$, the variable $x$ is bound.  In other words, ``$\forall x \, P(x)$'' is not a statement about $x$.

Possible (but poor style) to use the same variable name for a bound variable and a free variable.

The scope of a quantifier is the part of the proposition it applies to.

\section*{Example}

For all integers $x$, $x$ is even or odd.

(For all integers $x$, $x$ is even) or (For all integers $x$, $x$ is odd).

\section*{Nested quantifiers}

Does the order of quantifiers matter?

$P(x,y)$ means ``$x$ wants to date $y$.''

Then, $\exists y \forall x P(y,x)$ means there is someone who wants to date everybody.

Then, $\forall x \exists y P(y,x)$ means everybody has someone interested in them.

These are different.

How does $\forall x \exists y$ differ from $\exists y \forall x$?  The latter implies the former.

\section*{Algebraic example}

think about sum and products

\end{document}
