\documentclass[12pt]{midterm}

\title{Fake Midterm 2}
\course{Math 345}
\date{November 2010}

\excludeversion{solutiontext}

\newcommand{\abs}[1]{\left|#1\right|}
\DeclareMathOperator{\realpart}{Re}
\DeclareMathOperator{\imagpart}{Im}

\setlength{\introductionwidth}{0.6\paperwidth}
\introduction{%
Practical application is found by not looking for it, and one can say that the whole progress of civilization rests on that principle.
\null\hfill---Hadamard
}

\instructions{%
\begin{enumerate}
\item Write your name above.
\item Calculators are forbidden.
\item Look inside the fake exam before taking the real exam.
\item Justify your answers.
\item Show your work.
\item Write your answers down to practice.
\item Answer all questions.
\item To prevent fire, do not divide by zero.
\vfill
\end{enumerate}
}

\usepackage{xcolor}

\begin{document}
\begin{exam}

%%%%%%%%%%%%%%%%%%%%%%%%%%%%%%%%%%%%%%%%%%%%%%%%%%%%%%%%%%%%%%%%
\begin{problem}[360]
Is it the case for all integers $x$ that
$$
12 \mbox{ divides } x(x+1)(x+2)(x+3)?
$$
If so, prove it.  If not, provide a counterexample.
\end{problem}

\begin{solution}\begin{solutiontext}
\end{solutiontext}\end{solution}

%%%%%%%%%%%%%%%%%%%%%%%%%%%%%%%%%%%%%%%%%%%%%%%%%%%%%%%%%%%%%%%%
\begin{problem}[360]
Is it the case for all integers $x$ that
$$
16 \mbox{ divides } 17^n - 1?
$$
If so, prove it.  If not, provide a counterexample.
\end{problem}

\begin{solution}\begin{solutiontext}
\end{solutiontext}\end{solution}

%%%%%%%%%%%%%%%%%%%%%%%%%%%%%%%%%%%%%%%%%%%%%%%%%%%%%%%%%%%%%%%%
\begin{problem}[360]
  Suppose $S \subset \mathbb{Z}$ and that for all $x \in S$, $x >
  -10$.  Does $S$ have a least element?  If so, prove it.  If not,
  give a counterexample.
\end{problem}

\begin{solution}\begin{solutiontext}
\end{solutiontext}\end{solution}

%%%%%%%%%%%%%%%%%%%%%%%%%%%%%%%%%%%%%%%%%%%%%%%%%%%%%%%%%%%%%%%%
\begin{problem}[360]
  Suppose $a \equiv b \mod m$ and $c \equiv d \mod m$.  Is it the case
  that $ac \equiv bd \mod m$?  If so, prove it.  If not, give a
  counterexample.
\end{problem}

\begin{solution}\begin{solutiontext}
\end{solutiontext}\end{solution}

%%%%%%%%%%%%%%%%%%%%%%%%%%%%%%%%%%%%%%%%%%%%%%%%%%%%%%%%%%%%%%%%
\begin{problem}[360]
  Suppose $a \equiv b \mod m$ and $c \equiv d \mod m$.  Is it the case
  that $a+c \equiv b+d \mod m$?  If so, prove it.  If not, give a
  counterexample.
\end{problem}

\begin{solution}\begin{solutiontext}
\end{solutiontext}\end{solution}

%%%%%%%%%%%%%%%%%%%%%%%%%%%%%%%%%%%%%%%%%%%%%%%%%%%%%%%%%%%%%%%%
\begin{problem}[360]
  Suppose $a \equiv b \mod m$ and $c \equiv d \mod m$.  Is it the case
  that $a^c \equiv b^d \mod m$?  If so, prove it.  If not, give a
  counterexample.
\end{problem}

\begin{solution}\begin{solutiontext}
\end{solutiontext}\end{solution}

%%%%%%%%%%%%%%%%%%%%%%%%%%%%%%%%%%%%%%%%%%%%%%%%%%%%%%%%%%%%%%%%
\begin{problem}[360]
Find a polynomial $f(n)$ so that
$$
f(n) = \sum_{k=1}^n \left( k^2 + k \right).
$$
\end{problem}

\begin{solution}\begin{solutiontext}
\end{solutiontext}\end{solution}

%%%%%%%%%%%%%%%%%%%%%%%%%%%%%%%%%%%%%%%%%%%%%%%%%%%%%%%%%%%%%%%%
\begin{problem}[360]
Find a polynomial $f(n)$ so that
$$
f(n) = \sum_{k=1}^n k^3
$$
\end{problem}

\begin{solution}\begin{solutiontext}
\end{solutiontext}\end{solution}

%%%%%%%%%%%%%%%%%%%%%%%%%%%%%%%%%%%%%%%%%%%%%%%%%%%%%%%%%%%%%%%%
\begin{problem}[360]
Prove by induction that for all $n \in \mathbb{N}$,
$$
\binom{n}{2} = \frac{(n)(n-1)}{2}.
$$
(I thank Marilyn Rayner for correcting an error in a previous version of this problem.)
\end{problem}

\begin{solution}\begin{solutiontext}
\end{solutiontext}\end{solution}

%%%%%%%%%%%%%%%%%%%%%%%%%%%%%%%%%%%%%%%%%%%%%%%%%%%%%%%%%%%%%%%%
\begin{problem}[360]
Prove by induction that for all $n \in \mathbb{N}$ and for all integers $x$ and $y$,
$$
\frac{x^n - y^n}{x-y}
$$
is an integer.
\end{problem}

\begin{solution}\begin{solutiontext}
\end{solutiontext}\end{solution}

%%%%%%%%%%%%%%%%%%%%%%%%%%%%%%%%%%%%%%%%%%%%%%%%%%%%%%%%%%%%%%%%
\begin{problem}[360]
Prove by induction that, for all $n \in \mathbb{N}$,
$$
\sum_{k=0}^n (2k+1)
$$
is a perfect square.  (I thank Marilyn Rayner for correcting an error in a previous version of this problem.)
\end{problem}

\begin{solution}\begin{solutiontext}
\end{solutiontext}\end{solution}

%%%%%%%%%%%%%%%%%%%%%%%%%%%%%%%%%%%%%%%%%%%%%%%%%%%%%%%%%%%%%%%%
\begin{problem}[360]
Prove by induction that, for all $n \in \mathbb{N}$,
$$
\sqrt{2 \sqrt{3 \sqrt{4 \cdots \sqrt{(n-1) \sqrt{n}}}}} < 3.
$$
\end{problem}

\begin{solution}\begin{solutiontext}
\end{solutiontext}\end{solution}

%%%%%%%%%%%%%%%%%%%%%%%%%%%%%%%%%%%%%%%%%%%%%%%%%%%%%%%%%%%%%%%%
\begin{problem}[360]
Prove by induction that, for all $n \in \mathbb{N}$,
$$
\sum_{k=0}^n \binom{n+k}{k} \frac{1}{2^k} = 2^n.
$$
(This problem might be very hard)
\end{problem}

\begin{solution}\begin{solutiontext}
\end{solutiontext}\end{solution}


%%%%%%%%%%%%%%%%%%%%%%%%%%%%%%%%%%%%%%%%%%%%%%%%%%%%%%%%%%%%%%%%
\begin{problem}[360]
  Let $F_n$ be the Fibonacci numbers.  For wich values of $n$ does
  $F_n$ end in a zero?
\end{problem}

\begin{solution}\begin{solutiontext}
\end{solutiontext}\end{solution}

%%%%%%%%%%%%%%%%%%%%%%%%%%%%%%%%%%%%%%%%%%%%%%%%%%%%%%%%%%%%%%%%
\begin{problem}[360]
  Let $F_n$ be the Fibonacci numbers (here, $F_1 = 1$ and $F_2 = 1$,
  and $F_{n+2} = F_{n+1} + F_{n}$).  Suppose $x$ is a real number for
  which $x^2 = 1 - x$.  Is it the case that $x^{100} = F_{99} -
  F_{100} x$?  If so, prove it.  (This is a situation where you might
  want to prove a much stronger statement by induction).
\end{problem}

\begin{solution}\begin{solutiontext}
\end{solutiontext}\end{solution}

%%%%%%%%%%%%%%%%%%%%%%%%%%%%%%%%%%%%%%%%%%%%%%%%%%%%%%%%%%%%%%%%
\begin{problem}[360]
  Define a sequence $G_n$ so that $G_1 = 1$, $G_2 = 1$, $G_3 = 1$, and
  $G_{n+3} = G_{n+2} + G_{n+1} + G_{n}$.  For which $n$ is $G_n$ even?
  Prove your claim by using induction.
\end{problem}

\begin{solution}\begin{solutiontext}
\end{solutiontext}\end{solution}

%%%%%%%%%%%%%%%%%%%%%%%%%%%%%%%%%%%%%%%%%%%%%%%%%%%%%%%%%%%%%%%%
\begin{problem}[360]
Prove that there are infinitely many prime numbers.
\end{problem}

\begin{solution}\begin{solutiontext}
\end{solutiontext}\end{solution}

%%%%%%%%%%%%%%%%%%%%%%%%%%%%%%%%%%%%%%%%%%%%%%%%%%%%%%%%%%%%%%%%
\begin{problem}[360]
  State and prove the binomial theorem.
\end{problem}

\begin{solution}\begin{solutiontext}
\end{solutiontext}\end{solution}

%%%%%%%%%%%%%%%%%%%%%%%%%%%%%%%%%%%%%%%%%%%%%%%%%%%%%%%%%%%%%%%%
\begin{problem}[360]
  Use the Binomial theorem to expand $(1+x)^{8}$.
\end{problem}

\begin{solution}\begin{solutiontext}
\end{solutiontext}\end{solution}

%%%%%%%%%%%%%%%%%%%%%%%%%%%%%%%%%%%%%%%%%%%%%%%%%%%%%%%%%%%%%%%%
\begin{problem}[360]
  Show that for every integer $x \geq 2$, there is a prime number $p$
  so that $p$ divides $x$.
\end{problem}



% Complete induction
% Prime numbers
% Pascal's triangle
% Patterns in Pascal's triangle
% Binomial theorem
% Congruences (``mod'' arithmetic)

\end{exam}
\end{document}
