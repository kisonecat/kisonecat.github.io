\documentclass[12pt]{handout}

\title{Lecture 18: Pascal's triangle}
\author{Jim Fowler}
\course{Math 345}
\date{Monday, October 25, 2010}


\begin{document}
\maketitle

\section*{Textbook}

This lecture discusses section 5 of the textbook.

\section*{Homework} 

The homework is due Wednesday, October 27, 2010.
From Section 5 of the textbook, do exercises 9 and 10.

\section*{Midterm results}

The average score on the final exam was 1418/1800, or 78\%, \\
with a standard deviation of 230. \\
The median score on the final exam was 1420/1800, \\
which is close to the mean \\
which suggests a reasonable distribution.

the distribution is posted on the website.

let me please remind you that \\
neither your worth as a human being \\
nor as a mathematician \\
is connected to your performance on exams.

in fact, it is the opposite. \\
you ought not take your successes too seriously \\
nor your failures too devastatingly.

for, this too shall pass.  how can we make someone dissatisifed with
their work, yet always satisfied with working?  A joy it will be one
day, perhaps, to remember even this.  Sic transit gloria mundi.

\section*{summation notation}

\section*{definiition of pascal's triangle}

\section*{what patterns do you see?}

\section*{what patterns can you prove?}

\section*{binomial theorem examples}

\end{document}
