\documentclass[12pt]{handout}
\usepackage{nopageno}
\usepackage{add-copyright}

\def\thefootnote{\fnsymbol{footnote}}

\author{Jim Fowler}
\course{Math 345}
\title{Problem Set 3}
\date{Due Wednesday, September 29, 2010}

\begin{document}
\maketitle










\subsection*{From Section 2 of the textbook}



\begin{description}

\item[On page 13 of the textbook] do Exercise 14.

\item[On page 14 of the textbook] do Exercise 15.

\item[On page 16 of the textbook] do Exercise 17.

\end{description}









\subsection*{Feel free to do additional problems}
The listed homework is a \textit{lower bound} on the number of
problems you should be doing: doing more homework is a great idea!  It
is also a good idea to do more than just the problems from the book:
try your hand at inventing your own problems!  Try to push the limits
of the techniques we learn.  See what you can do with the mathematical
machinery.



\subsection*{Textbook}
Be sure to use \textit{The Fundamentals of Higher Mathematics} by Neil Falkner, Autumn 2010 edition; other editions will number the problems differently.




\end{document}
