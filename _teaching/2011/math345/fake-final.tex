\documentclass[12pt]{midterm}

\title{Fake Final}
\course{Math 345}
\date{December 2010}

\excludeversion{solutiontext}

\newcommand{\abs}[1]{\left|#1\right|}
\DeclareMathOperator{\realpart}{Re}
\DeclareMathOperator{\imagpart}{Im}

\setlength{\introductionwidth}{0.6\paperwidth}
\introduction{%
Das Unendliche hat wie keine andere Frage von jeher so tief das
Gem\"ut der Menschen bewegt. \textit{The infinite, has, like no other
  question, moved the human mind so deeply.} \\
\null\hfill---David Hilbert
}

\instructions{%
\begin{enumerate}
\item Write your name above.
\item Calculators are forbidden.
\item Look inside the fake exam before taking the real exam.
\item Justify your answers.
\item Show your work.
\item Write your answers down to practice.
\item Answer all questions.
\item To prevent fire, do not divide by zero.
\vfill
\end{enumerate}
}

\usepackage{xcolor}

\begin{document}
\begin{exam}

%  Propositional calculus
%  Truth tables
%  De Morgan's laws
%  Distributive laws
%  Tautology
%  Proof by contradiction
%  Converse of a conditional proposition
%  Contrapositives
%  Quantifiers
%  Bound and free variables
%  Nested quantifiers
%  Even and odd numbers
%  Rational and irrational numbers
%  Divisibility
%  Induction
%  Complete induction
%  Least elements
%  Divisibility
%  Prime numbers
%  Infinitude of the primes
%  Telescoping sums
%  Fibonacci numbers
%  Pascal's triangle
%  Patterns in Pascal's triangle
%  Binomial theorem
%  Proof of Binomial theorem
%  $\displaystyle\binom{n+1}{k} = \displaystyle\binom{n}{k-1} + \displaystyle\binom{n}{k}$
%  Congruences
%  Set theory
%  Set builder notation
%  Intersections, unions, subsets
%  Set theory and de Morgan's laws
%  Functions
%  Surjections, injections, bijections
%  Inverse functions
%  Compositions of functions
%  Families of sets
%  Binomial coefficients in combinatorics
%  Infinite sets 
%  Diagonalization 
%  Countability 

%%%%%%%%%%%%%%%%%%%%%%%%%%%%%%%%%%%%%%%%%%%%%%%%%%%%%%%%%%%%%%%%
%  Functions
\begin{problem}[350]
  Let $A$ be the set of English words, and $B$ be the set of words
  (nonsense or not) made up of (possibly zero!) letters in the Latin
  alphabet.  Consider the function $f : A \to B$ which sends an
  English word to the word without its last letter.

  Is $f$ injective?  Is $f$ surjective?
\end{problem}

\begin{solution}\begin{solutiontext}
\end{solutiontext}\end{solution}

%%%%%%%%%%%%%%%%%%%%%%%%%%%%%%%%%%%%%%%%%%%%%%%%%%%%%%%%%%%%%%%%
\begin{problem}[350]
  Let $A$ be the set of all strings of at least 2 letters in the Latin
  alphabet.  Let $r : A \to A$ be the function which reverses its
  input (so $r(\mbox{hello}) = \mbox{olleh})$ and $s : A \to A$ be the
  function which takes the first letter and makes it the last letter (so $s(\mbox{hello})
  = \mbox{elloh})$).

  By applying $r$ and $s$ repeatedly, is it possible to transform
  $\mbox{game}$ into $\mbox{mage}$?  Is it possible to transform
  $\mbox{pets}$ to $\mbox{pest}$?  Is it possible to transform
  $\mbox{maple}$ into $\mbox{ample}$?
\end{problem}

\begin{solution}\begin{solutiontext}
\end{solutiontext}\end{solution}


%%%%%%%%%%%%%%%%%%%%%%%%%%%%%%%%%%%%%%%%%%%%%%%%%%%%%%%%%%%%%%%%
%  Even and odd numbersik
\begin{problem}[350]
Let $A = \{ x \in \mathbb{Z} : \mbox{$x$ is even} \}$ and $B = \{ x
\in \mathbb{Z} : \mbox{$x$ is odd} \}$.  Describe a bijection $f : A
\to B$.

Be sure to justify your answer completely.
\end{problem}

\begin{solution}\begin{solutiontext}
\end{solutiontext}\end{solution}

%%%%%%%%%%%%%%%%%%%%%%%%%%%%%%%%%%%%%%%%%%%%%%%%%%%%%%%%%%%%%%%%
%  Complete induction
\begin{problem}[350]
  Use complete induction to prove that every natural number bigger
  than one is prime or can be written as a product of prime numbers.
\end{problem}

\begin{solution}\begin{solutiontext}
\end{solutiontext}\end{solution}

%%%%%%%%%%%%%%%%%%%%%%%%%%%%%%%%%%%%%%%%%%%%%%%%%%%%%%%%%%%%%%%%
%  Converse of a conditional proposition
\begin{problem}[350]
Consider the statement: if $f : A \to A$ is injective, then $f$ is
surjective.  What is the converse of this statement?  The
contrapositive of this statement?  If $A$ is a finite set, which of
these statements are true?
\end{problem}

\begin{solution}\begin{solutiontext}
\end{solutiontext}\end{solution}

%%%%%%%%%%%%%%%%%%%%%%%%%%%%%%%%%%%%%%%%%%%%%%%%%%%%%%%%%%%%%%%%
%  Quantifiers
\begin{problem}[350]
  Let $f : \mathbb{R} \to \mathbb{R}$, and suppose $\exists x \in
  \mathbb{R}, \forall y \in \mathbb{R} \exists z \in \mathbb{R} f(x +
  z) = y$.  Does it follow that $f$ is surjective?
\end{problem}

\begin{solution}\begin{solutiontext}
\end{solutiontext}\end{solution}

%%%%%%%%%%%%%%%%%%%%%%%%%%%%%%%%%%%%%%%%%%%%%%%%%%%%%%%%%%%%%%%%
%  Inverse functions
\begin{problem}[350]
Suppose $f : \mathbb{N} \to \mathbb{N}$ is given by
$$
f(n) = \begin{cases}
n+1 & \mbox{if $n$ is odd} \\
n-1 & \mbox{if $n$ is even}
\end{cases}
$$
Is $f$ bijective?  If so, find an inverse function.
\end{problem}

\begin{solution}\begin{solutiontext}
\end{solutiontext}\end{solution}

%%%%%%%%%%%%%%%%%%%%%%%%%%%%%%%%%%%%%%%%%%%%%%%%%%%%%%%%%%%%%%%%
%  Surjections, injections, bijections
\begin{problem}[350]
Suppose $A,B,C$ are sets, and $f : A\to B$ and $g : B \to C$ are
functions.  If $f$ is surjective, and $g$ is injective, what can you
say about the composition $g \circ f$?  Need it be injective?  Need it
be surjective?
\end{problem}

\begin{solution}\begin{solutiontext}
\end{solutiontext}\end{solution}

%%%%%%%%%%%%%%%%%%%%%%%%%%%%%%%%%%%%%%%%%%%%%%%%%%%%%%%%%%%%%%%%
%  Divisibility
\begin{problem}[350]
Suppose $f : \mathbb{N} \to \mathbb{N}$ is a function with the
properties that
\begin{itemize}
\item $f(xy) = f(x) \cdot f(y)$ and
\item $f(p) = 2$ if $p$ is prime.
\end{itemize}
Among two-digit numbers $x$, how large can $f(x)$ be?
\end{problem}

\begin{solution}\begin{solutiontext}
\end{solutiontext}\end{solution}

%%%%%%%%%%%%%%%%%%%%%%%%%%%%%%%%%%%%%%%%%%%%%%%%%%%%%%%%%%%%%%%%
%  Set theory
\begin{problem}[350]
Find three sets of real numbers, $A$, $B$, and $C$, so that $A \cap B
\neq \varnothing$, and $A \cap C
\neq \varnothing$, and $B \cap C
\neq \varnothing$, but $A \cap B \cap C = \varnothing$.
\end{problem}

\begin{solution}\begin{solutiontext}
\end{solutiontext}\end{solution}

%%%%%%%%%%%%%%%%%%%%%%%%%%%%%%%%%%%%%%%%%%%%%%%%%%%%%%%%%%%%%%%%
%  Binomial coefficients in combinatorics
\begin{problem}[350]
Let $P_2(A)$ be the set of two element subsets of $A$.  If $A$ is an
$n$-element set, show that $P_2(A)$ has $\displaystyle\binom{n}{2}$ elements.
\end{problem}

\begin{solution}\begin{solutiontext}
\end{solutiontext}\end{solution}

%%%%%%%%%%%%%%%%%%%%%%%%%%%%%%%%%%%%%%%%%%%%%%%%%%%%%%%%%%%%%%%%
%  Least elements
\begin{problem}[350]
Let $A$ be the set of natural numbers which only use the digits $1$
and $2$.  Define $B$ by
$$
\{ n \in \mathbb{N} : 9n \in A \}.
$$
Does $B$ have a least element?  % yes, it is 1358!
% 

\end{problem}

\begin{solution}\begin{solutiontext}
\end{solutiontext}\end{solution}

%%%%%%%%%%%%%%%%%%%%%%%%%%%%%%%%%%%%%%%%%%%%%%%%%%%%%%%%%%%%%%%%
%  Propositional calculus
\begin{problem}[350]
Let $A$ be a set.  Suppose $P(n)$ is the proposition $(n \in A)
\rightarrow ((n+1) \in A)$.  If $P(n)$ is true for all $n$, does it
follow that $A = \mathbb{N}$?
\end{problem}

\begin{solution}\begin{solutiontext}
\end{solutiontext}\end{solution}

%%%%%%%%%%%%%%%%%%%%%%%%%%%%%%%%%%%%%%%%%%%%%%%%%%%%%%%%%%%%%%%%
%  De Morgan's laws
\begin{problem}[350]
  Let $P, Q, R, S$ be propositions.

Rewrite $(\neg (P \vee (Q \wedge
  R))) \wedge S$ in a simpler way.

Also, rewrite $(\neg (P \vee (Q \wedge
  R))) \wedge P$ in a simpler way.
\end{problem}

\begin{solution}\begin{solutiontext}
\end{solutiontext}\end{solution}

%%%%%%%%%%%%%%%%%%%%%%%%%%%%%%%%%%%%%%%%%%%%%%%%%%%%%%%%%%%%%%%%
%  Fibonacci numbers
\begin{problem}[350]
Show that every $k^{\mbox{th}}$ Fibonacci number is a multiple of $F_k$.
\end{problem}

\begin{solution}\begin{solutiontext}
\end{solutiontext}\end{solution}

%%%%%%%%%%%%%%%%%%%%%%%%%%%%%%%%%%%%%%%%%%%%%%%%%%%%%%%%%%%%%%%%
%  Congruences
\begin{problem}[350]
For which $n \in \mathbb{Z}$ is there a number $x \in \mathbb{Z}$ so
that $x^2 \equiv n \pmod 11$?
\end{problem}

\begin{solution}\begin{solutiontext}
\end{solutiontext}\end{solution}

%%%%%%%%%%%%%%%%%%%%%%%%%%%%%%%%%%%%%%%%%%%%%%%%%%%%%%%%%%%%%%%%
%  Prime numbers
\begin{problem}[350]
For which prime numbers $p$ is it possible to find an integer $x$ so
that $x^2 \equiv -1 \pmod p$?  You might not be able to prove your
statement, but you will probably be able to come up with enough
evidence to support your conjecture.
\end{problem}

\begin{solution}\begin{solutiontext}
\end{solutiontext}\end{solution}

%%%%%%%%%%%%%%%%%%%%%%%%%%%%%%%%%%%%%%%%%%%%%%%%%%%%%%%%%%%%%%%%
%  Induction
\begin{problem}[350]
This was meant to be an induction problem.

Determine the number of ways of covering a $n \times 2$ chessboard
with $2 \times 1$ dominoes.
\end{problem}

\begin{solution}\begin{solutiontext}
\end{solutiontext}\end{solution}

%%%%%%%%%%%%%%%%%%%%%%%%%%%%%%%%%%%%%%%%%%%%%%%%%%%%%%%%%%%%%%%%
%  Infinitude of the primes
\begin{problem}[350]
Let $A$ be the set 
$$
\{ n \in \mathbb{N} : \mbox{$n$ is prime and $n \equiv 3 \pmod 4$}\}.
$$
Prove that $A$ is infinite.
\end{problem}

\begin{solution}\begin{solutiontext}
\end{solutiontext}\end{solution}

%%%%%%%%%%%%%%%%%%%%%%%%%%%%%%%%%%%%%%%%%%%%%%%%%%%%%%%%%%%%%%%%
%  Infinitude of the primes
\begin{problem}[350]
Let $A$ be the set of prime numbers.  Can you find a million
consecutive numbers in the set $\mathbb{N} \setminus A$?
\end{problem}

\begin{solution}\begin{solutiontext}
\end{solutiontext}\end{solution}


%%%%%%%%%%%%%%%%%%%%%%%%%%%%%%%%%%%%%%%%%%%%%%%%%%%%%%%%%%%%%%%%
%  Set builder notation
\begin{problem}[350]
Explainw why there is a bijection between $\mathbb{N}$ and $\mathbb{Q}$.
\end{problem}

\begin{solution}\begin{solutiontext}
\end{solutiontext}\end{solution}

%%%%%%%%%%%%%%%%%%%%%%%%%%%%%%%%%%%%%%%%%%%%%%%%%%%%%%%%%%%%%%%%
%  Infinite sets 
\begin{problem}[350]
Prove that, if $A$ is an infinite set, then there exists a proper
subset $B \subset A$ with $B$ equinumerous to $A$.
\end{problem}

\begin{solution}\begin{solutiontext}
\end{solutiontext}\end{solution}

%%%%%%%%%%%%%%%%%%%%%%%%%%%%%%%%%%%%%%%%%%%%%%%%%%%%%%%%%%%%%%%%
%  Distributive laws
\begin{problem}[350]
Let $A,B,C$ be sets.  Prove that $A \cap (B \cup C)$ is equal to $(A
\cap B) \cup (A \cap C)$.
\end{problem}

\begin{solution}\begin{solutiontext}
\end{solutiontext}\end{solution}

%%%%%%%%%%%%%%%%%%%%%%%%%%%%%%%%%%%%%%%%%%%%%%%%%%%%%%%%%%%%%%%%
%  Rational and irrational numbers
\begin{problem}[350]
  For $q \in \mathbb{Q}$, let $A_q$ be the interval $(q - \epsilon, q+
  \epsilon)$.  Prove that $\bigcup_{q \in \mathbb{Q}} A_q = \mathbb{R}$.
\end{problem}

\begin{solution}\begin{solutiontext}
\end{solutiontext}\end{solution}

%%%%%%%%%%%%%%%%%%%%%%%%%%%%%%%%%%%%%%%%%%%%%%%%%%%%%%%%%%%%%%%%
%  Binomial theorem
\begin{problem}[350]
Show that $101$ divides $(15 + 17)^{101} - 15 - 17$.
\end{problem}

\begin{solution}\begin{solutiontext}
\end{solutiontext}\end{solution}

%%%%%%%%%%%%%%%%%%%%%%%%%%%%%%%%%%%%%%%%%%%%%%%%%%%%%%%%%%%%%%%%
%  Intersections, unions, subsets
\begin{problem}[350]
  Describe a collection of eight sets, any seven of which intersect,
  but for which all eight do not intersect.

  Describe a collection of $k$ sets, any $(k-1)$ of which intersect,
  but for which all $k$ do not intersect.
\end{problem}

\begin{solution}\begin{solutiontext}
\end{solutiontext}\end{solution}

%%%%%%%%%%%%%%%%%%%%%%%%%%%%%%%%%%%%%%%%%%%%%%%%%%%%%%%%%%%%%%%%
%  Compositions of functions
\begin{problem}[350]
Define functions $f : \mathbb{Z} \to \mathbb{Z}$ and $g : \mathbb{Z}
\to \mathbb{Z}$ by $f(n) = n+1$ and
$$
g(n) = \begin{cases} 1 & \mbox{if $n = 2$}, \\
2 & \mbox{if $n = 1$}, \\
n & \mbox{otherwise.}
\end{cases}
$$
Describe a bijective function $h : \mathbb{Z} \to \mathbb{Z}$ which
cannot be written as a composition of $f$ and $g$ and $f^{-1}$.
\end{problem}

\begin{solution}\begin{solutiontext}
\end{solutiontext}\end{solution}

%%%%%%%%%%%%%%%%%%%%%%%%%%%%%%%%%%%%%%%%%%%%%%%%%%%%%%%%%%%%%%%%
%  Bound and free variables
\begin{problem}[350]
In the statement
$$
\forall x \in \mathbb{R}\, \left( 
\left( \exists y \in \mathbb{R}\, (x^2 > y) \right) \vee
\left( \exists z \in \mathbb{R}\, (z^2 < w) \right) \right),
$$
which variables are bound?  Which variables are free?
\end{problem}

\begin{solution}\begin{solutiontext}
\end{solutiontext}\end{solution}

%%%%%%%%%%%%%%%%%%%%%%%%%%%%%%%%%%%%%%%%%%%%%%%%%%%%%%%%%%%%%%%%
%  Proof of Binomial theorem
\begin{problem}[350]
Prove the binomial theorem.
\end{problem}

\begin{solution}\begin{solutiontext}
\end{solutiontext}\end{solution}

%%%%%%%%%%%%%%%%%%%%%%%%%%%%%%%%%%%%%%%%%%%%%%%%%%%%%%%%%%%%%%%%
%  Pascal's triangle
\begin{problem}[350]
Show that
$$
\sum_{i=0}^n \binom{n}{i}^2 = \binom{2n}{n}.
$$

\end{problem}

\begin{solution}\begin{solutiontext}
\end{solutiontext}\end{solution}

%%%%%%%%%%%%%%%%%%%%%%%%%%%%%%%%%%%%%%%%%%%%%%%%%%%%%%%%%%%%%%%%
%  Nested quantifiers
\begin{problem}[350]
Describe a proposition $P(x,y)$ so that
$$ \forall x \, \exists y\, P(x,y) $$
is true, but
$$ \exists x \, \forall y\, P(x,y) $$
is false.
\end{problem}

\begin{solution}\begin{solutiontext}
\end{solutiontext}\end{solution}

%%%%%%%%%%%%%%%%%%%%%%%%%%%%%%%%%%%%%%%%%%%%%%%%%%%%%%%%%%%%%%%%
%  Divisibility
\begin{problem}[350]
Let $\omega$ be the set of nonnegative integers, and define a function
$A : \omega \times \omega \to \omega$,
$$
 A(m, n) =
\begin{cases}
n+1 & \mbox{if } m = 0 \\
A(m-1, 1) & \mbox{if } m > 0 \mbox{ and } n = 0 \\
A(m-1, A(m, n-1)) & \mbox{if } m > 0 \mbox{ and } n > 0.
\end{cases}
$$
Compute $A(4,4)$.
\end{problem}

\begin{solution}\begin{solutiontext}
\end{solutiontext}\end{solution}

%%%%%%%%%%%%%%%%%%%%%%%%%%%%%%%%%%%%%%%%%%%%%%%%%%%%%%%%%%%%%%%%
%  Divisibility
\begin{problem}[350]
Define a sequence $a_n$ be the rule that $a_0 = 1$ and $a_1 = 2$ and
$a_{n+2} = 2 a_{n+1} + a_n$.  For which values of $n$ is $a_n$ a
multiple of 7?
\end{problem}

\begin{solution}\begin{solutiontext}
\end{solutiontext}\end{solution}


%%%%%%%%%%%%%%%%%%%%%%%%%%%%%%%%%%%%%%%%%%%%%%%%%%%%%%%%%%%%%%%%
%  Countability 
\begin{problem}[350]
  Show that the set of functions $f : \mathbb{N} \to \mathbb{N}$ is
  not countable.
\end{problem}

\begin{solution}\begin{solutiontext}
\end{solutiontext}\end{solution}

%%%%%%%%%%%%%%%%%%%%%%%%%%%%%%%%%%%%%%%%%%%%%%%%%%%%%%%%%%%%%%%%
\begin{problem}[350]
  Let $A$ be the set of functions $\mathbb{N} \to \mathbb{N}$, and let
  $g : \mathbb{N} \to A$ be a function.  Define a function $f :
  \mathbb{N} \to \mathbb{N}$ by $f(n) = g(n)\left( n \right) + 1$.
  Does there exists a number $F \in \mathbb{N}$ so that $g(F) = f$?
\end{problem}

\begin{solution}\begin{solutiontext}


\end{solutiontext}\end{solution}

%%%%%%%%%%%%%%%%%%%%%%%%%%%%%%%%%%%%%%%%%%%%%%%%%%%%%%%%%%%%%%%%
%  Families of sets
\begin{problem}[350]
  Define $A = \{ x \in x \in \mathbb{R} : 0 \leq x \leq 1 \}$, and
  suppose $B_x$ is the interval $(x,10)$.  Determine
$$
\bigcup_{x \in A} B_x \mbox{ and } \bigcap_{x \in A} B_x \mbox{ and }
$$

\end{problem}

\begin{solution}\begin{solutiontext}
\end{solutiontext}\end{solution}

%%%%%%%%%%%%%%%%%%%%%%%%%%%%%%%%%%%%%%%%%%%%%%%%%%%%%%%%%%%%%%%%
%  Set theory and de Morgan's laws
\begin{problem}[350]
Let $A, B, C$ be three subsets of $\mathbb{R}^2$.  Rewrite
$$
\mathbb{R}^2 \setminus ((\mathbb{R}^2 \setminus (A \cup B)) \cap C)
$$
in a shorter form.
\end{problem}

\begin{solution}\begin{solutiontext}
\end{solutiontext}\end{solution}

%%%%%%%%%%%%%%%%%%%%%%%%%%%%%%%%%%%%%%%%%%%%%%%%%%%%%%%%%%%%%%%%
%  Diagonalization 
\begin{problem}[350]
Let $A$ be the power set of $\mathbb{R}$; show that there is no
bijection between $A$ and $\mathbb{R}$.
\end{problem}

\begin{solution}\begin{solutiontext}
\end{solutiontext}\end{solution}


% i'm going to run through the 38 problems on the fake final very
% quickly, just so you can see the answers if you've been wondering.

% i should say, though, just my philosophy about the fake final: I
% wrote these questions with the intention that you'd really have to
% think about them; they're hard questions.   But that's what makes
% them sort of fun to think through, and some of them are the
% beginning of whole areas of mathematics.

% so let's start with the first problem.

% we're halfway done.

% that's it; that's the end of the fake final.  i hope you enjoyed %
% thinking through these problems, and I wish you the very best as you
% prepare to take the real exam next week.

\end{exam}
\end{document}
