\documentclass[12pt]{handout}

\title{Lecture 1: Propositional calculus}
\author{Jim Fowler}
\course{Math 345}
\date{Wednesday, September 22, 2010}



\begin{document}
\maketitle

\section*{Go through names}

\section*{Syllabus}

Inspirational message about the history of mathematics.

\section*{Textbook}

This lecture discusses section 2 of the textbook.

\section*{Homework} 

The homework is due Monday, September 27, 2010.

From Section 2 of the textbook, do exercises 1, 2, and 5.

\section*{Logic}

Assign truth values to sentences

not all sentences have a truth value

\section*{Propositional calculus}

assign truth values to compound sentences, based on truth values of simpler sentences

\section*{Symbols}

$$ \Huge
\neg \hspace{1em} \wedge \hspace{1em} \vee \hspace{1em} \Rightarrow \hspace{1em} \Leftrightarrow
$$

\section*{Not}

double negation

logical equivalence (denoted by $\equiv$, which is not a symbol of the
propositional calculus)

\section*{Conjunction $\wedge$}

make truth table

\section*{Disjunction $\vee$}

make truth table

compare this to English

\section*{DeMorgan's laws}

$\neg (P \wedge Q) \equiv \neg P \vee \neg Q$

$\neg (P \vee Q) \equiv \neg P \wedge \neg Q$

Proof via truth table

proof via words

which is better?

\section*{Distributive laws}

$P \wedge (Q \vee R) \equiv (P \wedge Q) \vee (P \wedge R)$

$P \vee (Q \wedge R) \equiv (P \vee Q) \wedge (P \vee R)$

\end{document}
