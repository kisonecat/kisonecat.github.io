\documentclass[12pt]{article}
\usepackage{fullpage}
\usepackage{nopageno}
\usepackage{ifthen}
\usepackage{amsmath}
\usepackage{amssymb}
\usepackage{graphicx} 
\usepackage{version}
\usepackage{amsthm}
\usepackage{multicol}
\usepackage{add-copyright}

\excludeversion{solution}

\DeclareMathOperator{\ft}{ft}

\newcommand{\R}{\mathbb{R}}

\title{Take-Home Quiz 4}
\author{Math 132 Section 22}
\date{Due Monday, February 6, 2006}

\newcounter{problem}
\setcounter{problem}{1}

\newenvironment{problem}[1][]
{\begin{flushleft}\hangindent=1em\hangafter=1\noindent\textbf{Problem \arabic{problem}.}
\ifthenelse{\equal{#1}{}}{}{
\textbf{(#1 \ifthenelse{\equal{#1}{1}}{point}{points}).}}
}
{\addtocounter{problem}{1}\end{flushleft}}

\begin{document}
\maketitle

%This quiz is worth 14 points---12 points are for ``correctness'' of
%answers, but an additional 2 points will be given out for
%``style''---awarded to answers which are clearly written, easy to
%follow, etc.

\begin{problem}[2]
Compute $\displaystyle\int x^2 - 2x + 1 \, dx$.  Be sure to show your work!
\end{problem}

\begin{problem}[2]
Compute $\displaystyle\int \frac{\theta^2 + \cos \theta}{3} \, d\theta$.
\end{problem}

\begin{problem}[2]
Compute $\displaystyle\int 2 x \cos (x^2) \, dx$.
\end{problem}

\begin{problem}[2]
Compute $\displaystyle\sum_{n=1}^6 n - 2$.
\end{problem}

\begin{problem}[2]
Compute $\displaystyle\sum_{n=1}^{17} (-1)^{n}$.
\end{problem}

\begin{problem}[2]
Compute $\displaystyle\sum_{n=0}^{100} (n+1)^{10} - n^{10}$.
\end{problem}

\begin{problem}[2]
Compute $\displaystyle\sum_{n=1}^{9} \frac{1}{n^2 + n}$.  \textit{Hint:}
Rewrite the fraction as the difference of two fractions.
\end{problem}

%\begin{problem}[3]
%There is a spring attached to the wall; unable to resist, you pull the
%free end, and let go.  Let $\ell(t)$ be difference between the
%spring's length at time $t$ and its original length.
%
%You note that $\ell''(t) = -\ell(t)$ because the speed with which the
%spring's length is changing is related to the current length of the
%spring---a longer spring makes the spring gets shorter faster, while a
%spring only a little longer than usual will try to get shorter, but
%not as quickly.  Conversely, a much shorter than usual spring is
%becoming longer more quickly than a spring only a little shorter than
%usual.
%
%Give a single example of a function with the given property, i.e.,
%give a function $\ell : \R \to \R$ such that $\ell''(t) = \ell(t)$.
%\end{problem}
%
%\begin{problem}[3]
%You throw a ball straight up into the air; let $p(t)$ be the ball's
%height in meters above the ground after $t$ seconds have elapsed.  The
%ball's initial speed is $5 \mbox{m}/\mbox{sec}$, so $p'(0) = 5$.  The
%ball is constantly accelerating at $-10 \mbox{m}/\mbox{sec}^2$ due to
%gravity, so $p''(t) = -10$.  When will the ball hit the ground?
%\end{problem}
%
%\begin{problem}[3]
%Find a solution to the differential equation:
%$$
%y = x \frac{dy}{dx}.
%$$
%\end{problem}


\end{document}
