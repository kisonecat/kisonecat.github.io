\documentclass[12pt]{article}
\usepackage{fullpage}
\usepackage{nopageno}
\usepackage{ifthen}
\usepackage{amsmath}
\usepackage{amssymb}
%\usepackage{add-copyright}

\newcommand{\R}{\mathbb{R}}

\title{Take-Home Quiz 2}
\author{Math 131 Section 22}
\date{Due Monday, October 17, 2005}

\newcounter{problem}
\setcounter{problem}{1}

\newenvironment{problem}[1][]
{\begin{flushleft}\hangindent=1em\hangafter=1\noindent\textbf{Problem \arabic{problem}.}
\ifthenelse{\equal{#1}{}}{}{
\textbf{(#1 \ifthenelse{\equal{#1}{1}}{point}{points}).}}
}
{\addtocounter{problem}{1}\end{flushleft}}

\begin{document}
\maketitle

You may use any resource (books, notes, even people), but you must
write down your solutions when you are alone.  Anytime that you are
asked to evaluate a limit, you must justify your calculation.

\begin{problem}[1]
Give the official definition of
$$
\lim_{x \to \infty} f(x) = L.
$$
\end{problem}

\begin{problem}[2]
Prove that
$$
\lim_{x \to \infty} 1/x = 0.
$$
\end{problem}

\begin{problem}[1]
Evaluate % and this is ridiculously easy.
$$
\lim_{x \to 2^{+}} \left( \frac{3x^2 - 2x + 2}{\lfloor x \rfloor^2 + 1} \right)^4.
$$
\end{problem}

\begin{problem}[2]
Evaluate
$$
\lim_{x \to 17^{-}} \lfloor x \rfloor + \lfloor -x \rfloor.
$$
\end{problem}

\begin{problem}[2]
Evaluate
$$
\lim_{x \to \infty} \frac{x^2}{x-1} - \frac{x^2}{x+1}.
$$
\end{problem}

\begin{problem}[2]
Evaluate
%\lim_{x \to 3^{+}} \frac{(x-4)(x+3)}{(x-3)(x-5)}.
$$
\lim_{x \to 3^{+}} \frac{x^2 - x - 12}{x^2 - 8x + 15}.
$$
\end{problem}

\begin{problem}[2]
Evaluate
$$
\lim_{x \to \infty} \frac{\sqrt[3]{ 2 x^3 + x + 1 }}{\sqrt{x^2 + 1}}.
$$
\end{problem}


\end{document}
