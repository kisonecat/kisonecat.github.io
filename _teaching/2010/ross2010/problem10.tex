\documentclass[12pt]{pset}
%\usepackage{add-copyright}

\geometry{margin=0.5in}
\title{Problem Set 10}
\course{Piecewise-Linear Topology}
\author{Jim Fowler}
\date{Summer 2010}

\usepackage{tikz}
\usetikzlibrary{matrix}

\newcommand{\defnword}[1]{\textbf{#1}}

\usepackage{stmaryrd}
\newcommand{\boundary}{\partial}
\newcommand{\collapses}{\searrow}
\newcommand{\expands}{\nearrow}
\newcommand{\she}{\ssearrow\nnearrow}
\newcommand{\join}{\ast}
\newcommand{\subdivided}{\triangleleft}
\geometry{margin=1in}%,top=0.5in,bottom=0.5in}
\DeclareMathOperator{\st}{st}
\DeclareMathOperator{\vertices}{vert}
\DeclareMathOperator{\lk}{lk}
\DeclareMathOperator{\cl}{cl}
\DeclareMathOperator{\interior}{int}
\newcommand{\fullsubcomplex}{\Subset}

\usepackage{hyperref}

\begin{document}
\maketitle

\subsubsection*{Homology, continued.} Today we continue studying
homology of simplicial complexes.  In particular, we introduce a
computational tool, the \textit{Mayer--Vietoris sequence.}

\parskip 0.45\baselineskip
\parindent 0pt

%Last time, Max Rabinovich asked the very good question of whether
%there were (say, compact) subsets of $\R^3$ having torsion in $H_1$.
%The answer is no: you can find a discussion of this question at
%\url{http://mathoverflow.net/questions/4478/}.

\begin{requiredproblem}
  Compute $H_n(CX)$, where $CX$ is the cone on a simplicial complex $X$.
\end{requiredproblem}

\begin{problem}
  Show that $H_n(A \sqcup B) = H_n(A) \oplus H_n(B)$.
\end{problem}

\begin{problem}
  If $X$ is path-connected (see Problem Set 3), show that $H_0(X) = \Z$.
\end{problem}

\begin{definition*}
  A \textbf{chain complex} is a sequence of abelian groups $C_n$ and
  homomorphisms $\partial : C_n \to C_{n-1}$, so that $\partial
  \circ \partial = 0$.  In particular, chains on a simplicial complex
  $X$ is an example of a chain complex.

  A map $A_\bullet \to B_\bullet$ of chain complexes, called a \textbf{chain
    map}, is a collection of homomorphisms $f_n : A_n \to B_n$
  intertwining with the boundary, i.e., $\partial \circ f_n = f_{n-1}
  \circ \partial$.
\end{definition*}

\begin{requiredproblem}
  Show that a chain map sends boundaries to boundaries and cycles to
  cycles, and therefore induces a map on homology.
\end{requiredproblem}

\begin{problem}
  Show that a simplicial map $f : X \to Y$ induces a chain map $C_\bullet(f) :
  C_\bullet(X) \to C_\bullet(Y)$, and therefore a map $H_\bullet(f) : H_\bullet(X)
  \to H_\bullet(Y)$.  Usually, $H_\bullet(f)$ is written $f_\star$, but I
  like to emphasize the \textbf{functoriality}.

  Be careful to describe what happens if $f$ crushes some simplexes.
\end{problem}

\begin{definition*}
  A sequence of abelian groups and homomorphisms 
  $$\cdots \to A \stackrel{f}{\longrightarrow} B
  \stackrel{g}{\longrightarrow} C \to \cdots$$ is \textbf{exact} at
  $B$ if the image of $f$ equals the kernel of $g$ (in other words,
  the kernel is ``exactly'' the image).

  If we say ``a sequence is exact'' we mean it is exact at each group
  in the sequence.  An exact sequence $0 \to A \to B \to C \to 0$ is
  usually called a \textbf{short exact sequence}.

  We can also talk about exact sequences of chain complexes, e.g., a
  short exact sequence $0 \to A_\bullet \to B_\bullet \to C_\bullet \to 0$ of
  chain complexes and chain maps.
\end{definition*}

\begin{requiredproblem}
  Suppose $X = A \cup B$.  Find chain maps so that
  $$
  0 \to C_\bullet(A \cap B) \to C_\bullet(A) \oplus C_\bullet(B) \to C_\bullet(A
  \cup B) \to 0
  $$
  is a short exact sequence of chain complexes.  \textit{Warning:} there is more than one way to do this.
\end{requiredproblem}

\begin{problem}
  Show that a short exact sequence of chain complexes yields a long
  exact sequence in homology; this is the \textbf{Zig-zag lemma},
  which transforms

\begin{tikzpicture}
    \matrix (m) [matrix of math nodes,row sep=3em,column sep=0.8em]
    {
               & \vdots  & \vdots & \vdots & \\
        0 & A_{n+1} & B_{n+1} & C_{n+1} & 0 \\
        0 & A_{n} & B_{n} & C_{n} & 0 \\
        0 & A_{n-1} & B_{n-1} & C_{n-1} & 0 \\
              & \vdots  & \vdots & \vdots & \\
   };
   \draw[->] (m-2-1) edge (m-2-2);
   \draw[->] (m-2-2) edge (m-2-3);
   \draw[->] (m-2-3) edge (m-2-4);
   \draw[->] (m-2-4) edge (m-2-5);

   \draw[->] (m-3-1) edge (m-3-2);
   \draw[->] (m-3-2) edge (m-3-3);
   \draw[->] (m-3-3) edge (m-3-4);
   \draw[->] (m-3-4) edge (m-3-5);

   \draw[->] (m-4-1) edge (m-4-2);
   \draw[->] (m-4-2) edge (m-4-3);
   \draw[->] (m-4-3) edge (m-4-4);
   \draw[->] (m-4-4) edge (m-4-5);

   \draw[->] (m-1-2) edge (m-2-2);
   \draw[->] (m-2-2) edge (m-3-2);
   \draw[->] (m-3-2) edge (m-4-2);
   \draw[->] (m-4-2) edge (m-5-2);

   \draw[->] (m-1-3) edge (m-2-3);
   \draw[->] (m-2-3) edge (m-3-3);
   \draw[->] (m-3-3) edge (m-4-3);
   \draw[->] (m-4-3) edge (m-5-3);

   \draw[->] (m-1-4) edge (m-2-4);
   \draw[->] (m-2-4) edge (m-3-4);
   \draw[->] (m-3-4) edge (m-4-4);
   \draw[->] (m-4-4) edge (m-5-4);
\end{tikzpicture}
\raisebox{10em}{into}
\hspace{-5em}
\begin{tikzpicture}
    \matrix (m) [matrix of math nodes,row sep=3em,column sep=0.8em]
   {
              & & \cdots \\
             H_{n+1}(A_\bullet) & H_{n+1}(B_\bullet) & H_{n+1}(C_\bullet)  \\
             H_{n}(A_\bullet) & H_{n}(B_\bullet) & H_{n}(C_\bullet)  \\
             H_{n-1}(A_\bullet) & H_{n-1}(B_\bullet) & H_{n-1}(C_\bullet)  \\
             \cdots  & &  \\
   };

   \draw[->] (m-1-3) edge[out=0,in=180] (m-2-1);

   \draw[->] (m-2-1) edge (m-2-2);
   \draw[->] (m-2-2) edge (m-2-3);
   \draw[->] (m-2-3) edge[out=0,in=180] (m-3-1);

   \draw[->] (m-3-1) edge (m-3-2);
   \draw[->] (m-3-2) edge (m-3-3);
   \draw[->] (m-3-3) edge[out=0,in=180] (m-4-1);

  \draw[->] (m-4-1) edge (m-4-2);
  \draw[->] (m-4-2) edge (m-4-3);
  \draw[->] (m-4-3) edge[out=0,in=180] (m-5-1);

\end{tikzpicture}
\end{problem}

\begin{problem}
  Combine the previous problems to discover the \textbf{Mayer--Vietoris
    sequence}, a long exact sequence
$$
\cdots \to H_{n+1}(A \cup B) \to H_n(A \cap B) \to H_n(A) \oplus H_n(B) \to H_n(A
\cup B) \to H_{n-1}(A \cap B) \to \cdots
$$
One can think of this long exact sequence as a generalization of
inclusion-exclusion.
\end{problem}

\begin{requiredproblem}
  Use the Mayer--Vietoris sequence to compare $H_n(SX)$ with $H_{n-1}(X)$.
\end{requiredproblem}

\begin{problem}
  Assuming $H_n(S^1 \times I) = H_n(S^1)$, use the Mayer--Vietoris
  sequence to compute $H_n(T^2)$ by building $T^2$ by gluing together
  two copies of $S^1 \times I$.
\end{problem}

\begin{problem}
  Compute $H_n(M \times S^1)$ in terms of $H_n(M)$.
\end{problem}

\begin{requiredproblem}
  Calculate $H_n(T^2 - D^2)$ by using the Mayer--Vietoris sequence.
  Here, $T^2 - D^2$ denotes $T^2$ with the interior of a $2$-simplex
  removed.
\end{requiredproblem}

\begin{problem}
 Calculate $H_n(S^2 - 3D^2)$, i.e., the homology of $S^2$ with 3 (open) disks removed.
\end{problem}

\begin{requiredproblem}
Calculate $H_n(T^2 \# T^2)$ by gluing together two copies of $T^2 - D^2$.
\end{requiredproblem}

\begin{problem}
  Can you calculate $H_n(M \# N)$ in terms of $H_n(M)$ and $H_n(N)$?
\end{problem}

\begin{problem}
  Show that $H_n(\mbox{M\"obius band}) \cong H_n(S^1)$.
\end{problem}

\begin{requiredproblem}
Calculate $H_n(\RP^2)$; recall that $\RP^2$ is a M\"obius band attached to a disk along their common circle boundary.
\end{requiredproblem}

\begin{problem}
  Calculate $H_n(K)$, where $K$ is the Klein bottle.  Recall that the
  Klein bottle consists of two copies of the M\"obius band, attached
  along their boundary.
\end{problem}

\end{document}

