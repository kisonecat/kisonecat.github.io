\documentclass[12pt]{pset}
%\usepackage{add-copyright}

\title{Problem Set 5}
\course{Piecewise-Linear Topology}
\author{Jim Fowler}
\date{Summer 2010}

\newcommand{\defnword}[1]{\textbf{#1}}

\usepackage{stmaryrd}
\newcommand{\boundary}{\partial}
\newcommand{\collapses}{\searrow}
\newcommand{\expands}{\nearrow}
\newcommand{\she}{\ssearrow\nnearrow}
\newcommand{\join}{\ast}
\newcommand{\subdivided}{\triangleleft}
\geometry{margin=1in}%,top=0.5in,bottom=0.5in}
\DeclareMathOperator{\st}{st}
\DeclareMathOperator{\vertices}{vert}
\DeclareMathOperator{\lk}{lk}
\DeclareMathOperator{\cl}{cl}
\DeclareMathOperator{\interior}{int}
\newcommand{\fullsubcomplex}{\Subset}

\begin{document}
\maketitle

\subsubsection*{\textit{Ceci n'est pas une Problem Set.}} Instead of
questions, this sheet records the definitions we will be working with;
here, we leave the realm of abstract simplicial complexes, and embrace
the geometry---our simplexes are now situated in space.

\parskip 0.5\baselineskip
\parindent 0pt

\begin{definition*}
The \textbf{join} of $A, B \subset \R^n$ is $$
A \join B = \{ \lambda a + \mu b : a
\in A, \hspace{1em}b \in B, \hspace{1em}\lambda \in [0,1], \hspace{1em}\mu \in [0,1], \hspace{1em}\lambda + \mu = 1 \}
$$
Imagine this as connecting $A$ and $B$ by including all line segments
between points in $A$ and points in $B$.

$A$ and $B$ are \textbf{independent} if each point in $A \join B$ can
be written uniquely as $\lambda a + \mu b$, for $a \in A$, $b \in B$,
$\lambda, \mu \in [0,1]$, $\lambda + \mu = 1$.
\end{definition*}

% \begin{definition*}
%   A subset $A \subset \R^n$ is \textbf{bounded} if there is some $r
%   \in \R$ so that
%   $A \subset \{ v \in \R^n : |v| \leq r \}$.
% \end{definition*}

\subsection*{Polyhedra}

\begin{definition*}
  $P \subset \mathbb{R}^n$ is a \textbf{polyhedron} if, for each $p
  \in P$, there is a neighborhood $N \ni p$, so that $N = p \join L$,
  with $L$ closed and bounded (in other words, compact).
  
  In this case, $N$ is called a \textbf{closed star} around $p$, and
  $L$ is a \textbf{link} of $p$.

  A \textbf{subpolyhedron} of $P$ is a subset $Q \subset P$ which is
  also a polyhedron.
\end{definition*}

\begin{definition*}
  Let $P, Q$ be polyhedra. Then $f : P \to Q$ is a
  \textbf{piecewise-linear map} (a PL map, for short) if each point $p
  \in P$ has a closed star $N = p \join L$ so that $f(\lambda p + \mu
  x) = \lambda f(p) + \mu f(x)$ for $x \in L$ and $\lambda,\mu \in
  [0,1]$ with $\lambda + \mu = 1$.  A \textbf{PL homeomorphism} is a
  PL map with a PL inverse.
\end{definition*}

\begin{definition*}
  A PL map $f : P \to Q$ is a \textbf{piecewise linear embedding} if
  $f(P)$ is a subpolyhedron of $Q$, and $f : P \to f(P)$ is a PL
  homeomorphism.
\end{definition*}

\pagebreak
\subsection*{Manifolds}

\begin{definition*}
  A polyhedron $P$ is an $n$-dimensional \textbf{PL manifold} if each
  $p \in P$ has an open neighborhood $N \ni p$ which is PL
  homeomorphic to an open set in $\R^n$.  In this case, we call $N$
  with the PL homeomorphism a \textbf{coordinate neighborhood}.

  A polyhedron $P$ is an $n$-dimensional PL manifold \textbf{with
    boundary} if each point $p \in P$ has an open neighborhood $N \ni
  p$ which is PL homeomorphic to an open subset of $\R^{n-1} \times
  \R_{\geq 0}$.  The \textbf{boundary} of $P$ (written $\partial P$)
  consists of points $p \in P$ which are identified with $\R^{n-1}
  \times \{ 0 \} \subset \R^{n-1} \times \R_{\geq 0}$.

  A manifold $P$ is \textbf{closed} if $\partial P = \varnothing$ and
  $P$ is compact.
\end{definition*}

\begin{definition*}
  The $n$-ball $B^n$ (sometimes called the $n$-disk $D^n$) is any
  manifold PL homeomorphic to $[0,1]^n$.  The $n$-sphere is any
  manifold PL homeomorphic to $\partial [0,1]^{n+1}$.
\end{definition*}

\subsection*{Cell complexes}

\begin{definition*}
 A subset $C \in \R^n$ is \textbf{convex} if for any $p,q \in C$, the
 segment $\{p\} \join \{q\}$ is contained in $C$.
\end{definition*}

\begin{definition*}
  A compact convex subset $C \in \R^n$ is a $k$-dimensional
  \textbf{cell} if it spans a $k$-dimensional subspace.

  For $x \in C$, define $\langle x, C \rangle$ to be the union of
  $\{x\}$ and all lines $L$ such that $L \cap C$ contains $x$ in its
  interior.  The subset $C_x = C \cap \langle x, C \rangle$ is the
  \textbf{face} of $C$ containing $x$. Write $D < C$ if $D$ is a face
  of $C$.
\end{definition*}

\begin{definition*}
  A \textbf{cell complex} is a finite collection $K$ of cells such that
  \begin{itemize}
  \item If $C \in K$, and $D < C$, then $D \in K$.
  \item If $C, D \in K$, then $C \cap D$ is a face of $C$ and $D$.
  \end{itemize}
  The \textbf{underlying polyhedron}, $|K|$ is the union of all cells
  in $K$.

  A \textbf{cellular map} $f : K \to L$ is a PL map $|f| : |K| \to
  |L|$ which is linear on cells of $K$, and sends cells to cells.
\end{definition*}

\begin{definition*}
  A cell complex $L$ is a \textbf{subdivision} of $K$ if $|L| = |K|$
  and each cell of $L$ is contained in a cell $K$.  We write $L
  \subdivided K$ if $L$ is a subdivision of $K$.
\end{definition*}

\pagebreak
\subsection*{Simplicial complexes}

Our original notion of simplicial complex should now be called an
\textbf{abstract simplicial complex}, to emphasize the fact that we
originally did not place our simplexes in $\R^n$.  In an abstract
simplicial complex, only the relationship between the vertices and the
faces is recorded.

\begin{definition*}
  A cell complex is a \textbf{simplicial complex} if each $C \in K$ is
  a \textbf{simplex} (i.e., an $n$-cell which is the join of $n+1$
  independent points).  A \textbf{triangulation} of a polyhedron $P$
  is a simplicial complex $K$ with a PL homeomorphism $f : |K| \to P$.
\end{definition*}

\begin{definition*}
  Suppose $L \subset K$ are simplicial complexes.  Define $f : K \to
  [0,1]$ on vertices by
  $$
f(v) = \begin{cases}
0 & \mbox{ if $v \in L$ } \\
1 & \mbox{ if $v \not\in L$ } 
\end{cases}
  $$
  and extending linearly to simplexes.  If $L = f^{-1}(0)$ we say $L$
  is a \textbf{full subcomplex} of $K$, and write $L \fullsubcomplex K$.
\end{definition*}

\end{document}

