\documentclass[12pt]{article}
\usepackage{fullpage}
\usepackage{amsthm}
\usepackage{amsmath}

\newtheorem*{example}{Example}
\newtheorem*{thm}{Theorem}

\title{Lecture 2: Coordinate systems}
\author{Math 195 Section 91}
\date{Wednesday June 24, 2009}

\begin{document}
\maketitle

Goal: section 11.3 and 13.1 and 13.2.

What is a ``coordinate system''?  why are these important?  the greatness of analytic geometry!

\subsection{Polar coordinates}

Examples of polar coordinates

$x = r \cos \theta, y = r \sin \theta$

converting back and forth.  same point may have many names.

graphing polar equations $r = f(\theta)$.  it helps to graph the function in cartesian coordinates first!

example: $r = 17$.

example: $r = 2 \cos \theta$ and then rearrange to find that
this is a circle

example: $r = 1 + \sin \theta$ cardioid

example: $r = \cos (2 \theta)$ four-petaled rose

\subsection{Calculus in polar coordinates}

find slopes of tangent lines $dy/dx$.  same trick as in parametric case.

\subsection{Three dimensional systems}

three coordinates!

draw some pictures

draw the $xz$, $xy$, and $yz$ planes.

draw all the points where an equation holds: $z=3$, $y = 3$, $x = 3$, $x = y$, $x^2 + y^2 = 4$.  $x^2 + y^2 + z^2 = 1$.

this reminds me about the distance formula!

equation of a sphere with center $(a,b,c)$ and radius $r$.

complete-the-square to transform complicated looking equations into the equation for a sphere.

doing calculus in this situation---it will be coming later.

\subsection{Vectors}

a ``vector'' means different things: a tuple of numbers, a direction with a magn itude, an arrow, a\ldots

add vectors with triangles (and componentwise)

does it matter what order you add?

scale a vector by multiplying by a scalar (hence the name).  do this algebraically and geometrically.

subtract vectors

say a bit about the ``span'' of vectors $(1,0,0)$ and $(0,1,0)$.  What about the span of two other 3-vectors?

\end{document}
