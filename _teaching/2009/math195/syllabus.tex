\documentclass[12pt,letterpaper]{article}
\usepackage{add-copyright}

\title{Math195 Syllabus}
\author{Jim Fowler}

\usepackage{multicol}
\usepackage{fullpage}
\usepackage{palatino}
\pagestyle{empty}

\usepackage{html}

%\usepackage[T1]{fontenc}
%\usepackage{textcomp}
\usepackage{lmodern}
%\newcommand{\ditto}{\textquotedbl}

%\setlength{\parindent}{0pt}

\newcommand{\peem}{\textsc{p.m.}}
\newcommand{\ayem}{\textsc{a.m.}}

\begin{document}

%%%%%%%%%%%%%%%%%%%%%%%%%%%%%%%%%%%%%%%%%%%%%%%%%%%%%%%%%%%%%%%%
\section*{\Large\sf Syllabus\hfill
Math 195, Section 91\hfill
Summer 2009}

A complicated model might have many inputs and many outputs.  How do
small changes to those inputs affect the outputs?  In this class, you
will learn to use calculus to analyze such situations.

\begin{htmlonly}
You can also download this syllabus as a \htmladdnormallink{PDF
file}{http://math.uchicago.edu/~fowler/teaching/math195/syllabus.pdf}.
\end{htmlonly}

%%%%%%%%%%%%%%%%%%%%%%%%%%%%%%%%%%%%%%%%%%%%%%%%%%%%%%%%%%%%%%%%
\section*{Resources}

We present six resources to help you to learn multivariable calculus.

\subsection*{Office Hours}
If you have questions, want to work through problems, or just talk
about mathematics, I invite you to come to my office hours.
\begin{multicols}{2}
\begin{tabular}{ll}
\textbf{Name:} & Jim Fowler \\
\textbf{Office:} & Math/Stat 102 \\
%\textbf{Phone:} & 773--573--5659 \\
\textbf{Email:} & \texttt{jim@uchicago.edu}
\end{tabular}

\begin{tabular}{ll}
\textbf{Office Hours:} & Tuesday 4:30--6:00\peem \\
& Thursday 4:30--6:00\peem \\
& and by appointment
\end{tabular}
\end{multicols}
\noindent
Please email me with any concerns you have; the success of this course
depends on open communication.

%The VIGRE Course Assistant, Brooke Ullery
%(\texttt{bullery@uchicago.edu}), will also office hours on
%Wednesdays from 6:00--7:00\peem\ in the C-Shop.

\subsection*{Textbook}
Our text is \textit{Multivariable Calculus} by James Stewart (sixth edition).  Somewhat confusingly, the first chapter in this book is numbered ``11.''

\subsection*{Website}
The website is on chalk; I will post assignments, notes, handouts.

\subsection*{Lectures}
We will meet Mondays, Wednesdays, and Fridays, 9:00--11:00\ayem\ in room
108 of the Social Sciences Research Building for an interactive
lecture.  There will be no lecture on July 3, 2009.

\subsection*{Take-home quizzes}
Your quizzes will be graded by Jim Fowler.

\subsection*{Recommended Homework}
Homework will be assigned, but not graded.

\pagebreak

%%%%%%%%%%%%%%%%%%%%%%%%%%%%%%%%%%%%%%%%%%%%%%%%%%%%%%%%%%%%%%%%
\section*{Requirements}

There are one thousand points possible in this course, broken down as
follows:
\begin{description}
\item[Quizzes (100 points).] After most classes, a short take-home
  quiz will be assigned.
\item[2 midterms (225 points each).]  The midterms are in class.  The
  first midterm is on Monday, July 6, 2009; the second will be on
  Friday, July 17, 2009.  The first midterm will replace the first
  hour of lecture; the second hour will be an interactive lecture as
  usual.  The second midterm will be an oral exam scheduled outside of
  class.
\item[Participation in lectures (50 points).] The lectures will be
  interactive: you should participate.
\item[1 final exam (400 points).]  The final exam will be at
  9:00--11:00\ayem\ on Friday, July 24, 2009, in room 108 of the
  Social Sciences Research Building.
\end{description}

%%%%%%%%%%%%%%%%%%%%%%%%%%%%%%%%%%%%%%%%%%%%%%%%%%%%%%%%%%%%%%%%
\subsection*{Department Policy on Final Exams}

\textit{It is the policy of the Department of Mathematics that the
following rules apply to final exams in all undergraduate mathematics
courses:}
\begin{enumerate}
\item \textit{The final exam must occur at the time and place designated on
the College Final Exam Schedule.}  In particular, \textit{no} final examinations
may be given during the tenth week of the quarter, except in the case
of graduating seniors.
\item Any student who wishes to depart from the scheduled final exam
time for the course must receive permission from Paul Sally (office is
Ryerson 350, phone is 2-7388, email is
\texttt{sally@math.uchicago.edu}).  Instructors are not permitted to
excuse students from the scheduled time of the final exam except in
the cases of an Incomplete.
\end{enumerate}

%%%%%%%%%%%%%%%%%%%%%%%%%%%%%%%%%%%%%%%%%%%%%%%%%%%%%%%%%%%%%%%%
\subsection*{Requesting to reschedule a midterm}

Contact Jim Fowler as soon as possible if you will not be able to take
a midterm on the scheduled day.

%%%%%%%%%%%%%%%%%%%%%%%%%%%%%%%%%%%%%%%%%%%%%%%%%%%%%%%%%%%%%%%%
\subsection*{Late quizzes, skipping class, and so forth}

You absolutely must stay caught up.  It is tempting to fall behind,
but difficult to catch up again---this is true of all courses, but
especially true of a course in mathematics.  That said, I understand
your schedules are very busy, so I will not penalize you for
\textit{infrequently} turning in your work \textit{a day or two late.}
Do not make a habit of it!

\end{document}
