\documentclass[12pt]{article}
\usepackage{fullpage}
\usepackage{nopageno}
\usepackage{amsmath}
\usepackage{amssymb}
\usepackage{amsthm}
\usepackage{multicol} 
\newcommand{\R}{\mathbb{R}}

%\usepackage{add-copyright}
\renewcommand{\labelitemi}{$\qed$}

\newtheorem*{example}{Example}
\newtheorem*{thm}{Theorem}

\title{Final Exam: Frequently Asked Questions}
\author{Math 195 Section 91}
\date{Wednesday July 22, 2009}

\begin{document}
\maketitle

\section*{What will the final exam be like?}

There will be \textbf{18 questions} on the exam and it will last
\textbf{120 minutes}.  It will be worth 400 points.

As on the first exam, the final question will be \textbf{extra credit}
with true/false questions.

\section*{Should I bring a calculator?}

Absolutely not.  Calculators are \textbf{forbidden}.  You may bring
the official Math 195 Paper Slide Rule, if you really want to.

\section*{How should I write down my answers?}

Your task is not merely to find an answer---it is to provide an
explanation.  You will lose points if you surround an otherwise
convincing argument with false statements (after all, once I have
proved $2 = 1$, I can prove anything!).  \textbf{Erase} untrue
statements for full credit.

\section*{Is the exam cumulative?}

Yes; the final exam will cover---and combine!---everything we have
done so far.  I have provided little boxes so you can check off the tasks that you feel prepared to do.
\begin{multicols}{2}
\begin{itemize}
\item Find the slope of a tangent line to a curve
\item Find the slope of a tangent line to a curve given in polar coordinates
\item Find distance between points in $\R^2$, $\R^3$, and $\R^n$
\item Write vectors as a linear combination of other vectors
\item Compute dot products
\item Determine when vectors are orthogonal
\item Find the angle between two vectors
\item Normalize a vector
\item Compute the norm of a vector
\item Compute the cross product of two vectors in $\R^3$
\item Find an equation for a line through a given point and in a given direction
\item Find an equation for a plane through a given point and with a given normal vector
\item Determine whether two lines intersect, are parallel, or are skew
\item Find the point of intersection between a line and a plane
\item Differentiate and integrate vector-valued functions
\item Caclulate unit tangent vectors to a curve
\item Find the angle of intersection between curves
\item Differentiate a vector-valued function
\item Compute a limit of a function of several variables
\item Convert a limit from cartesian coordinates to polar coordinates
\item Define continuity for functions of several variables
\item Give an example of a limit that does not exist
\item Compute partial derivatives
\item Compute higher partial derivatives
\item Give an example in which mixed partials commute
\item Write down a linear approximation to a function
\item Find the tangent plane to a function at a point
\item Compute partial derivatives with the chain rule
\item Compute the gradient of a function
\item Compute directional derivatives
\item Find critical points
\item Find maximum and minimum values
\item Use the second derivative test
\item Optimize a function given a constraint with Lagrange multipliers
\item Evaluate double and triple integrals
\item Integrate a function over a rectangular region
\item Integrate a function over a general region
\item Apply Fubini's theorem
\item Convert polar coordinates to cartesian coordinates.
\item Evaluate integrals using polar coordinates
\item Use cylindrical coordinates
\item Convert cartesian coordinates to spherical coordinates.
\item Use spherical coordinates
\item Compute a Jacobian
\item Find the area of a given region
\item Compute the volume of a 3-dimensional region
\end{itemize}
\end{multicols}

\end{document}