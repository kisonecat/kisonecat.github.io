\documentclass[12pt]{article}
\usepackage{fullpage}
\usepackage{amsthm}
\usepackage{amsmath}

\newtheorem*{example}{Example}
\newtheorem*{thm}{Theorem}

\title{Lecture 28: Differential equations}
\author{Math 153 Section 57}
\date{Monday December  1, 2008}

\begin{document}
\maketitle

Following chapter 9.

Differential equations are important: we often know something about the derivatives of a function, and we want to recover the function.

Easy to verify a solution; hard to find.

\subsection*{The technique from the homework}

You already solved a diff equ on your homework; this was using the
technique of ``separation of variables'' which we will talk about more
on Wednesday.

\subsection*{Melting icecube}

The change in volume of a melting object is proportional to its
surface area.  That is,
$$
\frac{dV}{dt} = k \, S(t)
$$
For a cube of side length $r$, $V = r^3$ and $S = 6r^2$, so
$$
3 \, r(t)^2 \frac{dr}{dt} = k \cdot 6 r(t)^2
$$
Divide both sides by $r(t)$ and get
$$
3 \, \frac{dr}{dt} = 6k
$$
so $\frac{dr}{dt} = 2k$.

\subsection*{Homogeneous first-order linear equations}

Consider:
$$
a(x) \, f'(x) + b(x) \, f(x) = 0
$$
Might as well divide everything by $q(x)$, so we are left with
$$
f'(x) + h(x) \, f(x) = 0.
$$

Note that if $f_1$ and $f_2$ are both solutions, then $f_1 + f_2$ is a solution.

Note that if $f_1$ is a solution, so is $\alpha \cdot f_1$.

Can we solve this equation?  Yes.  Rewrite it as
$$
\frac{f'(x)}{f(x)} = -h(x)
$$
We integrate both sides
$$
\log f(x) = - \int h(x) \, dx + C
$$
But then
$$
f(x) = e^{- \int h(x) \, dx + C} = k e^{-H(x)}.
$$
This is a solution, and indeed, every solution has this form.

Example: the change in the value of my company is proportional to its
current value.
$$
f'(x) = k \, f(x)
$$
which we can solve.  Get $e^{kx + C}$.

\subsection*{Nonlinear equations}

Solve something like
$$
f''(x) + f(x)^2 = 0.
$$
Note that if $f_1$ and $f_2$ are solutions, there is no reason to
believe $f_1 + f_2$ would be a solution.

You could solve these by finding the Taylor coefficients!

\subsection*{Homogeneous second-order linear diff equ}

Suppose we want to solve
$$
f''(x) + b(x) \, f'(x) + c(x) \, f(x) = 0
$$
In other words
$$
\left( \frac{d^2}{dx^2} + b(x) \, \frac{d}{dx} + c(x) \right) f(x) = 0
$$
Suppose we could ``factor'' the operator into
$$
\left( \frac{d}{dx} - r_1 \right) \left( \frac{d}{dx} - r_2 \right) f(x) = 0
$$
But then solutions include
$$
f(x) = e^{\int r_1(x) \, dx} \mbox{ and } f(x) = e^{\int r_2(x) \, dx}
$$

Very concretely: 
$$
f''(x) - 5f(x) + 6 = 0
$$
This can be factored as
$$
\left( \frac{d}{dx} - 2 \right) \left( \frac{d}{dx} - 3 \right) f(x) = 0
$$
So solutions include $e^{2x}$ and $e^{3x}$.  Check?

More surprisingly, the spring equation:
$$
f''(x) = -f(x)
$$
can be written as
$$
\left( \frac{d^2}{dx^2} + 1 \right) f(x) = 0
$$
and then written as
$$
\left( \frac{d}{dx} - i \right) \left( \frac{d}{dx} + i \right) f(x) = 0
$$
where $i^2 = -1$.  But then the solutions we might guess would include
$$
f(x) = C \cdot e^{\pm ix}
$$
This includes sine and cosine!  Why?

(Problems appear if there are repeated roots)

\subsection*{And the third degree case?  And fourth?}

\subsection*{What about the nonhomogeneous case?}

What if we have
$$
f'(x) + h(x) \, f(x) = g(x)?
$$
Can we find a solution?

Note that if $f_1$ and $f_2$ are solutions, it is no longer the case
that $f_1 + f_2$ is a solution.

Trick: integrating factor.  Set $H(x) = \int h(x) \, dx$, and multiply
by $e^{H(x)}$.

Then we want to solve
$$
e^{H(x)} f'(x) + h(x) \, e^{H(x)} \, f(x) = e^{H(x)} \, g(x).
$$
But this is the same as
$$
\frac{d}{dx} \left( e^{H(x)} f(x) \right) = e^{H(x)} \, g(x).
$$
And so
$$
e^{H(x)} f(x) = \int e^{H(x)} \, g(x) \, dx + C
$$

General trick: sometimes multiplying by something (the ``integrating
factor'') lets us solve a previously unsolvable differential equation.

What about nonhomogeneous second order equations?  The method of
``undetermined coefficients'' (i.e., make an educated guess).



\end{document}
