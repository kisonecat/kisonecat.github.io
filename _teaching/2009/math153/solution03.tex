\documentclass[12pt]{article}
\usepackage{add-copyright}

\usepackage{fullpage}
\usepackage{nopageno}
\usepackage{amsthm}
\usepackage{amsmath}
\usepackage{amssymb}
\newcommand{\R}{\mathbb{R}}
\newcommand{\N}{\mathbb{N}}
\usepackage[margin=0.5in]{geometry}

\title{Homework 3}
\author{Jim Fowler}
\date{Due Wednesday, October 8, 2008}

\long\def\symbolfootnote[#1]#2{\begingroup%
\def\thefootnote{\fnsymbol{footnote}}\footnote[#1]{#2}\endgroup}

\begin{document}
\maketitle

\begin{enumerate}

\item State whether the following sequences converge (and, if so,
  state the limit).

\begin{description}
\item[(a)] $a_n = \sqrt{n}$.  Since $a_n$ is not bounded above, it diverges.
\vfill
\item[(b)] $b_n = 2^{-n} + \frac{1}{n}$.
$$
\lim_{n \to \infty} b_n = \lim_{n \to \infty} 2^{-n} + \lim_{n \to \infty} \frac{1}{n} = 0 + 0 = 0.
$$
\vfill
\item[(c)] $c_n = \displaystyle\frac{n+1}{n^2}$.
$$
\lim_{n \to \infty} c_n = \lim_{n \to \infty} \frac{n+1}{n^2} = \lim_{n \to \infty} \left( \frac{n}{n^2} + \frac{1}{n^2} \right) = \lim_{n \to \infty} \left( \frac{1}{n} + \frac{1}{n^2} \right) = 0 + 0 = 0.
$$
\vfill
\item[(d)] $d_n = \displaystyle\frac{n + (-1)^n}{n}$.
$$
\lim_{n \to \infty} d_n = \lim_{n \to \infty} \frac{n + (-1)^n}{n} = \lim_{n \to \infty} \left( \frac{n}{n} + \frac{(-1)^n}{n} \right) = \lim_{n \to \infty} \left( 1 + \frac{(-1)^n}{n} \right) = 1.
$$
\vfill
\item[(e)] $e_n = (-1)^n \cdot n^3$.  Since $e_n$ is not bounded (because, for instance, $e_{2n} > 2n$), $e_n$ diverges.
\vfill
\item[(f)] $f_n = \cos \displaystyle\frac{1}{n}$.  Because $\cos$ is continuous,
$$
\lim_{n \to \infty} f_n = \lim_{n \to \infty} \cos \displaystyle\frac{1}{n} = \cos \left( \lim_{n \to \infty} \displaystyle\frac{1}{n} \right) = \cos 0 = 1.
$$
\vfill
\item[(g)] $g_n = \left|10 - n\right| - n$.  If $n > 10$, then $10 - n
  < 0$, so $\left| 10 - n \right| = n - 10$.  But then $g_n = (n - 10)
  - n = -10$.  So for large values of $n$, $g_n = -10$, and therefore, $\displaystyle\lim_{n \to \infty} g_n = -10$.
 \vfill
\item[(h)] $h_n = \displaystyle\frac{3^n}{4^n + 1}$.  We can use squeezing here.  For all $n \in \N$,
$$
0 \leq \frac{3^n}{4^n + 1} \leq \frac{3^n}{4^n} = \left(\frac{3}{4}\right)^n
$$
But the left-hand and right-hand sequences both converge to zero, so $\displaystyle\lim_{n \to \infty} h_n = 0$.
\vfill
\item[(i)] $i_n = \log \left(n+1\right) - \log n$.
Note that $\displaystyle\lim_{n \to \infty} \displaystyle\frac{n+1}{n} = 1$.  But $\log$ is continuous, so $\displaystyle\lim_{n \to \infty} \log \left(\displaystyle\frac{n+1}{n}\right) = \log 1 = 0$.  Since
$$
i_n = \log \left(n+1\right) - \log n = \log \left(\displaystyle\frac{n+1}{n}\right),
$$
we may conclude $\lim_{n \to \infty} i_n = 0$.
\vfill
\item[(j)] $j_n = \sqrt{n^2 + n} - n$.\symbolfootnote[3]{This is
    rather tricky; if you don't answer it, you will not be penalized.}
  The limit is $1/2$, which I will prove by squeezing.  First, for all
  $n \in \N$,
$$
j_n = \sqrt{n^2 + n} - n < \sqrt{n^2 + n + 1/4} - n = \sqrt{\left(n+\frac{1}{2}\right)^2} - n = n + \frac{1}{2} - n = \frac{1}{2}.
$$
On the other hand,

%\frac{n - 2}{2n} 

%\frac{1}{2} - \frac{1}{n}

% $$
% \frac{1}{4} + \frac{(n^2 - 1)^2}{n^2} + \frac{n^2 - 1}{n} \leq n^2 + n

% \left( \frac{1}{2} + \frac{n^2-1}{n} \right)^2 \leq n^2 + n

% \left( \frac{1}{2} - \frac{1}{n} + n \right)^2 \leq n^2 + n

% \left( \frac{1}{2} - \frac{1}{n} + n \right)^2 \leq n^2 + n

% \frac{1}{2} - \frac{1}{n} + n \leq \sqrt{n^2 + n}

% $$

% $$
% \frac{1}{2} - \frac{1}{n} \leq j_n = \sqrt{n^2 + n} - n
% $$
% because

\begin{eqnarray*}
\lim_{n \to \infty} j_n
&=&  \lim_{n \to \infty} \sqrt{n^2 + n} - n \\
&=&  \lim_{n \to \infty} \sqrt{n^2 + n} - n \\
\end{eqnarray*}
\vfill
\end{description}

\item Give an $\epsilon$-$K$ proof that the sequence $a_n =
  \displaystyle\frac{3}{n}$ converges,

\item Prove that if $\displaystyle\lim_{n \to \infty} a_n = L$ and
  $\displaystyle\lim_{n \to \infty} b_n = M$, then
  $\displaystyle\lim_{n \to \infty} a_n + b_n = L + M$.

\item Let $a_n$ be a sequence of real numbers, and set $b_n = |a_n|$.
  If $b_n$ converges, does $a_n$ converge?  If so, prove it; if not,
  provide a counterexample.

\item Let $a_n$ and $b_n$ be sequences of real numbers; suppose
  $\displaystyle\lim_{n \to \infty} a_n = 0$.  Is it the case that
$$
\lim_{n \to \infty} a_n \cdot b_n = 0?
$$
If so, prove it; if not, provide a counterexample.

\end{enumerate}

\end{document}
