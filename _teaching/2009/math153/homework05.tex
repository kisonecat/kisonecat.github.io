\documentclass[12pt]{article}

\usepackage{fullpage}
\usepackage{nopageno}
\usepackage{amsthm}
\usepackage{amsmath}
\usepackage{amssymb}
\newcommand{\R}{\mathbb{R}}
\newcommand{\N}{\mathbb{N}}
\usepackage[margin=1in]{geometry}
\usepackage{add-copyright}

\title{Homework 5}
\date{Due Wednesday, October 15, 2008}

\long\def\symbolfootnote[#1]#2{\begingroup%
\def\thefootnote{\fnsymbol{footnote}}\footnote[#1]{#2}\endgroup}

\begin{document}
\maketitle

\textbf{Remember:} October 20 is the date of the first midterm!

\begin{description}

\item[(a)] On page 558, section 11.5, do problems: 2, 10, 13, 17, 23, 27.

\vfill

\item[(b)] On page 563, section 11.6, do problems: 3, 9, 14, 15, 21, 31.

\vfill

\item[(c)] Define two functions as follows:
\begin{eqnarray*}
f(x) &=& x + \cos x \, \sin x \\
g(x) &=& e^{\sin x} \left( x + \cos x \, \sin x \right).
\end{eqnarray*}
We want to calculate $\displaystyle\lim_{x \to \infty} \displaystyle\frac{f(x)}{g(x)}$.  Since this is the indeterminate form $\infty/\infty$, we apply l'H\^opital's rule as follows:
\begin{eqnarray*}
\lim_{x \to \infty} \frac{f'(x)}{g'(x)}
&=& \lim_{x \to \infty} \frac{1 + \cos^2 x - \sin^2 x}{e^{\sin x} \, \cos x \, \left( x + \sin x \, \cos x + 2 \, \cos x \right)} \\
&=& \lim_{x \to \infty} \frac{\cos^2 x + \sin^2 x + \cos^2 x - \sin^2 x}{e^{\sin x} \, \cos x \, \left( x + \sin x \, \cos x + 2 \, \cos x \right)} \\
&=& \lim_{x \to \infty} \frac{2 \cos^2 x}{e^{\sin x} \, \cos x \, \left( x + \sin x \, \cos x + 2 \, \cos x \right)} \\
&=& \lim_{x \to \infty} \frac{2 \cos x}{e^{\sin x} \, \left( x + \sin x \, \cos x + 2 \, \cos x \right)} = 0.
\end{eqnarray*}
(I left out some steps in computing the derivative of the
denominator.)  Therefore, by l'H\^opital's rule, the limit vanishes.
Or\ldots does it?  Note that $g(x) = e^{\sin x} \, f(x)$, hence
$$
\lim_{x \to \infty} \frac{f(x)}{g(x)} = \lim_{x \to \infty} \frac{1}{e^{\sin x}}
$$
which does not exist.  Where did I make a mistake?

\end{description}

\end{document}
