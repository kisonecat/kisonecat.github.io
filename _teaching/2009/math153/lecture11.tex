\documentclass[12pt]{article}
\usepackage{fullpage}
\usepackage{amsthm}
\usepackage{amsmath}

\newtheorem*{example}{Example}
\newtheorem*{thm}{Theorem}

\title{Lecture 11: Series}
\author{Math 153 Section 57}
\date{Wednesday October 22, 2008}

\begin{document}
\maketitle

Following chapters 12.1 and 12.2.

\section*{Sigma notation}

Write
$$
\sum_{n=0}^N f_n
$$
or
$$
\sum_{n=a}^b f_n
$$
Draw an analogy with integration.

Some properties:
$$
\sum_{n=a}^b (\alpha f_n + \beta g_n) = \alpha \sum_{n=a}^b f_n + \beta \sum_{n=a}^b g_n
$$
and
$$
\sum_{n=a}^b f_n + \sum_{n=b+1}^c f_n = \sum_{n=a}^c f_n
$$

Classic example: $\sum_{n=1}^b n = (b)(b+1)/2$

\section*{Infinite series}

Consider a series $\sum_{k=0}^\infty a_n$.

A partial sum is
$$
s_n = \sum_{k=0}^n a_n.
$$
If $\lim_{n \to \infty} s_n = L$, then we write
$$
\sum_{k=0}^\infty a_k = L
$$
We call $L$ the sum of the series.

This is not the same as ``adding up all the terms in the sequence.''

\subsection*{Example}

The series $\sum_{n=1}^\infty 1$ diverges.

The series $\sum_{n=1}^\infty 1/2^n$ converges to 1 (show the
picture).  Mention Zeno's paradox, and the Singularity.

The series $\sum_{n=1}^\infty (-1)^n$ diverges.  Thompson's Lamp
(every interval, the lamp switches on or off).

\subsection*{The most important example}

Geometric series.

If $-1 < x < 1$, then $\sum_{n=1}^\infty x^n = 1/(1-x)$.  For other values of $x$, the series diverges.

Proof: compute $s_k = \sum_{n=1}^k x^n$.  Multiply by $(1-x)$ to get $(1-x) s_k = 1 - x^{k+1}$, so
$$
s_k = \frac{1-x^{k+1}}{1-x}
$$
And take the limt as $k \to \infty$.

\subsection*{Application}

$\sum_{n=1}^\infty \frac{9}{10^n} = 1$.

\subsection*{Wrong applications}

$\sum_{n=0}^\infty 2^n = -1$?  Uhm, no.

\section*{Theorems}

Sum of convergent series are convergent.

Products of convergent series by a constant are convergent.

\section*{Simplest criterion}

If $\sum_{k=0}^\infty a_k$ converges, then $\lim a_k = 0$.

\end{document}
