\documentclass[12pt]{article}
\usepackage{fullpage}
\usepackage{amsthm}
\usepackage{amsmath}

\usepackage{amssymb}

\newcommand{\N}{\mathbb{N}}
\newcommand{\Q}{\mathbb{Q}}
\newcommand{\Z}{\mathbb{Z}}
\newcommand{\R}{\mathbb{R}}

\newtheorem*{example}{Example}
\newtheorem*{thm}{Theorem}

\title{Lecture 2: Sequences}
\author{Math 153 Section 57}
\date{Wednesday October  1, 2008}

\begin{document}
\maketitle

We will be following chapter 11.2.

Here we introduce sequences (some ``nouns''), define some properties
(bounded above, bounded below, increasing, decreasing, non-decreasing,
non-increasing; the ``adjectives''), and prove that some sequences
satisfy some of the properties.

Remember: $\N$ is for the natural numbers, and $\R$ is for the real numbers.

\section{Sequences}

Intuitive definition: a list of numbers that goes on forever.

Precise definition: a function $a : \N \to \R$, but instead of writing $a(n)$, we normally write it with subscripts, like $a_1, a_2, \ldots$.

Some terminology: the numbers in the sequence are ``terms,'' e.g., the first term, second term, third term.

\subsection{Building sequences}

Define a sequence by giving a formula for the $n$-th term.

Examples of sequences: $a_n = n$.  $b_n = 2n$.  $c_n = 2^n$.  $d_n = (-1)^n$.

Modify sequences: $f_n = 17 \cdot a_n$.

Combine sequences: $g_n = b_n + c_n$, or $h_n = c_n \cdot d_n$.  Then $g_n = n + 2n$, and $h_n = (-2)^n$.

\subsection{Why do we care?}

Mathematically: A fun object to play with, sort of like numbers (we
can add, subtract, multiply, divide).

Practically: sequences come up all the time: $a_n$ might be how much
money I have on the $n$ day of trading stocks (decreasing), the number
of rabbits in a field after $n$ generations (unbounded, increasing),
etc.

\subsection{Boundedness}

A sequence $a_n$ is \textbf{bounded above} if there is a number $b$
such that $b \geq a_n$ for all $n \in \N$.

A sequence $a_n$ is \textbf{bounded below} if there is a number $b$
such that $b \leq a_n$ for all $n \in \N$.

A sequence is \textbf{bounded} if it is bounded above, and bounded
below.

Examples: $a_n = 17$.  $a_n = 2^n$.  $a_n = -n$.  $a_n = (-1)^n$.
$a_n = (-1)^n \cdot n$.

\subsection{Monotonicity}

Increasing?  Decreasing?  Nonincreasing?  Nondecreasing?

Increasing means for all $n \in \N$ that $a_n < a_{n+1}$.

Decreasing means for all $n \in \N$ that $a_n > a_{n+1}$.

Non-increasing means for all $n \in \N$ that $a_n \geq a_{n+1}$.

Non-decreasing means for all $n \in \N$ that $a_n \leq a_{n+1}$.

Examples of increasing sequences: $a_n = 2n$, $a_n = n^2$.

Proofs.

Example of a decreasing sequence: $a_n = 1/n$.

Example of a decreasing sequence: $a_n = 1/n^2$.

Proofs.

Examples of non-increasing sequences: $a_n = 17$.

Examples of non-decreasing sequences:
$$
a_n = \begin{cases}
(n+1)/2 & \mbox{if $n$ is odd,} \\
n/2 & \mbox{if $n$ is even,}
\end{cases}
$$

Another example: $a_n = (-1)^n + n$.

\subsection{Some remarks}

To prove something is increasing, you must check infinitely many things.

To prove something is not increasing, you need a single counterexample.

``Not increasing is not the same thing as non-increasing.''

\subsection{Some basic theorems.}

The sum of two bounded sequences is bounded.

The sum of two unbounded sequences is unbounded?

\end{document}
