\documentclass[11pt,letterpaper]{article}

\title{A cool trick for limits of sequences}
\author{Jim Fowler}

\usepackage{fullpage}
\usepackage{nopageno}
\usepackage{amsmath}
\usepackage[margin=1in]{geometry}
\usepackage{add-copyright}

\usepackage{amsthm}
\newtheorem{theorem}{Theorem}
\newtheorem*{theorem*}{Theorem}
\newtheorem*{slogan*}{Slogan}
\newtheorem{corollary}[theorem]{Corollary}
\newtheorem{lemmma}[theorem]{Lemma}
\newtheorem{proposition}[theorem]{Proposition}

\begin{document}

\section*{A cool trick for limits of sequences}

\begin{slogan*}
  Taking successive differences of the terms in a sequence is vaguely
  like differentiating a function.
\end{slogan*}
Taking this slogan seriously, the following might be called a
``l'H\^opital's rule for sequences.''  Here, the differences of
successive terms take the place of the derivative:
\begin{theorem*}[Stolz-Ces\`aro]
  Let $a_n$ and $b_n$ be sequences of real numbers, both unbounded.
  Assume $b_n$ is positive and increasing.  If
$$
\lim_{n \to \infty} \frac{a_{n+1} - a_n}{b_{n+1} - b_n} = L
$$
then $\displaystyle\lim_{n\to\infty} a_n/b_n = L$.
\end{theorem*}

\section*{A cleaner proof to an old problem}

You might remember
$$
L = \lim_{x \to 0} \left( \frac{1}{\sin^2 x} - \frac{1}{x^2} \right).
$$
This problem, done in the straightforward way, required four
applications of l'H\^opital to increasingly ugly denominators.  There
is a trick---it doesn't require any less differentiation, but the
differentiation is much easier.  Since
$$
\lim_{x \to 0} \frac{\sin x}{x} = 1
$$
we can multiply the above by this, without changing the limit $L$.  So
$$
L = \lim_{x \to 0} \left( \frac{\sin^2 x}{x^2} \right) \cdot \left( \frac{1}{\sin^2 x} - \frac{1}{x^2} \right)
$$
Gathering these terms together gives
$$
L = \lim_{x \to 0} \left( \frac{1}{x^2} - \frac{\sin^2 x}{x^4} \right),
$$
which we put over a common denominator
$$
L = \lim_{x \to 0} \frac{x^2 - \sin^2 x}{x^4}.
$$
But this is pretty easy to apply l'H\^opital to:
$$
L = \lim_{x \to 0} \frac{2 x - 2 \sin x \cos x}{4 x^3} = 
\lim_{x \to 0} \frac{2 x - \sin (2x)}{4 x^3}.
$$
And again,
$$
L = \lim_{x \to 0} \frac{2 - 2 \cos (2x)}{12 x^2}.
$$
And again,
$$
L = \lim_{x \to 0} \frac{4 \sin (2x)}{24 x}.
$$
And once more,
$$
L = \lim_{x \to 0} \frac{8 \cos (2x)}{24} = \frac{8}{24} = \frac{1}{3}.
$$

\end{document}