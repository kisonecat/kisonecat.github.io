\documentclass[11pt]{article}
\usepackage{fullpage}
\usepackage{nopageno}
\usepackage[margin=1.5cm]{geometry}
\usepackage{add-copyright}

\usepackage{amssymb}
\newcommand{\R}{\mathbb{R}}
\newcommand{\N}{\mathbb{N}}

\title{Taylor Series You Should Know}

\begin{document}
\section*{Taylor Series You Should Know}

You should have the following series memorized
\begin{eqnarray*}
\frac{1}{1-x} &=& \sum^{\infty}_{n=0} x^n = 1 + x + x^2 + x^3 + \cdots \hspace{1em}(\mbox{when $-1 < x < 1$)} \\
e^{x} &=& \sum^{\infty}_{n=0} \frac{x^n}{n!} = 1 + x + \frac{x^2}{2!} + \frac{x^3}{3!} + \cdots \\
\sin x &=& \sum^{\infty}_{n=0} \frac{(-1)^n}{(2n+1)!} x^{2n+1}\quad =  x - \frac{x^3}{3!} + \frac{x^5}{5!} - \cdots \\
\cos x &=& \sum^{\infty}_{n=0} \frac{(-1)^n}{(2n)!} x^{2n}\quad =  1 - \frac{x^2}{2!} + \frac{x^4}{4!} - \cdots
\end{eqnarray*}
Note the similarity between $\sin x$, $\cos x$ and $e^x$.

\vfill

\section*{How to Find a Taylor Series}

Given a smooth function $f$, we can always write down a Taylor series;
there is no guarantee that the series converges to anything, let alone
to the function.  Given a smooth function $f : \R \to \R$, its Taylor
series (around $0$) is
$$
\sum_{n=0}^\infty \frac{f^{(n)}(0) \, x^n}{n!}
$$
A common mistake is to use $f^{(n)}(x)$ instead of $f^{(n)}(0)$.

Given a smooth function $f : \R \to \R$, its Taylor series expanded
around $a$ is
$$
\sum_{n=0}^\infty \frac{f^{(n)}(a) \, (x-a)^n}{n!}
$$
The most important example of this sort of Taylor series is
$$
\log x = \sum_{n=1}^\infty \frac{(-1)^{n+1} \, (x-1)^n}{n} = (x-1) - \frac{(x-1)^2}{2} + \frac{(x-1)^3}{3} - \frac{(x-1)^4}{4} + \cdots
$$

\vfill

\section*{How to \textit{More Easily} Find a Taylor Series}

There are tricks: you can add, subtract, multiply, and divide (!)
power series.  You can substitute one series into another.

\vfill

\section*{Why to Find a Taylor Series}

Taylor series are good for:
\begin{description}
\item[Estimating values] by cutting the series off after a few terms and bounding the remainder;
\item[Seeing a snapshot] by looking at the first few terms, you can get a sense of what the function is doing near zero;
\item[Proving facts about functions] like $\cos (-x) = \cos x$, which follows from the fact that the Taylor series for cosine only includes even degree terms;
\item[Doing calculus] because it is easy to differentiate and integrate term-by-term;
\item[Solving differential equations] for which it is often easier to find a Taylor series for a solution.
\end{description}

\end{document}
