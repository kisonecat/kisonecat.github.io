\documentclass[12pt]{article}
\usepackage{fullpage}
\usepackage{amsthm}
\usepackage{amsmath}

\newtheorem*{example}{Example}
\newtheorem*{thm}{Theorem}

\title{Lecture 23: Integration by parts}
\author{Math 153 Section 57}
\date{Wednesday November 19, 2008}

\begin{document}
\maketitle

Following chapter 8.2.

Jeopardy---integration as an inverse operation.  295927 is 541 times 547.

Overview about integration---emphasize again the fundamental theorem
and its amazing power

antidifferentiation and the challenge of inverse operations

$u$ substitution is the chain rule---backwards!

parts is the product rule---backwards!  not a rule, but a
technique---something like a skilled warrior who transforms the
situation into something more advantengeous rather than outright solving it

what is $u$?  what is $v$?  Try to make choices to have easy
antiderivatives.

$$
\int \frac{\ln x}{x^2}\, dx = -\frac{\ln x}{x} - \int (1/x)(-1/x)\, dx.
$$
$$
\int \frac{\ln(\sin x)}{(\cos x)^2}\, dx = \ln(\sin x)\tan x - \int \frac{1}{\tan x} \tan x\, dx.
$$

$$
\int x\, dx = x^2 - \int x\, dx
$$
so we can find an antiderivative for $x^2$.  Same trick works for $\log x$.

$$
\int x^3e^{x^2}\, dx=\frac{1}{2}e^{x^2}(x^2-1)+C.
$$

\subsection*{Repeated integration}

$$\int uv = u v_1 - u' v_2 + u'' v_3 - \cdots + (-1)^{n}\ u^{(n)} \ v_{n+1}
$$

\subsection*{Integrals of products of sines and cosines}

Do
$$
\int \sin(ax) \, \cos(bx) \, dx
$$
by using an angle sum formula? No!  Use parts twice.

Powers of trig functions?  Replace $\sin^2 x$ with $1 - \cos^2 x$ as follows
$$
\int \cos^3 x \, \sin^5 x \, dx = 
\int \cos x (1 - \sin^2 x) \, \sin^5 x \, dx = 
\int  (1 - u^2) \, u^5 \, du = 
$$

\end{document}
