\documentclass[12pt]{article}

\usepackage{hyperref}
\usepackage{fullpage}
\usepackage{nopageno}
\usepackage{amsthm}
\usepackage{amsmath}
\usepackage{amssymb}
\newcommand{\R}{\mathbb{R}}
\newcommand{\N}{\mathbb{N}}
\usepackage[margin=0.5in]{geometry}
\usepackage{wasysym}

\usepackage{graphicx}

\title{Homework 14}
\date{Due Monday, November 24, 2008}

\long\def\symbolfootnote[#1]#2{\begingroup%
\def\thefootnote{\fnsymbol{footnote}}\footnote[#1]{#2}\endgroup}

\begin{document}
\maketitle

\begin{center}
  A common question is: what should I choose for $u$?  What should I
  choose for $dv$?  You should choose something easier to
  differentiate for $u$, and easy to antidifferentiate for $dv$.
\end{center}

\begin{description}

\item[(a)] Use integration by parts twice to rewrite 
$$
\int \cos (2x) \, \sin (3x) \, dx
$$
in terms of itself---and therefore, find an antiderivative.

Such problems as these are usually solved by using a trigonometric
substitution, but to illustrate the usefulness of integration by
parts, refrain from using the angle sum formulas.

\vfill

\item[(b)] Here is our first foray into the world of \textbf{Fourier
    series}.  For real numbers $a_1, a_2, \ldots, a_k$, define
$$
f(x) = \sum_{n=1}^k a_n \sin (nx).
$$
For each integer $m$ with $1 \leq m \leq k$, compute
$$
\frac{1}{\pi} \int_{-\pi}^\pi f(x) \, \sin (mx) \, dx
$$
by using the same trick you used in problem~\textbf{(a)}.

\textit{Hint:} you will be helped by first calculating
$$
\int_{-\pi}^\pi \sin (nx) \, \sin (mx) \, dx.
$$
There's a lovely pattern for you to discover.

\vfill

\item[(c)] Use integration by parts to compute, for each natural
  number $n \in \N$,
$$
\int_{-\pi}^\pi x \, \sin (nx) \, dx.
$$

\vfill

\item[(d)] Use integration by parts to evaluate
$$
\int_0^1 \arctan x \, dx.
$$
\textit{Hint:} here you should set $u = \arctan x$, and $dv = dx$.

\vfill

\item[(e)] Replace $\cos^2 x$ with $1 - \sin^2 x$ in order to find
$$
\int \cos^5 x \, \sin^4 x \, dx = \int \cos x \, \left( \cos^2 x \right)^2 \, \sin^4 x \, dx.
$$

\vfill

\end{description}


\end{document}
