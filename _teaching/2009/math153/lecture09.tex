\documentclass[12pt]{article}
\usepackage{fullpage}
\usepackage{amsthm}
\usepackage{amsmath}

\newtheorem*{example}{Example}
\newtheorem*{thm}{Theorem}

\title{Lecture 9: More improper integrals}
\author{Math 153 Section 57}
\date{Friday October 17, 2008}

\begin{document}
\maketitle

Continuing in chapter 11.7.

Also, review for the test.

\section{Insufficiently impressed with logarithms?}

How about quartersquares?

\section{Review for the test}

\section{Existence without computation}

Sometimes we can show that an improper integral converges without actually doing the calculation.  This is the comparison test.

If $f$ and $g$ are cts on $[a,\infty)$ and $0 \leq f(x) \leq g(x)$,
then $\int_a^\infty g(x) \, dx$ converges implies $\int_a^\infty f(x)
\, dx$ converges.  And the contrapositive.

Example: $\int_1^\infty dx/\sqrt{1+x^{100}}$ converges because
$$
\frac{1}{\sqrt{1 + x^{100}}} < 1/x^{50}
$$
as long as $x \geq 1$.

Example: $\int_1^\infty dx/\sqrt{1+x^{100} + \sin^2 x}$ converges because
$$
\frac{1}{\sqrt{1 + x^{100} + \sin^2 x}} < \frac{1}{\sqrt{1 + x^{100}}}
$$
as long as $x \geq 1$.

Example: $\int_2^\infty dx/\sqrt{x^2 - \sin^2 x}$ diverges because
$$
\frac{1}{x} < \frac{1}{\sqrt{x^2 - \sin^2 x}}
$$

\section{If only I could l'H\^opital sequences\ldots}

Now you can!

The following might be called a ``l'H\^opital's rule for sequences.''
Here, the differences of successive terms take the place of the
derivative:
\begin{thm}[Stolz-Ces\`aro]
  Let $a_n$ and $b_n$ be sequences of real numbers, both unbounded.
  Assume $b_n$ is positive and increasing.  If
$$
\lim_{n \to \infty} \frac{a_{n+1} - a_n}{b_{n+1} - b_n} = L
$$
then $\displaystyle\lim_{n\to\infty} a_n/b_n = L$.
\end{thm}

\section{Review $\epsilon$-$K$ proofs}

\end{document}
