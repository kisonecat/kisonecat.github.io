\documentclass[12pt]{article}
\usepackage[margin=0.8in,top=1in,bottom=1in]{geometry}
\usepackage{multicol}
%\usepackage{add-copyright}

\usepackage{amsthm}
\theoremstyle{definition}
\newtheorem*{example*}{Example}
\newtheorem{question}{Question}
\newcommand{\limn}{\displaystyle\lim_{n \to \infty}}
\newcommand{\sumn}{\displaystyle\sum_{n=1}^{\infty}}

\title{Frequently Asked Questions about and on Final Exams}

\usepackage{amssymb}
\newcommand{\R}{\mathbb{R}}
\newcommand{\N}{\mathbb{N}}

\begin{document}
\renewcommand{\labelitemi}{$\qed$}

\section*{Frequently Asked Questions about the Final Exam}

\subsection*{When is the final exam?}

The exam is on \textbf{Tuesday, December 9, 2008}.  The exam starts at
\textbf{4:00pm}, and ends at \textbf{6:00pm}, providing for
\textbf{120 minutes} of examination.

\subsection*{Where is the exam?}

The exam will be in the usual lecture location, namely \textbf{Kent 101}.

\subsection*{What are some questions that will be on the final exam?}

These questions will absolutely appear on the exam.
\begin{itemize}
\item What is the definition of the limit of a sequence, $\limn a_n = L$?
\item What is the definition of the limit of a series, $\displaystyle\sum_{n=1}^\infty a_n = L$?
\item Use an $\epsilon$-$K$ argument to prove that the sequence \rule{1in}{12pt} converges.
\item What letter grade do you believe you have earned this quarter?
\end{itemize}
There will, as usual, be extra credit questions at the end.

% \item There will be \textbf{one improper integral} on the exam; it
%   will not involve a challenging integral: I would rather that you
%   make sure you are extremely comfortable with
%   \textit{differentiation} on this exam, so you can handle the
%   questions involving l'H\^opital's rule---future exams will give us
%   plenty of practice with the fancier techniques of integration.

% \item The last question will be \textbf{extra credit} with true/false
%   questions.  I think these are \textbf{terribly enjoyable!}  They
%   will ask you to agree and disagree with statements of theorems,
%   slightly modified statements of theorems, existence of sequences
%   exhibiting certain kinds of phenomena, etc.

% \item You will lose points if you surround an otherwise convincing
%   argument with false statements (after all, once I have proved $2 =
%   1$, I can prove anything!).  \textbf{Erase} or cross-out untrue
%   statements \textbf{for full credit.}

% \item Style matters: if you are taking a limit, write $\lim$.  Do not
%   use an ``$=$'' between two expressions unless they are, in fact,
%   equal.  Do not \textbf{under any circumstances} divide by zero
%   during the exam.

\subsection*{What do these boxes mean?}

I have drawn boxes next to everything you should know, so you can
easily check things off when you believe you know it.

\subsection*{How ought I to write down my answers?}

You must not merely write down the answer; you must give the whole
story, showing all the steps you took to arrive at your answer.  You
must \textbf{justify the arguments} you make in order to receive full
credit.

\subsection*{What definitions must I know?}

You may be asked to give definitions of the following terms: 
\begin{multicols}{3}
\begin{itemize}
\item bounded below,
\item bounded above,
\item increasing,
\item decreasing,
\item non-increasing,
\item non-decreasing,
\item conditional convergence,
\item absolute convergence,
\item Taylor's theorem,
\item Lagrange's theorem.
\end{itemize}
\end{multicols}

\pagebreak

\subsection*{What convergence tests must I know?}

You must be able to both \textbf{apply} and \textbf{state precise descriptions of} the following tests:
\begin{multicols}{2}
\begin{itemize}
\item the $n^{\mbox{th}}$ term test,
\item comparison test,
\item limit comparison test,
\item $p$-series test,
\item geometric series test,
\item harmonic series test,
\item the ratio test,
\item alternating series test,
\item the root test,
\item the integral test.
\end{itemize}
\end{multicols}
\noindent Use the \textbf{limit comparison test} to determine whether
$
\displaystyle\sum \displaystyle\frac{\mbox{polynomial}}{\mbox{polynomial}}
$
converges.

\subsection*{What must I determine about \textit{sequences} on the exam?}

You must be able to determine whether a sequence is
\begin{multicols}{2}
\begin{itemize}
\item bounded, and
\item monotone.
\end{itemize}
\end{multicols}


\subsection*{How will I evaluate \textit{limits} on the exam?}

In addition to giving an $\epsilon$-$K$ proof in certain cases, you must be able to evaluate limits by using
\begin{multicols}{2}
\begin{itemize}
\item algebraic manipulation,
\item composition with continuous functions---i.e., $\limn f(a_n) = f(\limn a_n)$,
\item squeezing,
\item l'H\^opital's rule.
\end{itemize}
\end{multicols}
\noindent
In particular, you must be able to evaluate limits having the following indeterminate forms:
\begin{multicols}{7}
\small
\begin{itemize}
\item $\displaystyle\frac{0}{0}$,
\item $\displaystyle\frac{\infty}{\infty}$,
\item $0 \cdot \infty$,
\item $0^0$,
\item $1^\infty$,
\item $\infty^0$,
\item $\infty - \infty$
\end{itemize}
\end{multicols}

\subsection*{What must I do with \textit{series} on the exam?}

Given a series you must be able to do the following:
\begin{itemize}
\item Determine whether the series converges absolutely,
\item Determine whether the series converges conditionally,
\item Evaluate the series in certain cases, like $\displaystyle\sum_{n=0}^\infty x^n = \frac{1}{1-x}$ provided $|x| < 1$.
\end{itemize}
\noindent
Given a power series, you must be able to:
\begin{itemize}
\item Find the interval on which the power series converges,
\item Differentiate and integrate the power series term-by-term.
\end{itemize}

\subsection*{With \textit{Taylor series}, what must I be able to do?}

Given a function $f : \R \to \R$, you must be able to:
\begin{itemize}
\item Write down the first few terms of the Taylor series for $f$ around $0$,
\item Write down the first few terms of the Taylor series for $f$ around $a \neq 0$,
\item Find terms in a Taylor series using tricks like substitution,
\item Find the Taylor series for $f$---i.e., if $f(x) = \sum_{n=0}^\infty a_n x^n$, find a formula for $a_n$,
\item Use the Taylor series to find the values of derivatives of the function.
\end{itemize}
\noindent
There may be problems about approximation.
\begin{itemize}
\item Given an alternating series, find an approximate value and estimate the error.
\item Given a Taylor series, find an approximate value and estimate the error.
\end{itemize}

\subsection*{What sorts of \textit{integration} must I perform?}

There will be integrals and differential equations on the final exam.  You must:
\begin{itemize}
\item Evaluate improper integrals,
\item Compute $\int \sin(ax) \, \cos(bx) \, dx$ by using integration by parts,
\item Compute $\int \sin^n (x) \, \cos^m (x) \, dx$ for natural numbers $n$ and $m$,
\item Use partial fractions to compute $\displaystyle\int \frac{\mbox{polynomial}}{\mbox{polynomial}}$,
\item Solve inhomogeneous first-order linear differential equations by using an integrating factor,
\item Solve homogeneous second-order linear differential equations by factoring the derivative.
\end{itemize}

\pagebreak

%################################################################
\section*{Frequently Asked Questions on the Final Exam}

\begin{question}
Give an $\epsilon$-$K$ argument to prove that $\limn \displaystyle\frac{4}{n} = 0$.
\end{question}

\begin{question}
Give an $\epsilon$-$K$ argument to prove that $\limn \displaystyle\frac{4}{n^2} = 0$.
\end{question}

\begin{question}
Give an $\epsilon$-$K$ argument to prove that $\limn \displaystyle\frac{(-1)^n}{n^2} = 0$.
\end{question}

\begin{question}
Give an $\epsilon$-$K$ argument to prove that $\limn \displaystyle\frac{4 + 2\,n^2}{n^2} = 2$.
\end{question}

\begin{question}
Give an $\epsilon$-$K$ argument to prove that $\limn \displaystyle\frac{1 + 2\,n + 3\,n^2}{n^2} = 3$.
\end{question}

\begin{question}
Is the sequence $a_n = n^2$ bounded?  Is the sequence monotone?
\end{question}

\begin{question}
Is the sequence $a_n = n \, \cos n$ bounded?  Is the sequence monotone?
\end{question}

\begin{question}
Is the sequence $a_n = \cos (\pi n)$ bounded?  Is the sequence monotone?
\end{question}

\begin{question}
Is the sequence $a_n = n \cos^2 n + n \sin^2 n$ bounded?  Is the sequence monotone?
\end{question}

\begin{question}
Is the sequence $a_n = \sin (\pi n)$ bounded?  Is the sequence monotone?
\end{question}

\begin{question}
Is the sequence $a_n = \sin (\pi n)$ bounded?  Is the sequence monotone?
\end{question}

\begin{question}
Is the sequence $a_n = \sin \left( \frac{1}{n} \right)$ bounded?  Is the sequence monotone?
\end{question}

\begin{question}
Suppose $a_n$ and $b_n$ are bounded sequences.  Is the sequence $c_n = a_n + b_n$ necessarily bounded?
\end{question}

\begin{question}
Suppose $a_n$ and $b_n$ are bounded sequences.  Is the sequence $c_n = a_n \cdot b_n$ necessarily bounded?
\end{question}

\begin{question}
Suppose $a_n$ and $b_n$ are monotone sequences.  Is the sequence $c_n = a_n + b_n$ necessarily monotone?
\end{question}

\begin{question}
Evaluate $\limn \frac{n + \sqrt{n}}{1+n^2}$.
\end{question}

\begin{question}
Evaluate $\limn \frac{n^2 + \sin n + n \, \cos n}{(1+n)^3 - n^3}$.
\end{question}

\begin{question}
Evaluate $\limn \cos\left( \sin \left( 1/n^n \right) \right)$.
\end{question}

\begin{question}
Evaluate $\limn \cos \left( \left( 1 + \displaystyle\frac{1}{n} \right)^n \right)$.
\end{question}

\begin{question}
Evaluate $\limn (\pi/4)^{1/n}$.
\end{question}

\begin{question}
Evaluate $\limn \left(1/2\right)^n$.
\end{question}

\begin{question}
Evaluate $\limn \left(\sin (1/n)\right)^n$.
\end{question}

\begin{question}
Evaluate $\limn \left(\sin (1/n) + 1\right)^n$.
\end{question}

\begin{question}
Evaluate $\limn \left(\sin^2 (1/n) + 1\right)^n$.
\end{question}

\begin{question}
Evaluate $\limn \left(\sin^2 (1/n) + 2 \sin (1/n) + 1\right)^n$.
\end{question}

\begin{question}
Evaluate $\limn \left( 1 + \frac{\log 123456789}{n} \right)^n$.
\end{question}

\begin{question}
Search for the value of $\limn \left( 1 + \frac{50 \log 10}{n} \right)^{2n}$ on the Internet.
\end{question}

\begin{question}
Evaluate $\limn \frac{1}{\sqrt{n}}$.
\end{question}

\begin{question}
Evaluate $\limn \frac{\sqrt{n}}{\sqrt{n} + \sqrt[3]{n}}$.
\end{question}

\begin{question}
Evaluate $\limn \frac{ \left(10^{\left(10^{\left( 10^{10} \right)}\right)}\right)^n }{n!}$.
\end{question}

\begin{question}
Evaluate $\limn \frac{ \left(100^{\left(100^{\left( 100^{100} \right)}\right)}\right)^n }{\sqrt{n!}}$.
\end{question}

\begin{question}
Evaluate $\limn \displaystyle\frac{\log n}{n}$.
\end{question}

\begin{question}
Evaluate $\limn (2n)^{3/n}$.
\end{question}

\begin{question}
For each of the seven indeterminate forms:
$$ \displaystyle\frac{0}{0},\hspace{1ex}
 \displaystyle\frac{\infty}{\infty},\hspace{1ex}
0 \cdot \infty,\hspace{1ex}
 0^0,\hspace{1ex}
 1^\infty,\hspace{1ex}
 \infty^0,\hspace{1ex}
 \infty - \infty
$$
find a limit exhibiting the form.
\end{question}

\begin{question}
Evaluate $\limn \left(\displaystyle\frac{\cos n + \sin n + \sin^2 n}{4}\right)^n$.
\end{question}

%\item l'H\^opital's rule.
%\end{itemize}
%\begin{itemize}
%\item $\displaystyle\frac{0}{0}$,
%\item $\displaystyle\frac{\infty}{\infty}$,
%\item $0 \cdot \infty$,
%\item $0^0$,
%\item $1^\infty$,
%\item $\infty^0$,
%\item $\infty - \infty$
%\end{itemize}

\begin{question}
By Taylor's theorem,
$$
\sin \frac{1}{2} = (1/2) - \frac{(1/2)^3}{3!} + R_3(1/2) = \frac{23}{48} + R_3(1/2)
$$
I have built a right triangle, having a hypotenuse $48\,\mbox{cm}$,
and height $23\,\mbox{cm}$.  Use Lagrange's theorem to bound
$R_3(1/2)$, and thereby justify that this is a good way to build a
triangle having an angle of $1/2$ radians.
\end{question}

\begin{question}
Use Taylor's theorem to show that
$\log 2 \approx 7/12$ and estimate the error.
\end{question}

\begin{question}
Recall that (as you should have memorized for the exam)
$$
\log (1+x) = \sum_{n=1}^\infty \frac{(-1)^{n+1} \, x^n}{n}.
$$
Musically, an important fact is that $2^{19} \approx 3^{12}$.  Let's
see if we can show this using Taylor series.  Taking logarithms, we find
$$
19 \cdot \log 2 \approx 12 \cdot \log 3
$$
But $\log 3 = \log 1.5 + \log 2$, so
$$
19 \cdot \log 2 \approx 12 \cdot (\log 1.5 + \log 2)
$$
which means that
$$
7 \cdot \log 2 \approx 12 \cdot \log 1.5
$$
or by the Taylor series above
$$
7 \left( 1 - 1/2 + 1/3 + R_3(1) \right) \approx 12 \left( (1/2) - (1/2)^2/2 + (1/2)^2/3 + R_3(1/2) \right).
$$
Use a theorem on alternating series to bound $R_3(1)$ and $R_3(1/2)$ to show that this is possible.
\end{question}

\begin{question}
Does the series $\sumn n^4 / 4^n$ converge?
\end{question}

\begin{question}
Does the series $\sumn n!$ converge?
\end{question}

\begin{question}
Does the series $\sumn ( n + 1 )/(n)$ converge?
\end{question}

\begin{question}
Does the series $\sumn ( n! + 1 )/(n!)$ converge?
\end{question}

\begin{question}
Does the series $\sumn (1/3)^n$ converge?
\end{question}

\begin{question}
Does the series $\sumn \left( (1/3)^n + (1/4)^n \right)$ converge?
\end{question}

\begin{question}
  Does the series $\sumn \left( (1/3)^n + (1/3)^{n-1} \,(n/4) +
    (1/4)^n \right)$ converge?  For a very different method, try comparison with $\left( (1/3) +
    (1/4) \right)^n$.
\end{question}

\begin{question}
  Does the series $\sumn \left( n \, (1/3)^n \right)$ converge?  Did
  you do it without using the ratio test?  Can you estimate its limit?
\end{question}

\begin{question}
Does the series $\sumn \displaystyle\frac{n^2 + n + 1}{n^3 + n + 1}$ converge?
\end{question}

\begin{question}
Does the series $\sumn \displaystyle\frac{4\,n^{10} + n + 1}{(n^2 + 1)^6}$ converge?
\end{question}

\begin{question}
Does the series $\sumn \left( \displaystyle\frac{n}{n+10} \right)^n$ converge?
\end{question}

\begin{question}
Does the series $\sumn \displaystyle\frac{n+1}{n!}$ converge?
\end{question}

\begin{question}
Does the series $\sumn \displaystyle\frac{3\,n^6 + n^4}{(n!)^2}$ converge?
\end{question}

\begin{question}
Does the series $\sumn \displaystyle\frac{(2n)!}{(3n)!}$ converge?
\end{question}

\begin{question}
Does the series $\sumn \displaystyle\frac{(2n)! + 13}{(3n)! + 12}$ converge?
\end{question}

\begin{question}
Does the series $\sumn \displaystyle\frac{(2n)! - 13}{(3n)! + 12}$ converge?
\end{question}

\begin{question}
Does the series $\sumn \displaystyle\frac{1}{n}$ converge?
\end{question}

\begin{question}
Does the series $\sumn \displaystyle\frac{1}{n}$ converge?
\end{question}

\begin{question}
If $a_n$ converges absolutely, does $a_n$ converge?
\end{question}

\begin{question}
Does the series $\sumn (1/n! - 1/n)$ converge absolutely?  Converge conditionally?
\end{question}

\begin{question}
Does the series $\sumn 2^{-n}$ converge absolutely?  Converge conditionally?
\end{question}

\begin{question}
Does the series $\sumn (-2)^{-n}$ converge absolutely?  Converge conditionally?
\end{question}

\begin{question}
Does the series $\sumn (-1)^n / n$ converge absolutely?  Converge conditionally?
\end{question}

\begin{question}
Does the series $\sumn (-1)^n \displaystyle\frac{\log n}{n}$ converge absolutely?  Converge conditionally?
\end{question}

\begin{question}
Does the series $\sumn (-1)^n / \sqrt{n}$ converge absolutely?  Converge conditionally?
\end{question}

\begin{question}
Does the series $\sumn (-1)^n / n^2$ converge absolutely?  Converge conditionally?
\end{question}

\begin{question}
Does the series $\sumn \left(\sin n\right)/ n^2$ converge absolutely?  Converge conditionally?
\end{question}

\begin{question}
What are the first five terms of the Taylor series for
$f(x) = \sin x \cos x$ around zero?
\end{question}

\begin{question}
What are the first four terms of the Taylor series for
$f(x) = e^{\sin x}$ around zero?
\end{question}

\begin{question}
What are the first five terms of the Taylor series for
$f(x) = \cos^2 x$ around $\pi$?
\end{question}

\begin{question}
Find the Taylor series for $f(x) = e^{3x}$ around $x = 0$.
\end{question}

\begin{question}
Find the Taylor series for $f(x) = e^x - e^{-x}$ around $x = 0$.
\end{question}

\begin{question}
Find the Taylor series for $f(x) = e^x - e^{-x}$ around $x = 0$.
\end{question}

\begin{question}
Find the Taylor series for $f(x) = (1 + x)^{2}$ around $x = 0$.
\end{question}

\begin{question}
Find the Taylor series for $f(x) = (1 + x)^{100}$ around $x = 0$.
\end{question}

\begin{question}
Find the Taylor series for $f(x) = - x \cos x + \sin x$ around $x = 0$.
\end{question}

\begin{question}
Find the Taylor series for $f(x) = - x \cos x + \sin x$ around $x = 0$.
\end{question}

\begin{question}
Find the Taylor series for $f(x) = \cos^2 x \sin^3 x$ around $x = 0$.
\end{question}

\begin{question}
What are the first five terms of the Taylor series for
$f(x) = \sin x \cos x$ around $\pi$?
\end{question}

\begin{question}
Evaluate the integral
$$
\int_3^\infty \frac{dx}{(1-x)^2}
$$
\end{question}

\begin{question}
Evaluate the integral
$$
\int \frac{1}{x^2 - 9x + 20} \, dx
$$
\end{question}

\begin{question}
Evaluate the integral
$$
\int \frac{x^3 - 1}{1 + x + x^2} \, dx
$$
\end{question}

\begin{question}
Evaluate the integral
$$
\int \frac{x^3 - 2}{1 + x + x^2} \, dx
$$
\end{question}

\begin{question}
Evaluate the integral
$$
\int \sin (3x) \, \cos (3x) \, dx.
$$
\end{question}

\begin{question}
Evaluate the integral
$$
\int \sin^3 x \, \cos^{10} x \, dx.
$$
\end{question}

\begin{question}
Evaluate the integral
$$
\int \sin^5 x \, \cos^5 x \, dx.
$$
\end{question}

\begin{question}
Find a two nonconstant polynomials $p(x)$ and $q(x)$ so that
$$
\int_0^1 p(x) \, (1 + x^2) dx = 0 \mbox{ and } \int_0^1 q(x) \, (1 + x^2) dx = 0
$$
and $p(x)$ is not a multiple of $q(x)$.
\end{question}

\begin{question}
Find the general solution to the differential equation
$$
f'(x) + \frac{1}{x} f(x) = \sin x
$$
\end{question}

\begin{question}
Find the general solution to the differential equation
$$
f'(x) - \tan x f(x) = \sin x
$$
\end{question}

\end{document}