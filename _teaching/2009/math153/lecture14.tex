\documentclass[12pt]{article}
\usepackage{fullpage}
\usepackage{amsthm}
\usepackage{amsmath}

\newtheorem*{example}{Example}
\newtheorem*{thm}{Theorem}

\title{Lecture 14: Root and ratio tests}
\author{Math 153 Section 57}
\date{Wednesday October 29, 2008}

\begin{document}
\maketitle

Following chapter 12.4.

\subsection*{Ratio Test}

This is a test I do like, and it is easy to apply!

If $a_n > 0$, and $\lim a_{n+1}/a_n = L$, then if $L < 1$, $\sum a_n$
converges, if $L > 1$, then $\sum a_n$ diverges, and if $L = 1$, the
test is inconclusive.

Proof: If $L < 1$, choose $\epsilon$ so small that $L + \epsilon < 1$.  Then there exists $K$ so that for $n \geq K$, we have $| a_{n+1}/a_n - L | < \epsilon$.  In other words, $a_{n+1} / a_n < L + \epsilon < 1$.

So $a_{n+1} < (L + \epsilon) \cdot a_n$.  Therefore, $a_{n+k} < (L +
\epsilon)^n {a_k}$.  But $\sum (L + \epsilon)^n a_k$ converges, and
therefore, by comparison, so does $\sum_{n=k}^\infty a_n$.

\subsection{Best use of the ratio test}

We know that $\sum 1/n!$ converges, because for $n$ large
$$
\frac{1}{n!} < \frac{1}{n \cdot (n-1)} < \frac{1}{(n-1)^2}
$$
But we can also check this using the ratio test.

Set $a_n = 1/n!$.  Then $a_{n+1}/a_n = 1/(n+1)$ which converges to
zero, and we are done.

What do we know about $\sum 1/n!$?  It equals $e$.  We'll see that soon.

Anti-example: $\sum n^{n}/n!$.  But we already knew that since $\lim n^{n}/n! = \infty$.

Example: $\sum n^{n/2}/n!$.  Apply ratio test:
$$
\frac{\sqrt{(n+1)^{n+1}}}{(n+1)!} \cdot \frac{n!}{\sqrt{n^n}} =
\frac{\sqrt{n+1} \cdot \sqrt{(\frac{n+1}{n})^n}}{n+1} = 
\frac{\sqrt{(\frac{n+1}{n})^n}}{\sqrt{n+1}}
$$
which converges.

\subsection{Bad uses of the ratio test}

example: $\sum 1/(3n + 17)$.

\subsection{Sometuimes we have a parameter}

\subsection*{Which test to apply?}

$\lim a_n = 0$ is always a good thing to check!

Integral test is good for things you can integrate.

Root test is really only good for things with a powers of $n$.

Limit comparison takes care of all rational functions (i.e., polynomial
over polynomial).

Ratio test is good for factorials.

\subsection*{Absolute and conditional convergence}

Sometimes series have both positive and negative terms.

Theorem: If $\sum |a_n|$ converges, then $\sum a_n$ converges.

Proof: If $\sum |a_n|$, then $\sum 2|a_n|$ conv, then $\sum a_n + |a_n|$ conv, so $\sum a_n = \sum (a_n + |a_n|) - \sum |a_n|$ conv.

The proof doesn't work backwards!

Definition: $\sum a_n$ is absolutely convergent if $\sum |a_n|$ conv.

Absolute convergence implies convergence, but it is not the same.

Example: alternating series.

\end{document}
