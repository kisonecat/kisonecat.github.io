\documentclass[12pt]{article}

\usepackage{hyperref}
\usepackage{fullpage}
\usepackage{nopageno}
\usepackage{amsthm}
\usepackage{amsmath}
\usepackage{amssymb}
\newcommand{\R}{\mathbb{R}}
\newcommand{\N}{\mathbb{N}}
\usepackage[margin=1.5in]{geometry}
\usepackage{wasysym}
\usepackage{add-copyright}

\usepackage{graphicx}

\title{Homework 13}
\date{Due Friday, November 14, 2008}

\long\def\symbolfootnote[#1]#2{\begingroup%
\def\thefootnote{\fnsymbol{footnote}}\footnote[#1]{#2}\endgroup}

\begin{document}
\maketitle

\begin{center}
{\sc Nota Bene:} The second midterm is scheduled for Monday, November 17.
\end{center}

\vfil

\begin{description}

\item[(a)] Study for the midterm.  Prepare yourself to do your very best.

\vfill

\item[(b)] On page 632, in section 12.9, do problems: 4, 14, 22, 28, 44.

\vfill

\item[(c)] Here is another way to get a series equal to $\pi$.  Geometrically, we know that
$$
\pi = 4 \cdot \int_0^1 \sqrt{1 - x^2} \, dx.
$$
We also know that for $|x| < 1$,
$$
\sqrt{1-x} = \sum_{n=0}^\infty \frac{(2n)!}{(1-2n) \, (n!)^2 \, 4^n}x^n.
$$
Combine these facts to find a series converging to $\pi$.

\vfill

\end{description}


\end{document}
