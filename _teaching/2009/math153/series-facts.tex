\documentclass[12pt]{article}
\usepackage{add-copyright}

\usepackage{fullpage}
\usepackage{nopageno}

\title{Facts about series}

\begin{document}

\section*{Facts about series}

We are comfortable with numbers that ``go all the way to the right''
(i.e., non-terminating decimals like $0.3333\cdots$) so why not
numbers that go all the way to the \textbf{left}?

I mean, consider a ``number'' like $\cdots 999999$, meaning
$\sum_{n=0}^\infty 9 \cdot 10^n$.  Of course, this is meaningless, but
if we \textbf{ignore convergence issues} and apply the formula for
geometric series, we might be fooled into thinking $\cdots 999999 =
-1$.  This is possibly less ridiculous than it seems, because
$$
\begin{array}{cr}
  & \cdots 99999999 \\
+ & 1 \\
\hline
 & \cdots  00000000
\end{array}
$$ which we see by adding $9 + 1$, getting $10$, writing down a $0$
and carrying the $1$---and repeating.  A number which vanishes when we
add $1$ to it ought to be called ``-1.''  For similar reasons, we
might believe $\cdots 11111111 = -1/9$, because if we multiply $\cdots
11111111$ by $9$, we get the number for $-1$.

We can show $-1 \times -1 = 1$, because
$$
\begin{array}{cr}
       & \cdots 99999999 \\
\times & \cdots 99999999 \\
\hline
 & \cdots 99999991 \\
 & \cdots 99999910 \\
 & \cdots 99999100 \\
 & \cdots 99991000 \\
 & \vdots \hspace{1em} \\
\hline
 & \cdots 00000001
\end{array}
$$

\subsection*{How about one third?}

There are fancier examples in this crazy world, too.  Because
$$
\begin{array}{cr}
       & \cdots 66666667 \\
\times & 3 \\
\hline
 & \cdots 00000001
\end{array}
$$ so $\cdots 66666667$ deserves to be called $1/3$, since it is a
multiplicative inverse for $3$.  But there is another reason why
$\cdots 66666667 = 1/3$.  After all, if $\cdots 11111111 = -1/9$, then
$\cdots 66666666$ is $-6/9 = -2/3$.  And therefore,
$$
\begin{array}{crl}
       & \cdots 66666666 & \mbox{(``$-2/3$'')}\\
\times & 1 \\
\hline
 & \cdots 66666667 & \mbox{(``$1/3$'')}
\end{array}
$$

\subsection*{How about one seventh?}

I wanted to write down $1/7$, so I started with a $3$ (since $3 \times
7 = 21$, and this will give me the $1$ on the right hand side).  The
next digit should be a $4$, because $4 \times 7 = 28$, and since I had
to carry that $2$, I will get $30$, which means I will write down a
zero.  Now I am carrying a $3$; but if I put a $1$ as the next digit,
then $1 \times 7 + 3 = 10$, so I will write down a zero, and carry a
$1$.  Each time the next digit is designed so that I write down a
zero, and carry something.  I discover:
$$
\begin{array}{cr}
       & \cdots 2857142857142857142857143 \\
\times &  7 \\
\hline
 & \cdots 0000000000000000000000001 \\
\end{array}
$$ This is a repeating decimal: we might write it as
$\overline{285714}3$, though here the digits repeat to the left.
This means we could also write it as a series, formally:
$$
3 + 10 \cdot \left( 285714 \cdot \sum_{n=0}^\infty 1000000^n \right)
$$
If we ignore convergence, and apply the formula for geometric
series here, where it does not apply, we might be fooled into thinking
$$
3 + 10 \cdot \left( 285714 \cdot \frac{1}{1 - 1000000} \right) = 1/7
$$ And even though this series does not converge, it does have the
appearance of being equal to $1/7$.

\subsection*{How about $\sqrt{3}$?}


$$
\begin{array}{cr}
       & \cdots 2857142857142857142857143 \\
\times & \cdots 7 \\
\hline
 & \cdots 0000000000000000000000001 \\
\end{array}
$$

\end{document}