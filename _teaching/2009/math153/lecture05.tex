\documentclass[12pt]{article}
\usepackage{fullpage}
\usepackage{amsthm}
\usepackage{amsmath}

\newtheorem*{example}{Example}
\newtheorem*{thm}{Theorem}

\usepackage{amssymb}

\newcommand{\N}{\mathbb{N}}
\newcommand{\Q}{\mathbb{Q}}
\newcommand{\Z}{\mathbb{Z}}
\newcommand{\R}{\mathbb{R}}

\title{Lecture 5: Popular limits}
\author{Math 153 Section 57}
\date{Wednesday October  8, 2008}

\begin{document}
\maketitle

Following chapter 11.4.

Theme: important limits

Key idea: how mathematics gets done.  the definition.  a few easy cases.  and then theorems that leverage the easy cases into harder cases.

In this section, seven basic limits that we will be able to use over
and over again.

\section{Divgergence}

Example: $a_n = 3\cdot n$ diverges.  Why?

Example: $a_n = (-1)^n$ diverges.  Why?

Two different reasons for divergence.

\subsection{Sometimes we get convergence for free.}

Theorem: Nondecreasing bounded above sequences converge.

Theorem: Nonincreasing bounded below sequences converge.

Example: $a_1 = a_2 = 1$, $a_{n+2} = a_{n+1} + a_n$.  Then $r_n =
a_{n+1} / a_n$ is bounded above by $2$, and increasing.  So the limit
exists, but it is hard to see what it is (turns out to be $(\sqrt{5} +
1)/2$).

\subsection{Squeezing}

State theorem (and sketch proof).

Apply it in some cases.

\subsection{The usefulness of these results}

Moral: we do not have to resort back to the definition of limit: by
using the squeezing theorem and such, we can make things easier.

Maybe laziness can be virtuous (not so moral!), if it inspires us to
find better proofs, more compelling, more memorable arguments.

\subsection{Continuous functions}

If $f : \R \to \R$ is continuous at $L$, and $\lim a_n = L$, then
$\lim f(a_n) = f(L)$.

Proof.

Either take the limit and apply the function, or apply the function
and take the limit.

Not true if $f$ is not continuous.

\subsection{State the important limits}

\end{document}
