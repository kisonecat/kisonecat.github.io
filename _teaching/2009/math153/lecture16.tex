\documentclass[12pt]{article}
\usepackage{fullpage}
\usepackage{amsthm}
\usepackage{amsmath}

\newtheorem*{example}{Example}
\newtheorem*{thm}{Theorem}

\title{Lecture 16: Taylor series}
\author{Math 153 Section 57}
\date{Monday November  3, 2008}

\begin{document}
\maketitle

Following chapter 12.6.

\subsection{Administrative stuff}

Many thanks to those who filled out an online anonymous survey; very
helpful!  I especially noticed a consensus on doing trickier problems
during lecture.

\subsection{Recall}

Series are not piles of numbers that we add up---they are
\textbf{lists} of numbers that we add up, in order.

\subsection{Overview}

We have paid a high price (limits, series) and now we reap the rewards
(Taylor series).

\subsection{Taylor series}

Goal: approximate $e^x$ (really, any function) by a polynomial.

$p_0(x) = 1$ is a good approximation (gets the value correct at 0).

$p_1(x) = 1 + x$ is better approximation (gets the value and first derivative correct at 0).

$p_2(x) = 1 + x + x^2/2$ is yet better approximation (gets the value and first and second derivative correct at 0).

$p_3(x) = 1 + x + x^2/2 + x^3/6$ is even yet a better approximation (gets the value and first and second and third derivative correct at 0).

General pattern: $p_k(x) = \sum_{n=0}^k \frac{x^n}{n!}$.

Very generally:
$$
p_k(x) = \sum_{n=0}^k \frac{f^{(n)}(0) \, x^n}{n!}
$$

\subsection{Is this useful?}

Approximations are useless without error estimates.  People often say
$\pi \approx 3.14$, but without knowing how good or bad an
approximation this is, it is useless.  I could just as well say $\pi
\approx 5$.

\subsection{Theorem}

If $f(x) : (-a,a) \to R$, with $n+1$ cts derivatives, then $f(x) = p_n(x) + R_n(x)$, where
$$
R_n(x) = \frac{1}{n!} \, \int_0^x f^{(n+1)}(t) \, (x-t)^n \, dt
$$

This is the exact error term---and if we know the error exactly, it
isn't much of an error---if we could calculate $R_n(x)$ we could
calculate the function.

\subsection{Bounding the remainder}

Lagrange's theorem:
$$
R_n(x) = \frac{f^{(n+1)}(c)}{(n+1)!} x^{n+1}
$$
where $c$ is some number between $0$ and $x$.

\subsection{Remainder vanishes for large $n$}

For instance, for $f(x) = e^x$, the remainder looks like $e^c \cdot x^{n+1}/(n+1)!$, which goes to zero.

Consequence: $e^x = \sum_{n=0}^\infty x^n/n!$ for all $x$.

\subsection{Other functions}

We can find series for $\sin x$ and $\cos x$---and these series converge to the corresponding function for all $x$.

\subsection{Horrible truth}

For any smooth function, we can write down a Taylor series---but we
can't be sure that it will converge unless we can show that the
remainder term gets small.

Example: $f(x) = 1/(1-x)$.


\end{document}
