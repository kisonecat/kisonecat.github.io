\documentclass[12pt]{article}
\usepackage{fullpage}
\usepackage{amsthm}
\usepackage{amsmath}

\newtheorem*{example}{Example}
\newtheorem*{thm}{Theorem}

\title{Lecture 8: Improper integrals}
\author{Math 153 Section 57}
\date{Wednesday October 15, 2008}

\begin{document}
\maketitle

Following chapter 11.7.

\section{Reminder about integrals}

The importance of the fundamental theorem.

\section{Logarithms}

Define $\log a = \int_1^a dx/x$.  What is $\lim_{a \to \infty} \log
a$?  This is an improper integral.

Incidentally, why is $\log a + \log b = \log (ab)$?  On the second
integral in
$$
\int_1^a \frac{dx}{x} + \int_1^b \frac{dx}{x}
$$
do a substitution: $u = ax$.  Then $dx = u/a$ and $x = u/a$, so $dx/x
= du/u$.  But
$$
\int_1^a \frac{dx}{x} + \int_{a}^{ab} \frac{du}{u} = \int_1^{ab} \frac{dx}{x}
$$
Hand out slide rules to demonstrate the power of this fact.

\section{Two bad things}

Either unbounded width, or unbounded height.  Both are ``improper.''

\section{Unbounded intervals}

When we write $\int_a^\infty$ we mean $\lim_{b \to \infty} \int_a^b$.

If the limit exists, the improper integral converges.

If not, it ``diverges.''

Examples: $\int_1^\infty dx/x$.  $\int_1^\infty dx/x^2$.  $\int_0^\infty \sin x\, dx$.  Different reasons for why these integrals diverge.

Remark: $\int_1^\infty dx / x^p$ converges if $p > 1$, and diverges if $0 < p \leq 1$.

\section{Unbounded on both sides}

When we write $\int_{-\infty}^\infty$ we mean $\lim_{b \to \infty} \int_0^\infty + \lim_{b \to -\infty} \int_{b}^0$.

This is not the same as $\lim_{b \to \infty} \int_{-b}^b$.  Example: $\sin x$.

\section{Unbounded integrand}

If $f$ is continuous on $[a,b)$ but not defined at $b$, we can still
integrate, by defining
$$
\int_{a}^b f(x) \, dx = \lim_{c \to b^{-}} \int_a^b f(x) \, dx
$$
Because for each number $c < b$, the integral over $[a,c] \subset
[a,b)$ makes sense.

Example: $\int_0^2 dx / x$.


\end{document}
