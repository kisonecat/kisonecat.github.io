\documentclass[12pt,letterpaper]{article}

\title{Facts about Limits}
\author{Jim Fowler}

\usepackage{fullpage}
\usepackage{nopageno}
\usepackage{amsmath}
\usepackage[margin=1in]{geometry}
\usepackage{add-copyright}

\usepackage{amsthm}
\newtheorem{theorem}{Theorem}
\newtheorem*{theorem*}{Theorem}
\newtheorem{corollary}[theorem]{Corollary}
\newtheorem{lemmma}[theorem]{Lemma}
\newtheorem{proposition}[theorem]{Proposition}

\theoremstyle{definition}
\newtheorem{remark}[theorem]{Remark}
\newtheorem{example}[theorem]{Example}
\newtheorem{definition}[theorem]{Definition}
\newtheorem*{definition*}{Definition}

\usepackage{amssymb}
\newcommand{\R}{\mathbb{R}}
\newcommand{\N}{\mathbb{N}}

\newcommand{\limn}{\displaystyle\lim_{n \to \infty}}

\begin{document}

\section*{More organized facts about limits}

\subsection*{Bare hands}

\begin{definition*}
We say $\limn a_n = L$ if
\begin{verse}
  for all $\epsilon > 0$, there exists $K \in \N$, \\
  so that if $n \geq K$, then $|a_n - L| < \epsilon$.
\end{verse}
\end{definition*}
We can give explicit $\epsilon$-K arguments to prove
$$
\lim_{n \to \infty} \frac{1}{n} = 0, \hspace{1em} \mbox{ and }
\lim_{n \to \infty} x^n = 0 \mbox{ if $-1 < x < 1$}.
$$

\subsection*{Algebraic operations}

If $\limn a_n = L$ and $\limn b_n = M$, then 
$$
\limn \left( a_n + b_n \right) = L + M, \hspace{1em}
\limn \left( a_n - b_n \right) = L - M, \hspace{1em}
\limn \left( a_n \cdot b_n \right) = L \cdot M, 
$$
and if $M \neq 0$ and all $b_n \neq 0$, then $\limn \left( a_n / b_n
\right) = L / M$.

\subsection*{Squeezing theorem}

If $a_n$, $b_n$, and $c_n$ are sequences of real numbers, and for all
$n \in \N$, we have $a_n \leq b_n \leq c_n$, and $\limn a_n = L$ and
$\limn c_n = L$, then $\limn b_n = L$.

\subsection*{Continuous functions}

If a function $f : \R \to \R$ is continuous at $L$, and $\limn a_n =
L$, then $\limn f(a_n) = f(L)$.

\subsection*{Subsequences}

If $a_n$ is a sequence of real numbers, and $b_n$ is a subsequence
produced from $a_n$ by throwing away finitely many terms, then $a_n$
converges if and only if $b_n$ converges.

\vspace{1ex}
\noindent
If $a_n$ is a convergent sequence of real numbers, and $b_n$ is a
subsequence, then $b_n$ converges and $\limn a_n = \limn b_n$.

\subsection*{Existence without construction}

\begin{theorem*}
If $a_n$ is a nondecreasing bounded above sequence, then $a_n$ converges.
\end{theorem*}

\begin{theorem*}
If $a_n$ is a nonincreasing bounded below sequence, then $a_n$ converges.
\end{theorem*}

\end{document}
