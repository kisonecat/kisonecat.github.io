\documentclass[12pt]{article}
\usepackage{fullpage}
\usepackage{amsthm}
\usepackage{amsmath}

\newtheorem*{example}{Example}
\newtheorem*{thm}{Theorem}

\title{Lecture 25: Partial fractions}
\author{Math 153 Section 57}
\date{Monday November 24, 2008}

\begin{document}
\maketitle

\subsection*{Overview}

Recall that the limit comparison test allows us to determine whether
$$
\sum \frac{\mbox{polynomial}}{\mbox{polynomial}}
$$
converges.  These quotients of polynomials are called rational
functions.

Rational functions are maybe the ``next simplest'' class o f functions
after polynomials (which we understand very well).

Derivative of a rational function is a rational function, so we can
take all the derivatives we would like---no problem!

But to antidifferentiate a rational function?  For this, we use
\textbf{partial fractions}.  This is, in fact, how the computers do
these integrals.

\subsection*{Theory}

We can integrate polynomials.

What are some rational functions we can integrate?
$$
\int \frac{1}{ax + b} \, dx = \frac{\log (a x + b)}{a} + C
$$
or somewhat more generally, for $n > 1$,
$$
\int \frac{1}{(ax + b)^n} \, dx = \frac{1}{a} \cdot \frac{(a x + b)^{-n+1}}{-n+1} + C
$$
For quadratics:
$$
\int \frac{x}{x^2 + b^2} \, dx = \frac{1}{2} \log \left(x^2 + b^2\right) + C
$$
or even
$$
\int \frac{x - a}{(x-a)^2 + b^2} \, dx = \frac{1}{2} \log \left((x-a)^2 + b^2\right) + C
$$
We can also do
$$
\int \frac{1}{(x-a)^2 + b^2} \, dx = \frac{1}{b} \arctan \left(\frac{x-a}{b}\right) + C
$$
We can also do these if the denominators are raised to a power (with
more difficulty).

\textbf{These functions suffice to do all rational functions}.  But we
need a machine for transforming the general rational function into
something of this form.  What would that be?

\subsection*{Enter partial fractions}

[ ``I would like to see a proof of such and such.''  ``How many
alphabets are you familiar with?'' ]

The theorem.  Let $f(x) = p(x)/q(x)$, a ratio of polynomials.  We can
assume that $q$'s leading term has a $1$ coefficient.

The fundamental theorem of algebra: factor denominator as
$$
q(x) =  \cdot \prod_{i=1}^m (x-a_i)^{j_i} \cdot \prod_{i=1}^n (x^2 + b_i x + c_i)^{k_i}
$$
that is, into linear and quadratic terms.

Then we can write $f(x)$ as
$$
f(x) = \frac{p(x)}{q(x)} = P(x) + \sum_{i=1}^m\sum_{r=1}^{j_i} \frac{A_{ir}}{(x-a_i)^r} + \sum_{i=1}^n\sum_{r=1}^{k_i} \frac{B_{ir}x+C_{ir}}{(x^2+b_ix+c_i)^r}
$$

This seems pretty awful.  How can we ever hope to find all these constants?

\subsection*{An example}

Suppose
$$
\frac{1}{(x-3)^2 (x-2)} = \frac{A}{x-3} + \frac{B}{(x-3)^2} + \frac{C}{x-2}
$$
The least clever (and most involved) way is to multiply through by the denominator:
$$
1 = A(x-2)(x-3) + B(x-2) + C(x-3)^2
$$
This gives
$$
1 = (A+C) x^2 + (-5 A + B - 6 C)x+(6 A-2 B+9 C)
$$
So $A + C = 0$, and $-5A + B - 6C = 0$, and $6A - 2B + 9C = 1$.
Solve the three equations in three unknowns (the resulting equations
will always have a unique solution).
Get $A = -C$.  So $5C + B - 6C = 0$, so $B - C = 0$, so $B = C$.
Then, $-6C - 2C + 9C = 1$, so $C = 1$, and therefore $A = -1$ and $B = 1$.

Check!
$$
\frac{1}{(x-3)^2 (x-2)} = \frac{-1}{x-3} + \frac{1}{(x-3)^2} + \frac{1}{x-2}
$$
This is true.

This lets us do the integral
$$
\int \frac{1}{(x-3)^2 (x-2)} \, dx
$$
because you can do
$$
\int \left( \frac{-1}{x-3} + \frac{1}{(x-3)^2} + \frac{1}{x-2} \right) \, dx
$$

\subsection*{So many problems}

The problems involved here are many.

You are not likely to see $(x-3)^2 (x-2)$.  You are more likely to see
$$
x^3-8 x^2+21 x-18
$$
How are you supposed to \textbf{solve the factorization problem}?

\subsection*{A tool in our toolbox, not a rule for solving all problems!}

It is silly for me to give you lists of rules.

\subsection*{A second example}

$$
f(x)=\frac{x^3+16}{x^3-4x^2+8x}
$$
Do long division
$$
f(x)=1+\frac{4x^2-8x+16}{x^3-4x^2+8x}=1+\frac{4x^2-8x+16}{x(x^2-4x+8)}
$$
$$
\frac{4x^2-8x+16}{x(x^2-4x+8)}=\frac{A}{x}+\frac{Bx+C}{x^2-4x+8}
$$
Multiply through by $x(x^2-4x+8)$ to get
$$
4x^2 - 8x + 16 = A (x^2 - 4x + 8) + (Bx + C)x
$$
So $A = 2$.  So $B = 2$.  And $C = 0$.  Therefore,
$$
f(x)=1+2\left(\frac{1}{x}+\frac{x}{x^2-4x+8}\right)
$$

\subsection*{Some tricks}

Multiply the partial fraction decomposition by $(x - a_i)$ and take the limit $x \to a_i$.  Then only the term $A_{i,1}$ survives, so
$$
A_{i,1} = \lim_{x \to a_i} (x - a_i) f(x)
$$
In the case of
$$
f(x) = \frac{1}{(x-3)^2 (x-2)}
$$
this tells us that 
$$
\lim_{x \to 2} (x-2) \cdot \frac{1}{(x-3)^2 (x-2)} = 1
$$
but of course is useless for to find the coefficients on $1/(x-3)$ and $1/(x-3)^2$.

We can find
$$
\lim_{x \to 3} (x-3)^2 \cdot \frac{1}{(x-3)^2 (x-2)} = 1
$$
to discover that the coefficient on $1/(x-3)^2$ is $1$.

Then we know that
$$
\frac{1}{(x-3)^2 (x-2)} = \frac{1}{x-2} + \frac{1}{(x-3)^2} + \frac{A}{x-3}
$$
We could plug in a value like $x = 4$ to find:
$$
\frac{1}{(4-3)^2 (4-2)} = \frac{1}{4-2} + \frac{1}{(4-3)^2} + \frac{A}{4-3}
$$
which becomes
$$
\frac{1}{2} = - \frac{1}{2} + \frac{1}{1^2} + \frac{A}{1}
$$
so $A = -1$.

\subsection*{Underview}

Some some tricks include: taking limits, and plugging in values.

\end{document}
