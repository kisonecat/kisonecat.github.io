\documentclass[12pt]{article}
\usepackage{fullpage}


\usepackage{amsmath}
\DeclareMathOperator{\Alt}{Alt}
\DeclareMathOperator{\id}{id}

\usepackage{amsthm}
\newtheorem*{theorem}{Theorem}

\theoremstyle{definition}
\newtheorem*{quotedefinition}{``Definition''}
\newtheorem*{definition}{Definition}

\newtheorem*{exercise}{Exercise}

\newcommand{\Tensor}[2]{\mathcal{T}^{#2} #1}

\usepackage{amssymb}
\newcommand{\R}{\mathbb{R}}

\long\def\symbolfootnote[#1]#2{\begingroup%
\def\thefootnote{\fnsymbol{footnote}}\footnote[#1]{#2}\endgroup}


\begin{document}

Define an operator $\star$ (called the ``Hodge star''),
$$
\star : \bigwedge^k \R^n \to \bigwedge^{n-k} \R^n.
$$
defined by
$$\star \left( e_1 \wedge \cdots \wedge e_k \right) = e_{k+1} \wedge \cdots \wedge
e_n.$$
Our definition depends on choosing a basis.

\begin{exercise}
Compute $\dim \bigwedge^k \R^n$ and $\dim \bigwedge^{n-k} \R^n$.
\end{exercise}

\begin{exercise}
Is $\star : \bigwedge^k \R^n \to \bigwedge^{n-k} \R^n$ an isomorphism?
\end{exercise}

\begin{exercise}
Can you find a basis-free definition of $\star$?
\end{exercise}

\begin{exercise}
Compute $\star \star \omega$ in terms of $\omega$.
\end{exercise}

explain more here

For $n = 3$, the cross-product of vectors $v$ and $w$ is
$$
v \times w := \star \left( v \wedge w \right).
$$
The inner product can also be interpreted in this language, as
$$
\star \langle v, w \rangle = v \wedge (\star w)
$$

\subsection*{A certain three tensor.}

Here we define a map that takes three vectors in $\R^3$ and produces a number:
$$
\left( a, b, c \right) \mapsto \langle a, b \times c \rangle.
$$
\begin{exercise}
Check that this is trilinear (so it is, in fact, a $3$-tensor).
\end{exercise}
We can deduce the true nature of this 3-tensor by using the Hodge
star.  Note that
\begin{align*}
\langle a, b \times c \rangle
&= \star \left( a \wedge \star \left( b \times c \right) \right) \\
&= \star \left( a \wedge \star \star \left (b \wedge c \right ) \right) \\
&= \star \left( a \wedge b \wedge c \right),
\end{align*}
and therefore $\langle a, b \times c \rangle$ is antisymmetric in the
inputs $a,b,c$.  In fact, it is
$$
\det \begin{pmatrix}
a_1 & b_1 & c_1 \\
a_2 & b_2 & c_2 \\
a_3 & b_3 & c_3 \\
\end{pmatrix},
$$
which explains a popular method for computing the cross product (by
taking the determinant of a particular matrix).

\subsection*{A certain four tensor.}

An interesing four tensor on $\R^3$ is 
$$
\left( a, b, c, d \right) \mapsto \langle a \times (b \times c), d \rangle.
$$
\begin{exercise}
Check that this is a four tensor.
\end{exercise}
In fact, this four tensor can be written in a different way.
\begin{align*}
\star \langle a \times (b \times c), d \rangle
&= \left( \star \left( a \wedge \star \left( b \wedge c \right) \right) \right) \wedge \left( \star d \right) \\
&= \left( a \wedge \star \left( b \wedge c \right) \right)\wedge d \\
&= a \wedge \left( \star \left( b \wedge c \right) \right) \wedge d \\
&= - a \wedge d \wedge \star \left( b \wedge c \right) \\
&= - \left( \star \star \left ( a \wedge d \right) \right) \wedge \star \left( b \wedge c \right) \\
&= - \langle a \times d, b \times c \rangle
\end{align*}

$$
(a \wedge \star c) \wedge (\star (\star b \wedge d)) 
+
(a \wedge \star b) \wedge (\star (\star c \wedge d)) 
$$


$$
\star a \wedge \left( c \cdot \star (b \wedge \star d) - b \cdot \star (c \wedge \star d) \right)
$$

$$
\star a \wedge \left( c \cdot \star (\star b \wedge d) - b \cdot \star (\star c \wedge d) \right)
$$


$$
a \wedge \left( \star c \cdot \star (\star b \wedge d) - \star b \cdot \star (\star c \wedge d) \right)
$$

$$
(a \wedge \star c) \cdot \star (\star b \wedge d) - (a \wedge \star b) \cdot \star (\star c \wedge d)
$$

$$
(a \wedge \star c) \cdot \star (\star b \wedge d) - (a \wedge \star b) \cdot \star (\star c \wedge d)
$$

$$
(a \wedge \star c) (b \wedge \star d) - (a \wedge \star b) (c \wedge \star d)
$$

\subsection*{reduction}

\begin{align*}
(\star ((\star b) \wedge (\star c))) \wedge d
&= 
c \wedge \star (b \wedge d) +
b \wedge \star (c \wedge d)
\end{align*}

\begin{align*}
(\star (b \wedge c)) \wedge d
&= 
(\star c) \wedge \star ((\star b) \wedge d) +
(\star b) \wedge \star ((\star c) \wedge d)
\end{align*}

\section*{Birdtracks}

Levi-Civita relation says

$$
(a,b,c,d) \mapsto \left( \tr (v \mapsto 
$$




\end{document}
