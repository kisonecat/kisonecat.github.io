\documentclass[12pt]{article}

\usepackage{amsmath}
\usepackage{amsthm}
\usepackage{amssymb}

\newcommand{\R}{\mathbb{R}}

\usepackage{nopageno}


\begin{document}

\section*{Higher derivatives}

Suppose $\gamma : \R \to \R^n$ linear, $\gamma(t) = tv$ and $f : \R^n
\to \R$ is smooth.  We want to show
$$
D^k f (0)(v,\ldots,v) = D^k (f \circ \gamma)(0)(1,\ldots,1)
$$

In fact, we will show the stronger result
$$
D^k f (\gamma(t))(v,\ldots,v) = D^k (f \circ \gamma)(t)(1,\ldots,1)
$$

This is true for $k = 0$.  Suppose it holds for $k$; we will show it
holds for $k+1$.

Note that
$$
D^k f (\gamma(t))(v,\ldots,v) = D^k (f \circ \gamma)(t)(1,\ldots,1)
$$
is the same as
$$
\mbox{ev}_v \circ D^k f \circ \gamma = \mbox{ev}_1 \circ D^k (f \circ \gamma)
$$
Apply $D$ to both sides
$$
D(\mbox{ev}_v \circ D^k f \circ \gamma)(t) = D(\mbox{ev}_1 \circ D^k (f \circ \gamma))(t)
$$
By the chain rule
$$
\left( \mbox{ev}_v \circ D(D^k f \circ \gamma) \right)(t) = \mbox{ev}_1 \circ D(D^k (f \circ \gamma))(t)
$$
And applying the chain rule to $D(D^k f \circ \gamma$ gives
$$
D^{k+1} f(\gamma(t))(v,\ldots v) \circ \gamma = D^{k+1} (f \circ \gamma)(t)(1,\ldots,1)
$$
Both sides are linaer maps; evaluate both sides at $1$ to get
$$
D^{k+1} f(\gamma(t))(v,\ldots v)(\gamma(1)) = D^{k+1} (f \circ \gamma)(t)(1,\ldots,1)(1),
$$
but this is just
$$
D^{k+1} f(\gamma(t))(v,\ldots v)(v) = D^{k+1} (f \circ \gamma)(t)(1,\ldots,1)(1)
$$
which is what we wanted.


\end{document}