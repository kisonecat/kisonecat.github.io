\documentclass[12pt]{article}
\usepackage{nopageno}
\usepackage{fullpage}
\usepackage{add-copyright}

\title{Quarter Overview}
\author{Undergraduate Lie Theory Seminar}
%\date{October 3, 2006}
\date{}

\begin{document}

\maketitle

A \textbf{Lie group} is a manifold which is also a group, and
consequently Lie groups combine algebra, topology, geometry, and
analysis.  They appear everywhere, and our goal this quarter (or
longer) is to understand these ubiquitous and beautiful objects
more deeply.

Our goal will be to classify the simple compact Lie groups, and to do
so in as geometric (as opposed to algebraic) a way as possible.

\section*{Plan of Attack}

Right now, our plan might go as follows.

\subsection*{Combinatorial background.}

We will work through the definitions and basic properties of Coxeter
groups and root systems.  A good reference is
\begin{quote}
Grove and Benson, \textit{Finite Reflection Groups.}
\end{quote}
This will introduce \textbf{Dynkin diagrams} and the \textbf{ADE
  classification}---a classification that appears throughout
mathematics, from regular polytopes, to Lie algebras, Lie groups,
singularities, quivers\ldots

\subsection*{Lie groups.}

A nice references for Lie groups, done very geometrically, is
\begin{quote}
Frank Adams, \textit{Lectures on Lie Groups.}
\end{quote}
This will introduce us to \textbf{Lie groups} and we will see the
importance of \textbf{maximal tori}.

\subsection*{Representations.}

A nice reference here is
\begin{quote}
Fulton and Harris, \textit{Representation Theory, A First Course.}
\end{quote}
This book is problem-based, and works through examples in fantastic
detail, with lovely pictures of the root systems.

\end{document}
