\documentclass[12pt]{handout}
\usepackage{nopageno}
\course{Math 758}
\title{Intersections, Linking}
\author{Jim Fowler}

\usepackage{amsmath}
\newtheorem*{theorem*}{Theorem}
\newtheorem*{corollary*}{Corollary}
\theoremstyle{definition}
\newtheorem{question}[theorem]{Question}
\newtheorem*{remark*}{Remark}
\newtheorem*{definition*}{Definition}
\newtheorem*{example*}{Example}
\newtheorem{problem}[theorem]{Problem}
\newtheorem{subproblem}[theorem]{Subproblem}
\newtheorem{requiredproblem}[theorem]{\makebox[0in][r]{$\bullet$\hspace{0.25em}}Problem}

\DeclareMathOperator{\projection}{proj}
\DeclareMathOperator{\link}{link}

\begin{document}

\maketitle

First, something that came up during office hours.

\begin{problem}
  Let $M^n$ be a closed manifold with a fixed triangulation; let $M'$
  be its barycentric subdivision, and let $w_1$ be the $\mathbb{Z}/2$-chain
  given by adding up all the edges in $M'$.  Is $w_1$ a cycle?  When
  is $w_1$ a boundary?
\end{problem}

Next, we'll talk about submanifolds, intersections, and linking.  At
this point, the ``special topics'' start, so if there are particular
things you're eager to learn about, we should talk about those things.

\begin{remark}
For an oriented manifold $M^m$ with submanifolds ${A}^a$ and ${B}^{b}$ intersecting transversely, then
$$
[A \cap B] = \left( [A]^\star \smallsmile [B]^\star \right)^\star
$$
where $\star$ denotes Poincar\'e duality, and the inclusion maps are
not mentioned.
\end{remark}
\noindent
One proof of this goes via the \textbf{Thom isomorphism theorem} which
is the purview of a future course---we haven't defined what
``transversely'' means, even.

\begin{problem}
  When $a + b = m$, we can regard $[A \cap B]$ as a number.  Is this
  number the same as the cardinality of the set $A \cap B$?  If we
  count with orientation?  Can we isotope $A$ and $B$ to make $|A \cap
  B| = [A \cap B]$?
\end{problem}

\begin{problem}
  Compute $\pi_1$ of the complement of the ``unlink'' and the ``Hopf
  link.''  Why might this fact put a mountain climber in danger?
\end{problem}

\begin{problem}
  Use Alexander duality to compute $H_1(S^3 - A)$ for a curve $A$.
  Can you describe a generator?
\end{problem}

\begin{definition}
  For two curves $A$ and $B$ in $S^3$, their \textbf{linking number} $\link(A,B)$ is defined to be
  $$
  [A] = \link(A,B) [m] \in H_1(S^3 - B)
  $$
  where $m$ is the \textbf{meridian} of the curve $B$.
\end{definition}

\begin{problem}
  Let $N(B)$ be a thickened, closed neighborhood of the curve $B$, and
  let $\Sigma$ be an oriented connected surface with $\partial \Sigma
  = B$.  Show that $\link(A,B) = [A \cap \Sigma]$ by using a certain
  geometric fact about the meridian.
\end{problem}

\begin{problem}
  How does $\link(A,B)$ relate to $\link(B,A)$?
\end{problem}

\begin{problem}
  Compute the linking number of the Hopf link.
\end{problem}

\begin{problem}
  Compute the linking number of the Whitehead link.
\end{problem}

\begin{problem}
  Suppose $B$ is a curve, and that there is a surface $\Sigma$ in $S^3
  - B$ with boundary $A_1 \sqcup A_2$ with $A_1$ and $A_2$ connected
  curves; how does $\link(A_1,B)$ relate to $\link(A_2,B)$?
\end{problem}

\pagebreak
\null

\end{document}

%%% Local Variables: 
%%% mode: latex
%%% TeX-master: t
%%% End: 
