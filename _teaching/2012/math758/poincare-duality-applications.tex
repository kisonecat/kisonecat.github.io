\documentclass[12pt]{handout}
\usepackage{nopageno}
\course{Math 758}
\title{Poincar\'e Duality}
\author{Jim Fowler}

\usepackage{amsmath}
\newtheorem*{theorem*}{Theorem}
\newtheorem*{corollary*}{Corollary}
\theoremstyle{definition}
\newtheorem{question}[theorem]{Question}
\newtheorem*{remark*}{Remark}
\newtheorem*{definition*}{Definition}
\newtheorem*{example*}{Example}
\newtheorem{problem}[theorem]{Problem}
\newtheorem{subproblem}[theorem]{Subproblem}
\newtheorem{requiredproblem}[theorem]{\makebox[0in][r]{$\bullet$\hspace{0.25em}}Problem}

\DeclareMathOperator{\projection}{proj}
\DeclareMathOperator{\Hom}{Hom}
\newcommand{\Q}{\mathbb{Q}}
\newcommand{\Z}{\mathbb{Z}}
\newcommand{\RP}{\mathbb{R}P}

\begin{document}

\maketitle

\begin{problem}
  The \textbf{Euler characteristic} of a space $X$ with finite dimensional homology is
  $$
  \chi(X) = \sum_{n} (-1)^n \, \dim H_n(X;\Q).
  $$
  If $X$ is a finite simplicial complex, show that this is the same as
  $$
  \chi(X) = \sum_{n} (-1)^n \, \dim C_n(X;\Q),
  $$
  where $C_n(M;\Q)$ is the $n$-dimensional simplicial chain groups with rational coefficients.
\end{problem}

\begin{problem}
  Let $M^3$ be a closed oriented 3-manifold with a given PL
  triangulation (so the \textbf{link} of each vertex is a sphere);
  show, by using the formula for $\chi$ in terms of chains, that
  $\chi(M) = 0$.
\end{problem}

\begin{problem}
  If $M$ is an odd-dimensional closed oriented manifold, show that
  $\chi(M) = 0$.
\end{problem}


\begin{problem}
  Let $M^n$ be an oriented closed $n$-manifold.  Show that the pairing
  $$
  \smallsmile : H^k(M;\Q) \times H^{n-k}(M;\Q) \to H^n(M;\Q) = \Q
  $$
  is nonsingular form by considering the adjoint map
  $$
  H^k(M;\Q) \to \Hom(H^{n-k}(M;\Q), H^n(M;\Q)).
  $$
\end{problem}


\begin{problem}
  Suppose $M^{6}$ is an oriented closed $6$-manifold.  Show that the pairing
$$
\smallsmile : H^{3}(M;\Q) \to H^{3}(M;\Q) \to H^{6}(M;\Q)
$$
is a nonsingular skew-symmetric bilinear form over $\Q$.  What does this imply about $\dim H^3(M;\Q)$?
\end{problem}


\begin{problem}
  Use Poincar\'e duality to compute the cup product structure on
$$
H^\star(\RP^m;\Z/2).
$$
\end{problem}


\begin{problem}
  Let $X$ and $Y$ be simplicial complexes with disjoint sets of
  vertices; in what follows, we regard a simplicial complex as simply
  a set of subsets of a vertex set, closed under taking of subsets.

  The \textbf{join} of $X$ and $Y$ is denoted by $X \star Y$, and is
  defined as follows
  $$
  X \star Y = \{ \sigma \cup \tau : \sigma \in X \mbox{ and } \tau \in Y
  \},
  $$
  If $X$ is homeomorphic to the $n$-sphere $S^n$ and $Y$ is homeomorphic
  to the $m$-sphere $S^m$, describe the homeomorphism type of $X \star
  Y$.
\end{problem}


\begin{problem}
  Consider a $\Z/5$ action on $S^1$; by taking the join of $S^1$ with
  itself, produce a $\Z/5$ action on $S^{2n+1}$.  The quotient of this
  sphere by the $\Z/5$ action is a \textbf{lens space} $L^{2n+1}$.

  Pick a generator $\alpha \in H^1(L^{2n+1};\Z/5)$ and a generator $\beta \in
  H^2(L^{2n+1};\Z/5)$.  Show that $\alpha$ and $\beta$ generate $H^\star(L^{2n+1};\Z/5)$ as a ring.
\end{problem}

\begin{problem}
  Exhibit two noncompact surfaces which are not homeomorphic.
\end{problem}

\pagebreak
\null

\end{document}

%%% Local Variables: 
%%% mode: latex
%%% TeX-master: t
%%% End: 
