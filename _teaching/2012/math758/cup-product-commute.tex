\documentclass[12pt]{handout}
\usepackage{nopageno}
\course{Math 758}
\title{Cup Products Commute}
\author{Jim Fowler}

\usepackage{amsmath}
\newtheorem*{theorem*}{Theorem}
\newtheorem*{corollary*}{Corollary}
\theoremstyle{definition}
\newtheorem{question}[theorem]{Question}
\newtheorem*{remark*}{Remark}
\newtheorem*{definition*}{Definition}
\newtheorem*{example*}{Example}
\newtheorem{problem}[theorem]{Problem}
\newtheorem{subproblem}[theorem]{Subproblem}
\newtheorem{requiredproblem}[theorem]{\makebox[0in][r]{$\bullet$\hspace{0.25em}}Problem}

\DeclareMathOperator{\projection}{proj}

\begin{document}

\maketitle

I'll update the calendar soon to account for the changes in the schedule.

\begin{remark}
  Our goal is to show that for $\alpha \in H^i(X)$ and $\beta \in H^j(X)$, show that the cup product satisfies
  \[
  \alpha \smallsmile \beta = (-1)^{ij} \beta \smallsmile \alpha.
  \]
\end{remark}

\begin{definition}
Let $\epsilon_n = (-1)^{n \cdot (n+1)/2}$.
\end{definition}

\begin{subproblem}
Verify $\epsilon_{i+j} = (-1)^{ij} \epsilon_i \epsilon_j$.
\end{subproblem}

\begin{definition}
Let $\tau : C_n(X) \to C_n(X)$ be the chain map which, on a simplex $\sigma$, is defined as
\[
\tau(\sigma) = \epsilon_n (\sigma \circ \mathrm{reverse}),
\]
where $\mathrm{reverse}$ is the affine map $[v_0,\ldots,v_n] \to
[v_n,\ldots,v_0]$.
\end{definition}

\begin{subproblem}
Verify that $\tau$ is a chain map.
\end{subproblem}

\begin{subproblem}
  Verify $\tau^\star \alpha \smallsmile \tau^\star \beta = \pm
  \,\tau^\star(\beta \smallsmile \alpha)$ and determine the sign.
\end{subproblem}

\begin{subproblem}
  Consider $X \times I$ as a simplicial complex; in particular, for
  $\Delta^n \times I$,
  \begin{align*}
    [v_0,\ldots,v_n] &= \Delta^n \times \{0\} \\
    [w_0,\ldots,w_n] &= \Delta^n \times \{1\} \\
    \Delta^n \times I &= \bigcup [v_0,\ldots,v_i,w_n,\ldots,w_i].
    \end{align*}
    Draw a picture to illustrate the resulting simplicial structure on the prism $\Delta^2 \times I$.
\end{subproblem}

\begin{subproblem}
 Recall, from Math~757, the \textbf{prism
    operator} $P : C_n(X) \to C_{n+1}(X)$ defined via
  \[
  P(\sigma) = \sum_i (-1)^i \epsilon_{n-i} (\sigma \circ \projection)
  |_{[v_0,\ldots,v_i,w_n,\ldots, w_i]},
  \]
  where $\mathrm{proj}$ is the projection $\Delta^n \times I \to \Delta^n$.
  Show that $\partial \circ P + P \circ \partial = \tau - \mathrm{Id}$.
\end{subproblem}

\begin{subproblem}
  Conclude that $\smallsmile$ is (graded) commutative.
\end{subproblem}

\begin{problem}
  Let $\Sigma_g$ be the oriented surface of genus $g$; describe a
  four-fold covering map $f : \Sigma_5 \to \Sigma_2$, and explain how
  the map $f_\star$ maps the ring $H^\star(\Sigma_2)$ to the ring
  $H^\star(\Sigma_5)$.
\end{problem}

\begin{problem}
  For which $i$ and $j$ are there maps $f : \Sigma_i \to \Sigma_j$ and
  $g : \Sigma_j \to \Sigma_i$ so that $f \circ g$ is the identity?
\end{problem}

\pagebreak
\null

\end{document}

%%% Local Variables: 
%%% mode: latex
%%% TeX-master: t
%%% End: 
