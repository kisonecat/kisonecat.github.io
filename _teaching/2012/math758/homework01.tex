\documentclass[12pt]{pset}

\course{Math 758}
\author{Jim Fowler}
\date{Spring 2012}

\usepackage{tikz}
\usetikzlibrary{arrows,chains,matrix,positioning,scopes}

\makeatletter
\tikzset{join/.code=\tikzset{after node path={%
\ifx\tikzchainprevious\pgfutil@empty\else(\tikzchainprevious)%
edge[every join]#1(\tikzchaincurrent)\fi}}}
\makeatother

\tikzset{>=stealth',every on chain/.append style={join},
         every join/.style={->}}

\newcommand{\Z}{\mathbb{Z}}
\newcommand{\Q}{\mathbb{Q}}
\newcommand{\CC}{\mathbb{C}}
\newcommand{\CP}{\mathbb{C}P}
\newcommand{\RP}{\mathbb{R}P}

\usepackage{enumerate}
\usepackage{nopageno}
\usepackage{hyperref}

\DeclareMathOperator{\Ext}{Ext}
\DeclareMathOperator{\Hom}{Hom}

\begin{document}
\maketitle

\noindent\textbf{\href{http://en.wikipedia.org/wiki/List_of_games_with_concealed_rules}{The only way to learn the game is to play the game.}}
The following represents a \textit{lower bound} on the number of
exercises you should be doing; the textbook is full of great
exercises, so I encourage you to do as many as possible.

%\textit{The exercises below should be handed in on Monday.}

%%%%%%%%%%%%%%%%%%%%%%%%%%%%%%%%%%%%%%%%%%%%%%%%%%%%%%%%%%%%%%%%
\begin{problem}[Hom is sometimes exact]
  For which abelian groups $G$ is $\Hom(G,-)$ an exact functor?
\end{problem}

\begin{problem}[Ext is functorial]
  Let $G$ be an abelian group; show that $\Ext(-,G)$ and $\Ext(G,-)$ are functors.
\end{problem}

\begin{problem}[Ext for cyclic groups]
  Compute $\Ext(\Z/m,\Z/n)$.
\end{problem}

\begin{problem}[Ext is extensions]
  An extension of $A$ by $B$ is a short exact sequence
  $$
  0 \to A \to E \to B \to 0.
  $$
  Two extensions $0 \to A \to E \to B \to 0$ and $0 \to A \to E' \to B
  \to 0$ are said to be equivalent if there is a map $f : E \to E'$ so
  that the following diagram
  \begin{center}
  \begin{tikzpicture}
  \matrix (m) [matrix of math nodes, row sep=3em,
    column sep=3em]
    { 0 & A  & E  & B  & 0 \\
      0 & A & E' & B & 0 \\ };
  { [start chain] \chainin (m-1-1);
    \chainin (m-1-2);
    { [start branch=A] \chainin (m-2-2)
        [join={node[right] {$\mathrm{id}$}}];}
%    \chainin (m-1-3) %[join={node[above]
                     % {$\scriptstyle\varphi$}}];
    \chainin (m-1-3);
    { [start branch=B] \chainin (m-2-3)
        [join={node[right] {$f$}}];}
%    \chainin (m-1-4)% [join={node[above]
                    %  {$\scriptstyle\psi$}}];
    \chainin (m-1-4);
    { [start branch=C] \chainin (m-2-4)
        [join={node[right] {$\mathrm{id}$}}];}
    \chainin (m-1-5); }
  { [start chain] \chainin (m-2-1);
    \chainin (m-2-2);
    \chainin (m-2-3);% [join={node[above]
%                      {$\scriptstyle\varphi'$}}];
    \chainin (m-2-4);% [join={node[above]
%                      {$\scriptstyle\psi'$}}];
    \chainin (m-2-5); 
}
\end{tikzpicture}
\end{center}
commutes.  Show that the set of equivalence classes of extensions of
$A$ by $B$ is naturally isomorphic to $\Ext(B,A)$.
\end{problem}

\begin{problem}[Real projective plane]
  Let $X$ be $\RP^2$.  Compute $H_\star(X;\Z)$ and $H^\star(X;\Z)$ via
  (simplicial or) cellular (co)homology.
\end{problem}

\begin{problem}[Jacob's ladder]
  Consider the simplicial graph $X$ with a vertex $a_i$ and $b_i$ for each $i \in \Z$, and edges
  \begin{itemize}
  \item between $a_k$ and $a_{k+1}$,
  \item between $b_k$ and $b_{k+1}$,
  \item between $a_k$ and $b_k$, for each $k \in K$.
  \end{itemize}
  Compute $H^\star(X;\Z)$.
\end{problem}

\begin{problem}[Isomorphic homology, isomorphic comohology?]
  Let $f : X \to Y$ be a map of CW complexes and let $G$ be an abelian
  group.  Is it the case that if $f_\star : H_\star(X;G) \to
  H_\star(Y;G)$ is an isomorphism, then $f_\star : H^\star(Y;G) \to
  H^\star(X;G)$?  If yes, prove it.  If not, salvage the statement to
  make it true.
\end{problem}

\begin{problem}[Cohomology with coefficients and tensor product]
  Suppose $H^\star(X;\Z)$ is torsion-free and $G$ is an abelian group.
  Is it then the case that $H^n(X;G) = H^n(X;\Z) \otimes G$?  If yes,
  prove it; if not, salvage the statement to make it true.
\end{problem}




\end{document}
