\documentclass[12pt]{pset}

\course{Math 758}
\author{Jim Fowler}
\date{Spring 2012}

\newcommand{\CC}{\mathbb{C}}
\newcommand{\CP}{\mathbb{C}P}
\newcommand{\RP}{\mathbb{R}P}

\usepackage{enumerate}
\usepackage{nopageno}
\usepackage{hyperref}

\begin{document}
\maketitle

\noindent\textit{The exercises below should be handed in on Monday.}

%%%%%%%%%%%%%%%%%%%%%%%%%%%%%%%%%%%%%%%%%%%%%%%%%%%%%%%%%%%%%%%%
\begin{problem}[Surfaces]
Let $\Sigma_g$ be the closed surface of genus~$g$; compute the ring structure on $H^\star(\Sigma_g)$.
\end{problem}

\begin{problem}[Surface automorphisms]
  A homeomorphism $f : \Sigma_g \to \Sigma_g$ induces a map $f^\star :
  H^1(\Sigma_g) \to H^1(\Sigma_g)$; can every automorphism of the
  abelian groups $H^1(\Sigma_g)$ be realized as $f^\star$ for some
  homeomorphism $f$?
\end{problem}

\begin{problem}[Not a wedge]
  Show that $\CP^2 \not\simeq S^4 \vee S^2$ by using cup products.
\end{problem}

\begin{problem}[Hatcher page 229, problem 4]
  Use the Lefschetz fixed point theorem (did you do this last
  quarter?) to show that every map $f : \CP^n \to \CP^n$ has a fixed
  point if $n$ is even, using the fact that $f^\star$ is a ring
  isomorphism; when $n$ is odd, show that there is a fixed point
  unless $f^\star(\alpha) = -\alpha$ for $\alpha$ a generator of
  $H^2(\CP^n)$.
\end{problem}

\begin{problem}[Hatcher page 229 problem 6]
  Use cup products to compute the map $f^\star : H^\star(\CP^n) \to
  H^\star(\CP^n)$ induced by the map $f : \CP^n \to \CP^n$ which is
  the quotient of the map $\CC^{n+1} \to \CC^{n+1}$ given by raising
  each coordinate to the $d$th power.
\end{problem}

\begin{problem}[Suspension kills cup products]
  Let $\Sigma X$ be the suspension of $X$ (recall this is the union of
  two cones on $X$ glued together along $X$); for $\alpha \in
  H^a(\Sigma X)$ and $\beta \in H^b(\Sigma X)$, show that $a \smile b$
  vanishes.
\end{problem}

\end{document}
