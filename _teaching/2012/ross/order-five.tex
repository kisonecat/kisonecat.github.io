\documentclass{article}
\usepackage{fullpage}
\usepackage{geometry}
\geometry{margin=0.3in,left=1in,right=1in}
\usepackage{nopageno}
\usepackage{amsmath}
\usepackage{amssymb}
\DeclareMathOperator{\SL}{SL}
\DeclareMathOperator{\trace}{trace}
\DeclareMathOperator{\Id}{Id}
\newcommand{\Z}{\mathbb{Z}}
\title{No elements of order five in $\SL_2(\Z)$}
\author{Jim Fowler}

\begin{document}

\maketitle

Here are various methods by which one can see there are no elements of
order five in $\SL_2(\Z)$.  Morally, the ``reason'' is that pentagons
don't tile the plane.

\section*{Cayley--Hamilton theorem}

If $A \in \SL_2(\Z)$, then the characteristic polynomial for $A$ is
$\lambda^2 - (\trace A)\lambda + 1$ and the trace is an integer.  By
the Cayley--Hamilton theorem,
\[
A^2 - n\, A + \Id = 0 \mbox{ where $n = \trace A$.}
\]
And also $A^5 = \Id$.  So
\begin{align*}
A^5 &= (A^2)^2 A \\
&= \left(n A - \Id\right)^2 A \\
%&= (n^2 A^2 - 2 n A + \Id\right) A \\
&= n^2 A^3 - 2 n A^2 + A \\
%&= n^2 (n A - \Id) A - 2 n (n A - \Id) + A \\
%&= n^3 A^2 - 3n^2 A + 2n \Id + A
%&= n^3 (n A - \Id) - 3n^2 A + 2n \Id + A \\
&= (n^4 - 3 n^2 + 1)A + (2n - n^3) \Id = \Id.
\end{align*}
In order to ensure $A$ is not a multiple of the identity, $n = \trace
A \in \Z$ must satisfy
\[
n^4 - 3 n^2 + 1 = 0
\]
but there are no integer solutions to that polynomial.

\section*{No regular pentagon with vertexes on lattice points}

If there is a matrix $M \in \SL_2(\Z)$ of order five, then for a vector $v \in \Z^2$ the five points
\[
v,\hspace{1em}
Mv,\hspace{1em}
MMv,\hspace{1em}
MMMv,\hspace{1em}
MMMMv
\]
are also lattice points.  These five points form a regular pentagon,
but there is no regular pentagon with vertices on lattice points
(e.g., by a \textit{shrinking} argument).

\section*{Trigonometry}

\section*{Orbifolds}

There are no five-fold covers of the modular surface.

\section*{Finding a free kernel}

The abelianization map $\SL_2(\Z) \to \Z/12\Z$ has kernel a free
group.  So the orders of elements of $\SL_2(\Z)$ divide 12.

\section*{Recognizing it is as an amalgamated product}

The group $\SL_2(\Z)$ is $\Z/6\Z \star_{\Z/2\Z} \Z/4\Z$.  Serre's book
\textit{Trees} would help here.

\section*{Counting modulo powers of two}

The special linear group over the field with two elements is
$$
\SL_2(\Z/2\Z) = \left\{
\left(\begin{array}{rr}
0 & 1 \\
1 & 0
\end{array}\right), \left(\begin{array}{rr}
0 & 1 \\
1 & 1
\end{array}\right), \left(\begin{array}{rr}
1 & 0 \\
0 & 1
\end{array}\right), \left(\begin{array}{rr}
1 & 0 \\
1 & 1
\end{array}\right), \left(\begin{array}{rr}
1 & 1 \\
0 & 1
\end{array}\right), \left(\begin{array}{rr}
1 & 1 \\
1 & 0
\end{array}\right) \right\}.
$$
Note that there are six elements.  By more careful counting,
$|\SL_2(\Z/2^n \Z)| = 3 \cdot 2^{3n-2}$, so there are no elements of
order five in $\SL_2(\Z/2^n \Z)$.  If $M \in \SL_2(\Z)$ had order
five, then by choosing $n$ so large that the image of $M$ in
$\SL_2(\Z/2^n \Z)$ is nontrivial, we would have a contradiction.

\section*{Counting modulo primes}

The group $\SL_2(\Z/p\Z)$ has $(p-1)(p)(p+1)$ elements.  There are
infinitely many primes so that $p \not\equiv \pm 1 \pmod 5$.  For any
$M \in \SL_2(\Z)$, one can choose $p$ large enough so that $M \bmod p$
is nontrivial, and so that $5$ does not divide $(p-1)(p)(p+1)$.  Then
$M$ cannot have order five.



\end{document}


%%% Local Variables: 
%%% mode: latex
%%% TeX-master: t
%%% End: 
