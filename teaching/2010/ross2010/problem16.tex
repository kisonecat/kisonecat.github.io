\documentclass[12pt]{pset}
%\usepackage{add-copyright}

\geometry{margin=0.5in}
\title{Problem Set 16}
\course{Piecewise-Linear Topology}
\author{Jim Fowler}
\date{Summer 2010}

\usepackage{tikz}
\usetikzlibrary{matrix}
\usepackage{enumerate}

\newcommand{\defnword}[1]{\textbf{#1}}

\usepackage{stmaryrd}
\newcommand{\boundary}{\partial}
\newcommand{\collapses}{\searrow}
\newcommand{\expands}{\nearrow}
\newcommand{\she}{\ssearrow\nnearrow}
\newcommand{\join}{\ast}
\newcommand{\subdivided}{\triangleleft}
\geometry{margin=1in}%,top=0.5in,bottom=0.5in}
\DeclareMathOperator{\st}{st}
\DeclareMathOperator{\trace}{trace}
\DeclareMathOperator{\vertices}{vert}
\DeclareMathOperator{\lk}{lk}
\DeclareMathOperator{\cl}{cl}
\DeclareMathOperator{\interior}{int}
\DeclareMathOperator{\id}{id}
\newcommand{\fullsubcomplex}{\Subset}
\newcommand{\cupp}{\smallsmile}
\newcommand{\Ct}{C_{\mbox{\scriptsize ord}}}
\newcommand{\Chaint}{C^{\mbox{\scriptsize ord}}}

\usepackage{hyperref}

\begin{document}
\maketitle

% [[skim:///Users/jim/Papers/MR755006.pdf?page=2][page 2 of MR755006.pdf]]

\subsubsection*{High dimensional knots.}  We show that there exist
locally flat embeddings of $S^n$ in $S^{n+2}$ which are knotted.  In
this way, we have come full-circle: we saw (via sunny collapse) that
any embedding of $S^n \subset S^{\geq n+3}$ is unknotted (in the PL
sense), and now we see that in codimension two, knotting is possible.

\begin{definition*}
  An embedding $X^n \subset Y^{m}$ is \textbf{locally flat} if for
  each vertex $v \in X$, the embedding
  $$
  \lk(v,X) = S^{n-1} \subset \lk(v,Y) = S^{m-1}
  $$
  is an unknotted embedding of $S^{n-1} \subset S^{m-1}$.
\end{definition*}

\begin{problem}[from Problem Set 7]
  Suppose $S^1 \subset S^3$ is a knot; show that the suspension $SS^1
  \subset SS^3$ gives a knotted $S^2$ in $S^4$.  How can we see this?
  The easiest way to show that the $S^2 \subset S^4$ is knotted is to
  show that the embedding is not locally flat!
\end{problem}

\begin{requiredproblem}
  Find a proper, locally flat embedding $D^2 \subset D^4$, with
  boundary $\partial D^2 \subset S^3$ given by the stevedore knot.  The
  following sequence of pictures might be inspirational:
  \begin{center}
  \begin{tabular}{ccc}
    \includegraphics[width=0.25\textwidth]{stevedore.pdf} &
    \includegraphics[width=0.25\textwidth]{stevedore2.pdf} &
    \includegraphics[width=0.25\textwidth]{stevedore3.pdf} \\
    stevedore knot &
    stevedore knot &
    two unlinked circles
  \end{tabular}
\end{center}
\end{requiredproblem}

\begin{problem}
  Let $K \subset S^3$ be the stevedore knot; then the suspension $SK
  \subset S^4$ is not locally flat around the cone points.  Explain
  why it is possible to replace neighborhoods of the cone points with
  the locally flat embedding $D^2 \subset D^4$ we constructed in the
  previous problem.
\end{problem}

\begin{problem}
  Find another proper, locally flat embedding $D^2 \subset D^4$, with
  boundary $\partial D^2 \subset S^3$ neither the unknot nor the
  stevedore knot.  Such knots are called \textbf{topologically slice}.
\end{problem}

\begin{definition*}
  Here we describe a procedure called \textbf{spinning a knot}.  Start
  with a knotted, properly embedded arc $K \subset D^3$.  Then $K
  \times S^1 \subset D^3 \times S^1 \subset S^4$, because
  $$S^4 = \partial D^5 = \partial(D^3 \times D^2) = (D^3
  \times S^1) \cup (S^2 \times D^2).$$
  The boundary of $K \times S^1$
  consists of two circles, which we can cap off with two disks.  This
  produces a locally flat embedding of $S^2 \subset S^4$.

  There are quite a few variations on this basic technique (e.g.,
  Zeeman's ``twist-spinning'' which involves, well, twisting the arc
  $K \subset D^3$ while it spins around the $S^1$).
\end{definition*}

\begin{requiredproblem}
  Spin a knotted arc $K \subset B^3$ to produce a locally flat
  embedding $S^2 \subset S^4$ so that the pair $(S^4,S^2)$ is not
  homeomorphic to the standard sphere pair.  Can you find a way to use
  (co)homology to distinguish this sphere pair from the unknot?
\end{requiredproblem}

\end{document}

