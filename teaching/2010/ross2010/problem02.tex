\documentclass[12pt]{pset}
%\usepackage{add-copyright}

\title{Problem Set 2}
\course{Piecewise-Linear Topology}
\author{Jim Fowler}
\date{Summer 2010}

\newcommand{\defnword}[1]{\textbf{#1}}

\newcommand{\boundary}{\partial}
\newcommand{\collapses}{\searrow}
\newcommand{\expands}{\nearrow}
\newcommand{\she}{\ssearrow\nnearrow}
\newcommand{\join}{\ast}
\newcommand{\subdivided}{\triangleleft}
\geometry{margin=1in}%,top=0.5in,bottom=0.4in}
\DeclareMathOperator{\st}{st}
\DeclareMathOperator{\vertices}{vert}
\DeclareMathOperator{\lk}{lk}
\DeclareMathOperator{\cl}{cl}
\DeclareMathOperator{\interior}{int}

\begin{document}
\maketitle

\subsubsection*{Goal.} Problem Set 2 introduces \textit{abstract
  simplicial complexes,} formalizing the intuitive ideas suggested by
Problem Set 1.  You should be warned that what follows is not the only
way to formalize our intuition.  As usual, problems marked
  with a $\bullet$ should be written up.

\parskip 0.25\baselineskip
\parindent 0pt

\begin{definition*}
  Geometrically, an $n$-dimensional \defnword{simplex} (written
  $\Delta^n$, and usually called an $n$-simplex for short) is the
  $n$-dimensional analog of a triangle; just as a triangle is the
  smallest convex set containing 3~points which do not lie on a line,
  the $n$-simplex $\Delta^n$ is the smallest convex set containing
  $n+1$~points in ``general position.''
%\end{definition*}

  But for now, the important feature of a simplex is the relationship
  between the faces.  Indeed, topologically, we do not care about the
  size of the simplexes, or where they are sitting in space---so we
  will abstract away the geometry, leaving only the combinatorics
  behind.

%\begin{definition*}
  A simplex $A$ is a \defnword{face} of a simplex $B$ if the vertices
  determining $A$ are a subset of the vertices determining $B$.  We
  write $A < B$ if $A$ is a face of $B$.  Specifically, some $v$ for which $\{v\}
  \in B$ is called a \defnword{vertex} of $B$.
%\end{definition*}

%\begin{definition*}
%  We allow the \defnword{empty simplex}, which is determined
%  by no points at all, and might call it the $(-1)$-simplex (after all,
%  $-1 + 1 = 0$ points).  Note that the $(-1)$-simplex is a face of every
%  simplex.
%\end{definition*}

%\begin{definition*}

  A \defnword{simplicial complex} $K$ is a collection of finite sets
 (called the ``simplexes''), with the property that if $\sigma \in
 K$, and $\tau < \sigma$ (i.e., if $\tau \subset \sigma$ when thought
 of as finite sets), then $\tau \in K$.

  In words, a simplicial complex is a collection of simplexes, where
  any face of a simplex is also in the complex.  We can think of a
  simplicial complex as a geometric object by gluing together actual
  simplexes along their faces (the so-called ``geometric
  realization'') by following the pattern of the combinatorial data
  encoded by the finite sets.

  Anytime we define a mathematical object, we must also describe the
  maps between such objects.  Let $K$ and $L$ be complexes.  A
  \defnword{simplicial map} $f : K \to L$ is a function $f :
  \vertices(K) \to \vertices(L)$ with the property that if $\sigma \in
  K$, then $f(\sigma) \in L$.
\end{definition*}

\begin{requiredproblem}
  Find an injective simplicial map $f : K_7 \to T^2$; here, $T^2$ is
  the torus, and $K_7$ denotes $7$ points connected in pairs by all
  $\displaystyle\binom{7}{2} = 21$ edges.  Use this to triangulate
  $T^2$ as a simplicial complex with as few triangles as possible.
\end{requiredproblem}

\noindent\parbox{0.85\textwidth}{%
\begin{problem}
 Does there exist an injective map $f : K_{3,3} \to S^2$?  What about
$f : K_{3,3} \to T^2$?  Here, $K_{3,3}$ denotes six points,
connected by nine lines, as shown on the right.
\end{problem}}%
\hspace{10pt}%
\parbox{0.15\textwidth}{%
\includegraphics[width=0.15\textwidth-10pt]{bipartite-graph.pdf}%
}


%%%%%%%%%%%%%%%%%%%%%%%%%%%%%%%%%%%%%%%%%%%%%%%%%%%%%%%%%%%%%%%%
\subsection*{Subcomplexes}

\begin{definition*}
  If $K$ and $L$ are complexes, and every simplex in $K$ is a simplex
  of $L$, then we write $K \subset L$ and say that $K$ is a
  \defnword{subcomplex} of $L$.
\end{definition*}
%Here is an example of a subcomplex.
\begin{example*}
  The $n$-dimensional \defnword{skeleton} (usually called the
  $n$-skeleton) of a complex $K$ consists of all those simplexes which
  contain $n+1$ or fewer points (i.e., are at most $n$-dimensional).
  Write $K^{(n)}$ for the $n$-skeleton.  We say that a complex $K$ is
  an $n$-complex if $K = K^{(n)}$.
\end{example*}
\begin{requiredproblem}
  Prove that the $n$-skeleton of a simplicial complex $K$ is still a
  simplicial complex.
\end{requiredproblem}

%%%%%%%%%%%%%%%%%%%%%%%%%%%%%%%%%%%%%%%%%%%%%%%%%%%%%%%%%%%%%%%%
% \subsection*{Spheres and balls}

% \begin{example*}
% We regard $n$-simplex $\Delta^n$ itself as an example of a simplicial
% complex.  If we label the $n+1$ vertices of $\Delta^n$ using the set
% $V = \{ 0, 1, 2, \ldots, n \}$, then the simplexes of $\Delta^n$ are
% all $2^{n+1}$ subsets of $V$.

% When thinking of the $n$-simplex as a complex, we will usually call it
% $B^n$, thinking of it as the $n$-dimensional \defnword{ball}.
% \end{example*}

% \begin{example*}
% The \defnword{boundary} of a simplex $\sigma$ (written $\boundary \sigma$) is
% the complex consisting of all faces of $\sigma$ except $\sigma$
% itself.

% The $n$-dimensional boundary of the $(n+1)$-simplex $\Delta^{n+1}$
% is the $n$-dimensional \defnword{sphere} $S^n$.
% \end{example*}


%%%%%%%%%%%%%%%%%%%%%%%%%%%%%%%%%%%%%%%%%%%%%%%%%%%%%%%%%%%%%%%%
\subsection*{Joins}

One way to get a new complex from two old complexes is to take their
join.  Geometrically, the join of two complexes $K$ and $L$ is the
complex $K \join L$ consisting of all line segments from a point in
$K$ to a point in $L$.

\begin{definition*}
Let $K$ and $L$ be complexes with disjoint sets of vertices (we call
such complexes \defnword{joinable}).  We define a new complex called
the \defnword{join} of $K$ and $L$, written
$$
K \join L := \{ \sigma \cup \tau : \sigma \in K, \tau \in L \}.
$$
\end{definition*}

\begin{requiredproblem}
Prove that if $K$ and $L$ are joinable complexes, then $K \join L$ is
also a complex.
\end{requiredproblem}

\begin{problem}
If $K$, $L$, and $M$ are joinable complexes, then $(K \join
L) \join M = K \join (L \join M)$ and $K \join L = L \join K$.
\end{problem}

\begin{problem}
  Prove that $\Delta^0 \join \Delta^n = \Delta^{n+1}$, and therefore
  that $\Delta^n \join \Delta^m = \Delta^{n+m+1}$.  (In this problem,
  it looks as if $\Delta^0$ and $\Delta^n$ ought not be joinable,
  since they both have a vertex labeled $0$.  In this case, we
  tacitly rename vertices to make the two complexes joinable.  But
  then, what does that equal sign really mean?)
\end{problem}

\pagebreak

\subsection*{Euler characteristic}

\begin{definition*}
For a complex $K$, the Euler characteristic $\chi(K)$ is
$$
\sum_{n} (-1)^n c_n, \mbox{ where $c_n$ is the number of $n$-simplexes
  in $K$.}
$$
\end{definition*}

\begin{requiredproblem}
Can you calculate $\chi(K \join L)$ in terms of $\chi(K)$ and $\chi(L)$?
\end{requiredproblem}

\begin{requiredproblem}
Can you calculate $\chi(K_1 \cup K_2)$ in terms of $\chi(K_1)$ and
$\chi(K_2)$?%  What other information do you need?
\end{requiredproblem}

\begin{requiredproblem}
  Calculate $\chi(S^n)$, $\chi(\Delta^n)$, $\chi(T^2)$, and
  $\chi(T^3)$.  But consider: what does this mean?  There are distinct
  simplicial complexes which we want to refer to as $S^n$ and
  $T^n$\ldots
\end{requiredproblem}

\begin{problem}
  Let $K$ and $L$ be surfaces.  Suppose $f : K \to L$ is a simplicial
  map, so that
\begin{center}
\begin{tabular}{l@{ }l@{ }l@{ }l@{ }l}
the preimage of every & vertex & in $L$ consists of two & vertices & in $K$, \\
the preimage of every & edge & in $L$ consists of two & disjoint edges & in $K$, \\
the preimage of every & triangle & in $L$ consists of two & disjoint triangles & in $K$.
\end{tabular}
\end{center}
First, relate $\chi(L)$ and $\chi(K)$.  Then find an example of surfaces $K$
and $L$ and such a ``doubling'' map $f : K \to L$.
\end{problem}

\subsection*{Products of complexes}

\begin{problem}
  Describe the $n$-cube $I^n$ as a simplicial complex, by building a
  simplicial complex $K$ with $n$ coordinate maps $f_i : K \to I =
  \Delta^1$.  Show that each point $x \in K$ is uniquely determined by
  its image under the $n$ coordinate maps.
\end{problem}

\begin{problem}
  Describe the $n$-torus $T^n$ as a simplicial complex, by building a
  simplicial complex $K$ with $n$ coordinate maps $f_i : K \to S^1$,
  so that each point $x \in K$ is determined by its image under these
  coordinate maps.
\end{problem}


\end{document}

