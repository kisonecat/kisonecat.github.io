\documentclass[12pt]{pset}
%\usepackage{add-copyright}

\geometry{margin=0.5in}
\title{Problem Set 14}
\course{Piecewise-Linear Topology}
\author{Jim Fowler}
\date{Summer 2010}

\usepackage{tikz}
\usetikzlibrary{matrix}

\newcommand{\defnword}[1]{\textbf{#1}}

\usepackage{stmaryrd}
\newcommand{\boundary}{\partial}
\newcommand{\collapses}{\searrow}
\newcommand{\expands}{\nearrow}
\newcommand{\she}{\ssearrow\nnearrow}
\newcommand{\join}{\ast}
\newcommand{\subdivided}{\triangleleft}
\geometry{margin=1in}%,top=0.5in,bottom=0.5in}
\DeclareMathOperator{\st}{st}
\DeclareMathOperator{\trace}{trace}
\DeclareMathOperator{\vertices}{vert}
\DeclareMathOperator{\lk}{lk}
\DeclareMathOperator{\cl}{cl}
\DeclareMathOperator{\interior}{int}
\DeclareMathOperator{\id}{id}
\newcommand{\fullsubcomplex}{\Subset}
\newcommand{\cupp}{\smallsmile}

\usepackage{hyperref}

\begin{document}
\maketitle

% [[skim:///Users/jim/Papers/MR755006.pdf?page=2][page 2 of MR755006.pdf]]

\subsubsection*{Cohomology.} We introduce a dual version of homology,
called cohomology.  Reversing all the arrows, while not immediately
evident, is quite useful.

%Why does the sequence split?  last time we used the fact that $X
%\times I$ is the same as $X$ to compute that $H_n(X \times S^1) =
%H_n(X) \oplus H_{n-1}(X)$.

\parskip 0.45\baselineskip
\parindent 0pt

\begin{definition*}
  An integer-valued \textbf{cochain} is a homomorphism $\alpha :
  C_n(X) \to \Z$.  We denote the collection of all such homomorphisms
  by $C^n(X)$.  The \textbf{coboundary} is a map $d : C^n(X) \to
  C^{n+1}(X)$ defined by $(d \alpha)(\sigma) = \alpha(\partial
  \sigma)$ for a simplex $\sigma$.  If a cochain has zero coboundary,
  we call it a \textbf{cocycle}; let $Z^n(X)$ denote the collection of
  all cocycles.  Let $B^n(X)$ denote the coboundaries, that is,
  the image of $d : C^{n-1}(X) \to C^n(X)$.

  Just as homology was cycles modulo boundaries, cohomology is
  cocycles modulo coboundaries; in symbols, $H^n(X) = Z^n(X) / B^n(X)$.
\end{definition*}


\begin{problem}
  As before, if $X = A \cup B$, then we have a short exact sequence
  $$
  0 \to C^\bullet(X) \to C^\bullet(A) \oplus C^\bullet(B) \to C^\bullet(A \cap B) \to 0
  $$
  which, by the zig-zag lemma, gives a long exact sequence in
  cohomology.  Describe this long exact sequence.
\end{problem}

\begin{problem}
  Check that a map $f : X \to Y$ induces a homomorphism $H^n(Y) \to
  H^n(X)$, that is, in the wrong direction!  Because of this, we say
  that cohomology is \textbf{contravariantly functorial}, whereas
  homology is \textbf{covariantly functorial}.
\end{problem}

\begin{requiredproblem}
  Compute $H^\star(S^1)$.
\end{requiredproblem}

\begin{problem}
  Compute the cohomology of $T^2$ as before, by using the fact that
  cohomology is homotopy invariant, and $S^1 \times I \simeq S^1$.
  Compare $H^\bullet(T^2)$ with $H_\bullet(T^2)$.
\end{problem}

\begin{requiredproblem}
  Compute the cohomology of $\RP^2$.  Compare $H^\bullet(\RP^2)$ with $H_\bullet(\RP^2)$.
\end{requiredproblem}

\begin{problem}
  Given $\alpha \in C^n(X)$ and $x \in C_n(X)$, we get an integer
  $\alpha(x)$.  Does this descend to cohomology and homology?  In other words, if
  $\alpha \in H^n(X)$ and $x \in H_n(X)$, can you get a well-defined integer?
  Is it still well-defined if $\alpha \in H^n(X)$ but  $x \in C_n(X)$, or conversely if
  $\alpha \in C^n(X)$ but $x \in H_n(X)$?
\end{problem}

\pagebreak

\subsection*{Cup product}

\begin{definition*}
  For the time being, we replace $\Z$ by $\Q$; just as we can define $C_n(X;\Q)$, we define $C^n(X;\Q)$ by considering homomorphisms from $C_n(X;\Q)$ to $\Q$.  Now suppose $\alpha \in C^n(X;\Q)$ and $\beta \in C^m(X;\Q)$.  Define $\alpha \cupp \beta$ by
  \begin{align*}
  & (\alpha \cupp \beta) [v_0,\ldots,v_{n+m+1}] = \\
  &\quad\quad\quad \frac{1}{(n+m+1)!} \sum_{\sigma \in S_{n+m+1}} (-1)^\sigma \cdot \alpha [v_{\sigma(0)},\ldots,v_{\sigma(n)}] \cdot \beta [v_{\sigma(n)},\ldots,v_{\sigma(n+m+1)}] 
  \end{align*}
\end{definition*}

\begin{requiredproblem}
  If $\alpha \in Z^n(X;\Q)$ and $\beta \in Z^m(X;\Q)$, is it the case that $\alpha \cupp \beta \in Z^{n+m}(X;\Q)$?
\end{requiredproblem}

\begin{requiredproblem}
  If $\alpha \in H^n(X;\Q)$ and $\beta \in H^m(X;\Q)$, show that $\alpha
  \cupp \beta$ a well-defined cohomology class in $H^{n+m}(X;\Q)$.  Thus,
  we have defined the \textbf{cup product}, making $H^\bullet(X;\Q)$ into
  a ring, rather than a collection of abelian groups.
\end{requiredproblem}

\begin{problem}
  Show that a map $f : X \to Y$ induces a ring map $H^\bullet(Y;\Q) \to H^\bullet(X;\Q)$.
\end{problem}

\begin{problem}
  Compute the ring structure of $H^\bullet(T^2;\Q)$.
\end{problem}

\begin{problem}
  Is it possible for a map $f : T^2 \to T^2$ to induce an isomorphism
  on $H^1(T^2;\Q)$ but the zero map on $H^2(T^2;\Q)$?
\end{problem}

%\pagebreak

\subsection*{Spaces with the homology of a point}

Last time, Tim Campion asked for an example of a non-contractible
space with the homology of a point.  Here are a few exercises which
culiminate in the description of such a space.

\begin{problem}
  As a warm-up exercise, let $X = S^1 \times D^2$ be the solid torus.
  Because $\partial X = S^1 \times S^1 = T^2$, we can use a
  homeomorphism $h : T^2 \to T^2$ to glue together two copies of $X$.
  Let $h$ be the map that ``exchanges the two coordinate axes'' and
  show that the resulting space $X \cup_{\partial X} X$ is $S^3$.
\end{problem}

\begin{problem}
  Let $X$ be $S^3$ after removing a regular neighborhood of a trefoil
  knot.  Calculate $H_\bullet(X)$.  (Incidentally, what do you get if
  you perform the same procedure with a different knot?).
\end{problem}

\begin{problem}
  Again, let $X$ be $S^3$ after removing a regular neighborhood of a
  trefoil knot.  Let $P$ be the union of $X$ with the solid torus $S^1
  \times D^2$, along the common torus boundary, using the map that
  ``exchanges the two coordinate axes.''  Calculate $H_\bullet(P)$ via Mayer-Vietoris, and show
  $H_\bullet(P) \cong H_\bullet(S^3)$.

  In fact, what we have produced here is a \textbf{homology sphere}, a
  space with the homology of a sphere.  This space $P$ is the very
  same homology sphere that Poincar\'e discovered by gluing together
  the sides of a dodecahedron.

  The \textbf{Poincar\'e conjecture} asserts that if a manifold $M^n$
  is homotopy equivalent to $S^n$, then $M$ is homeomorphic to $S^n$;
  the space $P$, being distinct from $S^n$, shows that the Poincar\'e
  conjecture is false if the words ``homotopy equivalent to'' were
  replaced by ``having the homology of.''
\end{problem}

\begin{problem}
  Use Mayer-Vietoris to show that $H_\bullet(P - \interior D^3) \cong
  H_\bullet(\mbox{point})$.  By $P - \interior D^3$, we mean $P$ with the interior
  of a tetrahedron removed.  If we had studied the
  \textbf{fundamental group} $\pi_1$, we could further show that $P -
  \interior D^3 \not\simeq \mbox{point}$.

  Thus, $P - \interior D^3$ is an \textbf{acyclic space}, a space having the
  same homology as that of a point.
\end{problem}

\begin{problem}
  Repeated the previous problem for an arbitrary homology sphere.
\end{problem}

\begin{requiredproblem}
  We can make the preceding discussion a bit more concrete:
  triangulate a dodecahedron, by introducing a vertex at the center of
  each pentagon and dividing each pentagon into five triangles.  Then,
  identify opposite pentagons with a $\pi/3$ twist.  The result is a
  2-dimensional simplicial complex $X$ with 30 triangles and
  $H_\bullet(X) \cong H_\bullet(\mbox{point})$.

%  The hardest part of this problem, like so many in life, is just
%  believing that you can do the calculation.  You can!  To help you
%  get started, notice that any 1-chain is homologous to a 1-chain on
%  the edges of the original dodecahedron\ldots
\end{requiredproblem}

\end{document}

