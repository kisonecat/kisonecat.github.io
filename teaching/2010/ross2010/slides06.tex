\documentclass[14pt]{beamer}

\setbeamertemplate{navigation symbols}{}

\usepackage{pgf}
\usepackage{amsmath}
\usepackage{amsthm}
%\newtheorem*{problem}{Problem}
\usepackage{amssymb}
\newcommand{\N}{\mathbb{N}}
\newcommand{\R}{\mathbb{R}}

\usepackage{stmaryrd}
\newcommand{\boundary}{\partial}
\newcommand{\collapses}{\searrow}
\newcommand{\expands}{\nearrow}
\newcommand{\she}{\ssearrow\nnearrow}
\newcommand{\join}{\ast}
\newcommand{\subdivided}{\triangleleft}
\newcommand{\fullsubcomplex}{\Subset}

\DeclareMathOperator{\st}{st}
\DeclareMathOperator{\vertices}{vert}
\DeclareMathOperator{\lk}{lk}
\DeclareMathOperator{\cl}{cl}
\DeclareMathOperator{\interior}{int}

\title{Topology of  \\ Piecewise-Linear Manifolds}
\author{Jim Fowler}
\date{Lecture 6 \\ Summer 2010}

\newcommand{\setbackgroundpicture}[1]{%
\usebackgroundtemplate{
\begin{pgfpicture}{0in}{0in}{\paperwidth}{\paperheight}
\pgfputat{\pgfxy(0,0)}{\includegraphics[width=\paperwidth]{#1}}
\color{white}
\pgfsetfillopacity{0.8}
\pgfrect[fill]{\pgfxy(0,0)}{\pgfpoint{\paperwidth}{\paperheight}}
\end{pgfpicture}
}
}
\newcommand{\clearbackgroundpicture}{\usebackgroundtemplate{}}

\begin{document}

\begin{frame}
\maketitle
\end{frame}

%%%%%%%%%%%%%%%%%%%%%%%%%%%%%%%%%%%%%%%%%%%%%%%%%%%%%%%%%%%%%%%%

\setbackgroundpicture{A20_Ouest_km143.jpg}
\begin{frame}
\frametitle{Where are we now?}

\begin{pgfpicture}{0in}{0in}{\textwidth}{\textheight}
\pgfsetxvec{\pgfpoint{\textwidth}{0cm}}
\pgfsetyvec{\pgfpoint{0cm}{\textheight}}

\only<2->{\pgfputat{\pgfxy(0.25,0.80)}{\pgfbox[center,bottom]{\scalebox{1}{\Huge\textcolor{green!50!black}{Objects}}}}}

\only<2->{\pgfputat{\pgfxy(0.25,0.6)}{\pgfbox[center,bottom]{\scalebox{1}{\large\textcolor{green!50!black}{\parbox{0.5\textwidth}{\begin{center}\uncover<3->{polyhedra} \\ \uncover<4->{simplicial complexes}\end{center}}}}}}}

\only<5->{\pgfputat{\pgfxy(0.75,0.80)}{\pgfbox[center,bottom]{\scalebox{1}{\Huge\textcolor{red!50!black}{Maps}}}}}

\only<5->{\pgfputat{\pgfxy(0.75,0.6)}{\pgfbox[center,bottom]{\scalebox{1}{\large\textcolor{red!50!black}{\parbox{0.5\textwidth}{\begin{center}\uncover<6->{PL maps} \\ \uncover<7->{simplicial maps}\end{center}}}}}}}

\only<8->{\pgfputat{\pgfxy(0.5,0.35)}{\pgfbox[center,bottom]{\scalebox{1}{\Huge\textcolor{blue!50!black}{Invariants}}}}}

\only<9->{\pgfputat{\pgfxy(0.25,0.25)}{\pgfbox[center,bottom]{\scalebox{1}{\Large\textcolor{blue!50!black}{$\chi$}}}}}

\only<10->{\pgfputat{\pgfxy(0.5,0.25)}{\pgfbox[center,bottom]{\scalebox{1}{\Large\textcolor{blue!50!black}{$b_0$}}}}}

\only<11->{\pgfputat{\pgfxy(0.75,0.25)}{\pgfbox[center,bottom]{\scalebox{1}{\Large\textcolor{blue!50!black}{$w_1$}}}}}

\end{pgfpicture}
\end{frame}
\clearbackgroundpicture

\begin{frame}
  \frametitle{Piecewise linear map}
  Let $P, Q$ be polyhedra. \\
  $f : P \to Q$ is a PL map \\
  if each point $p \in P$ \\
  has a closed star $N = p \join L$ \\
  so that $f(\lambda p + (1-\lambda) x) = \lambda f(p) + (1 - \lambda)
  f(x)$ \\
  for $x \in L$ and $\lambda \in [0,1]$.

  \vfill

  It locally maps conical rays to conical rays.

\end{frame}

\begin{frame}
  \frametitle{Convexity}
  
  \begin{definition}
    A subset $C \in \R^n$ is \textbf{convex} \\
    if for any $p,q \in C$,\\
    the segment $\{p\} \join \{q\}$\\
    is contained in $C$.
  \end{definition}

\end{frame}

\begin{frame}
\frametitle{Cells}

 A compact convex subset $C \in \R^n$ \\
  is a $k$-dimensional \textbf{cell} \\
  if it spans a $k$-dimensional subspace.

  \vfill\pause

  For $x \in C$, define $\langle x, C \rangle$ to be \\
  the union of $\{x\}$ and \\
  all lines $L$ such that $L \cap C$ contains $x$ in its
  interior.

  \vfill\pause

  The subset $C_x = C \cap \langle x, C \rangle$ is \\
  the \textbf{face} of $C$ containing $x$.

  \vfill\pause

  Write $D < C$ if $D$ is a face  of $C$.


\end{frame}


\begin{frame}
\frametitle{Cell complex}

 A \textbf{cell complex} is \\
  a finite collection $K$ of cells such that
  \begin{itemize}
  \item If $C \in K$, and $D < C$, then $D \in K$.
  \item If $C, D \in K$, then $C \cap D$ is a face of $C$ and $D$.
  \end{itemize}

  \vfill\pause
  The \textbf{underlying polyhedron}, $|K|$ is \\
  the union of all cells in $K$.

  \vfill\pause

  A \textbf{cellular map} $f : K \to L$ is \\
  a PL map $|f| : |K| \to |L|$ \\
  which is linear on cells of $K$, \\
  and sends cells to cells.

\end{frame}

\begin{frame}
\frametitle{Subdivision}

 A cell complex $L$ is a \textbf{subdivision} of $K$ \\
  if $|L| = |K|$ \\
  and each cell of $L$ is contained in a cell $K$. \\

  \vfill\pause
  Write $L \subdivided K$ if $L$ is a subdivision of $K$.

\end{frame}

\begin{frame}
\frametitle{Simplicial complexes, not abstract}

 A cell complex is a \textbf{simplicial complex} \\
  if each $C \in K$ is a \textbf{simplex} \\
  \quad (i.e., an $n$-cell which is the\\
  \quad\quad join of $n+1$ independent points).  \\
  \vfill
  A \textbf{triangulation} of a polyhedron $P$\\
  is a simplicial complex $K$ \\
  with a PL homeomorphism $f : |K| \to P$.

\end{frame}


\begin{frame}
\vfill
\begin{center}
 From now on, \\
 {\Huge complex} \\
 means \\
 \textbf{simplicial complex,} \\
 \textit{(not abstract)}
\end{center}
\vfill
\end{frame}


\setbackgroundpicture{collapsedbarn.jpg}
\begin{frame}
  \vfill
  \begin{center}
    \Huge Simplicial \\
    Collapse
  \end{center}
  \vfill
\end{frame}
\clearbackgroundpicture

\begin{frame}
  \frametitle{Principal Simplexes}

  \begin{definition}
  Let $K$ be a complex, and \\
  $\sigma \in K$ a simplex.  \\
  Call $\sigma$ a \textbf{principal simplex} \\
  if the only simplex containing $\sigma$ \\
  is $\sigma$ itself \\
  \quad (i.e., it isn't contained in a larger simplex).
  \end{definition} 

  \pause
  \begin{problem}
  Does every complex have a principal simplex?
  \end{problem}

\end{frame}

\begin{frame}
  \frametitle{Free faces}

  \begin{definition}
  Let $K$ be a complex, \\
  and $\sigma \in K$ a simplex, \\
  and $\tau < \sigma$ a face. \\
  \vspace{1ex}Call $\tau$ a \textbf{free face} of $\sigma$ \\
  if the only simplexes containing $\tau$ \\
  are $\tau$ and $\sigma$.
  \end{definition}

  \pause
  \begin{problem}
    \begin{tabular}{@{}l@{ }c@{ }l}
      Does & every & complex have a simplex with a free face? \\\pause
      Does & any & complex have a simplex with a free face? \\
   \end{tabular}
 \end{problem}

\end{frame}

\begin{frame}
\frametitle{Elementary simplicial collapse}

\begin{definition}
  Let $L$ and $K = L \cup \cl \{ \sigma, \tau \}$ be complexes \\

  \vspace{1ex}If $\sigma$ is a principal simplex of $K$, and \\
  $\tau$ is a free face of $\sigma$, then \\
  $L$ is an \textbf{elementary simplicial collapse} of $K$.
\end{definition}

\end{frame}

\begin{frame}
\frametitle{Simplicial collapse}

\begin{definition}
  Let $K_1, K_2, \ldots, K_n$ be complexes, with \\
  $K_{i+1}$ an elementary simplicial collapse of $K_i$.

  \vspace{1ex}Call $K_n$ a \textbf{simplicial collapse} of $K_1$, and \\
  write $K_1 \collapses K_n$.

  \vspace{1ex}Call $K_1$ a \textbf{simplicial expansion} of $K_n$, and \\
  write $K_n \expands K_1$.
\end{definition}

\end{frame}

\begin{frame}
  \frametitle{Simple homotopy equivalence}

  \begin{definition}
    $K$ is \textbf{simple homotopy equivalent} to $L$
    \quad (sometimes abbreviated s.h.e.) \\
    if you can reach transform $K$ into $L$ \\
    via a sequence of
    \begin{itemize}
    \item PL homeomorphisms,
    \item simplicial collapses,
    \item simplicial expansions.
    \end{itemize}
    In this case, we write $K \she L$.
  \end{definition}

\end{frame}

% \setbackgroundpicture{Grafton.jpg}
% \begin{frame}
% \frametitle{Collars}

%   $P \subset Q$ polyhedra.

%   A \textbf{collar} on $P$ in $Q$ \\
%   is an embedding $c : P \times I \to Q$ \\
%   throwing $P \times \{ 0 \}$ onto $P$, \\
%   so that $c(P \times [0,1))$ is an open neighborhood of $P$.
  
% \end{frame}
% \clearbackgroundpicture

\begin{frame}
  \frametitle{Full subcomplex}
  
    $L \subset K$ are complexes.

    \vfill
    Define $f_L : K \to [0,1]$ on vertices by
    $$
    f_L(v) = \begin{cases}
      0 & \mbox{ if $v \in L$ } \\
      1 & \mbox{ if $v \not\in L$ } 
    \end{cases}
    $$
    and extending linearly to simplexes. 

    \vfill
    If $L = {f_L}^{-1}(0)$ \\
    we say $L$ is a \textbf{full subcomplex} of $K$, \\
    and write $L \fullsubcomplex K$.

\end{frame}

\begin{frame}
  \frametitle{Derived subdivision}

  $L \subset K$ are complexes.

  \vfill

  Choose $a_i \in \interior A_i$ for each $A_i \in K$ with $A_i
  \not\in L$.

  \vfill
  
  $K' \subdivided K$ is ``K derived away from $L$'' \\
  is defined inductively over skeleta by
  $$
  {A_i}'  = \begin{cases}
    \{ a_i \} \join \partial {A_i}' & \mbox{ if $A_i \not\in L$, } \\
    A_i & \mbox{ if $A_i \in L$.}
    \end{cases}
  $$


\end{frame}

\begin{frame}
  $L \subset K$ are complexes.

  \vfill

  Define the simplicial neighborhood of $L$ in $K$ as
  $$N(L,K) = \{ A \in K : A < B, B \cap |L| \neq \varnothing \}.$$

  \vfill

  The \textbf{simplicial complement} of $L$ in $K$ is
  $$
C(L,K) = \{ A \in K : A \cap |L| = \varnothing \}.
  $$

  \vfill\pause
  $K = N(L,K) \cup C(L,K)$
  
  \vfill\pause
  $\partial N(L,K) := N(L,K) \cap C(L,K)$
\end{frame}

\begin{frame}
  A derived of $K$ \textit{near} $L$ \\
  is obtained by \\
  deriving $K$ away from $L \cup C(L,K)$, \\
  i.e., deriving simplexes meeting $|L|$ but not in $L$.

  \vfill\pause
  $L \fullsubcomplex K'$ if $K'$ is $K$ derived near $L$.

\end{frame}

\begin{frame}
  \vfill
  \begin{center}
    \scalebox{9}{Why?}
  \end{center}
  \vfill
\end{frame}

\begin{frame}
  If $K'$ is $K$ derived near $L$, \\
  $N(L,K')$ is a \\
  \textbf{derived neighborhood} of $L$ in $K$.

  \vfill
  \begin{theorem}
    If $K_1$ and $K_2$ are deriveds of $K$ near $L$, \\
    then $s : K_1 \to K_2$ \\
    throws $N(L,K_1)$ onto $N(L,K_2)$ \\
    and is the identity on $L \cup C(L,K)$.
  \end{theorem}

\end{frame}

\setbackgroundpicture{Bangsar.JPG}
\begin{frame}
\frametitle{Regular neighborhoods}

Suppose $X \subset Y$ are polyhedra.

$|K|$ a neighborhood of $X$ in $Y$, \\
$|L| = X$ \\
$L \fullsubcomplex K$ \\
$K'$ derived of $K$ near $L$.

\vfill
$|N(L,K')|$ is called a \textbf{regular neighborhood} \\

\begin{theorem}
  If $N_1$ and $N_2$ are regular neighborhoods of $X$ in $Y$,\\
  then there is a homeomorphism $h : Y \to Y$ \\
  which throws $N_1$ onto $N_2$, and \\
  which is the identity on $X$ \\
  and the identity outside a compact subset of $Y$.
\end{theorem}

\end{frame}

\begin{frame}
  \frametitle{Regular neighborhoods in manifolds}

  \begin{theorem}
    A regular neighborhood $N$ of \\
    a polyhedron $X$ in a manifold $M$ \\
    is a manifold with boundary.
  \end{theorem}
\end{frame}

\begin{frame}
 \begin{theorem}[Simplicial neighborhood theorem]
    $X$ a compact polyhedron \\
   $M$ a manifold \\
   $X \subset \interior M$ \\
   $N$ a neighborhood of $X$ in $\interior M$. \\
   Then $N$ is a regular neighborhood iff
    \begin{itemize}
      \item $N$ is a compact manifold with boundary
      \item there are triangulations $(K,L,J)$ of $(N,X,\partial N)$
        with $L \fullsubcomplex K$, $K = N(L,K)$ and $J = \partial N(L,K)$.
    \end{itemize}
  \end{theorem}
\end{frame}

\begin{frame}
  \begin{theorem}[SNT version 2]
    $X$ a compact polyhedron \\
    $M$ a manifold \\
    $X \subset \interior M$ \\
    $N$ a polyhedral neighborhood of $X$ in $\interior M$. \\
    Then $N$ is a regular neighborhood iff
    \begin{itemize}
    \item $N$ is a compact manifold with boundary
    \item $N \collapses X$.
    \end{itemize}
  \end{theorem}

  \vfill\pause
  e.g., if $X \collapses \mbox{point}$, $N$ is a ball.
  \vfill\pause
  e.g., a collapsible manifold is a ball.
\end{frame}


\clearbackgroundpicture

\begin{frame}
  \begin{corollary}
    If $B^n \subset S^n$, then \\
    $\cl(S^n - B^n)$ is a ball.
  \end{corollary}

  \pause
  \begin{proof}
    $S^n = \partial \Delta^{n+1}$. \\
    A point $p \in B^n$ is contained in some $\Delta^n \subset
    \Delta^{n+1}$, \\
    and both $\Delta^n$ and $B$ are regular neighborhoods of $P$. \\
    Regular neighborhoods are unique.
  \end{proof}
\end{frame}

%\setbackgroundpicture{zeeman.pdf}
\begin{frame}
 \vfill
  \begin{center}
    next time \\
    we will introduce the work of \\
    \Huge E. C. Zeeman
  \end{center}
  \vfill
\end{frame}
%\clearbackgroundpicture

\end{document}


%%% Local Variables: 
%%% mode: latex
%%% TeX-master: t
%%% End: 
