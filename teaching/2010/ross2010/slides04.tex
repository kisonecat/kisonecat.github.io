\documentclass[14pt]{beamer}

\setbeamertemplate{navigation symbols}{}

\usepackage{pgf}
\usepackage{amsmath}
\usepackage{amsthm}
%\newtheorem*{problem}{Problem}
\usepackage{amssymb}
\newcommand{\N}{\mathbb{N}}
\newcommand{\R}{\mathbb{R}}

\usepackage{stmaryrd}
\newcommand{\boundary}{\partial}
\newcommand{\collapses}{\searrow}
\newcommand{\expands}{\nearrow}
\newcommand{\she}{\ssearrow\nnearrow}
\newcommand{\join}{\ast}
\newcommand{\subdivided}{\triangleleft}

\DeclareMathOperator{\st}{st}
\DeclareMathOperator{\vertices}{vert}
\DeclareMathOperator{\lk}{lk}
\DeclareMathOperator{\cl}{cl}
\DeclareMathOperator{\interior}{int}

\title{Topology of  \\ Piecewise-Linear Manifolds}
\author{Jim Fowler}
\date{Lecture 4 \\ Summer 2010}

\newcommand{\setbackgroundpicture}[1]{%
\usebackgroundtemplate{
\begin{pgfpicture}{0in}{0in}{\paperwidth}{\paperheight}
\pgfputat{\pgfxy(0,0)}{\includegraphics[width=\paperwidth]{#1}}
\color{white}
\pgfsetfillopacity{0.8}
\pgfrect[fill]{\pgfxy(0,0)}{\pgfpoint{\paperwidth}{\paperheight}}
\end{pgfpicture}
}
}
\newcommand{\clearbackgroundpicture}{\usebackgroundtemplate{}}

\begin{document}

\begin{frame}
\maketitle
\end{frame}

% \begin{frame}
%   \begin{columns}
% \begin{column}{0.35\textwidth}
% \includegraphics[width=\textwidth]{matt-wage.jpg}
% \end{column}
% \begin{column}{0.65\textwidth}
%  \begin{center}
%    \Huge
%     Thank You, Matt Wage! \\
%     \vspace{1ex}
%     \normalsize 
%     You have cleared up my confusion \\
%     about the \textbf{link} of a simplex
%  \end{center}
% \end{column}
% \end{columns}
% \end{frame}


\setbackgroundpicture{link.pdf}
\begin{frame}
\frametitle{Link}

\begin{definition}
Let $K$ be a complex, and $\sigma \in K$ a simplex.  \\
The \textbf{link} of $\sigma \in K$, \\
\quad written $\lk(\sigma,K)$, \\
consists of those simplexes in $K$ which are in $\cl(\st(\sigma,K))$
but not touching $\sigma$; in other words,
$$
\lk(\sigma,K) = \{ \tau \in \cl(\st(\sigma,K)) : \tau \cap \sigma =
\varnothing \}.
$$
\end{definition}

\textcolor{red}{\textbf{Warning for myself}} \\
The other definition also appears in the literature.
\end{frame}
\clearbackgroundpicture

\setbackgroundpicture{Crab_Nebula.jpg}
\begin{frame}
\frametitle{Stellar Subdivision} 

\begin{definition}
  Let $K$ be a complex, and $\sigma \in K$ a simplex.  \\
  The \textbf{stellar subdivision} of $K$ at $\sigma$ is a new complex
  $K_\sigma$ with:
\begin{itemize}
\item the vertices of $K$ with a new vertex $v$.
\item the simplexes of $K$ not in $\st(\sigma,K)$,\\
  along with the simplexes in $v \join (\boundary \sigma) \join \lk(\sigma,K)$.
\end{itemize}
We might say:
$$
K_\sigma := (K - \st(\sigma,K)) \cup (v \join (\boundary \sigma) \join
\lk(\sigma,K))
$$
\end{definition}

\end{frame}
\clearbackgroundpicture

\begin{frame}
  \frametitle{Some names}

  The join of $X$ with a disjoint vertex \\
  is called ``the \textbf{cone} on $X$,'' \\
  and is denoted $CX$.

  \pause
  \vfill

  Can you compute $\chi(CX)$?

  \pause
  \vfill

  $\chi(CX) = \chi(\mbox{point}) + \chi(X) -
  \chi(\mbox{point})\chi(X) \pause = 1$.
  

\end{frame}

\begin{frame}
  \frametitle{Some names}

  The join of $X$ with $S^0$ \\
  is called ``the \textbf{suspension} of $X$,'' \\
  and is denoted $SX$.

  \pause
  \vfill

  Suggestively, $SS^n = S^{n+1}$.

  \pause
  \vfill

  $ST^2$ is not a manifold.

\end{frame}

%%%%%%%%%%%%%%%%%%%%%%%%%%%%%%%%%%%%%%%%%%%%%%%%%%%%%%%%%%%%%%%%
%%%%%%%%%%%%%%%%%%%%%%%%%%%%%%%%%%%%%%%%%%%%%%%%%%%%%%%%%%%%%%%%
%%%%%%%%%%%%%%%%%%%%%%%%%%%%%%%%%%%%%%%%%%%%%%%%%%%%%%%%%%%%%%%%
%%%%%%%%%%%%%%%%%%%%%%%%%%%%%%%%%%%%%%%%%%%%%%%%%%%%%%%%%%%%%%%%

\setbackgroundpicture{problem03.pdf}
\begin{frame}
\vfill
\begin{center}
from the last \\
\scalebox{4}{homework}
\end{center}
\vfill
\end{frame}
\clearbackgroundpicture

\setbackgroundpicture{Louis-Nicolas_Clerambault.jpg}
\begin{frame}
  \frametitle{Composition is key}
  \pause

 The most important thing about functions \\
  is how they compose.
  \vfill
 \pause
 
  Evaluating a function $f : A \to B$ at a point $x \in A$ \\
  is really composition of a ``point function''
  $$
  \star \to A
  $$
  with $f$, to get a new point function $\star \to B$.
  \vfill
  \pause

  \textit{To know how the function composes \\
  is to know the function.}
\end{frame}

\begin{frame}
  \frametitle{Composing Maps}

  \begin{problem}
    Let $K, L, M$ be complexes. \\
    If $f : K \to L$ and $g : L \to M$ are PL maps, \\
    how should we define the PL map $g \circ f : K \to M$?    
  \end{problem}

  We need a more general notion of ``PL map'' \\
  in order to do this!

%   \pause
%   \textcolor{red!50!black}{\textbf{Common refinement}}\\
%   \pause
%   Start with underlying simplicial maps
%   \begin{align*}
%     f :& K_1 \to L_1 \\
%     g :& L_2 \to M_1,
%   \end{align*}
%   \pause
%   Find common subdivision $L_3$ of $L_1$ and $L_2$. \\
%   \pause
%   Refine $f : K_2 \to L_3$. \\
%   Refine $g : L_3 \to M_2$. \\
%   \pause
%   Compose $g \circ f : K_2 \to M_2$.

\end{frame}
\clearbackgroundpicture

\setbackgroundpicture{genus-two-surface.png}
\begin{frame}
\frametitle{Manifoldness preserved\only<1>{?}\only<2->{!}}
\begin{problem}
  If $M$ is a manifold, \\
  and $N \cong M$, \\
  is $N$ a manifold?
\end{problem}

\vfill
\uncover<2->{Links are preserved by PL homeomorphisms}
\vfill
\uncover<3->{``Preserved'' but in what sense?} \\
\uncover<4->{Up to PL homeomorphism!}
\vfill

\end{frame}
\clearbackgroundpicture

%%%%%%%%%%%%%%%%%%%%%%%%%%%%%%%%%%%%%%%%%%%%%%%%%%%%%%%%%%%%%%%%

\setbackgroundpicture{A20_Ouest_km143.jpg}
\begin{frame}
\frametitle{Where are we?}

\begin{pgfpicture}{0in}{0in}{\textwidth}{\textheight}
\pgfsetxvec{\pgfpoint{\textwidth}{0cm}}
\pgfsetyvec{\pgfpoint{0cm}{\textheight}}

\only<2->{\pgfputat{\pgfxy(0.25,0.80)}{\pgfbox[center,bottom]{\scalebox{1}{\Huge\textcolor{green!50!black}{Objects}}}}}

\only<2->{\pgfputat{\pgfxy(0.25,0.6)}{\pgfbox[center,bottom]{\scalebox{1}{\large\textcolor{green!50!black}{\parbox{0.5\textwidth}{\begin{center}\uncover<3->{PL
              manifolds} \\ \uncover<4->{simplicial complexes}\end{center}}}}}}}

\only<5->{\pgfputat{\pgfxy(0.75,0.80)}{\pgfbox[center,bottom]{\scalebox{1}{\Huge\textcolor{red!50!black}{Maps}}}}}

\only<5->{\pgfputat{\pgfxy(0.75,0.6)}{\pgfbox[center,bottom]{\scalebox{1}{\large\textcolor{red!50!black}{\parbox{0.5\textwidth}{\begin{center}\uncover<6->{PL
              maps} \\ \uncover<7->{simplicial maps}\end{center}}}}}}}

\only<8->{\pgfputat{\pgfxy(0.5,0.35)}{\pgfbox[center,bottom]{\scalebox{1}{\Huge\textcolor{blue!50!black}{Invariants}}}}}

\only<9->{\pgfputat{\pgfxy(0.3333,0.25)}{\pgfbox[center,bottom]{\scalebox{1}{\Large\textcolor{blue!50!black}{$\chi$}}}}}

\only<10->{\pgfputat{\pgfxy(0.6666,0.25)}{\pgfbox[center,bottom]{\scalebox{1}{\Large\textcolor{blue!50!black}{$b_0$}}}}}

\end{pgfpicture}
\end{frame}
\clearbackgroundpicture

\setbackgroundpicture{Stair10.jpg}
\begin{frame}
  \frametitle{Next steps}
  \pause
  \begin{itemize}
    \item New constructions \pause\\
      \textcolor{green!50!black}{Cartesian product} \pause and \\
      \textcolor{green!50!black}{connected sum}\pause
    \item New invariant \pause\\
      \textcolor{green!50!black}{orientability, $w_1$}\pause
    \item New kind of equivalence \pause\\
      \textcolor{green!50!black}{simple homotopy equivalence}
 \end{itemize}

\end{frame}
\clearbackgroundpicture

% need products to get ambient isotopy
% also high dimensional manifolds
% real projective plane


\begin{frame}
  \vfill
  \begin{center}
    \scalebox{5}{\Huge $\#$} \\
   \vspace{2ex}\hspace{-2em}connected sum
  \end{center}
  \vfill
\end{frame}

%%%%%%%%%%%%%%%%%%%%%%%%%%%%%%%%%%%%%%%%%%%%%%%%%%%%%%%%%%%%%%%%
\begin{frame}
  \frametitle{Connected sum}

  \begin{definition}
  Suppose $K$ and $L$ are two $n$-manifolds. \\
  \vspace{1ex}%
  \begin{tabular}{@{}l@{ }l@{ }l@{ }l@{ }l}
  Let & $K'$ & be & $K$ & with one $n$-simplex removed. \\
   & $L'$ &  & $L$ &
  \end{tabular} \\
  \vspace{1ex}Define $K \# L = K' \cup_{\boundary \Delta^n} L'$.
\end{definition}

\vfill
\begin{columns}
\begin{column}{0.5\textwidth}
\uncover<2->{
\begin{problem}
 What is $T^2 \# T^2$?
\end{problem}}
\end{column}
\begin{column}{0.5\textwidth}
\uncover<3->{
\begin{problem}
 What is $S^2 \# T^2$?
\end{problem}}
\end{column}
\end{columns}

\end{frame}

\begin{frame}
\vfill
$$
 \includegraphics[width=0.25\textwidth]{torus.png} \raisebox{0.06\textwidth}{$\#$}
 \includegraphics[width=0.25\textwidth]{torus.png}  \raisebox{0.06\textwidth}{$\cong$}
 \hspace{1ex}\includegraphics[width=0.25\textwidth]{genus-two-surface.png}
$$
\vfill
\end{frame}

\begin{frame}
\vfill
$$
 \includegraphics[width=0.25\textwidth]{torus.png} \raisebox{0.06\textwidth}{$\#$}
 \includegraphics[width=0.25\textwidth]{genus-two-surface.png}  \raisebox{0.06\textwidth}{$\cong$}
 \uncover<2->{\includegraphics[width=0.25\textwidth]{genus-three-surface.png}}
$$
\vfill
\uncover<3->{In general, $\Sigma_g = \mbox{genus $g$ surface}$,}\\
\vspace{1ex}
\uncover<4->{and $T^2 \# \Sigma_g  = \Sigma_{g+1}$.}
\end{frame}

\begin{frame}
  \frametitle{Covering map}

  \begin{definition}
    $f : M \to N$ is an $n$-fold \textbf{covering map} \\
    if there is a subdivision $N'$ of $N$, \\
    so that for every vertex $v \in N'$ \\
    $f^{-1}(\cl \st(v,N)) \cong$ $n$ copies of $\cl \st(v, N)$, \\
    and $f$ is a homeo onto each copy of $\cl \st(v, N)$.
  \end{definition}

\end{frame}

\begin{frame}
  \frametitle{Gift-wrapping surfaces}

  \begin{problem}
    When is there an $n$-fold covering map \\
    from a genus $g$ surface \\
    to a genus $g'$ surface?
  \end{problem}
  \pause
  \begin{theorem}
    If $f : M \to N$ is an $n$-fold covering map \\
    then $\chi(M) = n \cdot \chi(N)$.
  \end{theorem}
  \pause
  \begin{problem}
    What is $\chi(\mbox{genus $g$ surface})?$
  \end{problem}

\end{frame}

%%%%%%%%%%%%%%%%%%%%%%%%%%%%%%%%%%%%%%%%%%%%%%%%%%%%%%%%%%%%%%%%
\begin{frame}
  \vfill
  \begin{center}
    \scalebox{5}{\Huge $\times$} \\
   \vspace{2ex}Cartesian product
  \end{center}
  \vfill
\end{frame}


\begin{frame}
  \frametitle{Products, geometrically}

\begin{definition}
Suppose $K \subset \R^n$, and $L \subset R^m$. \\
The product of $K$ and $L$ is
$$
K \times L = \{ (k,\ell) \in \R^{n+m} : k \in K, \ell \in L \}.
$$
\end{definition}

\pause
\vfill
How to do this for a simplicial complex?

\end{frame}

%%%%%%%%%%%%%%%%%%%%%%%%%%%%%%%%%%%%%%%%%%%%%%%%%%%%%%%%%%%%%%%%
\begin{frame}
 \frametitle{Product of simplexes}
 How do we take a product of $\Delta^1$ and $\Delta^1$?
 \vfill
\end{frame}

\begin{frame}
\frametitle{Product of simplexes}
How do we take a product of $\Delta^1$ and $\Delta^2$?
\vfill
\end{frame}

\begin{frame}
\frametitle{Product of simplexes}
How do we take a product of $\Delta^2$ and $\Delta^2$? \\\pause
Hard to see, so let's take a step back.
\vfill
\end{frame}

%%%%%%%%%%%%%%%%%%%%%%%%%%%%%%%%%%%%%%%%%%%%%%%%%%%%%%%%%%%%%%%%
\begin{frame}
\frametitle{Properties of products}

The product of $\Delta^n$ and $\Delta^m$ should have projection maps
\begin{align*}
p_1 &: \Delta^n \times \Delta^m \to \Delta^n \\
p_2 &: \Delta^n \times \Delta^m \to \Delta^m
\end{align*}
so that a point in $\Delta^n \times \Delta^m$ is determined by where
it lands under the two projection maps.

\vfill

\end{frame}

\begin{frame}

  Vertices of $\Delta^n \times \Delta^m$ are pairs \\
  $(i,j)$ with $0 \leq i \leq n+1$ and $0 \leq j \leq m+1$.

  \pause
  \vfill
  A simplex of $\Delta^n \times \Delta^m$ is \rule{48pt}{12pt}

  \vfill
  A homework problem!

\end{frame}

\begin{frame}
  \frametitle{Some counting problems}

  \vfill

  $\Delta^1 \times \Delta^n$ has $(n+1)!$ triangulations. \\

  \vfill

  $\Delta^2 \times \Delta^n$ has \rule{48pt}{12pt} triangulations.

  \vfill
  
\end{frame}

%0a 0b 1a 1b
%0a 0b 1a
%0b 1a 1b

% K times L,
% get CK times L
% 
% CK times L is a copy of L attached to K times L

\begin{frame}
  \frametitle{Some 4-manifolds}
  \Huge
  \vfill
  \begin{center}\pause
  $S^3 \times S^1$\\\pause
  $S^2 \times S^2$\\\pause
  $\Sigma_g \times \Sigma_h$
  \end{center}
  \vfill

\end{frame}

%%%%%%%%%%%%%%%%%%%%%%%%%%%%%%%%%%%%%%%%%%%%%%%%%%%%%%%%%%%%%%%%
\setbackgroundpicture{Templo_de_Afaia3.JPG}
\begin{frame}
  \vfill
  \begin{center}
    \scalebox{5}{\Huge $w_1$} \\
   \vspace{2ex}orientability
  \end{center}
  \vfill
\end{frame}
\clearbackgroundpicture

\begin{frame}
  \frametitle{The torus versus the Klein bottle}

Since $\chi(T^2) = \chi(K) = 0$, \\
we can't distinguish $T^2$ and $K$ using $\chi$.

\pause
\vfill

We need a new invariant.

\vfill

\end{frame}

\begin{frame}
  \frametitle{Orientation}

  An orientation of a simplex is \\
  a choice of ordering of the vertices,\\
  up to an even permutation.

  \vfill
  \pause

  \begin{definition}
    An orientation on a manifold $M^n$ \\
    is an orientation of each $n$-simplex \\
    so that neighboring $n$-simplexes \\
    induce opposite orientations on the shared $(n-1)$-simplex.
  \end{definition}
  \vfill
  \pause

  If $M$ admits an orientation, $w_1(M) = 0$. \\
  Otherwise, $w_1(M) = 1$.

\end{frame}

\begin{frame}
  \frametitle{Orientability for surfaces}

  \begin{theorem}
    A surface $\Sigma$ is not orientable \\
    iff $\Sigma$ contains a M\"obius strip.
  \end{theorem}

  \begin{problem}
   What is $w_1(T^2)$?
 \end{problem}

  \begin{problem}
    What is $w_1(K)$?
  \end{problem}

 \begin{problem}
   Is $w_1$ a PL homeomorphism invariant?
 \end{problem}
  
\end{frame}

\begin{frame}
  \frametitle{Distinguishing other manifolds?}

  Can we distinguish $T^3$ and $S^3$ with Euler characteristic?

  \vfill
  \pause

  No\pause\ldots

  \vfill

  We need more invariants, \\
  \pause
  or a looser notion of ``the same.''
  
\end{frame}

% a surface comes with a naturally defined map 
% into a space of $n$-vertex circles with two marked points

% vertices are 3, 4, 5, ... (different circles)
% edges are unions of circles along two points
% 


\setbackgroundpicture{collapsedbarn.jpg}
\begin{frame}
  \vfill
  \begin{center}
    \Huge Simplicial \\
    Collapse
  \end{center}
  \vfill
\end{frame}
\clearbackgroundpicture

\begin{frame}
  \frametitle{Principal Simplexes}

  \begin{definition}
  Let $K$ be a complex, and \\
  $\sigma \in K$ a simplex.  \\
  Call $\sigma$ a \textbf{principal simplex} \\
  if the only simplex containing $\sigma$ \\
  is $\sigma$ itself \\
  \quad (i.e., it isn't contained in a larger simplex).
  \end{definition} 

  \pause
  \begin{problem}
  Does every complex have a principal simplex?
  \end{problem}

\end{frame}

\begin{frame}
  \frametitle{Free faces}

  \begin{definition}
  Let $K$ be a complex, \\
  and $\sigma \in K$ a simplex, \\
  and $\tau < \sigma$ a face. \\
  \vspace{1ex}Call $\tau$ a \textbf{free face} of $\sigma$ \\
  if the only simplexes containing $\tau$ \\
  are $\tau$ and $\sigma$.
  \end{definition}

  \pause
  \begin{problem}
    \begin{tabular}{@{}l@{ }c@{ }l}
      Does & every & complex have a simplex with a free face? \\\pause
      Does & any & complex have a simplex with a free face? \\
   \end{tabular}
 \end{problem}

\end{frame}

\begin{frame}
\frametitle{Elementary simplicial collapse}

\begin{definition}
  Let $L$ and $K = L \cup \cl \{ \sigma, \tau \}$ be complexes \\

  \vspace{1ex}If $\sigma$ is a principal simplex of $K$, and \\
  $\tau$ is a free face of $\sigma$, then \\
  $L$ is an \textbf{elementary simplicial collapse} of $K$.
\end{definition}

\end{frame}

\begin{frame}
\frametitle{Simplicial collapse}

\begin{definition}
  Let $K_1, K_2, \ldots, K_n$ be complexes, with \\
  $K_{i+1}$ an elementary simplicial collapse of $K_i$.

  \vspace{1ex}Call $K_n$ a \textbf{simplicial collapse} of $K_1$, and \\
  write $K_1 \collapses K_n$.

  \vspace{1ex}Call $K_1$ a \textbf{simplicial expansion} of $K_n$, and \\
  write $K_n \expands K_1$.
\end{definition}

\end{frame}

\begin{frame}
  \frametitle{Simple homotopy equivalence}

  \begin{definition}
    $K$ is \textbf{simple homotopy equivalent} to $L$
    \quad (sometimes abbreviated s.h.e.) \\
    if you can reach transform $K$ into $L$ \\
    via a sequence of
    \begin{itemize}
    \item PL homeomorphisms,
    \item simplicial collapses,
    \item simplicial expansions.
    \end{itemize}
    In this case, we write $K \she L$.
  \end{definition}

\end{frame}

\setbackgroundpicture{zeeman.pdf}
\begin{frame}
  \frametitle{Next time}
  \pause
  \begin{itemize}
    \item Regular neighborhoods \\
    \item Knotting $S^1$ in $S^3$ \\
    \item Unknotting $S^1$ in $S^4$ \\
    \item How to unknot with simplicial collapse
\end{itemize}

\end{frame}
\clearbackgroundpicture

\end{document}


%%% Local Variables: 
%%% mode: latex
%%% TeX-master: t
%%% End: 
