\documentclass[14pt]{beamer}

\setbeamertemplate{navigation symbols}{}

\usepackage{amsmath}
\usepackage{amsthm}
%\newtheorem*{problem}{Problem}
\usepackage{amssymb}
\newcommand{\N}{\mathbb{N}}

\newcommand{\boundary}{\partial}
\newcommand{\collapses}{\searrow}
\newcommand{\expands}{\nearrow}
\newcommand{\she}{\ssearrow\nnearrow}
\newcommand{\join}{\ast}
\newcommand{\subdivided}{\triangleleft}

\DeclareMathOperator{\st}{st}
\DeclareMathOperator{\vertices}{vert}
\DeclareMathOperator{\lk}{lk}
\DeclareMathOperator{\cl}{cl}
\DeclareMathOperator{\interior}{int}

\title{Topology of  \\ Piecewise-Linear Manifolds}
\author{Jim Fowler}
\date{Lecture 2 \\ Summer 2010}

\begin{document}

\begin{frame}
\maketitle
\end{frame}

\begin{frame}
\frametitle{Goal}

\begin{center}
\scalebox{4}{Definitions?}
\end{center}

\end{frame}

%%%%%%%%%%%%%%%%%%%%%%%%%%%%%%%%%%%%%%%%%%%%%%%%%%%%%%%%%%%%%%%%
\begin{frame}
\begin{center}
\vspace{-1ex}
\includegraphics[height=\textheight]{Stereographic_polytope_8cell.png}
\end{center}
\end{frame}

%%%%%%%%%%%%%%%%%%%%%%%%%%%%%%%%%%%%%%%%%%%%%%%%%%%%%%%%%%%%%%%%
\begin{frame}
\begin{center}
\includegraphics[height=\textheight]{Stereographic_polytope_5cell.png}
\end{center}
\end{frame}

%%%%%%%%%%%%%%%%%%%%%%%%%%%%%%%%%%%%%%%%%%%%%%%%%%%%%%%%%%%%%%%%
\begin{frame}
  A \textbf{simplicial complex} $K$ is \\
  a collection of finite sets (called the ``simplexes''), \\
  with the property that \\
  \begin{center}
  if $\sigma \in K$, and $\tau \subset \sigma$, \quad then $\tau \in K$.
  \end{center}

  \vfill
  \pause
  
  This definition is pure combinatorics, \\
  but we will think of this as a geometric object.

\end{frame}

%%%%%%%%%%%%%%%%%%%%%%%%%%%%%%%%%%%%%%%%%%%%%%%%%%%%%%%%%%%%%%%%
\begin{frame}
\frametitle{Examples}

\begin{align*}
\vspace{1ex}\uncover<2->{I} &= \left\{ \varnothing, \{0\}, \{1\}, \{0,1\} \right\} \\
\vspace{1ex}\uncover<4->{S^0} &\uncover<3->{= \left\{ \varnothing, \{0\}, \{1\} \right\}} \\
\vspace{1ex}\uncover<6->{S^1} &\uncover<5->{= \left\{ \varnothing, \{a\}, \{b\}, \{c\}, \{a,b\},\{b,c\},\{a,c\} \right\}} \\
\vspace{1ex}\uncover<8->{V} &\uncover<7->{= \left\{ \varnothing, \{a\}, \{b\}, \{c\}, \{a,b\},\{b,c\} \right\}}
\end{align*}
\uncover<9->{Shouldn't $V \cong I$?}
\end{frame}

%%%%%%%%%%%%%%%%%%%%%%%%%%%%%%%%%%%%%%%%%%%%%%%%%%%%%%%%%%%%%%%%
\begin{frame}
\frametitle{Non-examples}

\pause
$$ \left\{ \varnothing, \{0\}, \{0,1\} \right\} $$

\pause
$$ \left\{ \{0\}, \{1\}, \{0,1\} \right\} $$

\end{frame}

%%%%%%%%%%%%%%%%%%%%%%%%%%%%%%%%%%%%%%%%%%%%%%%%%%%%%%%%%%%%%%%%
\begin{frame}
\frametitle{Examples}

The $n$-simplex $\Delta^n$ is a complex: label the $n+1$ vertices of $\Delta^n$ using the set $V = \{ 0, 1, 2, \ldots, n \}$, then the simplexes of $\Delta^n$ are all $2^{n+1}$ subsets of $V$.

\vfill

The $n$-sphere $S^n$ consists of all simplexes in $\Delta^n$, except for the top dimensional simplex $\{0,1,\ldots,n\}$.

\vfill
\pause

\begin{problem}
Calculate $\chi(S^n)$.
\end{problem}

\vfill

\end{frame}

\begin{frame}
  \frametitle{Simplicial Maps}
  A \textbf{simplicial map} $f : K \to L$ is a function $f : \vertices(K) \to \vertices(L)$ so that, \\
  whenever $\sigma \in K$, then $f(\sigma) \in L$.

  \vspace{1ex}
  \pause
  This can crush simplexes \pause (e.g., a trivial example?)
  \pause

  \begin{problem}
    Find a simplicial map $f : T^2 \to S^2$ which doesn't crush any edges (i.e., edges are sent to edges, not to vertices).
  \end{problem}
  \pause
  Think ``4-color the vertices of $T^2$.''
\end{frame}

\begin{frame}
\frametitle{Conversely\ldots}

\begin{problem}
  Find a simplicial map $f : S^2 \to T^2$ which doesn't crush any edges.
\end{problem}
\pause
Think ``Skewer and fold.''

\end{frame}

\begin{frame}
\begin{problem}
 Do there exist simplicial maps $f : T^2 \to S^2$ and $g : S^2 \to T^2$ which are inverses of each other?
\end{problem}

\pause

No.  \textbf{And why not?}

\pause
\begin{itemize}
\item Euler characteristic.
\pause\item Separation via curves.
\end{itemize}

\end{frame}

%%%%%%%%%%%%%%%%%%%%%%%%%%%%%%%%%%%%%%%%%%%%%%%%%%%%%%%%%%%%%%%%
\begin{frame}
\begin{problem}
  Suppose $f : S^1 \to S^2$ is an injective simplicial map (i.e.,
  distinct vertices are sent to distinct vertices).  Does the image of
  $f$ necessarily separate $S^2$ into two pieces?
\end{problem}
\pause
\textbf{Yes.}\\
\pause
\textit{Proof}
\end{frame}


%%%%%%%%%%%%%%%%%%%%%%%%%%%%%%%%%%%%%%%%%%%%%%%%%%%%%%%%%%%%%%%%
\begin{frame}
A problem with our definitions: we want to talk about $S^2$, but there are so many simplicial complexes which deserve to be called $S^2$.
\vfill
Our notion of simplicial complex is too rigid to be the right notion topologically.
\vfill
\begin{example}
  Let $K$ be a circle with three arcs \\
  and $L$ be a circle with four arcs \\
  \quad(i.e., the boundary of a square). \\
  Then $K$ and $L$ are not simplicially isomorphic.
\end{example}

\end{frame}

\begin{frame}
\frametitle{Fixing the theory}

Need to define a few things first\ldots
\begin{itemize}
\item star
\item closure
\item link
\item subdivision
\end{itemize}

\end{frame}

\begin{frame}
\frametitle{Star}

\begin{definition}
Let $K$ be a complex, and $\sigma \in K$ a simplex.\\
The \textbf{star} of $\sigma$ in $K$, \\
\quad written $\st(\sigma,K)$, \\
is defined by
$$
\st(\sigma,K) = \{ \tau \in K : \sigma < \tau \},
$$
i.e., the star of $\sigma$ includes all the simplexes having $\sigma$
as a face.
\end{definition}
\pause
\begin{problem}
Is the star of a simplex a complex?
\end{problem}
\end{frame}

\begin{frame}
\frametitle{Closure}
\begin{definition}
Let $S$ be a collection of simplexes in $K$.  \\
The \textbf{closure} of $S$, \\
\quad written as $\cl(S)$, \\
is the smallest subcomplex of $K$ \\
containing the simplexes in $S$.
\end{definition}
\pause
\begin{problem}
Relate $\cl(\cl(S))$ and $\cl(S)$.
\end{problem}
\end{frame}

\begin{frame}
\frametitle{Link}

\begin{definition}
Let $K$ be a complex, and $\sigma \in K$ a simplex.  \\
The \textbf{link} of $\sigma \in K$, \\
\quad written $\lk(\sigma,K)$, \\
consists of those simplexes in $K$ which are in $\cl(\st(\sigma,K))$
but not touching $\sigma$; in other words,
$$
\lk(\sigma,K) = \{ \tau \in \cl(\st(\sigma,K)) : \tau \cap \sigma =
\varnothing \}.
$$
\end{definition}
\pause
\begin{problem}
Is the link of a simplex a complex?
\end{problem}
\end{frame}

\begin{frame}
\begin{definition}
  Let $K$ be a complex, and $\sigma \in K$ a simplex.  \\
  The \textbf{stellar subdivision} of $K$ at $\sigma$ is a new complex
  $K_\sigma$ with:
\begin{itemize}
\item the vertices of $K$ along with a brand new vertex $v$.
\item the simplexes of $K$ not in $\st(\sigma,K)$, along with the
  simplexes in $v \join (\boundary \sigma) \join \lk(\sigma,K)$.
\end{itemize}
We might say:
$$
K_\sigma := (K - \st(\sigma,K)) \cup (v \join (\boundary \sigma) \join
\lk(\sigma,K))
$$
\end{definition}
\pause
\begin{problem}
  Stellar subdivision of a complex is a complex?
\end{problem}
\end{frame}

\begin{frame}
\frametitle{subdivision = repeated stellar subdivision}

\begin{definition}
  Let $K, L$ be complexes.  \\
  If $K$ can be produced through a (possibly
  empty) sequence of stellar subdivisions of $L$, \\
  we say that $K$ is a   \textbf{subdivision} of $L$, \\
  and write $K \subdivided L$.
\end{definition}

\uncover<2>{\alert<2>{As we will see, the \textit{real} definition of
  subdivision is more general than this.}}
\end{frame}


\begin{frame}
\frametitle{Piecewise linear maps}

\begin{definition}
  Let $K, K', L, L'$ be complexes, with $K' \subdivided K$ and $L'
  \subdivided L$.  \\
  If $f : K' \to L'$ is a simplicial map, we call $f
  : K \to L$ a \textbf{piecewise linear map} (or a \textbf{PL map}
  for short).  We call $f : K' \to L'$ an \textbf{underlying
    simplicial map}.
\end{definition}

\uncover<2>{\alert<2>{We write $f : K \to L$ for a PL map, but such a map
does not send simplexes in $K$ to simplexes in $L$}}

\uncover<3>{\alert<3>{The \textit{real} definition of PL map is more
   general than this.}}

\end{frame}

% \begin{frame}
% \frametitle{Homeomorphism}
% \begin{definition}
%   A \textbf{piecewise linear homeomorphism} \\
%   \quad (a \textbf{PL homeomorphism} for short)\\
%   is a PL map $f : K \to L$ for which the underlying
%   simplicial map $f : K' \to L'$ is a simplicial isomorphism.
% \end{definition}

% If there exists a homeomorphism between $A$ and $B$, then we say that
% $A$ and $B$ are \textbf{homeomorphic}.  We write $A \cong B$ if $A$
% and $B$ are homeomorphic.
% \pause
% \begin{problem}
% Is this the same as having a PL map with a PL inverse?
% \end{problem}

% \end{frame}

\begin{frame}
\frametitle{Going back, rethinking everything\ldots}

\begin{center}
\scalebox{2}{Are $S^2$ and $T^2$ the same?}
\end{center}

\pause
We will check that $\chi$ is well-defined.
\end{frame}


\end{document}


%%% Local Variables: 
%%% mode: latex
%%% TeX-master: t
%%% End: 
