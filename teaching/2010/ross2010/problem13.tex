\documentclass[12pt]{pset}
%\usepackage{add-copyright}

\geometry{margin=0.5in}
\title{Problem Set 13}
\course{Piecewise-Linear Topology}
\author{Jim Fowler}
\date{Summer 2010}

\usepackage{tikz}
\usetikzlibrary{matrix}

\newcommand{\defnword}[1]{\textbf{#1}}

\usepackage{stmaryrd}
\newcommand{\boundary}{\partial}
\newcommand{\collapses}{\searrow}
\newcommand{\expands}{\nearrow}
\newcommand{\she}{\ssearrow\nnearrow}
\newcommand{\join}{\ast}
\newcommand{\subdivided}{\triangleleft}
\geometry{margin=1in}%,top=0.5in,bottom=0.5in}
\DeclareMathOperator{\st}{st}
\DeclareMathOperator{\trace}{trace}
\DeclareMathOperator{\vertices}{vert}
\DeclareMathOperator{\lk}{lk}
\DeclareMathOperator{\cl}{cl}
\DeclareMathOperator{\interior}{int}
\DeclareMathOperator{\id}{id}
\newcommand{\fullsubcomplex}{\Subset}

\usepackage{hyperref}

\begin{document}
\maketitle

% [[skim:///Users/jim/Papers/MR755006.pdf?page=2][page 2 of MR755006.pdf]]

\subsubsection*{More examples of homology.}  Today, we consider what
sorts of homomorphisms between homology groups are induced by PL maps.
Additionally, we see that homology can obstruct ``social choice.''

%Why does the sequence split?  last time we used the fact that $X
%\times I$ is the same as $X$ to compute that $H_n(X \times S^1) =
%H_n(X) \oplus H_{n-1}(X)$.

%\parskip 0.45\baselineskip
%\parindent 0pt

\begin{requiredproblem}
  Find a map $f : T^2 \to T^2$ so that $f^n \not\simeq f^m$ for $n \neq m$.  Here, $f^n$ denotes the $n^{\mbox{\scriptsize th}}$ iterate of $f$ with itself, i.e.,
$$f^n(x) = (\overbrace{f \circ \cdots \circ f}^{\mbox{$n$ times}})(x)$$
\end{requiredproblem}

\begin{problem}
  Given $N$, can you find a map $f : T^2 \to T^2$ so that $f^n \not\simeq f^m$ unless $n \equiv m \pmod N$?
\end{problem}

\begin{problem}
  Any map $f : T^2 \to T^2$ induces a map on $H_1(T^2) = \Z^2$ and
  $H_2(T^2) = \Z$.  The former can be represented as a $2$-by-$2$
  matrix with integer entries, and the latter can be represented by a
  single integer.  Which combinations of matrices and integers can be
  realized by maps $f : T^2 \to T^2$? \vspace{1ex} \\
  More concretely, is it possible for a map $f : T^2 \to T^2$ to induce an isomorphism on $H_1$ but the zero map on $H_2$? 

\end{problem}

\begin{problem}
  Any map $f : T^2 \# T^2 \to T^2 \# T^2$ induces a map on $H_1(T^2 \# T^2) = \Z^4$, which can be represented by a $4$-by-$4$ matrix.  Can all $4$-by-$4$ matrices with integer entries arise in this fashion?
\end{problem}

\begin{problem}
  Suppose $f : T^2 \# T^2 \to T^2 \# T^2$ induces an isomorphism on homology; is it possible that $f \not\simeq \id$?
\end{problem}

\subsection*{Fixed points}

\begin{problem}
  Let $\Sigma_g$ be a surface of genus~$g$.  Is there a fixed point free map $f : \Sigma_g \to \Sigma_g$ with $f^2 = \id$?
\end{problem}

\begin{requiredproblem}
  Let $\Sigma_g$ be a surface of genus~$g$, and let $f : \Sigma_g \to \Sigma_g$ be the map which cycles the $g$~handles, so that $f^g = \id$.  Does any map  homotopic to $f$ have a fixed point?
\end{requiredproblem}

\pagebreak
\subsection*{Fundamental class}

\begin{requiredproblem}
  On Problem Set 4, we discussed \textbf{orientations} for
  $n$-manifolds.  Show that an oriented $n$-manifold $M$ has a
  well-defined and non-zero homology class $[M] \in H_n(M)$.  The homology
  class will depend on the orientation, and is called the \textbf{fundamental class}.
\end{requiredproblem}

\begin{definition*}
  Given oriented $n$-manifolds $M$ and $N$, the \textbf{degree} of a map $f : M \to N$ is the integer $\deg f$ so that
  $$
  f_\star [M] = (\deg f) [N].
  $$
\end{definition*}

\begin{problem}
  For any $n \in \Z$, produce a map $f : S^n \to S^n$ so that $\deg f = n$.
\end{problem}

\begin{problem}
  For any oriented $n$-manifold $M$, produce a map $f : M \to S^n$ of degree $1$.
\end{problem}

\begin{problem}
  Given an oriented $n$-manifold $M$, is there necessarily a map $f : S^n \to M$ of degree $1$?
\end{problem}

\subsection*{Topological social choice}

Recall that a \textbf{two-player choice function} for a space $X$ is a function $f : X \times X \to X$ with the properties that
\begin{itemize}
\item $f(a,b) = f(b,a)$, i.e., the decision should be impartial,
\item $f(x,x) = x$, i.e., the decision should respect consensus, and
\item $f$ is continuous.
\end{itemize}
\begin{problem}
  During lecture, we used homology to show that there is no two-player choice function for $S^1$; come up with a definition of an $n$-player choice function, and show that $S^1$ does not admit an $n$-player choice function (unless $n = 1$).
\end{problem}

\begin{requiredproblem}
  Show that the interval $I$ admits two-player choice function.
\end{requiredproblem}

\begin{problem}
  Does the two-sphere $S^2$ admit a two-player choice function?
\end{problem}

\end{document}

