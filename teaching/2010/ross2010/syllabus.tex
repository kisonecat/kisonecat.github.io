\documentclass[12pt]{handout}
\usepackage{mathpazo}
\usepackage[no-math]{fontspec}
\defaultfontfeatures{Mapping=tex-text}

\usepackage{xunicode}% provides unicode character macros
 \usepackage{xltxtra} % provides some fixes/extras
\usepackage{xltxtra,xunicode}
\setmainfont{Palatino}
\linespread{1.05}      % Palatino needs more space between lines

%\setsansfont{Gill Sans}
\setmonofont[Scale=0.8]{Monaco}

\newcommand{\peem}{\textsc{p.m.}}
\newcommand{\ayem}{\textsc{a.m.}}

\geometry{margin=0.7in}
\usepackage{nopageno}

\usepackage{hyperref}
\usepackage{enumerate}
\usepackage{multicol}

\title{Syllabus}
\course{Piecewise-Linear Topology}
\author{Jim Fowler}
\date{Summer 2010}

\begin{document}
\maketitle

\noindent
Manifolds are spaces which are locally modeled on Euclidean space, but
might be globally twisted in some way; two-dimensional examples
include a sphere or a torus.  In contrast to the usual introduction to
manifolds based on calculus and charts (that is, smooth manifolds),
this course will study manifolds as combinatorial objects (that is,
piecewise-linear manifolds).  Piecewise-linear manifolds are more
general than smooth manifolds, and because the basic definitions
involve combinatorics instead of calculus, we will find it easier to
give rigorous proofs.

\subsection*{Homework}
Problem sets will be distributed during most lectures.

\subsection*{Website}
The course website is \url{http://www.math.osu.edu/~fowler/teaching/ross2010/}

\subsection*{Lectures}
We meet Mondays, Wednesdays, and Fridays,
1:30\peem--2:30\peem\ in CH240.

\subsection*{Instructor}
\vspace{1ex}%
\noindent\parbox{0.5\textwidth}{%
\noindent\begin{tabular}{@{}llll}
\textsf{Name:} & Jim Fowler & \textsf{Email:} & \href{mailto:fowler@math.osu.edu}{\texttt{fowler@math.osu.edu}} \\
\textsf{Office:} & MW658 Mathematics Tower & \textsf{Website:} & \url{http://www.math.osu.edu/~fowler/} \\
\textsf{Phone:} & (773) 809--5659 & & \\
\end{tabular}}

%%%%%%%%%%%%%%%%%%%%%%%%%%%%%%%%%%%%%%%%%%%%%%%%%%%%%%%%%%%%%%%%
%\pagebreak
\subsection*{Tentative Schedule}

This is an ambitious schedule, to say the least.  Depending on your
interest, we can spend more or less time on certain topics.  Let me know.
%\vfill

\begin{multicols}{2}
\begin{enumerate}[\bfseries Week 1]
\item Simplicial complexes and \\ piecewise-linear manifolds
\item Regular neighborhoods and \\ simplicial collapse
\item Sunny collapse and unknotting
\item Simplicial homology
\item Poincar\'e duality
\item General position
\item Embeddings
\item Handle theory
\end{enumerate}
\end{multicols}



\end{document}
