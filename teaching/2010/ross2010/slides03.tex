\documentclass[14pt]{beamer}

\setbeamertemplate{navigation symbols}{}

\usepackage{pgf}
\usepackage{amsmath}
\usepackage{amsthm}
%\newtheorem*{problem}{Problem}
\usepackage{amssymb}
\newcommand{\N}{\mathbb{N}}

\newcommand{\boundary}{\partial}
\newcommand{\collapses}{\searrow}
\newcommand{\expands}{\nearrow}
\newcommand{\she}{\ssearrow\nnearrow}
\newcommand{\join}{\ast}
\newcommand{\subdivided}{\triangleleft}

\DeclareMathOperator{\st}{st}
\DeclareMathOperator{\vertices}{vert}
\DeclareMathOperator{\lk}{lk}
\DeclareMathOperator{\cl}{cl}
\DeclareMathOperator{\interior}{int}

\title{Topology of  \\ Piecewise-Linear Manifolds}
\author{Jim Fowler}
\date{Lecture 3 \\ Summer 2010}

\newcommand{\setbackgroundpicture}[1]{%
\usebackgroundtemplate{
\begin{pgfpicture}{0in}{0in}{\paperwidth}{\paperheight}
\pgfputat{\pgfxy(0,0)}{\includegraphics[keepaspectratio=true,width=\paperwidth,height=\paperheight]{#1}}
\color{white}
\pgfsetfillopacity{0.8}
\pgfrect[fill]{\pgfxy(0,0)}{\pgfpoint{\paperwidth}{\paperheight}}
\end{pgfpicture}
}
}
\newcommand{\clearbackgroundpicture}{\usebackgroundtemplate{}}

\begin{document}

\begin{frame}
\maketitle
\end{frame}

\setbackgroundpicture{apples-and-pears.jpg}
%%%%%%%%%%%%%%%%%%%%%%%%%%%%%%%%%%%%%%%%%%%%%%%%%%%%%%%%%%%%%%%%
\begin{frame}
\frametitle{A simplicial complex joke}

Define the complex \textbf{Food} so that
\begin{itemize}
\item vertices are edible objects
\item objects $k_1, \ldots, k_n$ comprise a simplex if they taste good together.
\end{itemize}

\pause
\vfill 

Inside \textbf{Food}, find $\partial \Delta^n$ \\
which doesn't extend to $\Delta^n$.\\

\pause
\vfill

Three foods, \\
\quad any two of which taste good together, \\
\quad but the three aren't tasty altogether.

\end{frame}
\clearbackgroundpicture{}

\setbackgroundpicture{hundred-dollar-bill.jpg}
%%%%%%%%%%%%%%%%%%%%%%%%%%%%%%%%%%%%%%%%%%%%%%%%%%%%%%%%%%%%%%%%
\begin{frame}
\frametitle{A simplicial complex which isn't funny}

Define the complex \textbf{Market} so that
\begin{itemize}
\item vertices are securities \\
  (e.g., stocks, bonds, currencies)
\item objects $k_1, \ldots, k_n$ comprise a simplex \\
  if they can be traded for each other
\end{itemize}

\vfill 

Move your money through the vertices. \\
Come back to where you started with more! \\

\end{frame}
\clearbackgroundpicture


\begin{frame}
\frametitle{The torus}
\begin{center}
\raisebox{0.4\textheight}{$T^2 = $}
\begin{pgfpicture}{0in}{0in}{0.8\textheight}{0.8\textheight}
  \pgfsetxvec{\pgfpoint{0.8\textheight}{0cm}}
  \pgfsetyvec{\pgfpoint{0cm}{0.8\textheight}}

  \pgfsetlinewidth{3pt}

  \color[rgb]{0.5,0.85,0.5}
  \pgfrect[fill]{\pgfxy(0.1,0.1)}{\pgfxy(0.8,0.8)}
  \color{black}
  \pgfrect[stroke]{\pgfxy(0.1,0.1)}{\pgfxy(0.8,0.8)}

  \pgfsetlinewidth{3pt}

  \pgfsetendarrow{\pgfarrowto}
  \pgfmoveto{\pgfxy(0.1,0.45)}
  \pgflineto{\pgfxy(0.1,0.50)}
  \pgfstroke

  \pgfsetendarrow{\pgfarrowto}
  \pgfmoveto{\pgfxy(0.9,0.45)}
  \pgflineto{\pgfxy(0.9,0.50)}
  \pgfstroke

  \pgfsetendarrow{\pgfarrowto}
  \pgfmoveto{\pgfxy(0.45,0.1)}
  \pgflineto{\pgfxy(0.50,0.1)}
  \pgfstroke
  \pgfsetendarrow{\pgfarrowto}
  \pgfmoveto{\pgfxy(0.50,0.1)}
  \pgflineto{\pgfxy(0.54,0.1)}
  \pgfstroke

  \pgfsetendarrow{\pgfarrowto}
  \pgfmoveto{\pgfxy(0.45,0.9)}
  \pgflineto{\pgfxy(0.50,0.9)}
  \pgfstroke
  \pgfsetendarrow{\pgfarrowto}
  \pgfmoveto{\pgfxy(0.50,0.9)}
  \pgflineto{\pgfxy(0.54,0.9)}
  \pgfstroke
\end{pgfpicture}
\end{center}
\end{frame}

\begin{frame}
\frametitle{Klein Bottle}
\begin{center}
\raisebox{0.4\textheight}{$K = $}
\begin{pgfpicture}{0in}{0in}{0.8\textheight}{0.8\textheight}
  \pgfsetxvec{\pgfpoint{0.8\textheight}{0cm}}
  \pgfsetyvec{\pgfpoint{0cm}{0.8\textheight}}

  \pgfsetlinewidth{3pt}

  \color[rgb]{0.85,0.5,0.5}
  \pgfrect[fill]{\pgfxy(0.1,0.1)}{\pgfxy(0.8,0.8)}
  \color{black}
  \pgfrect[stroke]{\pgfxy(0.1,0.1)}{\pgfxy(0.8,0.8)}

  \pgfsetlinewidth{3pt}

  \pgfsetendarrow{\pgfarrowto}
  \pgfmoveto{\pgfxy(0.1,0.50)}
  \pgflineto{\pgfxy(0.1,0.45)}
  \pgfstroke

  \pgfsetendarrow{\pgfarrowto}
  \pgfmoveto{\pgfxy(0.9,0.45)}
  \pgflineto{\pgfxy(0.9,0.50)}
  \pgfstroke

  \pgfsetendarrow{\pgfarrowto}
  \pgfmoveto{\pgfxy(0.45,0.1)}
  \pgflineto{\pgfxy(0.50,0.1)}
  \pgfstroke
  \pgfsetendarrow{\pgfarrowto}
  \pgfmoveto{\pgfxy(0.50,0.1)}
  \pgflineto{\pgfxy(0.54,0.1)}
  \pgfstroke

  \pgfsetendarrow{\pgfarrowto}
  \pgfmoveto{\pgfxy(0.45,0.9)}
  \pgflineto{\pgfxy(0.50,0.9)}
  \pgfstroke
  \pgfsetendarrow{\pgfarrowto}
  \pgfmoveto{\pgfxy(0.50,0.9)}
  \pgflineto{\pgfxy(0.54,0.9)}
  \pgfstroke
\end{pgfpicture}
\end{center}
\end{frame}

\begin{frame}
\frametitle{Klein Bottle}
\vfill
\begin{center}
\includegraphics[height=0.8\textheight]{klein-bottle.pdf}
\end{center}
\vfill
\end{frame}


\usebackgroundtemplate{%
\begin{pgfpicture}{0in}{0in}{\paperwidth}{\paperheight}
\pgfputat{\pgfxy(0,0)}{\includegraphics[,width=\paperwidth,height=\paperheight]{sampleMerry_0043_Lasalle.jpg}}
\end{pgfpicture}}
\begin{frame}
\end{frame}
\usebackgroundtemplate{%
\begin{pgfpicture}{0in}{0in}{\paperwidth}{\paperheight}
\pgfputat{\pgfxy(0,0)}{\includegraphics[,width=\paperwidth,height=\paperheight]{samplepippin0053.jpg}}
\end{pgfpicture}}
\begin{frame}
\end{frame}
\usebackgroundtemplate{%
\begin{pgfpicture}{0in}{0in}{\paperwidth}{\paperheight}
\pgfputat{\pgfxy(0,0)}{\includegraphics[,width=\paperwidth,height=\paperheight]{samplepippin0053-gray.jpg}}
\end{pgfpicture}}
\begin{frame}
\end{frame}
\usebackgroundtemplate{%
\begin{pgfpicture}{0in}{0in}{\paperwidth}{\paperheight}
\pgfputat{\pgfxy(0,0)}{\includegraphics[,width=\paperwidth,height=\paperheight]{samplepippin0053-zoomed.jpg}}
\end{pgfpicture}}
\begin{frame}
\huge
\pause
\vfill
\textcolor{white}{Look at $3 \times 3$ pixel subsets}
\vfill
\pause
\textcolor{white}{Get points in $\mathbf{R}^9$}
\vfill
\end{frame}
\usebackgroundtemplate{}

\begin{frame}
  \includegraphics[width=\textwidth]{space-of-natural-images.pdf} \\
  \hfill\scriptsize {From a paper of Gunnar Carlsson and Tigran Ishkhanov}
\end{frame}

\begin{frame}
\frametitle{I'm a cheerleader for geometry!}

\setbeamersize{description width=0.25\textwidth}
\begin{description}
\item[Dynamics] and mixing taffy
\item[Biology] and yeast
\item[Chemistry] and isomers
\item[Physics] and symmetry
\item[Neurology] and Klein bottles
\item[Economics] and ``least action''
\item[Engineering] and robots
\item[Astronomy] and the shape of space
\item[Linguistics] and document clustering
\end{description}
\end{frame}

\setbackgroundpicture{torus.png}
%%%%%%%%%%%%%%%%%%%%%%%%%%%%%%%%%%%%%%%%%%%%%%%%%%%%%%%%%%%%%%%%
\begin{frame}
\frametitle{Triangulate a torus}
\begin{center}
\includegraphics[height=0.8\textheight]{k7-on-torus/torus.pdf}
\end{center}
\end{frame}
\clearbackgroundpicture

%%%%%%%%%%%%%%%%%%%%%%%%%%%%%%%%%%%%%%%%%%%%%%%%%%%%%%%%%%%%%%%%

\setbackgroundpicture{Simplicial_complex_example.pdf}
\begin{frame}
\frametitle{Simplicial Complex}
\vfill
  A \textbf{simplicial complex} $K$ is \\
  a collection of finite sets (called the ``simplexes''), \\
  with the property that \\
  \begin{center}
  if $\sigma \in K$, and $\tau \subset \sigma$, \quad then $\tau \in K$.
  \end{center}
\vfill
\end{frame}
\clearbackgroundpicture{}

\setbackgroundpicture{star.pdf}
\begin{frame}
\frametitle{Star}

\begin{definition}
Let $K$ be a complex, and $\sigma \in K$ a simplex.\\
The \textbf{star} of $\sigma$ in $K$, \\
\quad written $\st(\sigma,K)$, \\
is defined by
$$
\st(\sigma,K) = \{ \tau \in K : \sigma < \tau \},
$$
i.e., the star of $\sigma$ includes all the simplexes having $\sigma$
as a face.
\end{definition}
\end{frame}
\clearbackgroundpicture

\setbackgroundpicture{closed-star.pdf}
\begin{frame}
\frametitle{Closure}
\begin{definition}
Let $S$ be a collection of simplexes in $K$.  \\
The \textbf{closure} of $S$, \\
\quad written as $\cl(S)$, \\
is the smallest subcomplex of $K$ \\
containing the simplexes in $S$.
\end{definition}
\end{frame}
\clearbackgroundpicture

\setbackgroundpicture{link.pdf}
\begin{frame}
\frametitle{Link}

\begin{definition}
Let $K$ be a complex, and $\sigma \in K$ a simplex.  \\
The \textbf{link} of $\sigma \in K$, \\
\quad written $\lk(\sigma,K)$, \\
consists of those simplexes in $K$ which are in $\cl(\st(\sigma,K))$
but not touching $\sigma$; in other words,
$$
\lk(\sigma,K) = \{ \tau \in \cl(\st(\sigma,K)) : \tau \cap \sigma =
\varnothing \}.
$$
\end{definition}
\end{frame}
\clearbackgroundpicture

\begin{frame}
\frametitle{Stellar Subdivision} 

\begin{definition}
  Let $K$ be a complex, and $\sigma \in K$ a simplex.  \\
  The \textbf{stellar subdivision} of $K$ at $\sigma$ is a new complex
  $K_\sigma$ with:
\begin{itemize}
\item the vertices of $K$ with a new vertex $v$.
\item the simplexes of $K$ not in $\st(\sigma,K)$,\\
  along with the simplexes in $v \join (\boundary \sigma) \join \lk(\sigma,K)$.
\end{itemize}
We might say:
$$
K_\sigma := (K - \st(\sigma,K)) \cup (v \join (\boundary \sigma) \join
\lk(\sigma,K))
$$
\end{definition}
\end{frame}

\begin{frame}
\frametitle{subdivision = repeated stellar subdivision}

\begin{definition}
  Let $K, L$ be complexes.  \\
  If $K$ can be produced through a (possibly
  empty) sequence of stellar subdivisions of $L$, \\
  we say that $K$ is a   \textbf{subdivision} of $L$, \\
  and write $K \subdivided L$.
\end{definition}
\end{frame}


\begin{frame}
\frametitle{Piecewise linear maps}

\begin{definition}
  Let $K, K', L, L'$ be complexes, \\
  \quad with $K' \subdivided K$ and $L' \subdivided L$.  \\
  \vspace{1ex}
  If $f : K' \to L'$ is a simplicial map, \\
  we call $f : K \to L$ a \textbf{piecewise linear map} \\
  \quad (or a \textbf{PL map} for short). \\
  \vspace{1ex}
  We call $f : K' \to L'$ an \textbf{underlying simplicial map}.
  
  \uncover<2>{\alert<2>{As we will see, the \textit{real} definition of
      subdivision is more general than this.}}
\end{definition}

\end{frame}

%%%%%%%%%%%%%%%%%%%%%%%%%%%%%%%%%%%%%%%%%%%%%%%%%%%%%%%%%%%%%%%%
%%%%%%%%%%%%%%%%%%%%%%%%%%%%%%%%%%%%%%%%%%%%%%%%%%%%%%%%%%%%%%%%
%%%%%%%%%%%%%%%%%%%%%%%%%%%%%%%%%%%%%%%%%%%%%%%%%%%%%%%%%%%%%%%%
%%%%%%%%%%%%%%%%%%%%%%%%%%%%%%%%%%%%%%%%%%%%%%%%%%%%%%%%%%%%%%%%

\setbackgroundpicture{problem02.pdf}
\begin{frame}
\vfill
\begin{center}
from the last \\
\scalebox{4}{homework}
\end{center}
\vfill
\end{frame}
\clearbackgroundpicture

\begin{frame}
  \frametitle{Joins}

  \begin{definition}
Let $K$ and $L$ be complexes\\
with disjoint sets of vertices\\
\quad (we call such complexes \textbf{joinable}).\\
Define a new complex, the \textbf{join} of $K$ and $L$, by 
$$
K \join L := \{ \sigma \cup \tau : \sigma \in K, \tau \in L \}.
$$
\end{definition}
\pause

\begin{problem}
What is $S^0 \join S^0$?
\end{problem}

\pause
\begin{problem}
What is $S^n \join S^m$?
\end{problem}

\end{frame}

\setbackgroundpicture{problem03.pdf}
\begin{frame}
\vfill
\begin{center}
on today's \\
\scalebox{4}{homework} \\
\includegraphics[width=0.5\textwidth]{genus-two-surface-with-curves.pdf}
\end{center}
\vfill
\end{frame}
\clearbackgroundpicture

\begin{frame}
\frametitle{PL Homeomorphism}
\begin{definition}
  A \textbf{piecewise linear homeomorphism} \\
  \quad (a \textbf{PL homeomorphism} for short)\\
  is a PL map $f : K \to L$ \\
  with a PL inverse.
\end{definition}

\pause
If there exists a homeomorphism between $A$ and $B$, \\
then $A$ and $B$ are \textbf{homeomorphic}.  \\
Write $A \cong B$ if $A$ and $B$ are homeomorphic.
% \pause
% \begin{problem}
% Is this the same as having a PL map $f$ so that an underlying
% simplicial map is an isomorphism?
% \end{problem}


\end{frame}

\begin{frame}
  \frametitle{PL Manifold}
  
    A complex $M$ is an $n$-dimensional \textbf{PL manifold}\\
    \quad(for short, an $n$-manifold)\\
    if for every vertex $v$ of $M$, \\
    $\lk(v,M)$ is PL homeomorphic to $S^{n-1}$.

    \pause
   \vfill

   \begin{problem}
     Is $S^2$ a manifold?
   \end{problem}

   \pause
   \vfill

   \begin{problem}
     Is $T^2$ a manifold?
   \end{problem}

\end{frame}

\begin{frame}
\frametitle{Going back, rethinking everything\ldots}

\begin{center}
\scalebox{1.76}{Is there a PL homeomorphism} \\
\scalebox{1.76}{between $S^2$ and $T^2$?}
\end{center}
\begin{center}
\includegraphics[height=0.25\textheight]{torus.png} \raisebox{0.125\textheight}{$\cong$}
\includegraphics[height=0.25\textheight]{sphere.png}\raisebox{0.125\textheight}{?}
\end{center}
\pause
Check $\chi$ is unchanged after stellar subdivision \\
(we must also check more general subdivisions!)
\end{frame}


\begin{frame}
  \frametitle{Euler characteristic}

$$
K_\sigma := (K - \st(\sigma,K)) \cup (v \join (\boundary \sigma) \join
\lk(\sigma,K))
$$
Need to check $\chi(K) = \chi(K_\sigma)$.

%Invariants should respect $\cup$ and $\join$.

\end{frame}

\begin{frame}
\frametitle{Euler characteristic and join}

\begin{align*}
p_K(x) &= 1/x + k_0 + k_1 x + \cdots + k_n x^n \\
\uncover<2->{p_L(x) &= 1/x + \ell_0 + \ell_1 x + \cdots + \ell_m x^m} \\
\uncover<3->{p_{K \join L}(x) &= p_K(x) \cdot p_L(x) \cdot x}
\end{align*}
\uncover<4->{Since $\chi(K) = p_K(-1) + 1$,} \\
\begin{align*}
\uncover<5->{\chi(K \join L) &= p_{K \join L}(-1) + 1} \\
\uncover<6->{&= \left( \chi(K) - 1 \right)\left( \chi(L) - 1 \right) \cdot \left(-1\right) + 1} \\
\uncover<7->{&= \chi(K) + \chi(L) - \chi(K) \chi(L).}
\end{align*}
\end{frame}

\begin{frame}
\frametitle{More joins!}

\begin{align*}
\chi( K \join L \join M )
\uncover<2->{&= \chi( K \join L ) + \chi(M) - \chi(K \join L) \chi(M)} \\
\uncover<3->{&= \chi(K) + \chi(L) - \chi(K) \chi(L) + \chi(M) - \\
& \quad\left( \chi(K) + \chi(L) - \chi(K) \chi(L) \right) \chi(M)} \\
\uncover<4->{&= \chi(K) + \chi(L) + \chi(M) \\
&\quad- \chi(K) \chi(L) - \chi(K) \chi(M) - \chi(L) \chi(M) \\
&\quad+\chi(K) \chi(L) \chi(M)} \\
\end{align*}

\end{frame}

\begin{frame}
  \frametitle{Euler characteristic and union}
  
\vfill
The usual inclusion-exclusion business gives
$$
\chi(K \cup L) = \chi(K) + \chi(L) - \chi(K \cap L)
$$
\vfill
\end{frame}

\begin{frame}
  \frametitle{Euler characteristic}

$\chi(K_\sigma)$
\begin{align*}
\uncover<2->{&= \chi \left( (K - \st(\sigma,K)) \cup (v \join (\boundary \sigma) \join
\lk(\sigma,K)) \right)} \\
\uncover<3->{&= \chi \left( K - \st(\sigma,K) \right) \\
&\quad+ \chi\left( v \join (\boundary \sigma) \join
\lk(\sigma,K) \right) \\
&\quad- \chi\left( \lk(\sigma,K) \right)} \\
\uncover<4->{&= \chi \left( K - \st(\sigma,K) \right) + \chi\left( v \join
 (\boundary \sigma) \right) \\
&\quad- 
\chi\left( v \join (\boundary \sigma) \right) \cdot \chi\left( \lk(\sigma,K) \right)} \\
\uncover<5->{&= \chi \left( K - \st(\sigma,K) \right) + 1 - \chi\left( \lk(\sigma,K) \right)} \\
\uncover<6->{&= \chi \left( K - \st(\sigma,K) \right) + \chi(\cl(\st(\sigma,K))) - \chi\left( \lk(\sigma,K) \right)} \\
&= \chi(K).
\end{align*}

\end{frame}


\begin{frame}
\begin{pgfpicture}{0in}{0in}{\textwidth}{\textheight}
\pgfsetxvec{\pgfpoint{\textwidth}{0cm}}
\pgfsetyvec{\pgfpoint{0cm}{\textheight}}

\only<1->{\pgfputat{\pgfxy(0.2,0.75)}{\pgfbox[center,center]{\scalebox{1}{\Huge\textcolor{green!50!black}{Upshot}}}}}
\only<3->{\pgfputat{\pgfxy(0.2,0.25)}{\pgfbox[center,center]{\scalebox{1}{\Huge\textcolor{green!50!black}{Haiku}}}}}

\only<4->{\pgfputat{\pgfxy(0.75,0.25)}{\pgfbox[center,center]{\parbox{0.75\textwidth}{\begin{center}
The sphere and torus, \\
what with their differing $\chi$, \\
are not the same space.
\end{center}
}}}}

\only<2->{\pgfputat{\pgfxy(0.75,0.75)}{\pgfbox[center,center]{\parbox{0.75\textwidth}{\begin{center}
$\chi(S^2) = 2$ but $\chi(T^2) = 0$, \\ 
and $\chi$ is a PL homeo invariant, \\ 
so $S^2 \not\cong T^2$. 
\end{center}
}}}}
\end{pgfpicture}
\end{frame}

\begin{frame}
  \frametitle{The torus versus the Klein bottle}

Since $\chi(T^2) = \chi(K) = 0$, \\
\pause
and \ldots  um \ldots

\vfill
\pause
$\chi$ is not a complete invariant.

\vfill

\end{frame}

\begin{frame}
\vfill
\vfill
\begin{center}
\scalebox{15}{$\chi$} \\
\vfill
Let's think about Euler characteristic
\end{center}
\vfill
\vfill
\end{frame}

\setbackgroundpicture{AdhesivesForHouseUse004.jpg}
\begin{frame}
  \begin{center}
\scalebox{9}{$\cup$} \\
union
\end{center}
\vfill
\begin{center}
\scalebox{1.25}{$\chi(K \cup L) = \chi(K) + \chi(L) - \chi(K \cap L)$}
\end{center}
\vfill
\end{frame}

\clearbackgroundpicture{}

\begin{frame}
  \frametitle{A new invariant}

  We want a new invariant. \\
  Call it $F$.
  \pause
  
  Since $I \cong V$, we must have $F(I) = F(V)$.
  \pause

  If our invariant is additive, we should have
  $$
  F(V) = F(I) + F(I) - F(\mbox{point}).
  $$
  \pause

  So $F(I) = F(\mbox{point})$.

\end{frame}

\begin{frame}
  \frametitle{A new invariant}

  Since $\Delta^2 \cup_I \Delta^2 \cong \Delta^2$, \\
  we must have $F(\Delta^2 \cup_I \Delta^2 ) = F(\Delta^2)$.
  \pause

  If our invariant is additive, we should have
  $$
  F(\Delta^2 \cup_I \Delta^2) = F(\Delta^2) + F(\Delta^2) - F(I).
  $$
  \pause

  So $F(\Delta^2) = F(I)$. \\
  \pause

  Similarly, $F(\Delta^n) = F(I)$.

\end{frame}

\begin{frame}
  \frametitle{A \textcolor{red!50!black}{``new''} invariant}

  Congratulations!  \pause You have invented $F$, \pause\\
  a rescaled version of the Euler characteristic.
  \pause
  \vfill
  \begin{theorem}
    An additive topological invariant is, \\
    up to rescaling, \\
    the Euler characteristic.
  \end{theorem}
  \pause
  \vfill

  New invariants cannot be precisely additive.

  \vfill

\end{frame}

\begin{frame}
\frametitle{Connected components}

Define $b_0(K)$ to be the \\
number of connected components of $K$.

\vfill
\pause
Vertices $v, w$ of $K$ belong to the same component \\
if there exists a PL map $f : I \to K$ \\
so that $f(0) = v$ and $f(1) = w$.

\vfill
\pause

\begin{problem}
Is ``belong to the same component'' \\
an equivalence relation \\
on the vertices of $K$?
\end{problem}

\end{frame}

\begin{frame}
\frametitle{Connected components}
\only<1>{\includegraphics[height=0.8\textheight]{components-1.png}}%
\only<2>{\includegraphics[height=0.8\textheight]{components-2.png}}%
\only<3>{\includegraphics[height=0.8\textheight]{components-3.png}}%
\only<4>{\includegraphics[height=0.8\textheight]{components-4.png}}%
$$
\mbox{\textcolor{red!50!black}{three}}
\uncover<2->{+ \mbox{\textcolor{blue!50!black}{one}}}
\uncover<3->{- \mbox{\textcolor{green!50!black}{three}}}
\uncover<4->{= \mbox{\textcolor{magenta!50!black}{one}}}
$$
\end{frame}

\begin{frame}
\vfill
$$
b_0(K \cup L) = b_0(K) + b_0(L) - b_0(K \cap L)?
$$
\vfill
\end{frame}

\begin{frame}
\frametitle{Connected components}
\only<1>{\includegraphics[height=0.8\textheight]{two-components-1.png}}%
\only<2>{\includegraphics[height=0.8\textheight]{two-components-2.png}}%
\only<3>{\includegraphics[height=0.8\textheight]{two-components-3.png}}%
\only<4>{\includegraphics[height=0.8\textheight]{two-components-4.png}}%
$$
\mbox{\textcolor{red!50!black}{four}}
\uncover<2->{+ \mbox{\textcolor{blue!50!black}{two}}}
\uncover<3->{- \mbox{\textcolor{green!50!black}{four}}}
\uncover<4->{= \mbox{\textcolor{magenta!50!black}{two}}}
$$
\end{frame}

\begin{frame}
\frametitle{Connected components}
\only<1>{\includegraphics[height=0.8\textheight]{circle-components-1.png}}%
\only<2>{\includegraphics[height=0.8\textheight]{circle-components-2.png}}%
\only<3>{\includegraphics[height=0.8\textheight]{circle-components-3.png}}%
\only<4>{\includegraphics[height=0.8\textheight]{circle-components-4.png}}%
$$
\mbox{\textcolor{red!50!black}{four}}
\uncover<2->{+ \mbox{\textcolor{blue!50!black}{four}}}
\uncover<3->{- \mbox{\textcolor{green!50!black}{eight}}}
\uncover<4->{= \mbox{zero} \neq \mbox{\textcolor{magenta!50!black}{one}}}
$$
\end{frame}

\begin{frame}

  We will invent new invariants. \\
  These invariants will not \textit{quite} be additive. \\
  The failure of additivity gives a new invariant, \\
  \quad itself not quite additive. \\
  The failure will be captured by a deeper invariant, \\
  \quad itself not quite additive. \\
  The failure will be captured by \ldots

  \vfill

  This is \textbf{homology}.
 
\end{frame}

\begin{frame}
\frametitle{The end of the beginning}

What's next?
\begin{itemize}
\item manifolds
\item simplicial collapse
\item simple homotopy equivalence
\item knots
\item unknotting theorems
\item return to (co)homology
\end{itemize}

\end{frame}

% for instance, try to define H_0...
% H_1(A cup B) -> H_0(A cap B) -> H_0(A) + H_0(B) -> H_0(A cup B) -> 0

\end{document}


%%% Local Variables: 
%%% mode: latex
%%% TeX-master: t
%%% End: 
