\documentclass[14pt]{beamer}

\setbeamertemplate{navigation symbols}{}

\usepackage{pgf}
\usepackage{amsmath}
\usepackage{amsthm}
%\newtheorem*{problem}{Problem}
\usepackage{amssymb}
\newcommand{\N}{\mathbb{N}}
\newcommand{\R}{\mathbb{R}}

\usepackage{stmaryrd}
\newcommand{\boundary}{\partial}
\newcommand{\collapses}{\searrow}
\newcommand{\expands}{\nearrow}
\newcommand{\she}{\ssearrow\nnearrow}
\newcommand{\join}{\ast}
\newcommand{\subdivided}{\triangleleft}

\DeclareMathOperator{\st}{st}
\DeclareMathOperator{\vertices}{vert}
\DeclareMathOperator{\lk}{lk}
\DeclareMathOperator{\cl}{cl}
\DeclareMathOperator{\interior}{int}

\title{Topology of  \\ Piecewise-Linear Manifolds}
\author{Jim Fowler}
\date{Lecture 5 \\ Summer 2010}

\newcommand{\setbackgroundpicture}[1]{%
\usebackgroundtemplate{
\begin{pgfpicture}{0in}{0in}{\paperwidth}{\paperheight}
\pgfputat{\pgfxy(0,0)}{\includegraphics[width=\paperwidth]{#1}}
\color{white}
\pgfsetfillopacity{0.8}
\pgfrect[fill]{\pgfxy(0,0)}{\pgfpoint{\paperwidth}{\paperheight}}
\end{pgfpicture}
}
}
\newcommand{\clearbackgroundpicture}{\usebackgroundtemplate{}}

\begin{document}

\begin{frame}
\maketitle
\end{frame}

%%%%%%%%%%%%%%%%%%%%%%%%%%%%%%%%%%%%%%%%%%%%%%%%%%%%%%%%%%%%%%%%

\setbackgroundpicture{hammer.jpg}
\begin{frame}
 \vfill
  \begin{center}
    Why do you keep hitting your hand with a hammer?
 \end{center}
  \vfill
\end{frame}
\clearbackgroundpicture

\setbackgroundpicture{A20_Ouest_km143.jpg}
\begin{frame}
\frametitle{Where are we?}

\begin{pgfpicture}{0in}{0in}{\textwidth}{\textheight}
\pgfsetxvec{\pgfpoint{\textwidth}{0cm}}
\pgfsetyvec{\pgfpoint{0cm}{\textheight}}

\only<2->{\pgfputat{\pgfxy(0.25,0.80)}{\pgfbox[center,bottom]{\scalebox{1}{\Huge\textcolor{green!50!black}{Objects}}}}}

\only<2->{\pgfputat{\pgfxy(0.25,0.6)}{\pgfbox[center,bottom]{\scalebox{1}{\large\textcolor{green!50!black}{\parbox{0.5\textwidth}{\begin{center}\uncover<3->{PL manifolds?} \\ \uncover<4->{simplicial complexes}\end{center}}}}}}}

\only<5->{\pgfputat{\pgfxy(0.75,0.80)}{\pgfbox[center,bottom]{\scalebox{1}{\Huge\textcolor{red!50!black}{Maps}}}}}

\only<5->{\pgfputat{\pgfxy(0.75,0.6)}{\pgfbox[center,bottom]{\scalebox{1}{\large\textcolor{red!50!black}{\parbox{0.5\textwidth}{\begin{center}\uncover<6->{PL maps?} \\ \uncover<7->{simplicial maps}\end{center}}}}}}}

\only<8->{\pgfputat{\pgfxy(0.5,0.35)}{\pgfbox[center,bottom]{\scalebox{1}{\Huge\textcolor{blue!50!black}{Invariants}}}}}

\only<9->{\pgfputat{\pgfxy(0.3333,0.25)}{\pgfbox[center,bottom]{\scalebox{1}{\Large\textcolor{blue!50!black}{$\chi$}}}}}

\only<10->{\pgfputat{\pgfxy(0.6666,0.25)}{\pgfbox[center,bottom]{\scalebox{1}{\Large\textcolor{blue!50!black}{$b_0$}}}}}

\end{pgfpicture}
\end{frame}
\clearbackgroundpicture

\setbackgroundpicture{Planisphaeri_celeste.jpg}
  \begin{frame}
    \vfill
    \begin{center}
   \large How to define \\
   \vspace{2ex}\scalebox{3}{\Huge PL} \\
   \vspace{1ex}\scalebox{3}{\large maps}
   \end{center}
    \vfill
  \end{frame}
\clearbackgroundpicture

\setbackgroundpicture{salad-fork.jpg}
\begin{frame}
  \vfill
  \begin{center}
    \textbf{a fork in the road}  \\
    \vspace{0.25\textheight}
    our notion of subdivision is, thus far, \\
    restricted to stellar subdivision
  \end{center}
  \vfill
\end{frame}
\clearbackgroundpicture

\setbackgroundpicture{Aristotle_Altemps_Inv8575.jpg}
\begin{frame}
  \vfill
  \begin{center}
   Why restrict to stellar subdivision?
    \vfill
    \Huge Some \\
    Philosophy \\
   \vfill
    \normalsize ``Lines are made up of points? Really?''
    \vfill
  \end{center}
  \vfill
\end{frame}
\clearbackgroundpicture

\begin{frame}
  \frametitle{State of the art}

  \begin{theorem}
    Two $n$-dimensional simplicial complexes\\
    are PL homeomorphic \\
    if and only if \\
    they are stellar equivalent.
  \end{theorem}

  \vfill
  \pause
  \begin{theorem}
    A PL map $f : K \to L$ is a simplicial map
    \\ from a subdivision of $K$ \\
    to a stellar subdivision of $L$.
  \end{theorem}

  \textbf{Warning:} The subdivision of $K$ needn't be stellar. \\

  \pause
  \textbf{Question:} What does ``subdivision'' mean?

\end{frame}

% \setbackgroundpicture{broken-logs.jpg}
% \begin{frame}
%   \frametitle{Help put my shattered dreams together}

%   Can't we set up PL theory in a way that doesn't involve, well, the
%   \textbf{real numbers?}  I want a wholly combinatorial framework
%   befitting the combinatorial objects!

%   \vfill\pause

%   How could we do this?

%   \vfill\pause

%   Mathematical reality versus definition-chasing

%   \vfill\pause

%   \textbf{Goal:} develop a theory of ``regular neighborhoods'' and
%   ``collapse'' which will mostly obviate the need for the
%   combinatorics, anyhow.
  
% \end{frame}
% \clearbackgroundpicture

\begin{frame}
\frametitle{The combinatorial part}

\begin{itemize}
\item  simplicial complex
\item  join
\item  link
\item  star
\item  Euler characteristic
\item PL manifold: a complex $K$ the link of each vertex PL
  homeomorphic (meaning stellar equivalent) to a sphere.
\end{itemize}

\end{frame}

\begin{frame}
\frametitle{The non-combinatorial part}

The notion of PL map is not quite general enough:\\
a simplicial map from a stellar subdivision of $K$ \\
to a stellar subdivision of $L$ is a PL map, \\
but we will need more general maps.

\vfill

To do this, we will place our simplexes in $\mathbb{R}^n$.

\end{frame}

% need products to get ambient isotopy
% also high dimensional manifolds
% real projective plane

% \begin{frame}
%  \vfill
%  \begin{center}
%    \large
%    I will have a handout prepared for Friday 
%    with the following definitions.
%  \end{center}
%  \vfill
% \end{frame}

\begin{frame}
\vfill
 $$\scalebox{5}{$\mathbb{R}^n$}$$
\vfill
\end{frame}

\begin{frame}
The join of $A, B \subset \mathbb{R}^n$ is
$$
\{ \lambda a + (1-\lambda) b : a \in A, b \in B, \lambda \in [0,1] \}
$$
\end{frame}

\begin{frame}
  \frametitle{Polyhedron}

  $P \subset \mathbb{R}^n$ is a \textbf{polyhedron} \\
  if for each $p \in P$ \\
  there is a neighborhood $N \ni p$ \\
  so that $N = p \join L$, \\
  with $L$ closed and bounded.

  \vfill
  \pause
  For example, $N = \{ q \in P : d(p,q) \leq \epsilon \}$,
  but we allow more general neighborhoods.

  \vfill
  \pause
  $N$ is called a \textbf{closed star} around $p$. \\
  $L$ is a \textbf{link} of $p$.

\end{frame}

\begin{frame}
  \frametitle{Piecewise linear map}
  Let $P, Q$ be polyhedra. \\
  $f : P \to Q$ is a PL map \\
  if each point $p \in P$ \\
  has a closed star $N = p \join L$ \\
  so that $f(\lambda p + (1-\lambda) x) = \lambda f(p) + (1 - \lambda)
  f(x)$ \\
  for $x \in L$ and $\lambda \in [0,1]$.

  \vfill

  In short, it locally maps conical rays to conical rays.

\end{frame}

\setbackgroundpicture{Stair10.jpg}
\begin{frame}
  \frametitle{Next steps}
  \pause
  \begin{itemize}
   \item New constructions \pause\\
     \textcolor{green!50!black}{connected sum}\pause and \\
      \textcolor{green!50!black}{Cartesian product}, using polyhedra \pause \\
    \item New invariant \pause\\
      \textcolor{green!50!black}{orientability, $w_1$}\pause
    \item New kind of equivalence \pause\\
      \textcolor{green!50!black}{simple homotopy equivalence}
 \end{itemize}

\end{frame}
\clearbackgroundpicture

%%%%%%%%%%%%%%%%%%%%%%%%%%%%%%%%%%%%%%%%%%%%%%%%%%%%%%%%%%%%%%%%

\begin{frame}
  \vfill
  \begin{center}
    \scalebox{5}{\Huge $\#$} \\
   \vspace{2ex}\hspace{-2em}connected sum
  \end{center}
  \vfill
\end{frame}

%%%%%%%%%%%%%%%%%%%%%%%%%%%%%%%%%%%%%%%%%%%%%%%%%%%%%%%%%%%%%%%%
\begin{frame}
  \frametitle{Connected sum}

  \begin{definition}
  Suppose $K$ and $L$ are two $n$-manifolds. \\
  \vspace{1ex}%
  \begin{tabular}{@{}l@{ }l@{ }l@{ }l@{ }l}
  Let & $K'$ & be & $K$ & with one $n$-simplex removed. \\
   & $L'$ &  & $L$ &
  \end{tabular} \\
  \vspace{1ex}Define $K \# L = K' \cup_{\boundary \Delta^n} L'$.
\end{definition}

\vfill
\begin{columns}
\begin{column}{0.5\textwidth}
\uncover<2->{
\begin{problem}
 What is $T^2 \# T^2$?
\end{problem}}
\end{column}
\begin{column}{0.5\textwidth}
\uncover<3->{
\begin{problem}
 What is $S^2 \# T^2$?
\end{problem}}
\end{column}
\end{columns}

\end{frame}

\begin{frame}
\vfill
$$
 \includegraphics[width=0.25\textwidth]{torus.png} \raisebox{0.06\textwidth}{$\#$}
 \includegraphics[width=0.25\textwidth]{torus.png}  \raisebox{0.06\textwidth}{$\cong$}
 \hspace{1ex}\includegraphics[width=0.25\textwidth]{genus-two-surface.png}
$$
\vfill
\end{frame}

\begin{frame}
\vfill
$$
 \includegraphics[width=0.25\textwidth]{torus.png} \raisebox{0.06\textwidth}{$\#$}
 \includegraphics[width=0.25\textwidth]{genus-two-surface.png}  \raisebox{0.06\textwidth}{$\cong$}
 \uncover<2->{\includegraphics[width=0.25\textwidth]{genus-three-surface.png}}
$$
\vfill
\uncover<3->{In general, $\Sigma_g = \mbox{genus $g$ surface}$,}\\
\vspace{1ex}
\uncover<4->{and $T^2 \# \Sigma_g  = \Sigma_{g+1}$.}
\end{frame}

\begin{frame}
  \frametitle{Covering map}

  \begin{definition}
    $f : M \to N$ is an $n$-fold \textbf{covering map} \\
    if for every $k$-simplex $\tau \in N$, \\
    $f^{-1}(\tau)$ consists of $n$ copies of a $k$-simplex.
%    if there is a subdivision $N'$ of $N$, \\
%    so that for every vertex $v \in N'$ \\
%    $f^{-1}(\cl \st(v,N)) \cong$ $n$ copies of $\cl \st(v, N)$, \\
%    and $f$ is a homeo onto each copy of $\cl \st(v, N)$.
  \end{definition}

\end{frame}

\begin{frame}
  \frametitle{Gift-wrapping surfaces}

  \begin{problem}
    When is there an $n$-fold covering map \\
    from a genus $g$ surface \\
    to a genus $g'$ surface?
  \end{problem}
  \pause
  \begin{theorem}
    If $f : M \to N$ is an $n$-fold covering map \\
    then $\chi(M) = n \cdot \chi(N)$.
  \end{theorem}
  \pause
  \begin{problem}
    What is $\chi(\mbox{genus $g$ surface})?$
  \end{problem}

\end{frame}

%%%%%%%%%%%%%%%%%%%%%%%%%%%%%%%%%%%%%%%%%%%%%%%%%%%%%%%%%%%%%%%%
\begin{frame}
  \vfill
  \begin{center}
    \scalebox{5}{\Huge $\times$} \\
   \vspace{2ex}Cartesian product
  \end{center}
  \vfill
\end{frame}


\begin{frame}
  \frametitle{Products, geometrically}

\begin{definition}
Suppose $K \subset \R^n$, and $L \subset R^m$. \\
The product of $K$ and $L$ is
$$
K \times L = \{ (k,\ell) \in \R^{n+m} : k \in K, \ell \in L \}.
$$
\end{definition}

\pause
\vfill
Could we do this for a simplicial complex?

\end{frame}

%%%%%%%%%%%%%%%%%%%%%%%%%%%%%%%%%%%%%%%%%%%%%%%%%%%%%%%%%%%%%%%%
\begin{frame}
 \frametitle{Product of simplexes}
 How do we take a product of $\Delta^1$ and $\Delta^1$?
 \vfill
\end{frame}

\begin{frame}
\frametitle{Product of simplexes}
How do we take a product of $\Delta^1$ and $\Delta^2$?
\vfill
\end{frame}

\begin{frame}
\frametitle{Product of simplexes}
How do we take a product of $\Delta^2$ and $\Delta^2$? \\\pause
Hard to see, so let's take a step back.
\vfill
\end{frame}

% %%%%%%%%%%%%%%%%%%%%%%%%%%%%%%%%%%%%%%%%%%%%%%%%%%%%%%%%%%%%%%%%
% \begin{frame}
% \frametitle{Properties of products}

% The product of $\Delta^n$ and $\Delta^m$ should have projection maps
% \begin{align*}
% p_1 &: \Delta^n \times \Delta^m \to \Delta^n \\
% p_2 &: \Delta^n \times \Delta^m \to \Delta^m
% \end{align*}
% so that a point in $\Delta^n \times \Delta^m$ is determined by where
% it lands under the two projection maps.

% \vfill

% \end{frame}

\begin{frame}

  Vertices of $\Delta^n \times \Delta^m$ are pairs \\
  $(i,j)$ with $0 \leq i \leq n+1$ and $0 \leq j \leq m+1$.

  \pause
  \vfill
  A simplex of $\Delta^n \times \Delta^m$ is \rule{48pt}{12pt}

\end{frame}

\begin{frame}
  \frametitle{Some counting problems}

  \vfill

  $\Delta^1 \times \Delta^n$ has $(n+1)!$ triangulations. \\

  \vfill\pause

  $\Delta^2 \times \Delta^n$ has \rule{48pt}{12pt} triangulations.

  \vfill\pause

  So maybe it would be nice if we could handle these cases without
  resorting to simplexes.
  
\end{frame}

\begin{frame}
  \frametitle{Product of polyhedra is a polyhedron}

  $P, Q$ polyhedra. \\
  Is $P \times Q$ a polyhedron?
  \vfill\pause
  Need neighborhood of $(p,q) \in P \times Q$.
  \vfill\pause
  $p \in P$ has a neighborhood $p \join L_p$ \\
  $q \in Q$ has a neighborhood $q \join L_q$ \\
  \vfill\pause
  Use $(p,q) \join (L_p \join L_q)$.

\end{frame}

%0a 0b 1a 1b
%0a 0b 1a
%0b 1a 1b

% K times L,
% get CK times L
% 
% CK times L is a copy of L attached to K times L

\begin{frame}
  \frametitle{Some 4-manifolds}
  \Huge
  \vfill
  \begin{center}\pause
  $S^3 \times S^1$\\\pause
  $S^2 \times S^2$\\\pause
  $\Sigma_g \times \Sigma_h$
  \end{center}
  \vfill

\end{frame}

%%%%%%%%%%%%%%%%%%%%%%%%%%%%%%%%%%%%%%%%%%%%%%%%%%%%%%%%%%%%%%%%
\setbackgroundpicture{Templo_de_Afaia3.JPG}
\begin{frame}
  \vfill
  \begin{center}
    \scalebox{5}{\Huge $w_1$} \\
   \vspace{2ex}orientability
  \end{center}
  \vfill
\end{frame}
\clearbackgroundpicture

\begin{frame}
  \frametitle{The torus versus the Klein bottle}

Since $\chi(T^2) = \chi(K) = 0$, \\
we can't distinguish $T^2$ and $K$ using $\chi$.

\pause
\vfill

We need a new invariant.

\vfill

\end{frame}

\begin{frame}
  \frametitle{Orientation}

  An orientation of a simplex is \\
  a choice of ordering of the vertices,\\
  up to an even permutation.

  \vfill
  \pause

  \begin{definition}
    An orientation on a manifold $M^n$ \\
    is an orientation of each $n$-simplex \\
    so that neighboring $n$-simplexes \\
    induce opposite orientations on the shared $(n-1)$-simplex.
  \end{definition}
  \vfill
  \pause

  If $M$ admits an orientation, $w_1(M) = 0$. \\
  Otherwise, $w_1(M) = 1$.

\end{frame}

\begin{frame}
  \frametitle{Orientability for surfaces}

  \begin{theorem}
    A surface $\Sigma$ is not orientable \\
    iff $\Sigma$ contains a M\"obius strip.
  \end{theorem}
  \pause
  \begin{problem}
   What is $w_1(T^2)$?
 \end{problem}
  \pause
  \begin{problem}
    What is $w_1(K)$?
  \end{problem}
  \pause
 \begin{problem}
   Is $w_1$ a PL homeomorphism invariant?
 \end{problem}
  
\end{frame}

\begin{frame}
  \frametitle{Distinguishing other manifolds?}

  Can we distinguish $T^3$ and $S^3$ with Euler characteristic?

  \vfill
  \pause

  No\pause\ldots

  \vfill

  We need more invariants, \\
  \pause
  or a looser notion of ``the same.''
  
\end{frame}

% a surface comes with a naturally defined map 
% into a space of $n$-vertex circles with two marked points

% vertices are 3, 4, 5, ... (different circles)
% edges are unions of circles along two points
% 


\setbackgroundpicture{collapsedbarn.jpg}
\begin{frame}
  \vfill
  \begin{center}
    \Huge Simplicial \\
    Collapse
  \end{center}
  \vfill
\end{frame}
\clearbackgroundpicture

\begin{frame}
  \frametitle{Principal Simplexes}

  \begin{definition}
  Let $K$ be a complex, and \\
  $\sigma \in K$ a simplex.  \\
  Call $\sigma$ a \textbf{principal simplex} \\
  if the only simplex containing $\sigma$ \\
  is $\sigma$ itself \\
  \quad (i.e., it isn't contained in a larger simplex).
  \end{definition} 

  \pause
  \begin{problem}
  Does every complex have a principal simplex?
  \end{problem}

\end{frame}

\begin{frame}
  \frametitle{Free faces}

  \begin{definition}
  Let $K$ be a complex, \\
  and $\sigma \in K$ a simplex, \\
  and $\tau < \sigma$ a face. \\
  \vspace{1ex}Call $\tau$ a \textbf{free face} of $\sigma$ \\
  if the only simplexes containing $\tau$ \\
  are $\tau$ and $\sigma$.
  \end{definition}

  \pause
  \begin{problem}
    \begin{tabular}{@{}l@{ }c@{ }l}
      Does & every & complex have a simplex with a free face? \\\pause
      Does & any & complex have a simplex with a free face? \\
   \end{tabular}
 \end{problem}

\end{frame}

\begin{frame}
\frametitle{Elementary simplicial collapse}

\begin{definition}
  Let $L$ and $K = L \cup \cl \{ \sigma, \tau \}$ be complexes \\

  \vspace{1ex}If $\sigma$ is a principal simplex of $K$, and \\
  $\tau$ is a free face of $\sigma$, then \\
  $L$ is an \textbf{elementary simplicial collapse} of $K$.
\end{definition}

\end{frame}

\begin{frame}
\frametitle{Simplicial collapse}

\begin{definition}
  Let $K_1, K_2, \ldots, K_n$ be complexes, with \\
  $K_{i+1}$ an elementary simplicial collapse of $K_i$.

  \vspace{1ex}Call $K_n$ a \textbf{simplicial collapse} of $K_1$, and \\
  write $K_1 \collapses K_n$.

  \vspace{1ex}Call $K_1$ a \textbf{simplicial expansion} of $K_n$, and \\
  write $K_n \expands K_1$.
\end{definition}

\end{frame}

\begin{frame}
  \frametitle{Simple homotopy equivalence}

  \begin{definition}
    $K$ is \textbf{simple homotopy equivalent} to $L$
    \quad (sometimes abbreviated s.h.e.) \\
    if you can reach transform $K$ into $L$ \\
    via a sequence of
    \begin{itemize}
    \item PL homeomorphisms,
    \item simplicial collapses,
    \item simplicial expansions.
    \end{itemize}
    In this case, we write $K \she L$.
  \end{definition}

\end{frame}



\setbackgroundpicture{zeeman.pdf}
\begin{frame}
  \frametitle{Next time}
  \pause
  \begin{itemize}
    \item Regular neighborhoods \\
    \item Knotting $S^1$ in $S^3$ \\
    \item Unknotting $S^1$ in $S^4$ \\
    \item How to unknot with simplicial collapse
\end{itemize}

\end{frame}
\clearbackgroundpicture

\end{document}


%%% Local Variables: 
%%% mode: latex
%%% TeX-master: t
%%% End: 
