\documentclass[12pt]{pset}
%\usepackage{add-copyright}

\title{Problem Set 7}
\course{Piecewise-Linear Topology}
\author{Jim Fowler}
\date{Summer 2010}

\newcommand{\defnword}[1]{\textbf{#1}}

\usepackage{stmaryrd}
\newcommand{\boundary}{\partial}
\newcommand{\collapses}{\searrow}
\newcommand{\expands}{\nearrow}
\newcommand{\she}{\ssearrow\nnearrow}
\newcommand{\join}{\ast}
\newcommand{\subdivided}{\triangleleft}
\geometry{margin=1in}%,top=0.5in,bottom=0.5in}
\DeclareMathOperator{\st}{st}
\DeclareMathOperator{\vertices}{vert}
\DeclareMathOperator{\lk}{lk}
\DeclareMathOperator{\cl}{cl}
\DeclareMathOperator{\interior}{int}
\newcommand{\fullsubcomplex}{\Subset}

\begin{document}
\maketitle

\subsubsection*{Regular neighborhoods.} Here we summarize the main
results on simplicial neighborhoods from the textbook
\textit{Introduction to Piecewise-Linear Topology}.  We will focus on
applications of this theory.

%The extra handout
%for today is a photocopy of the pages from this book, so you can see
%the proof.

%\parskip 0.25\baselineskip
%\parindent 0pt

\begin{definition*}[Some neighborhoods]
  The \textbf{simplicial neighborhood} of $L$ in $K$ is
  $$N(L,K) = \{ A \in K : A < B, B \cap |L| \neq \varnothing \}.$$
  The \textbf{simplicial complement} of $L$ in $K$ is
  $$  C(L,K) = \{ A \in K : A \cap |L| = \varnothing \}. $$
\end{definition*}

\begin{definition*}[Some subdivisions]
 Suppose $L \subset K$ are complexes, and $a_i \in \interior A_i$ for each $A_i \in K$ with $A_i \not\in L$.
 We define a subdivision $K' \subdivided K$ 
  inductively by
  $$
  {A_i}'  = \begin{cases}
    \{ a_i \} \join \partial {A_i} & \mbox{ if $A_i \not\in L$, } \\
    A_i & \mbox{ if $A_i \in L$.}
    \end{cases}
  $$
  This is called $K$ \textbf{derived away} from $L$.

  On the other hand, the derived of $K$ \textbf{near} $L$ is obtained
  by deriving $K$ away from $L \cup C(L,K)$, i.e., subdividing those
  simplexes meeting $|L|$ but not in $L$.

  If $K'$ is $K$ derived near $L$, then $N(L,K')$ is a \textbf{derived neighborhood} of $L$ in $K$.
\end{definition*}

\begin{definition*}[Regular neighborhood]
Suppose $X \subset Y$ are polyhedra.
\begin{itemize}
\item $|K|$ a neighborhood of $X$ in $Y$.
\item $|L| = X$.
\item $L \fullsubcomplex K$.
\item $K'$ derived of $K$ near $L$.
\end{itemize}
Then $|N(L,K')|$ is called a \textbf{regular neighborhood} of $X$ in $Y$.
\end{definition*}

\begin{theorem*}
  If $N_1$ and $N_2$ are regular neighborhoods of $X$ in $Y$, then
  there is a homeomorphism $h : Y \to Y$, which throws $N_1$ onto
  $N_2$, and which is the identity on $X$, and the identity outside a compact subset of $Y$.
\end{theorem*}

\begin{theorem*}
  A regular neighborhood $N$ of a polyhedron $X$ in a manifold $M$ is a manifold with boundary.
\end{theorem*}


 \begin{theorem*}[Simplicial neighborhood theorem]
   Suppose $X$ is a compact polyhedron,  $M$ is a manifold, and $X
   \subset \interior M$.   Then a polyhedral neighborhood $N$ of $X$ in $\interior
   M$ is a regular neighborhood if and only if
    \begin{itemize}
    \item $N$ is a compact manifold with boundary
    \item there are triangulations $(K,L,J)$ of $(N,X,\partial N)$
      with $L \fullsubcomplex K$, $K = N(L,K)$ and $J = \partial N(L,K)$.
    \end{itemize}
  \end{theorem*}

 \begin{theorem*}
    As before, suppose $X$ is a compact polyhedron, $M$ is a manifold,
    and $X \subset \interior M$.
    Then a polyhedral neighborhood $N$ of $X$ in $\interior
    M$ is a regular neighborhood if and only if
   \begin{itemize}
   \item $N$ is a compact manifold with boundary
   \item $N \collapses X$.
   \end{itemize}
 \end{theorem*}

\begin{corollary*}
 A collapsible manifold is a ball.
\end{corollary*}

\subsection*{Regular neighborhoods and Simplicial collapse}

\begin{problem}[From \textit{Introduction to Piecewise-Linear
    Topology}]  Show that $\mathbb{R}^n$ is PL homeomorphic to $S^n - \mbox{point}$.
\end{problem}

\begin{problem}
 Suppose $M^2$ is a 2-manifold, with $\partial M = S^1 \cup S^1$, and
 $M \collapses S^1 \subset \partial M$.  Identify $M$.
\end{problem}

\begin{requiredproblem}
  What are the possible regular neighborhoods of $S^1$ inside a
  2-manifold?  Use this to describe those surfaces which collapse to a circle.
\end{requiredproblem}

\begin{problem}
  Can you find two different 3-manifolds which both collapse to $S^1$?
  Which both collapse to $S^2$?
\end{problem}

\begin{requiredproblem}
 Is a subcomplex of a collapsible complex necessarily collapsible?
\end{requiredproblem}

\begin{problem}
  Let $X$ and $Y$ both be $S^1 \vee S^1 \vee S^1$, namely, three
  circles connected together at a single point; find embeddings of $X$
  and $Y$ into $T^3$, so that $T^3 - \mbox{regular neighborhood of
    $X$}$ collapses onto $Y$.
\end{problem}

\begin{requiredproblem}
  Show that $T^2 - \mbox{a disk}$ and $S^3 - \mbox{3 disks}$ collapse
  onto $S^1 \vee S^1$, the union of two circles along one point.
\end{requiredproblem}

\pagebreak
\subsection*{Knot theory}

If you want to try your hand at Reidemeister moves, here are some exercises
to help you to do so.

\begin{problem}
  Show that the figure eight knot 
    \raisebox{-0.05\textwidth}{\includegraphics[width=0.1\textwidth]{figure-eight.pdf}} is not the unknot.
\end{problem}

\begin{problem}
 Is the trefoil knot identical to its mirror image: is
  \raisebox{-0.05\textwidth}{\includegraphics[width=0.1\textwidth]{Trefoil_knot_left.pdf}}
  the same as
  \raisebox{-0.05\textwidth}{\includegraphics[width=0.1\textwidth]{TrefoilKnot_01.pdf}}?
\end{problem}

\begin{problem}
 Is the figure eight knot the same as its mirror image, i.e., is
 \begin{center}
   \includegraphics[width=0.1\textwidth]{figure-eight.pdf}\hspace{0.5em}\raisebox{0.05\textwidth}{the same as}\hspace{0.5em}%
   \includegraphics[width=0.1\textwidth]{figure-eight-flipped.pdf}\raisebox{0.05\textwidth}{?}
 \end{center}
\end{problem}

\begin{problem}
  Suppose $S^1 \subset S^3$ is a knot; show that the suspension $SS^1
  \subset SS^3$ gives a knotted $S^2$ in $S^4$.
\end{problem}

\end{document}

