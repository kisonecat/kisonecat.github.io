\documentclass[12pt]{handout}
%\usepackage{add-copyright}

\title{Remembering trigonometric formulas}
\author{Jim Fowler}
\course{Math 1181H}
\date{Autumn 2012} 

\usepackage{nopageno}

\begin{document}
\maketitle

\null\hfill\parbox{5.5in}{
\begin{verse}
\textit{Neo:} What are you trying to tell me? That I can memorize all trigonometric formulas? \\
\textit{Morpheus:} No, Neo. I'm trying to tell you that when you're ready, you won't have to. \\
\null\hfill ---An approximation to \textit{The Matrix}
\end{verse}}

\subsection*{Required Background}

You have to know about complex numbers (e.g., ``$a+bi$'' where $i^2 =
-1$) and about the rules for exponentiation.

\subsection*{Idea}

For now, we will bravely define
$$
e^{i\theta} = \cos \theta + i \sin \theta.
$$
We'll see in a couple weeks why this is a true statement (i.e., it
corresponds with what we already meant by $e^x$, $\sin x$, and $\cos
x$).  For now, we'll just accept it---accepting also that it is very
hard to understand what it should mean to raise a number to an
imaginary power!  Nevertheless, our brave definition gives an
interpretation.

We will ignore the conceptual concern, and simply manipulate the above
equation to translate facts about exponentiation to imaginary powers
into facts about trigonometric functions.  We assume that all the
usual rules for exponentiation apply.

\subsection*{Theory: A Famous Example}

Some of you might have heard that $e^{i\pi} + 1 = 0$, an equation
which relates five famous numbers: zero, one, $e$, $\pi$, and $i$.  In
other words, $e^{i\pi} = -1$, and we can see why by using the above
equation:
$$
e^{i \pi} = \cos \pi + i \sin \pi = -1,
$$
because $\cos \pi = -1$ and $\sin \pi = 0$.

\subsection*{Practice: Double-angle formula}

For instance, let's say we want to have a formula for $\cos 2\theta$ in
terms of $\cos \theta$ and $\sin \theta$.  We use the fact that
$$
e^{i\theta} \cdot e^{i\theta} = e^{i\theta + i\theta} = e^{i(\theta + \theta)} = e^{i (2\theta)}
$$
because exponents add.  Substitute in our brave definition to get
$$
(\cos \theta + i \sin \theta)(\cos \theta + i \sin \theta) = \cos (2\theta) + i \sin (2\theta).
$$
But we can expand the left-hand side to find
$$
\cos^2 \theta - \sin^2 \theta + 2 i \cos \theta \sin \theta = \cos (2 \theta) + i \sin(2\theta).
$$
Since real and imaginary parts must be equal, we then conclude
\begin{eqnarray*}
  \cos^2 \theta - \sin^2 \theta &=& \cos (2 \theta), \\
 2 \cos \theta \sin \theta &=& \sin(2\theta),
\end{eqnarray*}
giving us the double angle formulas.

\subsection*{Difference formula}

You can use this method to get a formula for $\sin (a - b)$ in terms
of sines and cosines of $a$ and $b$.  We have that
$$
e^{ia} \cdot e^{-ib} = e^{i(a-b)},
$$
and replacing the exponentials with trigonometric functions gives
$$
(\cos a + i \sin a)(\cos (-b) + i \sin (-b)) = \cos(a-b) + i \sin (a-b).
$$
Expanding the left-hand side gives
$$
\cos a \cos (-b) + i^2 \sin a \sin (-b) + i \sin a \cos (-b) + i \cos a \sin (-b) = \cos (a-b) + i \sin(a-b).
$$
But $i^2 = -1$, so
$$
\left( \cos a \cos (-b) - \sin a \sin (-b) \right) + i \left( \sin a \cos (-b) + \cos a \sin (-b) \right) = \cos (a-b) + i \sin(a-b).
$$
Again, real and imaginary parts must equal, so
\begin{eqnarray*}
  \cos a \cos (-b) - \sin a \sin (-b) &=& \cos (a-b), \\
  \sin a \cos (-b) + \cos a \sin (-b) &=& \sin(a-b).
\end{eqnarray*}
Since $\cos (-b) = \cos b$ and $\sin (-b) = -\sin b$, we could also write these as
\begin{eqnarray*}
  \cos a \cos b + \sin a \sin b &=& \cos (a-b), \\
  \sin a \cos b - \cos a \sin b &=& \sin(a-b).
\end{eqnarray*}
The moral: memorizing identities is difficult, and while rederiving
them when needed might be just as complicated, it requires less rote
memorization---and for some people, this is a huge help.

\end{document}
