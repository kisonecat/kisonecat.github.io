\documentclass[12pt]{handout}
%\usepackage{add-copyright}
\usepackage{geometry}
\geometry{margin=1in,top=0.4in,bottom=0.4in}

\title{Syllabus}
\author{Jim Fowler}
\course{Math 1181H}
\date{Autumn 2012}

\usepackage[T1]{fontenc}
\usepackage{lmodern}
\usepackage{hyperref}
\usepackage{nopageno}

\newcommand{\peem}{\textsc{p.m.}}
\newcommand{\ayem}{\textsc{a.m.}}

\titlespacing*{\section}{0in}{*0}{*1}
\titlespacing*{\subsection}{0in}{*0}{*1}

\begin{document}
\maketitle

\noindent Calculus is among the greatest achievements of humankind; this fast-moving course will explore single variable calculus in depth.

%\section*{Resources}

%\noindent%
%We present five resources to help you to understand calculus.

\subsection*{Office hours}
If you have questions, want to work through problems, or just talk
about mathematics, please attend office hours.

\vspace{1ex}%
\noindent\parbox{0.5\textwidth}{%
\noindent\begin{tabular}{@{}ll}
\textsf{Name:} & Jim Fowler \\
\textsf{Office:} & MW658 Mathematics Tower \\
\textsf{Phone:} & (773) 809--5659 \\
\textsf{Email:} & \href{mailto:fowler@math.osu.edu}{\texttt{fowler@math.osu.edu}} \\
\textsf{Website:} & \url{http://www.math.osu.edu/~fowler/}
\end{tabular}}
\noindent\parbox{0.5\textwidth}{%
\begin{tabular}{@{}ll}
\textsf{Office Hours:}
& Monday and Wednesday \\
& 4:00--5:30\peem \\
& and by appointment
\end{tabular}}

\vspace{1ex}\noindent
Please email me with any concerns you have; the success of this course
depends on open communication.

\subsection*{Textbook}
Our text is the \href{http://books.google.com/books?id=m1Q8AAAACAAJ}{second edition of Simmons' ``Calculus with Analytic Geometry''} published by McGraw--Hill.  The textbook's ISBN is 0070576424.

\subsection*{Website}

I will post handouts on Carmen.
%\url{http://www.math.ohio-state.edu/~fowler/teaching/math758/}.

\subsection*{Lectures}

We meet weekdays from 11:30\ayem\ until 12:25\peem\ in Smith Lab 1138 for an interactive lecture.

%\vfill
%\pagebreak
%%%%%%%%%%%%%%%%%%%%%%%%%%%%%%%%%%%%%%%%%%%%%%%%%%%%%%%%%%%%%%%%
\subsection*{Assessment}


There are 1000 points possible in this course; earning an $\mathrm{A}$ or $\mathrm{B}$ or $\mathrm{C}$ or $\mathrm{D}$ requires earning $900$ or $800$ or $700$ or $600$ points, respectively.  The 1000 points are broken down as follows.

\begin{description}
\item[\textsf{\textbf{15 problem sets (180 points; 12 points each).}}]
  Homework is assigned each week, usually assigned on Tuesday, and
  collected the following Tuesday.\vspace{1ex}\\
  You should work on the homework problems together, but you must
  write up your solutions independently and record the names of your collaborators. \vspace{1ex}\\
  You must stay caught up with the homework. But I understand your
  schedules are very busy, so I will not penalize you for
  \textit{very occasionally} turning in late homework.  Do not make a habit
  of it.

\item[\textsf{\textbf{3 midterms (420 points; 140 points each).}}]
The first midterm is Friday, September~21; the second midterm is Friday, October~19; the third midterm is Friday, November~9.

\item[\textsf{\textbf{1 final exam (400 points).}}]  The final examination will be held in our usual classroom at 12:00\peem\ on Monday, December~10, 2012.  Students who earn 368 points on the final exam will earn an $\mathrm{A}$ for the course as a whole.

\end{description}

\vspace{1ex}

\subsection*{Extra credit}

\noindent Occasionally, I may provide a worksheet you can complete for a few extra credit points.

\subsection*{No electronic calculators on exams}

You may not use electronic devices on the midterms or the final exam;
it is my responsibility to produce exam questions which can be
answered easily, and answered without resorting to a calculator.

``But how, without a calculator, can I take square roots?  Cosines?
Logs?''  Indeed, how?  This course, by bringing you deeper into
calculus, will address the question of how calculators perform these
very tasks.

\subsection*{General Education Requirement}

This mathematics course can be used, depending on your degree program, to satisfy the Math or Logical Skills category of the General Education Requirement. The goals and learning objectives for this category are as follows.
\begin{description}
\item[Goals.] Courses in quantitative and logical skills develop logical reasoning, including the ability to identify valid arguments. Math 1181H covers the essentials of one-variable Calculus and its applications, and develops corresponding computational  and problem solving skills. 	
\item[Learning objectives.] Students comprehend mathematical concepts and methods adequate to construct valid arguments and understand inductive and deductive reasoning, scientific inference and general problem solving.
\end{description}

\subsection*{Academic Misconduct}

It is the responsibility of the Committee on Academic Misconduct to
investigate or establish procedures for the investigation of all
reported cases of student academic misconduct. The term ``academic
misconduct'' includes all forms of student academic misconduct
wherever committed; illustrated by, but not limited to, cases of
plagiarism and dishonest practices in connection with
examinations. Instructors shall report all instances of alleged
academic misconduct to the committee. For additional information, see
the \href{http://studentaffairs.osu.edu/resource_csc.asp}{Code of Student
 Conduct} at \url{http://studentaffairs.osu.edu/resource_csc.asp}

\subsection*{Disabilities}

Students with disabilities that have been certified by the
\href{http://www.ods.ohio-state.edu}{Office for Disability Services}
will be appropriately accommodated, and should inform the instructor
as soon as possible of their needs. The Office for Disability Services
is located in 
\begin{quote}
\href{http://www.osu.edu/map/building.php?building=067}{150 Pomerene Hall}\\
1760 Neil Avenue\\
Columbus, OH\hspace{0.5em} 43210
\end{quote}
and can be reached by telephone at (614)~292--3307, by video relay
service at (614)~429--1334, and on the web at
\url{http://www.ods.ohio-state.edu/}.


\end{document}

