\documentclass[12pt]{article}
\usepackage[margin=1in]{geometry}
\usepackage{nopageno}
\usepackage{multicol}
\usepackage{add-copyright}

\usepackage{amsthm}
\theoremstyle{definition}
\newtheorem*{example*}{Example}
\newcommand{\limn}{\displaystyle\lim_{n \to \infty}}

\title{Midterm 1 Review Sheet}

\usepackage{amssymb}
\newcommand{\R}{\mathbb{R}}
\newcommand{\N}{\mathbb{N}}

\begin{document}

\section*{I wonder---what will the exam be like?}

\begin{itemize}
\item There will be \textbf{13 questions} on the exam and it will last
  \textbf{50 minutes}.  The first question will ask you to give the
  $\epsilon$-$K$ definition of the limit of a sequence.

\item Calculators are \textbf{forbidden}.

\item There will be \textbf{one improper integral} on the exam; it
  will not involve a challenging integral: I would rather that you
  make sure you are extremely comfortable with
  \textit{differentiation} on this exam, so you can handle the
  questions involving l'H\^opital's rule---future exams will give us
  plenty of practice with the fancier techniques of integration.

\item The last question will be \textbf{extra credit} with true/false
  questions.  I think these are \textbf{terribly enjoyable!}  They
  will ask you to agree and disagree with statements of theorems,
  slightly modified statements of theorems, existence of sequences
  exhibiting certain kinds of phenomena, etc.

\item You will lose points if you surround an otherwise convincing
  argument with false statements (after all, once I have proved $2 =
  1$, I can prove anything!).  \textbf{Erase} or cross-out untrue
  statements \textbf{for full credit.}

\item Style matters: if you are taking a limit, write $\lim$.  Do not
  use an ``$=$'' between two expressions unless they are, in fact,
  equal.  Do not \textbf{under any circumstances} divide by zero
  during the exam.

\item You can refer to our friend the ``natural logarithm'' by any
  nickname you prefer.
\end{itemize}

\section*{I wonder---what must I do on this ``exam''?}

\begin{multicols}{2}
\begin{itemize}
\item Give the definition of $\limn a_n = L$.
\item Prove that a sequence converges by providing an $\epsilon$-$K$ proof.
\item Give the definition for a sequence to be bounded above, bounded
  below, increasing, decreasing, non-increasing, non-decreasing
\item Recognize when a sequence is bounded
\item Recognize when a sequence is monotone
\item Find limits by using algebraic manipulation
\item Use $\limn f(a_n) = f(\limn a_n)$ provided $f$ is continuous.
\item Find limits by squeezing sequences between sequences with a
  common, known limit
\item Give the statement of l'H\^opital's rule for indeterminate forms $0/0$ and $\infty/\infty$
\item Use l'H\^opital's rule to compute limits (sometimes it takes more than one application!)
\item Handle indeterminate forms $0^0$, $\infty - \infty$
\item Compute an improper integrals by taking limits of definite integrals (and do so carefully, noting where you are taking limits)
\end{itemize}
\end{multicols}

\section*{How should I write down my answers?}

You are not giving answers: you are giving an explanation.  For full
credit, you must \textbf{justify your arguments} and \textbf{show the
  steps you took}: if the central issue in the problem is that the
limit of the sum is the sum of the limits, you should be sure to point
that out.

\section*{You must know the following limits}

\begin{multicols}{2}
\begin{example*}
For $x \in \R$, $x > 0$,
$\limn x^{1/n} = 1$.
\end{example*}

\begin{example*}
For $x \in \R$ with $-1 < x < 1$,
$\limn x^n = 0$.
\end{example*}

\begin{example*}
For $x \in \R$, $x > 0$,
$\limn \frac{1}{n^x} = 0$.
\end{example*}

\begin{example*}
For $x \in \R$,
$\limn \frac{x^n}{n!} = 0$.
\end{example*}

\begin{example*}
$\limn \displaystyle\frac{\log n}{n} = 0$.
\end{example*}

\begin{example*}
$\limn n^{1/n} = 1$.
\end{example*}

\begin{example*}
For $x \in \R$,
$\limn \left( 1 + \frac{x}{n} \right)^n = e^x$.
\end{example*}
\end{multicols}

\section*{You must know the following trigonometric facts}

$$
\sin (2x) = 2\,\sin x\,\cos x \hspace{1em} \cos (2x) = \cos^2 x - \sin^2 x
$$

If additional identities are needed, I will provide them (but be
warned: when you are one day stranded on desert island, you will have
wished you knew as many trigonometric identities as possible---or at least, how to derive them from $e^{i\theta}$).

You should know the derivatives of $\sin$, $\cos$, and $\tan$ (which
you can remember with the mnemonic ``you'll \textit{tan} in two
\textit{sec}onds using this product'' for $\displaystyle\frac{d}{dx} \tan x = \sec^2 x$).

\section*{You must be able to give $\epsilon$-$K$ proofs to show}

\begin{multicols}{2}
\begin{itemize}
\item $\limn \displaystyle\frac{1}{n} = 0$.
\item $\limn \displaystyle\frac{1+n}{n} = 1$.
\item $\limn \displaystyle\frac{1}{n^2} = 0$.
\item $\limn \displaystyle\frac{3}{n^2} = 0$.
\item $\limn \displaystyle\frac{(-1)^n}{n} = 0$.
\item $\limn \displaystyle\frac{\cos n}{n} = 0$.
\item $\limn \displaystyle\frac{\cos \left(n^3\right)}{n} = 0$.
\item $\limn \displaystyle\frac{2 + 3 \, n}{n^2} = 0$.
\end{itemize}
\end{multicols}

\end{document}