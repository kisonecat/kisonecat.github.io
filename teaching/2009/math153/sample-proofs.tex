\documentclass[12pt]{article}
\usepackage{fullpage}
\usepackage{add-copyright}

\title{Practice Proof Problems}

\usepackage{amsmath}
\usepackage[margin=1.25in]{geometry}

\usepackage{amsthm}
\newtheorem{theorem}{Theorem}
\newtheorem{corollary}[theorem]{Corollary}
\newtheorem{lemmma}[theorem]{Lemma}
\newtheorem{proposition}[theorem]{Proposition}

\theoremstyle{definition}
\newtheorem{remark}[theorem]{Remark}
\newtheorem{example}[theorem]{Example}
\newtheorem{definition}[theorem]{Definition}

\newtheorem{podsip}[theorem]{Prove or Disprove}

\usepackage{amssymb}
\newcommand{\R}{\mathbb{R}}
\newcommand{\N}{\mathbb{N}}

\newcommand{\limn}{\displaystyle\lim_{n \to \infty}}

\begin{document}

\section*{Practice Proof Problems}

Below are a list of statements that you can either prove (if the claim
is true) or disprove (by exhibiting a counterexample).  These are just
for fun---statements to ponder if you are looking for more things to think
about.

Incidentally, there is a pattern (dare I say the beginning of a
sequence!) to which statements are false, and which statements are
true.  And by my count, there are eighteen false statements in need of
counterexamples, and seventeen true statements in need of a proof.

\subsection*{The Statements}

\begin{podsip} % FALSE
Suppose $a_n$ is a sequence of real numbers, and $\limn n^2 \cdot a_n = 0$.  Then $\limn n \cdot a_n = 0$.
\end{podsip}

\begin{podsip} % TRUE
If $a_n$ and $b_n$ are sequences of real numbers, and $\limn a_n = 0$, and $b_n$ is bounded, then $\limn (a_n \cdot b_n) = 0$.
\end{podsip}

\begin{podsip} % TRUE
If $a_n$ and $b_n$ are bounded sequences of real numbers, then $c_n = a_n \cdot b_n$ is a bounded sequence.
\end{podsip}

\begin{podsip} % FALSE
If $a_n$ is an nondecreasing sequence of real numbers, then $a_n$ has a subsequence which is increasing.
\end{podsip}

\begin{podsip} % FALSE
If $a_n$ is a decreasing sequence of real numbers, then $a_n$ is bounded below.
\end{podsip}

\begin{podsip} % TRUE
If $a_n$ is an increasing sequence of real numbers, then $b_n = {a_n}^3$ is also increasing.
\end{podsip}

\begin{podsip} % FALSE
There exists a sequence $a_n$ of positive numbers, with $\limn a_n < 0$.
\end{podsip}

\begin{podsip} % TRUE
If $a_n$ is a decreasing sequence of real numbers, then $a_n$ is bounded above.
\end{podsip}

\begin{podsip} % TRUE
There exists a sequence $a_n$ of rational numbers, with $\limn a_n = \sqrt{2}$.
\end{podsip}

\begin{podsip} % FALSE
If $a_n$ is a sequence of real numbers, and $\limn \cos a_n = 1$, then $\limn a_n = 0$.
\end{podsip}

\begin{podsip} % TRUE
There exists a sequence $a_n$ of irrational numbers, with $\limn a_n = 3$.
\end{podsip}

\begin{podsip} % TRUE
If $a_n$ is a sequence of integers, and $\limn a_n = 6$, then for all but finitely many values of $n \in \N$, we have $a_n = 6$.
\end{podsip}

\begin{podsip} % FALSE
If $a_n$ is an unbounded sequence, then
$$
b_n = \frac{a_{n+1}}{\sqrt{n+1}}
$$
is bounded.
\end{podsip}

\begin{podsip} % FALSE
If $a_n$ is an unbounded sequence and $a_n \neq 0$ for all $n \in \N$, then
$$
b_n = \frac{a_{n+1}}{a_{n}}
$$
is bounded.
\end{podsip}

\begin{podsip} % TRUE
If $a_n$ is a bounded sequence, then $\limn a_n/n = 0$.
\end{podsip}

\begin{podsip} % FALSE
If $a_n$ is a sequence of real numbers, and $\limn {a_n}^2 = 1$, then $\limn a_n = 1$.
\end{podsip}

\begin{podsip} % FALSE
If $a_n$ is an unbounded sequence, then $b_n = \cos a_n$ is bounded, but does not converge.
\end{podsip}

\begin{podsip} % TRUE
If $a_n$ is a bounded sequence, and $b_n$ is an unbounded sequence, then $\limn a_n/b_n = 0$.
\end{podsip}

\begin{podsip} % TRUE
If $a_n$ is a sequence of positive real numbers, and $\limn \log a_n = 0$, then $\limn a_n = 1$.
\end{podsip}

\begin{podsip} % FALSE
If $a_n$ is an increasing sequence of positive real numbers, then $b_n = a_{n}/a_{n+1}$ is increasing.
\end{podsip}

\begin{podsip} % TRUE
If $a_n$ is a sequence of real numbers, and $\limn {a_n}^2 = 0$, then $\limn a_n = 0$.
\end{podsip}

\begin{podsip} % FALSE
Let $a_n$ be a sequence of non-zero numbers, and define $b_n = |a_n| / a_n$.  If $a_n$ converges, then $b_n$ converges.
\end{podsip}

\begin{podsip} % FALSE
If $a_n$ and $b_n$ are increasing sequences of positive real numbers, then
$$
c_n = \begin{cases}
a_{n} & \mbox{ if $n$ is even, } \\
b_{n} & \mbox{ if $n$ is odd. }
\end{cases}
$$
is also increasing.
\end{podsip}

\begin{podsip} % TRUE
If $b_n$ is the $n$-th digit of $\pi$, and $a_n = (b_n/10)^n$, then $\limn a_n = 0$.
\end{podsip}

\begin{podsip} % FALSE
  If $a_n$ is a sequence of real numbers, and $b_n = a_n/n$ is
  bounded, then $b_n$ converges.
\end{podsip}

\begin{podsip} % TRUE
If $a_n$ is an increasing sequence of positive real numbers, then $b_n = a_{2n}$ is increasing.
\end{podsip}

\begin{podsip} % TRUE
If $\limn a_n = L$, and there exists $A$ and $B$ so that for all $n \in \N$, we have $A \leq a_n \leq B$, then $A \leq L \leq B$.
\end{podsip}

\begin{podsip} % FALSE
If $a_n$ is an increasing sequence of positive real numbers, then $b_n = a_{n+1} - a_{n}$ is increasing.
\end{podsip}

\begin{podsip} % FALSE
There exists a bounded sequence $a_n$ so that $b_n = e^{a_n}$ converges to $0$.
\end{podsip}

\begin{podsip} % TRUE
Suppose $\limn a_n = L$, and $\limn b_n = L$.  Define
$$
c_n = \begin{cases}
a_{n} & \mbox{ if $n$ is even, } \\
b_{n} & \mbox{ if $n$ is odd. }
\end{cases}
$$
Then $\limn c_n = L$.
\end{podsip}

\begin{podsip} % FALSE
Define $a_n = n^n$.  Then $b_n = a_n / n!$ converges.
\end{podsip}

\begin{podsip} % FALSE
There is a $K \in \N$ so that for every $\epsilon > 0$, if $n \geq K$, then 
$$
\left| \frac{1}{n} - 0 \right| < \epsilon.
$$
\end{podsip}

\begin{podsip} % TRUE
  Let $a_n$ be the biggest number that can be written down using $n$
  letters from the ``alphabet'' containing
$$
0, 1, 2, 3, 4, 5, 6, 7, 8,
  9, +, -, \times.
$$
Then $a_n > n$.
\end{podsip}

\begin{podsip} % FALSE
There exists a sequence $a_n$ of integers, with $\limn a_n = \pi$.
\end{podsip}

\begin{podsip} % TRUE
Let $a_n$ be a sequence of real numbers, and define
$$
b_n = \begin{cases}
a_{n} & \mbox{ if $a_n < 1/n$, } \\
0 & \mbox{ if $a_n \geq 1/n$. }
\end{cases}
$$
Then $b_n$ converges.
\end{podsip}

\end{document}
