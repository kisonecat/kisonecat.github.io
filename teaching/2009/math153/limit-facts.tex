\documentclass[12pt,letterpaper,twocolumn]{article}

\title{Facts about Limits.}
\author{Jim Fowler}

\usepackage{fullpage}
\usepackage{nopageno}
\usepackage{amsmath}
\usepackage[margin=1cm]{geometry}
\usepackage{add-copyright}

\usepackage{amsthm}
\newtheorem{theorem}{Theorem}
\newtheorem{corollary}[theorem]{Corollary}
\newtheorem{lemmma}[theorem]{Lemma}
\newtheorem{proposition}[theorem]{Proposition}

\theoremstyle{definition}
\newtheorem{remark}[theorem]{Remark}
\newtheorem{example}[theorem]{Example}
\newtheorem{definition}[theorem]{Definition}

\usepackage{amssymb}
\newcommand{\R}{\mathbb{R}}
\newcommand{\N}{\mathbb{N}}

\newcommand{\limn}{\displaystyle\lim_{n \to \infty}}

\begin{document}

\section*{Facts about limits.}

\begin{definition}[Formal]
We say $\limn a_n = L$ if
\begin{verse}
  for all $\epsilon > 0$, \\
  there exists $K \in \N$, \\
  so that if $n \geq K$, \\
  then $|a_n - L| < \epsilon$.
\end{verse}
\end{definition}

\begin{definition}[Intuitive]
We say $\limn a_n = L$ if
\begin{verse}
  however close we want to be, \\
  there's a place we can go, \\
  so that beyond that place, \\
  we are that close.
\end{verse}
\end{definition}

\begin{proposition}[Limits are unique]
  If $\limn a_n = L$ and $\limn a_n = M$,
  then $L = M$.
\end{proposition}

\begin{proposition}
  If the sequence $a_n$ converges, then $a_n$ is bounded.
\end{proposition}

\begin{corollary}
  If the sequence $a_n$ is not bounded, then $a_n$ diverges.
\end{corollary}

\begin{proposition}
  If $\limn a_n = L$, and $c \in \R$, then $\lim_{n \to
    \infty} c \cdot a_n = c \cdot L$.
\end{proposition}

\begin{proposition}
  If $\limn a_n = L$ and $\limn b_n = M$,
  then $\limn a_n + b_n = L + M$.
\end{proposition}

\begin{proposition}
  If $\limn a_n = L$ and $\limn b_n = M$,
  then $\limn a_n \cdot b_n = L \cdot M$.
\end{proposition}

\begin{proposition}
  If $\limn a_n = L$ and $\limn b_n = M$, and $b_n \neq 0$ and $M \neq 0$, 
  then $\limn a_n/b_n = L/M$.
\end{proposition}

\begin{proposition}
  If $\limn a_n = L$ and $m \in \N$, then $\limn a_{n+m} = L$.
\end{proposition}

\begin{proposition}
  If $\limn a_n = L$, and $b_n$ is a sequence differing from $a_n$ in
  finitely many terms, then $\limn b_n = L$.
\end{proposition}

\begin{theorem}
If $a_n$ is a nondecreasing bounded above sequence, then $a_n$ converges.
\end{theorem}

\begin{theorem}
If $a_n$ is a nonincreasing bounded below sequence, then $a_n$ converges.
\end{theorem}

\begin{theorem}[Squeezing theorem]
  If $a_n$, $b_n$, and $c_n$ are sequences of real numbers, and for
  all $n \in \N$, we have $a_n \leq b_n \leq c_n$, and $\limn a_n = L$
  and $\limn c_n = L$, then $\limn b_n = L$.
\end{theorem}

\begin{theorem}[Sequences and continuity]
  If a function $f : \R \to \R$ is continuous at $L$, and $\limn a_n = L$, then
  $\limn f(a_n) = f(L)$.
\end{theorem}

\begin{example}
For a real number $x > 0$,
$$
\limn x^{1/n} = 1.
$$
\end{example}

\begin{example}
For a real number $x$ with $-1 < x < 1$,
$$
\limn x^n = 0.
$$
\end{example}

\begin{example}
For a real number $x > 0$,
$$
\limn \frac{1}{n^x} = 0.
$$
\end{example}

\begin{example}
For $x \in \R$,
$$
\limn \frac{x^n}{n!} = 0.
$$
\end{example}

\begin{example}
$\limn \displaystyle\frac{\log n}{n} = 0$.
\end{example}

\begin{example}
$\limn n^{1/n} = 0$.
\end{example}

\begin{example}
For $x \in \R$,
$$
\limn \left( 1 + \frac{x}{n} \right)^n = e^x.
$$
\end{example}

\end{document}
