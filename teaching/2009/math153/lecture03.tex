\documentclass[12pt]{article}
\usepackage{fullpage}
\usepackage{amsthm}
\usepackage{amsmath}

\usepackage{amssymb}

\newcommand{\N}{\mathbb{N}}
\newcommand{\Q}{\mathbb{Q}}
\newcommand{\Z}{\mathbb{Z}}
\newcommand{\R}{\mathbb{R}}

\newtheorem*{example}{Example}
\newtheorem*{thm}{Theorem}

\title{Lecture 3: Limits}
\author{Math 153 Section 57}
\date{Friday October  3, 2008}

\begin{document}
\maketitle

We will be following chapter 11.3.

Here, we introduce limits, the \textbf{most important idea in
  calculus}, and that which distinguishes calculus from mere algebra.

\section{Review of where we've been}

Defined ``bounded'' for sets and sequences.

Defined ``monotone'' for sequences.

\subsection{Some loose ends: recursively defined sequences}

One person asked: is a sequence just a function?  Yes!  You've
probably already studied functions $\R \to \R$, and a sequence is a
function $\N \to \R$.

A sequence (or a function) is not necessarily given by a formula: A
weirder example: $p_n$ is the $n$-th digit of $\pi$.  $x_n$ is the
number of digits in the decimal representation of $n$.

Another way to define sequences: define future terms by using past terms.

Example: $a_1 = 2$.  $a_{n+1} = 2 \cdot a_n$.  Increasing?  This is just $a_n = 2^n$.

Example: $a_1 = 16$.  $a_{n+1} = 16 - a_n$.  Increasing?  This
alternates between 0 and 16.

Example: $a_1 = 1, a_2 = 1$.  $a_{n+2} = a_{n+1} + a_n$.  Increasing?

\section{Limits}

Formal definition: $\lim_{n \to \infty} a_n = L$ means for every $\epsilon >
0$, there exists $K$ such that $n \geq K$ implies $|a_n - L| <
\epsilon$.

Intuitive definition: $\lim_{n \to \infty} a_n = L$ means that as
close as you want $a_n$ to get to $L$, you can go far enough out in
the sequence and stay that close.

If $a_n$ has a limit, then it is a \textbf{convergent} sequence, and
we say it \textbf{converges}.  If not, the sequence \textbf{diverges}
(or \textbf{is divergent}).

\subsection{Challenge-response}

Think of the definition of limit as a challenge response game: the
challenger gives you an $\epsilon$, and you must produce the $K$.

To prove that something converges to $L$, you must be find a $K$ for
every $\epsilon$.

To prove that something doesn't converge at all?  You have to show
that no matter what $L$ you pick, there is some $\epsilon$ for which
you can't find a suitable $K$.  That sounds difficult.

\subsection{Example}

Useful fact: for any $b \in \R$, there is an $n \in \N$ with $n > b$.

Useful fact: for any $b \in \R$ with $b > 0$, there is an $n \in \N$
with $1/n < b$.

If $\lim_{n \to \infty} (2n+1)/n = 2$.  Why?  Set $L = 2$.  We need to find $K$ so that for all $n > K$, we get $|2 + 1/n - 2| < \epsilon$.

If $a_n = 0.9999 \cdots 9$, i.e., the $n$-term has $n$ nines, then
$a_n = 1 - 10^{-n}$, and $\lim_{n \to \infty} a_n = 1$.  To be
precise?  Set $L = 1$.  Then need $K$ so that for all $n > K$, we get
$|1 - 10^{-n} - 1| < \epsilon$.  That is, we need $10^{-n} <
\epsilon$, so we take $\log_{10} 10^{-n} < \log_{10} \epsilon$, so we
take $n > -\log_{10} \epsilon$.

\subsection{Theorems}

Limits are unique: if $\lim_{n \to \infty} a_n = L$ and also $\lim_{n
  \to \infty} a_n = M$, then $L = M$.

Limits only depend on their tails: if $\lim_{n \to \infty} a_n = L$
and $m \in \N$, then $\lim_{n \to \infty} a_{n+m} = L$.

Convergent sequences are bounded.

Unbounded sequences are divergent (awesome---a situation where we can
show that no matter what $L$ we pick, there is an $\epsilon$ for which
we can't find a $K$).

\section{Survey with closed eyes}

One week is over: are we going too fast, too slowly, just right?

\end{document}
