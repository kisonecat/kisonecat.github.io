\documentclass[12pt,letterpaper]{article}
\usepackage{add-copyright}

\title{Math153 Syllabus}
\author{Jim Fowler}

\usepackage{multicol}
\usepackage{fullpage}
\usepackage{palatino}
\pagestyle{empty}

\usepackage{html}

%\usepackage[T1]{fontenc}
%\usepackage{textcomp}
\usepackage{lmodern}
%\newcommand{\ditto}{\textquotedbl}

%\setlength{\parindent}{0pt}

\newcommand{\peem}{\textsc{p.m.}}
\newcommand{\ayem}{\textsc{a.m.}}

\begin{document}

%%%%%%%%%%%%%%%%%%%%%%%%%%%%%%%%%%%%%%%%%%%%%%%%%%%%%%%%%%%%%%%%
\section*{\Large\sf Syllabus\hfill
Math 153, Section 57\hfill
Fall 2008}

This course covers sequences, series, techniques
of integration, and differential equations.

\begin{htmlonly}
You can also download this syllabus as a \htmladdnormallink{PDF
file}{http://math.uchicago.edu/~fowler/math153/syllabus.pdf}.
\end{htmlonly}

%%%%%%%%%%%%%%%%%%%%%%%%%%%%%%%%%%%%%%%%%%%%%%%%%%%%%%%%%%%%%%%%
\section*{Resources}

We present six resources to help you to learn Calculus.

\subsection*{Office Hours}
If you have questions, want to work through problems, or just talk
about mathematics, I invite you to come to my office hours.
\begin{multicols}{2}
\begin{tabular}{ll}
\textbf{Name:} & Jim Fowler \\
\textbf{Office:} & Math/Stat 102 \\
%\textbf{Phone:} & 773--573--5659 \\
\textbf{Email:} & \texttt{jim@uchicago.edu}
\end{tabular}

\begin{tabular}{ll}
\textbf{Office Hours:} & Tuesday 5:00--7:00\peem \\
& Thursday 5:00--6:00\peem \\
& and by appointment
\end{tabular}
\end{multicols}
\noindent
Please email me with any concerns you have; the success of this course
depends on open communication.

The VIGRE Course Assistant, Brooke Ullery
(\texttt{bullery@uchicago.edu}), will also office hours on
Wednesdays from 6:00--7:00\peem\ in the C-Shop.

\subsection*{Textbook}
Our text is \textit{Calculus} by Salas, Hille, and Etgen (10th
Edition).

\subsection*{Website}
The website is on chalk; we will post assignments, notes, and handouts.

\subsection*{Lectures}
We meet Mondays, Wednesdays, and Fridays, 12:30--1:20\peem\ in Kent 101 for an interactive lecture.

\subsection*{Recitation Sessions}
We meet Thursdays, 6:00--7:00\peem\ in Eckhart 312 for an interactive
recitation session.

\subsection*{Homework Assignments}
Your homework will be graded by Brooke Ullery and Jim Fowler.

\pagebreak

%%%%%%%%%%%%%%%%%%%%%%%%%%%%%%%%%%%%%%%%%%%%%%%%%%%%%%%%%%%%%%%%
\section*{Requirements}

There are one thousand points possible in this course, broken down as
follows:
\begin{description}
\item[Daily homework (150 points).]  Homework is assigned after most
  lectures.  You should start homework promptly.  You should work on
  the homework problems together, but you must write up your solutions
  independently.
\item[2 midterms (225 points each).]  The midterms are in class.  The
  first midterm is on Monday, October 20, at the beginning of week 4;
  the second midterm is on Monday, November 17, at the beginning of
  week 8.
\item[1 final exam (400 points).]  The final exam will be held
  4:00--6:00\peem\ on Tuesday, December 9, 2008.
\end{description}

%%%%%%%%%%%%%%%%%%%%%%%%%%%%%%%%%%%%%%%%%%%%%%%%%%%%%%%%%%%%%%%%
\subsection*{Department Policy on Final Exams}

\textit{It is the policy of the Department of Mathematics that the
following rules apply to final exams in all undergraduate mathematics
courses:}
\begin{enumerate}
\item \textit{The final exam must occur at the time and place designated on
the College Final Exam Schedule.}  In particular, \textit{no} final examinations
may be given during the tenth week of the quarter, except in the case
of graduating seniors.
\item Any student who wishes to depart from the scheduled final exam
time for the course must receive permission from Paul Sally (office is
Ryerson 350, phone is 2-7388, email is
\texttt{sally@math.uchicago.edu}).  Instructors are not permitted to
excuse students from the scheduled time of the final exam except in
the cases of an Incomplete.
\end{enumerate}

%%%%%%%%%%%%%%%%%%%%%%%%%%%%%%%%%%%%%%%%%%%%%%%%%%%%%%%%%%%%%%%%
\subsection*{Rescheduling a Midterm}

Contact Jim Fowler early in the quarter if you will not be able to
take a midterm on the scheduled day; we likely can accomodate you, but
we will need time to prepare.

%%%%%%%%%%%%%%%%%%%%%%%%%%%%%%%%%%%%%%%%%%%%%%%%%%%%%%%%%%%%%%%%
\subsection*{Late Homework}

You must stay caught up.  It is tempting to fall behind, but difficult
to catch up again---this is true of all courses, but especially true
of a course in mathematics.  That said, I understand your schedules
are very busy, so I will not penalize you for \textit{infrequently}
turning in work \textit{a day or two late.}  Do not make a habit of
it!

\end{document}
