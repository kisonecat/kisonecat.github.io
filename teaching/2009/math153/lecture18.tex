\documentclass[12pt]{article}
\usepackage{fullpage}
\usepackage{amsthm}
\usepackage{amsmath}

\newtheorem*{example}{Example}
\newtheorem*{thm}{Theorem}

\title{Lecture 18: Generalizing Taylor series}
\author{Math 153 Section 57}
\date{Friday November  7, 2008}

\begin{document}
\maketitle

Following chapter 12.7.

\section{how to expand Taylor series around other values.}

Two perspectives: the formal answer, and the trick by replacing $x$ by $x-a$.

\section{examples}

example: $\log$ around $x=1$

example: $1/x$ around $x=1$

\section{euler's formula}

exponentials and trig functions: $e^{i\theta} = \cos \theta + i \sin \theta$

\section{tricks}

tricks for Taylor series: do $2 \sin x \cos x$ and discover that this is $\sin (2x)$:
$$
2 \left( x - \frac{x^3}{6} + \frac{x^5}{120} - \cdots \right) \cdot 
\left( 1 - \frac{x^2}{2} + \frac{x^4}{24} - \cdots \right)
$$
which equals
$$
2 \left( x - \frac{x^3}{6} + \frac{x^5}{120} - \cdots \right) -
2 \left( \frac{x^3}{2} - \frac{x^5}{12} + \cdots \right) + 
2 \left( \frac{x^5}{24} - \cdots \right)
$$
which simplifies to
$$
2x - 2 \left(\frac{1}{6} + \frac{1}{2}\right) x^3 + 2 \left(\frac{1}{120} + \frac{1}{12} + \frac{1}{24}\right) x^5 - \cdots
$$
or equivalently
$$
2x - \frac{4}{3} x^3 + \left(\frac{4}{15} \right) x^5 - \cdots
$$
But this is the same as
$$
\frac{1}{2} \left( (2x) - \frac{(2x)^3}{6} + \frac{(2x)^5}{5!} - \cdots \right)
$$
which is the series for $\sin (2x)$

Very formally:
$$
\sum_{n=0}^\infty \frac{(-1)^n x^{2n+1}}{(2n+1)!} 
\sum_{m=0}^\infty \frac{(-1)^m x^{2m}}{(2m)!} =
\sum_{n=0}^\infty \sum_{m=0}^\infty \frac{(-1)^{n+m} x^{2n+2m+1}}{(2n+1)! \, (2m)!}
$$

\section{trick again}

$\sin x / x$ is very easy to do

What about $x / \sin x$?  We could differentiate, but that would be painful.  Instead, assume it has a Taylor series, and do long division to find it.
$$ \frac{x}{\sin x} = 1+\frac{x^2}{6}+\frac{7 x^4}{360}+\frac{31 x^6}{15120}+\frac{127
    x^8}{604800}+\frac{73 x^{10}}{3421440}+\cdots$$

same trick works on $1/(1-x)$.

\section{proving facts about functions}

$e^a \cdot e^b = e^{a+b}$

\end{document}
