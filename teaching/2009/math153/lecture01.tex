\documentclass[12pt]{article}
\usepackage{fullpage}
\usepackage{amsthm}
\usepackage{amsmath}
\usepackage{amssymb}

\newcommand{\N}{\mathbb{N}}
\newcommand{\Q}{\mathbb{Q}}
\newcommand{\Z}{\mathbb{Z}}
\newcommand{\R}{\mathbb{R}}

\newtheorem*{example}{Example}
\newtheorem*{thm}{Theorem}

\title{Lecture 1: Welcome}
\author{Math 153 Section 57}
\date{Monday September 29, 2008}

\begin{document}
\maketitle

\section{Hand out syllabus and survey}

A very brief overview of the course, and an inspirational message.

Problem session scheduling.

Grading system for homework.

Final grades.

\section{Review}

\subsection{Numbers}

Natural numbers. Integers. Rationals. Reals.

\subsection{Sets}

The $\in$ symbol.

How to define a set?

\subsection{Intervals}

Open intervals.  Closed intervals.  Half-open intervals.

\subsection{Bounds}

A number $b$ is an \textbf{upper bound} of a set $A$ if for all $a \in A$,
$a \leq b$.

A number $b$ is a \textbf{lower bound} of a set $A$ if for all $a \in A$,
$b \leq a$.

Example: $(0,5)$.  $(-\infty,12)$.  $(2,\infty)$.  $[0,5]$.

Example: $\{0, 1, 2, 3, 4, \ldots \}$.

Example: $\{0, 0.9, 0.99, 0.999, 0.9999, \ldots\}$.

There may be many upper bounds; there may be none.

A set is \textbf{bounded above} if it has an upper bound.

A set is \textbf{bounded below} if it has a lower bound.

A set is \textbf{bounded} if it has both a lower bound and an upper bound.

\section{Least upper bound axiom}

Axiom: every nonempty set of real numbers bounded above has a least
upper bound.

Example: $[0,3]$.  $(0,3)$.

Example: $\{0, 0.9, 0.99, 0.999, 0.9999, \ldots\}$.

Example: rational numbers?

\end{document}
