\documentclass[12pt]{article}
\usepackage{fullpage}
\usepackage{amsthm}
\usepackage{amsmath}

\newtheorem*{example}{Example}
\newtheorem*{thm}{Theorem}

\title{Lecture 13: Convergence tests}
\author{Math 153 Section 57}
\date{Monday October 27, 2008}

\begin{document}
\maketitle

Following chapter 12.3 and starting 12.4.

\subsection*{Bare hands proof that harmonic series diverges}

Estimate $\sum_{n = 1}^K 1/n$.  1/1.  1/2.  $1/3 + 1/4$ underestimated
by $1/2$.  $1/5 +1/6 + 1/7 + 1/8$ underestimates by $1/2$.  And so on.
Therefore, $s_{2^k} \geq 1 + k/2$.  This argument is about 700 years
old.

Use this to estimate how many terms we have to add up.  How big does
$K$ have to be in order for $\sum_{n=1}^K 1/n > 10$.  If I want $1 +
k/2 \geq 10$, I need $k \geq 18$.  But $2^{18} = 262144$.

This will do, but I actually only need $k = 12367$.  Better
approximation from Euler–Mascheroni constant:
$$
\gamma = \lim_{K \to \infty} \left(\left( \sum_{n=1}^K 1/n \right) - \log K\right) \approx 0.577.
$$
Since $e^{10 - 0.577} = 12369.6$, guess that $k = 12367$ suffices.

\subsection*{Be careful about series with negative terms}

For example, $\sum 1/n$ diverges, but as we will see soon, $\sum
(-1)^n/n$ converges---even though $1/n$ and $(-1)^n/n$ have the same
magnitude.  More about this on Friday.

\subsection*{Limit comparison test}

Limit comparison test: let $a_n \geq 0$ and $b_n \geq 0$.  If $\lim_{n \to \infty} a_n / b_n = L > 0$, then $\sum a_n$ converges iff $\sum b_n$ converges.

Proof: Choose $\epsilon$ so that $0 < \epsilon < L$.

Then there exist $K$ so that for $n \geq K$, $|a_n / b_n - L| < \epsilon$.

That means,
$$
(L - \epsilon) b_n < a_n < (L + \epsilon) b_n
$$
So if $\sum a_n$ converges, then $\sum (L - \epsilon) b_n$ converges.

\textbf{The slogan: converge is all about how quickly those terms are
  vanishing.}

\subsection*{Example}

$$
\sum_n \frac{3n^2 + n + 17}{n^4 + n^2 + 8}
$$
On the one hand, we could try to bound this and apply comparison
tes\ldots but even easier is to use the limit comparison test!  Use
$\sum 1/n^2$.

\subsection*{Buyer beware!}

The limit comparison test does not tell you what the series
equals---only whether it converges.

\subsection*{Like all tools, use it properly}

Limit comparison says nothing if $L = 0$, or $L = \infty$.

\section*{Review}

Does a series converge?

First, is it a series we recognize? Geometric series?  Harmonic series
or $p$-series?

Apply integral test (which involves coming up with a function),
comparison test (which involve coming up with another series), or
limit comparison test (which involves coming up with another series
and taking a limit).

\section*{More tests}

Let $a_n \geq 0$, and $\lim_{n \to \infty} {a_n}^{1/n} = L$.  Then if
$L < 1$, then $\sum a_n$ converges, and if $L > 1$, then $\sum a_n$
diverges.  If $L = 1$, inconclusive.

Proof: suppose $L < 1$.  Choose $\epsilon$ small so that $L + \epsilon
< 1$.  Then for $n$ large,
$$
{a_n}^{1/n} < L + \epsilon
$$
so for $n$ large, $a_n < (L + \epsilon)^n$.  By comparison with $\sum
(L + \epsilon)^n$, the series converges.

On the other hand, if $L > 1$, then choose $\epsilon$ small so that $L
- \epsilon > 1$.  Again, for $n$ large, ${a_n}^{1/n} > L - \epsilon$,
so ${a_n} > (L - \epsilon)^n$.  Since $\sum (L - \epsilon)^n$
diverges, comparison proves $\sum a_n$ diverges.

\subsection*{Example}

$$
\sum \frac{1}{n^n}.
$$
Or again,
$$\sum \frac{1}{(\cos n + \sin \log n + n)^n}.$$

counterexample: $\sum 1/n$ is not detected, since $\lim 1/n^{1/n} = 1$.  The same is true of $\sum 1/n^2$.

This is nothing more than an application of the comparison test (or,
if you like, a modification of the limit comparison test) with a
geometric series.

\end{document}
