\documentclass[12pt]{article}
\usepackage{fullpage}
\usepackage{amsthm}
\usepackage{amsmath}

\newtheorem*{example}{Example}
\newtheorem*{thm}{Theorem}

\title{Lecture 17: More Taylor series}
\author{Math 153 Section 57}
\date{Wednesday November  5, 2008}

\begin{document}
\maketitle

Continuing in chapter 12.6.

\subsection{Recall}

\begin{itemize}
\item Start with a function $f$
\item Write down a Taylor series
\item Does the series converge?
\item Does the series converge to $f$?
\end{itemize}

How to find a series?  Differentiate the function.  Or, maybe use a
trick: substitute (e.g., $1/(1+x^2)$) or multiply two well-known
series together.  As we see soon, differentiation also works.

\subsection{Taylor series}

promised proofs.

\subsection{Approximate $\pi$}

Here is a series
$$
\arctan x = \sum_{n=0}^\infty \frac{(-1)^n x^{2n+1}}{2n + 1}
$$
Since $\arctan 1 = \pi/4$, we might try to approximate $\pi$ this way.



\subsection{Things you can see from the series directly}

For example, $\sin (-x) = - \sin x$ and $\cos x = \cos (-x)$.

\subsection{Relationship between $e^x$ and trig functions}

$e^{ix} = \cos x + i \sin x$.

\subsection{$e$ is irrational}

Assume $e = a/b$.

Clever trick: define
$$
x = b! ( e - \sum_{n=0}^b 1/n! )
$$

Step 1: $x$ is an integer.  Substitute in $e = a/b$ to see this.

Step 2: $x$ is between 0 and $1$.  
Then,
$$
0 < x = \sum_{n=b+1}^\infty \frac{b!}{n!} \leq \sum_{n=b+1}^\infty \frac{1}{(b+1)^{n-b}} = 1/b
$$
But there is no integer between 0 and 1.

Challenge: do something similar for $e^2$.

\end{document}
