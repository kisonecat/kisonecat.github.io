\documentclass[12pt]{article}

\usepackage{hyperref}
\usepackage{fullpage}
\usepackage{nopageno}
\usepackage{amsthm}
\usepackage{amsmath}
\usepackage{amssymb}
\newcommand{\R}{\mathbb{R}}
\newcommand{\N}{\mathbb{N}}
\usepackage[margin=1in]{geometry}
\usepackage{wasysym}
\usepackage{add-copyright}

\usepackage{graphicx}

\title{Homework 7}
\date{Due Friday, October 24, 2008}

\long\def\symbolfootnote[#1]#2{\begingroup%
\def\thefootnote{\fnsymbol{footnote}}\footnote[#1]{#2}\endgroup}

\begin{document}
\maketitle

\begin{description}

\item[(a)] Give an example of sequences $a_n$ and $b_n$ so that
$\displaystyle\sum_{n=1}^\infty a_n$ and
$\displaystyle\sum_{n=1}^\infty b_n$ both diverge, but
$\displaystyle\sum_{n=1}^\infty \left( a_n + b_n \right)$ converges.

\vfill

\item[(b)] Dramatically, Alice and Bob are driving very slow cars, and they are
on a collision course, heading toward each other at the constant speed
of 5~mph.  Alice and Bob are currently 5~miles apart, when a bee
(travelling at 10~mph) begins to fly from Alice's bumper, to Bob's
bumper, back to Alice's bumper, back to Bob's bumper, and so
on---until that fateful moment when the bee will be squished.
\textbf{How far will the bee travel in its journey?}
\begin{description}
\item[Part (i)] Write down a \textit{geometric series} giving the
total distance the bee travels.
\item[Part (ii)] Use a trick to compute the answer instantly (i.e., use
  the speed of the bee, and the time for which the bee is flying, to
  calculate the value of the geometric series).
\end{description}

\vfill

\item[(c)] The government prints a new \$1 bill, and gives it to you.
  You deposit it with a bank; the bank keeps 10\cent\ on reserve, and
  loans 90\cent\ to another customer.  This customer deposits the
  90\cent\ in another bank, which again keeps 10\% on reserve (i.e.,
  keeps 9\cent), and loans out the remaining 81\cent\ldots

  Each time the government prints a dollar bill, how much more more
  ``money'' appears?

\vfill

\item[(d)] On page 584, section 12.2, problems 5, 7, 8, 33.

\end{description}

\end{document}
