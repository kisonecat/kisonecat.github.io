\documentclass[11pt]{article}

\usepackage{fullpage}
\usepackage{nopageno}
\usepackage{hyperref}
\usepackage{amsmath}
\usepackage[margin=1cm]{geometry}
\usepackage{add-copyright}

\title{The 10-adic numbers}

\begin{document}

\section*{The reasonableness of the ridiculous.}

If $\sum_{n=0}^\infty x^n$ converges to $L$, then consider the following.
$$
x \cdot L = x \cdot \sum_{n=0}^\infty x^n = \sum_{n=0}^\infty x^{n+1} = \sum_{n=1}^\infty x^n = L - 1.
$$
So we can solve for $L$, to find that $L = \displaystyle\frac{1}{1 -
  x}$.  Of course, this argument \textbf{assumes that the series
  converges}.

What if we did this when the series did not converge?  We might deduce that
$$
\sum_{n=0}^\infty 10^n = \frac{1}{1 - 10} = - \frac{1}{9}.
$$
Of course, this is incorrect, because the series $\sum_{n=0}^\infty
10^n$ diverges.  But even if this is ridiculous, there is a way in
which it makes sense.

\section*{The 10-adic numbers}

We are comfortable with numbers that ``go all the way to the right''
(i.e., non-terminating decimals like $0.3333\cdots$) so why not
numbers that go all the way to the \textbf{left}?

I mean, consider a ``number'' like $\cdots 999999$, meaning
$\sum_{n=0}^\infty 9 \cdot 10^n$.  Of course, this is meaningless, but
if we \textbf{ignore convergence issues} and apply the formula for
geometric series, we might be fooled into thinking $\cdots 999999 =
-1$.  After all, $\cdots 999999$ means $\sum_{n=0}^\infty 9 \cdot 10^n$.

This is less ridiculous than it seems, because
$$
\begin{array}{cr}
  & \cdots 99999999 \\
+ & 1 \\
\hline
 & \cdots  00000000
\end{array}
$$
This also looks ridiculous, but just apply the usual algorithm for
addition: add $9$ and $1$, get $10$, write down the $0$ and carry the
$1$---and repeat.  The answer is all zeroes.  A number which equal
zero when we add $1$ to it ought to be given a name: $-1$.  For
similar reasons, we might believe $\cdots 11111111 = -1/9$, because if
we multiply $\cdots 11111111$ by $9$, we get the number for $-1$.

We can show $-1 \times -1 = 1$, because
$$
\begin{array}{cr}
       & \cdots 99999999 \\
\times & \cdots 99999999 \\
\hline
 & \cdots 99999991 \\
 & \cdots 99999910 \\
 & \cdots 99999100 \\
 & \cdots 99991000 \\
 & \vdots \hspace{1em} \\
\hline
 & \cdots 00000001
\end{array}
$$

\subsection*{How about one third?}

There are other examples in this crazy world, too.  Because
$$
\begin{array}{cr}
       & \cdots 66666667 \\
\times & 3 \\
\hline
 & \cdots 00000001
\end{array}
$$
we decide that $\cdots 66666667$ deserves to be called $1/3$, since it
is a multiplicative inverse for $3$.  But there is another reason why
$\cdots 66666667$ deserves the name $1/3$.  After all, if $\cdots
11111111 = -1/9$, then $\cdots 66666666$ is $-6/9 = -2/3$.  And
therefore,
$$
\begin{array}{crl}
       & \cdots 66666666 & \mbox{(think ``$-2/3$'')}\\
\times & 1 \\
\hline
 & \cdots 66666667 & \mbox{(think ``$1/3$'')}
\end{array}
$$

\subsection*{How about other negative numbers?}

What happens if we multiply $-1$ by $17$.  We ought to get $-17$.  And
indeed, we do:
$$
\begin{array}{cr}
       & \cdots 99999999 \\
\times & 17 \\
\hline
  & \cdots 99999993 \\
+ & \cdots 99999990 \\
\hline
 & \cdots  99999983
\end{array}
$$
And of course, if we add $17$ to $\cdots 99999983$, we get zero.  This
trick of handling negative numbers is called two's complement
addition\footnote{For more, see \url{http://en.wikipedia.org/wiki/Twos_complement}}

\subsection*{How about one seventh?}

I wanted to write down $1/7$, so I started with a $3$ (since $3 \times
7 = 21$, and this will give me the $1$ on the right hand side).  The
next digit should be a $4$, because $4 \times 7 = 28$, and since I had
to carry that $2$, I will get $30$, which means I will write down a
zero.  Now I am carrying a $3$; but if I put a $1$ as the next digit,
then $1 \times 7 + 3 = 10$, so I will write down a zero, and carry a
$1$.  Each time the next digit is chosen so that I write down a zero,
and carry something.  Continuing in this way, I discover:
$$
\begin{array}{cr}
       & \cdots 2857142857142857142857143 \\
\times &  7 \\
\hline
 & \cdots 0000000000000000000000001 \\
\end{array}
$$
This is a repeating decimal: we might write it as
$\overline{285714}3$, though here the digits repeat to the left.
This means we could also write it as a series, formally:
$$
3 + 10 \cdot \left( 285714 \cdot \sum_{n=0}^\infty 1000000^n \right)
$$
If we ignore convergence, and apply the formula for geometric
series here, where it does not apply, we might be fooled into thinking
$$
\cdots 2857142857142857143 = 3 + 10 \cdot \left( 285714 \cdot \frac{1}{1 - 1000000} \right) = 1/7
$$
The trouble is that these series do not converge.  But if we changed
our notion of convergence\ldots That is a subject for a future course.

\subsection*{The point of all this?}

Many mathematical advances started by taking a ridiculous idea (e.g.,
negative numbers, imaginary numbers) and making sense of the absurd.
The numbers we have described here are called $10$-adic numbers, and,
I admit, they do not work very well (terrifyingly, there exist two
non-zero $10$-adic numbers which multiply to give zero); if we instead
worked in base $p$ for $p$ a prime, we would get the $p$-adic numbers,
and these numbers turn out to work much better\footnote{For more, see
  \url{http://en.wikipedia.org/wiki/P-adic_number}}

\end{document}
