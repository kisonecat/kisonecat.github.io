\documentclass[12pt]{article}
\usepackage{fullpage}
\usepackage{amsthm}
\usepackage{amsmath}

\newtheorem*{example}{Example}
\newtheorem*{thm}{Theorem}

\title{Lecture 7: More indeterminate forms}
\author{Math 153 Section 57}
\date{Monday October 13, 2008}

\begin{document}
\maketitle

Following chapter 11.6.

\section{Trig formulas}

Introduce $e^{i\theta} = \cos \theta + i \sin \theta$.  Derive
all the rules for sine and cosine.

With the correct perspective, you need to remember nothing, and derive
everything.  Practically, in your future lives, you probably wouldn't
remember tables of sines and cosines and derivatives and blah, but if
you learn the ideas, then you won't have.

\section{Trig functions and l'H\^opital's rule}

We want to calculate $\lim_{x \to 0} \frac{\sin x}{x}$.  Let's use
l'H\^opital's rule.

Now we need to know the derivative of sine.  So by definition,
$$
\lim_{h \to 0} \frac{\sin (x+h) - \sin x}{h} =
\lim_{h \to 0} \frac{\sin (h)}{h}
$$
I guess now we use l'H\^opital's rule again? Uhm...

\section{What does converge to $\infty$ mean?}

We say $f(x) \to \infty$ as $x \to a$ if for all $N$, there is $\delta$, so that if $0 < |x-a| < \delta$, then $f(x) > N$.

As big as we want $f$ to be, there's a $\delta$ so that if we're $\delta$ close to $a$, then $f(x)$ is that big.

\section{Tips about L'H\^opital's rule}

Only apply it to $0/0$ or $\infty/\infty$.  If you can't do the limit,
and it isn't in that form, transform it into that form.

Do not apply it to $0/1$ or $1/0$ or $0 / \infty$ or $\infty / 0$.

Sometimes you need to do it more than once: e.g., $\sin^2 x /x^2$ (though here you could use the product formula).

\section{L'H\^opital's rule backwards}

$f(x) = x + \sin x$ and $g(x) = x$.  Then $f(x)/g(x) \to 1$ as $x \to
\infty$, but $f'(x)/g'(x)$ does not converge.

\section{Review L'H\^opital's rule}

If $f(x) \to \pm \infty$ and $g(x) \to \pm \infty$, and $g'(x) \neq 0$
for all $x$ near $a$, then if $f'(x) / g'(x) \to L$, then $f(x)/g(x) \to
L$.

Intuition: limits of fractions measure how fast one is growing
compared to the other.

Examples: $\log x/x^n \to 0$ as $x \to \infty$.  Think: logrithms grow
very slowly.

Example: $e^x / x^n \to \infty$.  Apply L'H\^opital many times.

\section{$\infty - \infty$}

Example: $x - \log x = x(1 - \log x /x)$.

\section{$0^0$}

Show $\lim_{x \to 0^{+}} x^x = 1$.  Take logs to do $\lim_{x \to 0^{+}} x \log x = \lim \log x / (1/x) = \lim (1/x)/(-1/x^2) = \lim -x = 0$.

\section{Proof}

Apply mean-value theorem to
$$
H(x) = (g(b) - g(a)) ( f(x) - f(a) ) - (g(x) - g(a)) ( f(b) - f(a) )
$$
on the interval $(a,b)$.  Get a point where $H'(r) = 0$, which means
$$
f'(r)/g'(r) = (f(b) - f(a)) / (g(b) - g(a))
$$

Suppose $f(a) = 0$ and $g(a) = 0$.  Then on the interval $(a, a+\epsilon)$ we have some number $r \in (a,a+\epsilon)$ so that
$$
\frac{f(x)}{g(x)} = \frac{f(x) - f(a)}{g(x) - g(a)} = \frac{f'(r)}{g'(r)}
$$

\section{The $0^0$ controversy}

What does $0^0$ mean?  Well, what does $0/0$ mean...

\end{document}
