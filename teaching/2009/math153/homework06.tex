\documentclass[12pt]{article}

\usepackage{hyperref}
\usepackage{fullpage}
\usepackage{nopageno}
\usepackage{amsthm}
\usepackage{amsmath}
\usepackage{amssymb}
\newcommand{\R}{\mathbb{R}}
\newcommand{\N}{\mathbb{N}}
\usepackage[margin=1in]{geometry}
\usepackage{add-copyright}

\usepackage{graphicx}

\title{Homework 6}
\date{Due Friday, October 17, 2008}

\long\def\symbolfootnote[#1]#2{\begingroup%
\def\thefootnote{\fnsymbol{footnote}}\footnote[#1]{#2}\endgroup}

\begin{document}
\maketitle

\textbf{Remember:} October 20 is the date of the first midterm!

\begin{description}

\item[(a)] On page 571, section 11.7, do problems: 1, 6, 9, 10, 17, 31.

\vfill

\item[(b)] Consider the graph of $f(x) = 1/x$, restricted to $x \in
  [1,\infty)$.  Rotate this graph around the $x$-axis, to produce
\begin{center}
\includegraphics[width=4in]{GabrielHorn.pdf}
\end{center}
The surface area of this object is infinite, but what is its volume?

\vfill

\item[(c)] The \textbf{Gamma function} is defined as follows:
$$
\Gamma(z) = \int_0^\infty t^{z-1} \, e^{-t} \, dt.
$$
Compute $\Gamma(5)$.  \textit{Hint: first compute $\Gamma(1)$, and then integrate by parts to find $\Gamma(z)$ in terms of $\Gamma(z-1)$.}\symbolfootnote[2]{If you are very stuck, look at \texttt{\url{http://en.wikipedia.org/wiki/Gamma_function}}}

\vfill

\item[(d)] Evaluate $\displaystyle\int_0^{\pi/2} \displaystyle\frac{x}{\tan x} \, dx$.
This is an extremely difficult integral.

\vfill

% http://topologicalmusings.wordpress.com/2008/10/12/solution-to-pow-10-another-hard-integral/
% http://en.wikipedia.org/wiki/Differentiation_under_the_integral_sign#Other_problems

%\vfill
%
%\item[(d)] Evaluate the following improper integral
%$$
%\int_0^\infty \frac{\sin x}{x} \, dx
%$$
%by using the following trick.  Define
%$$
%f(t) = \int_0^\infty e^{-tx} \, \frac{\sin x}{x} \, dx
%$$
%But then
%\begin{eqnarray*}
%f'(t) &=& \frac{d}{dt} \int_0^\infty e^{-tx} \, \frac{\sin x}{x} \, dx \\
% &=& \int_0^\infty \frac{d}{dt} e^{-tx} \, \frac{\sin x}{x} \, dx \\
%&=& \int_0^\infty -x \cdot e^{-tx} \, \frac{\sin x}{x} \, dx \\
%&=& \int_0^\infty -e^{-tx} \cdot \sin x \, dx \\
%&=& \frac{1}{1+t^2}.
%\end{eqnarray*}
%So $f(t) = \arctan t + C$.  But $\lim_{t \to -\infty} f(t) = 0$, so
%$f(0) = \pi/2$.

\end{description}

\pagebreak
\null

\end{document}
