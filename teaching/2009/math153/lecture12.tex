\documentclass[12pt]{article}
\usepackage{fullpage}
\usepackage{amsthm}
\usepackage{amsmath}

\newtheorem*{example}{Example}
\newtheorem*{thm}{Theorem}

\title{Lecture 12: Convergence}
\author{Math 153 Section 57}
\date{Friday October 24, 2008}

\begin{document}
\maketitle

Following chapters 12.2 and 12.3.

\section*{Limit of terms of convergent series}

\section*{Convergence by boundedness}

Theorem: series with nonnegative terms converges iff bounded above

\section*{Integral test}

If $f$ continuous, positive, decreasing on $[1,\infty)$, then
$$
\sum_{k=1}^\infty f(k)
$$
iff $\int_1^\infty f(x) \, dx$ converges (i.e., as an improper integral).

Draw some pictures

apply integral test to prove harmonic series diverge.

prove harmonic series diverges another way.

apply integral test to $p$-series.

\section*{Comparison theorem}

if $a_k \geq 0$ and $b_k \geq 0$, and for $k$ large,
$$
\sum a_k \leq \sum b_k
$$
Then $\sum b_k$ converges implies $\sum a_k$ converges.

Example: $\sum 1/(2k^2 + 3)$

Example: $\sum 1/\log (k+5)$ by compariing with $1/2k$.

\end{document}
