\batchmode
\documentclass[12pt]{article}
\RequirePackage{ifthen}


\usepackage{fullpage}
\usepackage{amsthm}
\usepackage{amsmath}


\usepackage{amssymb}

%
\providecommand{\N}{\mathbb{N}}%
\providecommand{\Q}{\mathbb{Q}}%
\providecommand{\Z}{\mathbb{Z}}%
\providecommand{\R}{\mathbb{R}} 



\newtheorem{example}{Example} 
\newtheorem{thm}{Theorem} 

\title{Lecture 4: More about limits}
\author{Math 153 Section 57}
\date{Monday October  6, 2008}




\usepackage[dvips]{color}


\pagecolor[gray]{.7}

\usepackage[]{inputenc}



\makeatletter

\makeatletter
\count@=\the\catcode`\_ \catcode`\_=8 
\newenvironment{tex2html_wrap}{}{}%
\catcode`\<=12\catcode`\_=\count@
\newcommand{\providedcommand}[1]{\expandafter\providecommand\csname #1\endcsname}%
\newcommand{\renewedcommand}[1]{\expandafter\providecommand\csname #1\endcsname{}%
  \expandafter\renewcommand\csname #1\endcsname}%
\newcommand{\newedenvironment}[1]{\newenvironment{#1}{}{}\renewenvironment{#1}}%
\let\newedcommand\renewedcommand
\let\renewedenvironment\newedenvironment
\makeatother
\let\mathon=$
\let\mathoff=$
\ifx\AtBeginDocument\undefined \newcommand{\AtBeginDocument}[1]{}\fi
\newbox\sizebox
\setlength{\hoffset}{0pt}\setlength{\voffset}{0pt}
\addtolength{\textheight}{\footskip}\setlength{\footskip}{0pt}
\addtolength{\textheight}{\topmargin}\setlength{\topmargin}{0pt}
\addtolength{\textheight}{\headheight}\setlength{\headheight}{0pt}
\addtolength{\textheight}{\headsep}\setlength{\headsep}{0pt}
\setlength{\textwidth}{349pt}
\newwrite\lthtmlwrite
\makeatletter
\let\realnormalsize=\normalsize
\global\topskip=2sp
\def\preveqno{}\let\real@float=\@float \let\realend@float=\end@float
\def\@float{\let\@savefreelist\@freelist\real@float}
\def\liih@math{\ifmmode$\else\bad@math\fi}
\def\end@float{\realend@float\global\let\@freelist\@savefreelist}
\let\real@dbflt=\@dbflt \let\end@dblfloat=\end@float
\let\@largefloatcheck=\relax
\let\if@boxedmulticols=\iftrue
\def\@dbflt{\let\@savefreelist\@freelist\real@dbflt}
\def\adjustnormalsize{\def\normalsize{\mathsurround=0pt \realnormalsize
 \parindent=0pt\abovedisplayskip=0pt\belowdisplayskip=0pt}%
 \def\phantompar{\csname par\endcsname}\normalsize}%
\def\lthtmltypeout#1{{\let\protect\string \immediate\write\lthtmlwrite{#1}}}%
\newcommand\lthtmlhboxmathA{\adjustnormalsize\setbox\sizebox=\hbox\bgroup\kern.05em }%
\newcommand\lthtmlhboxmathB{\adjustnormalsize\setbox\sizebox=\hbox to\hsize\bgroup\hfill }%
\newcommand\lthtmlvboxmathA{\adjustnormalsize\setbox\sizebox=\vbox\bgroup %
 \let\ifinner=\iffalse \let\)\liih@math }%
\newcommand\lthtmlboxmathZ{\@next\next\@currlist{}{\def\next{\voidb@x}}%
 \expandafter\box\next\egroup}%
\newcommand\lthtmlmathtype[1]{\gdef\lthtmlmathenv{#1}}%
\newcommand\lthtmllogmath{\dimen0\ht\sizebox \advance\dimen0\dp\sizebox
  \ifdim\dimen0>.95\vsize
   \lthtmltypeout{%
*** image for \lthtmlmathenv\space is too tall at \the\dimen0, reducing to .95 vsize ***}%
   \ht\sizebox.95\vsize \dp\sizebox\z@ \fi
  \lthtmltypeout{l2hSize %
:\lthtmlmathenv:\the\ht\sizebox::\the\dp\sizebox::\the\wd\sizebox.\preveqno}}%
\newcommand\lthtmlfigureA[1]{\let\@savefreelist\@freelist
       \lthtmlmathtype{#1}\lthtmlvboxmathA}%
\newcommand\lthtmlpictureA{\bgroup\catcode`\_=8 \lthtmlpictureB}%
\newcommand\lthtmlpictureB[1]{\lthtmlmathtype{#1}\egroup
       \let\@savefreelist\@freelist \lthtmlhboxmathB}%
\newcommand\lthtmlpictureZ[1]{\hfill\lthtmlfigureZ}%
\newcommand\lthtmlfigureZ{\lthtmlboxmathZ\lthtmllogmath\copy\sizebox
       \global\let\@freelist\@savefreelist}%
\newcommand\lthtmldisplayA{\bgroup\catcode`\_=8 \lthtmldisplayAi}%
\newcommand\lthtmldisplayAi[1]{\lthtmlmathtype{#1}\egroup\lthtmlvboxmathA}%
\newcommand\lthtmldisplayB[1]{\edef\preveqno{(\theequation)}%
  \lthtmldisplayA{#1}\let\@eqnnum\relax}%
\newcommand\lthtmldisplayZ{\lthtmlboxmathZ\lthtmllogmath\lthtmlsetmath}%
\newcommand\lthtmlinlinemathA{\bgroup\catcode`\_=8 \lthtmlinlinemathB}
\newcommand\lthtmlinlinemathB[1]{\lthtmlmathtype{#1}\egroup\lthtmlhboxmathA
  \vrule height1.5ex width0pt }%
\newcommand\lthtmlinlineA{\bgroup\catcode`\_=8 \lthtmlinlineB}%
\newcommand\lthtmlinlineB[1]{\lthtmlmathtype{#1}\egroup\lthtmlhboxmathA}%
\newcommand\lthtmlinlineZ{\egroup\expandafter\ifdim\dp\sizebox>0pt %
  \expandafter\centerinlinemath\fi\lthtmllogmath\lthtmlsetinline}
\newcommand\lthtmlinlinemathZ{\egroup\expandafter\ifdim\dp\sizebox>0pt %
  \expandafter\centerinlinemath\fi\lthtmllogmath\lthtmlsetmath}
\newcommand\lthtmlindisplaymathZ{\egroup %
  \centerinlinemath\lthtmllogmath\lthtmlsetmath}
\def\lthtmlsetinline{\hbox{\vrule width.1em \vtop{\vbox{%
  \kern.1em\copy\sizebox}\ifdim\dp\sizebox>0pt\kern.1em\else\kern.3pt\fi
  \ifdim\hsize>\wd\sizebox \hrule depth1pt\fi}}}
\def\lthtmlsetmath{\hbox{\vrule width.1em\kern-.05em\vtop{\vbox{%
  \kern.1em\kern0.8 pt\hbox{\hglue.17em\copy\sizebox\hglue0.8 pt}}\kern.3pt%
  \ifdim\dp\sizebox>0pt\kern.1em\fi \kern0.8 pt%
  \ifdim\hsize>\wd\sizebox \hrule depth1pt\fi}}}
\def\centerinlinemath{%
  \dimen1=\ifdim\ht\sizebox<\dp\sizebox \dp\sizebox\else\ht\sizebox\fi
  \advance\dimen1by.5pt \vrule width0pt height\dimen1 depth\dimen1 
 \dp\sizebox=\dimen1\ht\sizebox=\dimen1\relax}

\def\lthtmlcheckvsize{\ifdim\ht\sizebox<\vsize 
  \ifdim\wd\sizebox<\hsize\expandafter\hfill\fi \expandafter\vfill
  \else\expandafter\vss\fi}%
\providecommand{\selectlanguage}[1]{}%
\makeatletter \tracingstats = 1 
\providecommand{\Beta}{\textrm{B}}
\providecommand{\Mu}{\textrm{M}}
\providecommand{\Kappa}{\textrm{K}}
\providecommand{\Rho}{\textrm{R}}
\providecommand{\Epsilon}{\textrm{E}}
\providecommand{\Chi}{\textrm{X}}
\providecommand{\Iota}{\textrm{J}}
\providecommand{\omicron}{\textrm{o}}
\providecommand{\Zeta}{\textrm{Z}}
\providecommand{\Eta}{\textrm{H}}
\providecommand{\Nu}{\textrm{N}}
\providecommand{\Omicron}{\textrm{O}}
\providecommand{\Tau}{\textrm{T}}
\providecommand{\Alpha}{\textrm{A}}


\begin{document}
\pagestyle{empty}\thispagestyle{empty}\lthtmltypeout{}%
\lthtmltypeout{latex2htmlLength hsize=\the\hsize}\lthtmltypeout{}%
\lthtmltypeout{latex2htmlLength vsize=\the\vsize}\lthtmltypeout{}%
\lthtmltypeout{latex2htmlLength hoffset=\the\hoffset}\lthtmltypeout{}%
\lthtmltypeout{latex2htmlLength voffset=\the\voffset}\lthtmltypeout{}%
\lthtmltypeout{latex2htmlLength topmargin=\the\topmargin}\lthtmltypeout{}%
\lthtmltypeout{latex2htmlLength topskip=\the\topskip}\lthtmltypeout{}%
\lthtmltypeout{latex2htmlLength headheight=\the\headheight}\lthtmltypeout{}%
\lthtmltypeout{latex2htmlLength headsep=\the\headsep}\lthtmltypeout{}%
\lthtmltypeout{latex2htmlLength parskip=\the\parskip}\lthtmltypeout{}%
\lthtmltypeout{latex2htmlLength oddsidemargin=\the\oddsidemargin}\lthtmltypeout{}%
\makeatletter
\if@twoside\lthtmltypeout{latex2htmlLength evensidemargin=\the\evensidemargin}%
\else\lthtmltypeout{latex2htmlLength evensidemargin=\the\oddsidemargin}\fi%
\lthtmltypeout{}%
\makeatother
\setcounter{page}{1}
\onecolumn

% !!! IMAGES START HERE !!!

\stepcounter{section}
{\newpage\clearpage
\lthtmlinlinemathA{tex2html_wrap_inline242}%
$ \lim_{n \to \infty} a_n =
L$%
\lthtmlinlinemathZ
\lthtmlcheckvsize\clearpage}

{\newpage\clearpage
\lthtmlinlinemathA{tex2html_wrap_inline244}%
$ \epsilon > 0$%
\lthtmlinlinemathZ
\lthtmlcheckvsize\clearpage}

{\newpage\clearpage
\lthtmlinlinemathA{tex2html_wrap_inline246}%
$ K$%
\lthtmlinlinemathZ
\lthtmlcheckvsize\clearpage}

{\newpage\clearpage
\lthtmlinlinemathA{tex2html_wrap_inline248}%
$ n \geq
K$%
\lthtmlinlinemathZ
\lthtmlcheckvsize\clearpage}

{\newpage\clearpage
\lthtmlinlinemathA{tex2html_wrap_inline250}%
$ |a_n - L| < \epsilon$%
\lthtmlinlinemathZ
\lthtmlcheckvsize\clearpage}

\stepcounter{section}
{\newpage\clearpage
\lthtmlinlinemathA{tex2html_wrap_inline253}%
$ a_n = (-1)^n$%
\lthtmlinlinemathZ
\lthtmlcheckvsize\clearpage}

\stepcounter{section}
{\newpage\clearpage
\lthtmlinlinemathA{tex2html_wrap_inline258}%
$ \lim_{n
  \to \infty} a_n = M$%
\lthtmlinlinemathZ
\lthtmlcheckvsize\clearpage}

{\newpage\clearpage
\lthtmlinlinemathA{tex2html_wrap_inline260}%
$ L = M$%
\lthtmlinlinemathZ
\lthtmlcheckvsize\clearpage}

{\newpage\clearpage
\lthtmlinlinemathA{tex2html_wrap_inline264}%
$ m \in \mathbb{N}$%
\lthtmlinlinemathZ
\lthtmlcheckvsize\clearpage}

{\newpage\clearpage
\lthtmlinlinemathA{tex2html_wrap_inline266}%
$ \lim_{n \to \infty} a_{n+m} = L$%
\lthtmlinlinemathZ
\lthtmlcheckvsize\clearpage}

\stepcounter{subsection}
{\newpage\clearpage
\lthtmlinlinemathA{tex2html_wrap_inline269}%
$ a_1 = 2$%
\lthtmlinlinemathZ
\lthtmlcheckvsize\clearpage}

{\newpage\clearpage
\lthtmlinlinemathA{tex2html_wrap_inline271}%
$ a_{n+1} = a_n/2 + 1/a_n$%
\lthtmlinlinemathZ
\lthtmlcheckvsize\clearpage}

{\newpage\clearpage
\lthtmlinlinemathA{tex2html_wrap_inline273}%
$ \lim a_n = \sqrt{2}$%
\lthtmlinlinemathZ
\lthtmlcheckvsize\clearpage}

{\newpage\clearpage
\lthtmlinlinemathA{tex2html_wrap_inline275}%
$ a_1 \in (0,1)$%
\lthtmlinlinemathZ
\lthtmlcheckvsize\clearpage}

{\newpage\clearpage
\lthtmlinlinemathA{tex2html_wrap_inline277}%
$ a_{n+1} = a_n (1-a_n)$%
\lthtmlinlinemathZ
\lthtmlcheckvsize\clearpage}

{\newpage\clearpage
\lthtmlinlinemathA{tex2html_wrap_inline279}%
$ \lim a_n = 0$%
\lthtmlinlinemathZ
\lthtmlcheckvsize\clearpage}

{\newpage\clearpage
\lthtmlinlinemathA{tex2html_wrap_inline283}%
$ a_{n+1} = 3.5 a_n (1-a_n)$%
\lthtmlinlinemathZ
\lthtmlcheckvsize\clearpage}

\stepcounter{subsection}
{\newpage\clearpage
\lthtmlinlinemathA{tex2html_wrap_inline286}%
$ \lim a_n = L$%
\lthtmlinlinemathZ
\lthtmlcheckvsize\clearpage}

{\newpage\clearpage
\lthtmlinlinemathA{tex2html_wrap_inline288}%
$ \lim b_n = M$%
\lthtmlinlinemathZ
\lthtmlcheckvsize\clearpage}

{\newpage\clearpage
\lthtmlinlinemathA{tex2html_wrap_inline290}%
$ \lim (a_n + b_n) = L + M$%
\lthtmlinlinemathZ
\lthtmlcheckvsize\clearpage}

{\newpage\clearpage
\lthtmlinlinemathA{tex2html_wrap_inline294}%
$ \lim (c \cdot a_n) = c \cdot L$%
\lthtmlinlinemathZ
\lthtmlcheckvsize\clearpage}

{\newpage\clearpage
\lthtmlinlinemathA{tex2html_wrap_inline300}%
$ \lim (a_n \cdot b_n) = L \cdot M$%
\lthtmlinlinemathZ
\lthtmlcheckvsize\clearpage}

\stepcounter{subsection}
{\newpage\clearpage
\lthtmlinlinemathA{tex2html_wrap_inline307}%
$ a_n \leq b_n \leq
c_n$%
\lthtmlinlinemathZ
\lthtmlcheckvsize\clearpage}

{\newpage\clearpage
\lthtmlinlinemathA{tex2html_wrap_inline309}%
$ \lim a_n = \lim c_n = L$%
\lthtmlinlinemathZ
\lthtmlcheckvsize\clearpage}

{\newpage\clearpage
\lthtmlinlinemathA{tex2html_wrap_inline311}%
$ \lim b_n = L$%
\lthtmlinlinemathZ
\lthtmlcheckvsize\clearpage}

{\newpage\clearpage
\lthtmlinlinemathA{tex2html_wrap_inline313}%
$ \lim_{n \to \infty} 1/n^2 = 0$%
\lthtmlinlinemathZ
\lthtmlcheckvsize\clearpage}

\stepcounter{subsection}
\stepcounter{subsection}
{\newpage\clearpage
\lthtmlinlinemathA{tex2html_wrap_inline317}%
$ f : \mathbb{R}\to \mathbb{R}$%
\lthtmlinlinemathZ
\lthtmlcheckvsize\clearpage}

{\newpage\clearpage
\lthtmlinlinemathA{tex2html_wrap_inline319}%
$ L$%
\lthtmlinlinemathZ
\lthtmlcheckvsize\clearpage}

{\newpage\clearpage
\lthtmlinlinemathA{tex2html_wrap_inline323}%
$ \lim f(a_n) = f(L)$%
\lthtmlinlinemathZ
\lthtmlcheckvsize\clearpage}

{\newpage\clearpage
\lthtmlinlinemathA{tex2html_wrap_inline325}%
$ f$%
\lthtmlinlinemathZ
\lthtmlcheckvsize\clearpage}


\end{document}
