\documentclass[12pt]{article}
\usepackage{fullpage}
\usepackage{nopageno}
\usepackage{amsmath}
\usepackage{amssymb}
\usepackage{amsthm}
\usepackage{multicol} 
\newcommand{\R}{\mathbb{R}}

\usepackage{add-copyright}

\newtheorem*{example}{Example}
\newtheorem*{thm}{Theorem}

\title{Midterm 1: Frequently Asked Questions}
\author{Math 195 Section 91}
\date{Wednesday July 1, 2009}

\begin{document}
\maketitle

\section*{What will the exam be like?}

There will be \textbf{11 questions} on the exam and it will last
\textbf{60 minutes}.  It will be worth 225 points.

The last question will be \textbf{extra credit} with true/false
questions.  I think these are fun; I will ask you to agree and
disagree with mathematical statements.

\section*{Can I use my calculator?}

Calculators are \textbf{forbidden}; I will write the exam in such a
way that calculators will not be necessary.

\section*{How should I write down my answers?}

You are not giving answers: you are giving an explanation.  For full
credit, \textbf{justify} your arguments by \textbf{showing} the steps
you took: if the central issue in the problem is that the derivative
of the sum is the sum of the derivatives, you should be sure to point
that out.

%You will lose points if you surround an otherwise convincing argument
%with false statements (after all, once I have proved $2 = 1$, I can
%prove anything!).  \textbf{Erase} untrue statements for full credit.

Style matters is important: do not surround an otherwise convincing
argument with false statements.  Do not use an ``$=$'' between two
expressions unless they are, in fact, equal.  Do not \textbf{under any
  circumstances} divide by zero during the exam.

\section*{What might I have to do on this exam?}

I could ask you about anything we have covered thus far, and that covers a lot of mathematics!
\begin{multicols}{2}
\begin{itemize}
\item Sketch a graph of a curve by finding a few points and differentiating.
\item Find the slope of a tangent line to a curve given
  in terms of a parameter.
\item Find the arc length of a curve.
\item Convert polar coordinates to cartesian coordinates.
\item Find the slope of a tangent line to a curve given in polar coordinates.
\item Find distance between points in $\R^2$, $\R^3$, and $\R^n$.
\item Given an equation for a sphere, find its center and radius.
\item Write vectors as a linear combination of other vectors.
\item Give a geometric interpretation of adding, subtracting, and scaling vectors.
\item Compute the dot product of two vectors.
\item Perform algebra with the dot product.
\item Interpret the dot product geometrically.
\item Normalize a vector (so that it has unit length).
\item Find the length of a vector.
\item Determine whether vectors are orthogonal.
\item Find the angle between two vectors.
\item Compute the cross product of two vectors in $\R^3$.
\item Interpret the cross product as the area of a parallelogram.
\item Find an equation for a line through a given point and in a given direction.
\item Find an equation for a plane through a given point and with a given normal vector.
\item Determine whether two lines intersect, are parallel, or are skew.
\item Find the point of intersection between a line and a plane.
\item Sketch the graph of a vector-valued function.
\item Differentiate and integrate vector-valued functions.
\item Caclulate unit tangent vectors to a curve.
\item Find the angle of intersection between curves.
\end{itemize}
\end{multicols}

\subsection*{Can you give me a hint?}

Here is a hint: I will not ask you to sketch any graphs on this
exam (sketching a graph takes a long time!), but I will ask you to
identify the graph of a function from among a collection of graphs.

\section*{What trigonometric formulas should I memorize?}

You will not need to memorize any trigonometric formulas; you should know, however, that
$$
\frac{d}{dx} \sin x = \cos x, \hspace{1em} 
\frac{d}{dx} \cos x = -\cos x, \hspace{1em} 
$$
and that $\sin^2 x + \cos^2 x = 1$.

\end{document}