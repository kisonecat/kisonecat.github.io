\documentclass[12pt]{article}
\usepackage{fullpage}
\usepackage{amsthm}
\usepackage{amsmath}

\newtheorem*{example}{Example}
\newtheorem*{thm}{Theorem}

\title{Lecture 1: Welcome, parametric equations}
\author{Math 195 Section 91}
\date{Monday June 22, 2009}

\begin{document}
\maketitle

Goal: Sections 11.1, and 11.2.

\subsection*{Welcome to the course}

\subsection*{Syllabus}

\subsection*{Goal}

Differentiation as a ``wiggle''

Machine with $n$ inputs and $k$ outputs---wiggle one input.

Goal: Expand both our intuition and our technical skill from one
variable to many variables.

Stop using your calculator.  You are Neo, and when you are good enough...

The book does a lot of great examples which are more complicated than
what I intend to do in class.  You should look at these examples.  You
should read the book.  In contrast, I will tell you the secrets that
this book---that every book---leaves out.

\subsection*{Objects}

Review: functions---what is a function?

We often broad the class of objects under consideration (history of
``number'')

Introduce section 11.1: Parametric curve (i.e., curve defind using a parameter)

example: $x(t) = 3t$ and $y(t) = 4t$---remark the goal of calculus to reduce evertying to a straight line.

example: $x(t) = t^2$ and $y(t) = t^3$---what happens at 0?

example: $x(t) = \sin t$ and $y(t) = \cos t$---why is that a circle?

concerned more with patterns, with qualitative data, than quantative
results.  A common problem is the following: you have a parametric
equation, and you must find a cartesian equation for part of the
curve.

Better terminology than the books: reparameterization---sweep out the same curve, but in a different amount of time

\subsection*{Thinking about parameters}

Think about the parameter as looking down at a 3-dimensional object
from above...

sometimes we might have more than one parameter: example: $x(u,t) = u \cdot \sin t$ and $y(u,t) = u \cdot \cos t$---a family of circles!

there's a great deal of flexibility about how you explore these objects

\section*{Calculus on curves}

take derivatives at a point

elliptic curve $x(t) = t^2$ and $y(t) = t^3 - 3t$.  graph through
$(0,0)$ and $(3,0)$ and $(1,2)$ and $(1,-2)$.

$dy/dx = (dy/dt)/(dx/dt)$.  Why?

calculate derivatives (and second derivatives) elsewhere to finish the sketch

better than using a calculator---we're not just sampling points!  We're using calculus to understand the qualitative features of a graph

what is the deriative at 0 when $x(t) = t^2$ and $y(t) = t^3$?

\section*{arc length}

under suitable conditions
$$
L = \int_a^b \sqrt{ x'(t)^2 + y'(t)^2 } \, dt
$$
confess: i could never remember this formula

challenge: try to find situations where the formula is incorrect.  this is the ``understand biology by killing things'' or ``learn how cars work by removing parts until the car stops working'' method

why should this be true?

example...  can we calculate the circumference of a circle?

\end{document}
