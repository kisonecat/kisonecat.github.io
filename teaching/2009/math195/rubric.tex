\documentclass[11pt]{article}

\usepackage{fullpage}
\usepackage{nopageno}

\title{Oral Exam Rubric}

\newcommand{\points}[1]{\fbox{\parbox{2em}{\null\vspace{1em}\hspace{2em}}}/\parbox{1em}{#1}}

\begin{document}

\section*{Midterm 2\hfill Name: \hspace{5em}\null}

\noindent
Compute a limit of a function of several variables
\dotfill
\points{20}
\vfill

\noindent
Convert a limit in cartesian coordinates to polar coordinates
\dotfill
\points{5}
\vfill

\noindent
Define continuity for functions of several variables
\dotfill
\points{5}
\vfill

\noindent
Give an example of a limit that does not exist
\dotfill
\points{5}
\vfill

\noindent
Compute partial derivatives
\dotfill
\points{25}
\vfill

\noindent
Compute the gradient
\dotfill
\points{10}
\vfill

\noindent
Compute the directional derivative
\dotfill
\points{10}
\vfill

\noindent
Compute higher partial derivatives
\dotfill
\points{10}
\vfill

\noindent
Illustrate that higher partials commute
\dotfill
\points{5}
\vfill

\noindent
Describe what a partial derivative is measuring
\dotfill
\points{5}
\vfill

\noindent
Write down a linear approximation to a function
\dotfill
\points{20}
\vfill

\noindent
Chain rule
\dotfill
\points{35}
\vfill

\noindent
Define global/local  max/min
\dotfill
\points{5}
\vfill

\noindent
Find critical points
\dotfill
\points{20}
\vfill

\noindent
Find max and min values
\dotfill
\points{20}
\vfill

\noindent
Apply second derivative test
\dotfill
\points{5}
\vfill

\noindent
Use Lagrange multipliers
\dotfill
\points{20}
\vfill

\vspace{1ex}
\hrule
\vspace{1ex}

\noindent
Total:
\dotfill
\points{225}
\vfill

\end{document}
