\documentclass[12pt]{article}
\usepackage{fullpage}
\usepackage{amsthm}
\usepackage{amsmath}

\newtheorem*{example}{Example}
\newtheorem*{thm}{Theorem}

\title{Lecture 4: Lines and planes}
\author{Math 195 Section 91}
\date{Monday June 29, 2009}

\begin{document}
\maketitle

Section 13.5, and 14.1 and 14.2.

parametric equation for line: $v = t w + u$.  parametric equation for line segment.

\section{Lines}

find line through two points

check if two lines intersect

skew lines versus parallel lines.

\section{Planes}

point on plane, and vector normal to the plane

equation: $n \cdot (v - w) = 0$ for $w$ on the plane and $n$ normal.

write as scalar equation by substitting in coordinates

find normal vector by taking cross product

two planes are parallel if their normals are parallel

find angle between two planes

find distance from point to a plane

\section{Calculus}

vector-valued function---in coordinates, a parametric equation!

take limits componentwise

sketch curve: $(\cos t, \sin t, t)$.

\section{Derivatives}

define derivative

take derivatives componentwise

find unit tangent vector

second derivative (on circle!)

how to differentiate: dot products, cross products.  it is the leibniz rule again!

\section{Integrating vectors}

Indeed we can do it.


\end{document}
