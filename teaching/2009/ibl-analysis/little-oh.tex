\documentclass[12pt]{article}

\usepackage{amsmath}
\usepackage{amsthm}
\usepackage{amssymb}

\usepackage{add-copyright}

\newcommand{\R}{\mathbb{R}}

\theoremstyle{plain}% default
\newtheorem{theorem}{Theorem}[section]
\newtheorem{lemma}[theorem]{Lemma}
\newtheorem{proposition}[theorem]{Proposition}
\newtheorem{corollary}{Corollary}
\theoremstyle{definition}
\newtheorem{definition}{Definition}[theorem]
\newtheorem{conjecture}{Conjecture}[theorem]
\newtheorem{example}{Example}[theorem]
\newtheorem{exercise}{Exercise}[theorem]
\newtheorem{remark}{Remark}
\newtheorem*{warning}{Warning}

\title{Little $o$ notation}
\author{Math 205}
\date{}

\begin{document}

\maketitle

\section{Overview}

There is no \textit{new} mathematics in this handout, but there is a
slightly different perspective.  We think of the linear approximation
(not the value of the derivative) as the fundamental object.

\section{Introduction}

Something that can make differentiation more beautiful is
\textit{little $o$ notation}.
\begin{definition}
We say ``$f$ is little-oh of $h$ as $x$
approaches $x_0$'' and write
$$
f(x) = o(h(x)) \mbox{ as $x \to x_0$}
$$
to mean that
$$
\lim_{x \to x_0} \frac{f(x)}{h(x)} = 0.
$$
\end{definition}
Intuitively, this means $f(x)$ is \textit{much} smaller than $h(x)$
for $x$ near $x_0$.

\begin{warning}
If $f(x) = o(h(x))$, it is not the case that $h(x) = o(f(x))$.
\end{warning}

\begin{example}
For instance,
\begin{itemize}
\item $x^2 = o(x)$ as $x \to 0$.
\item $x \neq o(x^2)$ as $x \to 0$.
\item $x - \sin x = o(x)$ as $x \to 0$.
\item $x - \sin x = o(x^2)$ as $x \to 0$.
\item $x - \sin x = o(x^2)$ as $x \to 0$.
\item $x - \sin x \neq o(x^3)$ as $x \to 0$.
\end{itemize}
\end{example}

\begin{definition}
For two functions $f, g : \R \to \R$, we say that $f(x) \sim_{x_0} g(x)$ provided
$$
f(x) - g(x) = o(x - x_0) \mbox{ as $x \to x_0$.}
$$
\end{definition}
\noindent
Unwrapping the definitions, this just means that
$$
\lim_{x \to x_0} \frac{f(x) - g(x)}{x - x_0} = 0
$$
Intuitively, $f(x) \sim_{x_0} g(x)$ means $f$ and $g$ look
``linearly'' the same around $x_0$.  More formally (or when trying to
intimidate others), we might say that the ``linear germ of $f$ and $g$
around $x_0$ is the same.''

\begin{exercise}
Prove that $\sim_{x_0}$ is an equivalence relation.
\end{exercise}

\begin{definition}
A \textbf{linear approximation} for $f(x)$ at $x_0$ is a linear function
$$
L(x) = ax + b
$$
so that $f(x) \sim_{x_0} L(x)$.
\end{definition}

\begin{exercise}
If $ax + b \sim_{x_0} cx + d$, then $a = c$ and $b = d$.
\end{exercise}

\begin{exercise}
Let $f : (a,b) \to \R$.  If $f$ has a linear approximation at $c$, then it is unique.
\end{exercise}

\begin{exercise}
Let $f : (a,b) \to \R$.  Then $f$ is differentiable at $c$ if and only if $f$ has a linear approximation at $c$.
\end{exercise}

\begin{exercise}
Let $f_i$ be functions, and $L_i$ be linear functions, so that $f_1 \sim_{x_0} L_1$ and $f_2 \sim_{x_0} L_2$.

Then,
\begin{itemize}
\item $f_1 + f_2 \sim_{x_0} L_1 + L_2$.
\item $f_1 \cdot f_2 \sim_{x_0} L_1 \cdot L_2$.
\end{itemize}
\end{exercise}

\begin{remark}
The latter ``explains'' the product rule:
\begin{align*}
f_1(x) &\sim_{x_0} {f_1}'(x_0) \, (x - x_0) + f_1(x_0) \\
f_2(x) &\sim_{x_0} {f_2}'(x_0) \, (x - x_0) + f_2(x_0), \\
\intertext{and therefore,}
f_1(x) \, f_2(x) &\sim_{x_0} \left( {f_1}'(x_0) \, \left(x - x_0\right) + f_1(x_0) \right) \left( {f_2}'(x_0) \, \left(x - x_0\right) + f_2(x_0) \right) \\
f_1(x) \, f_2(x) &\sim_{x_0} \left( {f_1}'(x_0) f_2(x_0) + f_1(x_0) {f_2}'(x_0) \right) \, \left(x - x_0\right) + f_1(x_0) f_2(x_0).
\end{align*}
You might have noticed that the ${f_1}'(x_0) {f_2}'(x_0) (x - x_0)^2$ term is missing---why are we justified in ignoring it?
\end{remark}

\section{Higher derivatives}

\begin{definition}
For two functions $f, g : \R \to \R$, we say that $f(x) \sim^n_{x_0} g(x)$ provided
$$
f(x) - g(x) = o\left( \left(x - x_0\right)^n \right) \mbox{ as $x \to x_0$.}
$$
\end{definition}
\noindent
Intuitively, this means that $f$ and $g$ are the ``same'' through degree $n$.
Note that, as before, this is an equivalence relation.

\begin{definition}
A \textbf{quadratic approximation} for $f(x)$ at $x_0$ is a quadratic function
$$
L(x) = ax^2 + bx + c
$$
so that $f(x) \sim^2_{x_0} L(x)$.
\end{definition}

\begin{remark}
The quadratic approximation packages together both the first and second derivative of $f$ at $x_0$.
\end{remark}

\begin{exercise}
Let $L_1$ and $L_2$ be quadratic functions such that $L_1 \sim^2_{x_0} L_2$.  Prove $L_1 = L_2$.
\end{exercise}

\begin{exercise}
Let $f : (a,b) \to \R$.  If $f$ has a quadratic approximation at $c$, then it is unique.
\end{exercise}

\end{document}
