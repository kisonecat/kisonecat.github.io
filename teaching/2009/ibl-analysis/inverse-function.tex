\documentclass[12pt]{article}
\usepackage{fullpage}

\title{Inverse Function Theorem}
\author{IBL Analysis}

\usepackage{amsmath}
\usepackage{amssymb}
\newcommand{\R}{\mathbb{R}}
\newcommand{\transpose}{{\mbox{\scriptsize\sf T}}}

\usepackage{fullpage}
\usepackage{add-copyright}

\begin{document}

\maketitle

\section{Nodal singularity}

Consider the function $f : \R^2 \to \R$ given by $f(x,y) = xy$.

Draw a picture of the level set where $f(x,y) = 0$.  For which $(x,y)$
does the implicit function theorem apply?  What (not so terribly) terrifying
event happens at $(x,y) = (0,0)$?

\section{Cuspidal singularity}

Consider the function $f : \R^2 \to \R$ given by $f(x,y) = x^3 - y^2$.

Draw a picture of the level set where $f(x,y) = 0$.  For which $(x,y)$
does the implicit function theorem apply?  What truly terrifying event
happens at $(x,y) = (0,0)$?

\section{Lines}

Consider the function
$$
f : \R^2 \times \R \to \R^2
$$
given by
$$
f(x,y,t) = (x t, y (1-t))
$$
That is, $f$ sends $(x,y,t)$ to the point in $\R^2$ which is $t$ of the way between $(x,0)$ and $(0,y)$.

Qualitatively describe $f^{-1}(x,y)$ using the implicit function
theorem.  Are there choices of $(x,y)$ with drastically different
behavior?

\section{Polar coordinates}

Consider the function $f : \R^2 \to \R^2$ given by
$$
f(r,\theta) = (r \cos \theta, r \sin \theta).
$$
For which $(r,\theta)$ is $f$ locally bijective?

\section{Polynomials}

Let $P_{n-1}$ be the (vector) space of degree $n-1$ polynomials in a
variable $t$; note that $\dim P_n = n$.  Define a function $f : \R^n
\to P_{n-1}$ given by
$$
f(x_1,\ldots,x_n) = (t - x_1) \cdots (t - x_n) - t^n
$$
So $f$ takes $n$ points in $\R$ to a polynomial having those points as
its roots (well, that polynomial minus $t^n$).

Find points $(x_i) \in \R^n$ for which there exists a neighborhood $U
\ni (x_i)$ with $f(U)$ open.  Our goal can can be described more
colorfully: we are seeking polynomals with $n$ roots, which, when
wiggled, still have $n$ roots.

For concreteness, it will help tremendously to explore the special
cases $f : \R^2 \to P_1$ and then $f : \R^3 \to P_2$.

\section{How to (not) flatten a sphere}

Write $D^2$ for the open unit disk, i.e.,
$$
D^2 = \{ v \in \R^2 : ||v|| < 1 \}.
$$
A function from the sphere $S^2$ to $X$ is given by a \textit{pair} of
functions $f : D^2 \to X$ and $g : D^2 \to X$ so that
$$
f(v) = g\left( \frac{\frac{3}{2} - ||v||}{||v||} \, v \right) \hspace{1em} \mbox{for $\frac{1}{2} < ||v|| < 1$}
$$
You should think of $f$ as defining the function on a bit more than
the top hemisphere, and $g$ as defining the function on a bit more
than the bottom hemisphere; the functions must agree on the overlap.

Does there exist a function $S^2 \to \R^2$ with everywhere
nonvanishing derivative?  (Equivalently, is it possible to put
coordinates on the Earth so that at every point, four ``cardinal
directions'' are defined?)

\section{Square roots}

Let $M_2$ be the four-dimensional space of two-by-two matrices; let $f
: M_2 \to M_2$ be the function $f(A) = A \cdot A$.

Compute $D f^{-1}(I)$.  Use this to approximate
$$
\sqrt{ \begin{bmatrix}
0.9 & 0.1  \\
0.1 & 1.1
\end{bmatrix}}
$$
For extra fun, when is $D f$ invertible?  This will tell us when
perturbations of matrices having square roots also have square roots.

\section{Orthogonal matrices}

Recall that a matrix is \textbf{orthogonal} if $A \cdot A^\transpose =
I$.

Let $M_3$ be the nine-dimensional vector space of 3-by-3 matrices; let
$S_3$ be the six-dimensional vector space of symmetric 3-by-3
matrices.  Define a function $f : M_3 \to S_3$ by $f(A) = A \cdot
A^\transpose$.  A matrix is orthogonal if it is in $f^{-1}(I)$.

Note that $f(I) = I$; along how many degrees of freedom may we perturb
$I$ and still have an orthogonal matrix?  (You will have to compute
$Df(I)$ to analyze this).

Repeat this problem for the analogously defined function $M_2 \to S_2$
to show that, in two dimensions, there is only one degree of freedom
for rotation.  For rotations in four dimensions, how many degrees of
freedom do you have?  (It would indeed be very confusing to be
tumbling in four dimensionsal space).

\section{Segre variety}

Consider $f : \R^2 \times \R^2 \to \R^4$ given by
$$
f(v_1,v_2,w_1,w_2) = (v_1 w_1, v_1 w_2, v_2 w_1, v_2 w_2).
$$
For the linear algebra cognoscenti, this is a special case of a
certain (nonlinear!) function $V \oplus W \to V \otimes W$.

Globally, $f$ is not injective---after all,
$$
f(0,0,w_1,w_2) = (0,0,0,0).
$$
But are there points for which $f$ is \textit{locally} bijective?

\section{Characteristic polynomials}

Let $M_2$ be the four-dimensional vector space of two-by-two matrices;
let $P_2$ be the three-dimensional vector space of degree two
polynomials in a variable $\lambda$.

The characteristic polynomial of a matrix $A$ is $\det (A - \lambda
I)$; let $f : M_2 \to P_2$ be the function sending a matrix to its
characteristic polynomial.  (If you are very fancy, you might have
learned that the roots of the characteristic polynomial are the
eigenvalues of the matrix).

Tell me a interesting story involving the implicit function theorem and
the function $f$.

\end{document}