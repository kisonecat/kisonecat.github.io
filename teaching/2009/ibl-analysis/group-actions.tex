\documentclass[12pt]{article}
\usepackage{fullpage}
\usepackage{add-copyright}

\title{Group actions on tensor products}
\author{Math 205}
\date{May 12, 2009}

\usepackage{amsmath}
\DeclareMathOperator{\Alt}{Alt}
\DeclareMathOperator{\id}{id}

\usepackage{amsthm}
\newtheorem*{theorem}{Theorem}

\theoremstyle{definition}
\newtheorem*{quotedefinition}{``Definition''}
\newtheorem*{definition}{Definition}

\newtheorem*{exercise}{Exercise}

\newcommand{\Tensor}[2]{\mathcal{T}^{#2} #1}

\usepackage{amssymb}
\newcommand{\R}{\mathbb{R}}

\long\def\symbolfootnote[#1]#2{\begingroup%
\def\thefootnote{\fnsymbol{footnote}}\footnote[#1]{#2}\endgroup}

\usepackage{rotating}
\usepackage{xypic}

\newcommand{\AN}{\mathbf{e}}
\newcommand{\AX}{\mathbf{X}}
\newcommand{\AY}{\mathbf{Y}}
\newcommand{\AR}{\mathbf{R}}

\newcommand{\squareN}{\framebox{\raisebox{0in}[0.3in][0.3in]{\parbox{0.6in}{\hfill\textsf{FRONT}\hfill\null\vspace{1ex}\\\null\hfill{\Huge$\uparrow$}\hfill\null}}}}

\newcommand{\squareF}{\framebox{\raisebox{0in}[0.3in][0.3in]{\parbox{0.6in}{\hfill\textsf{BACK}\hfill\null\vspace{1ex}\\\null\hfill{\Huge$\uparrow$}\hfill\null}}}}

\newcommand{\squareRR}{\begin{turn}{180}\framebox{\raisebox{0in}[0.3in][0.3in]{\parbox{0.6in}{\hfill\textsf{FRONT}\hfill\null\vspace{1ex}\\\null\hfill{\Huge$\uparrow$}\hfill\null}}}\end{turn}}

\newcommand{\squareFRR}{\begin{turn}{180}\framebox{\raisebox{0in}[0.3in][0.3in]{\parbox{0.6in}{\hfill\textsf{BACK}\hfill\null\vspace{1ex}\\\null\hfill{\Huge$\uparrow$}\hfill\null}}}\end{turn}}

\newcommand{\squareRRR}{\raisebox{-3.5pt}{\raisebox{-0.3in}{\begin{turn}{90}\framebox{\raisebox{0in}[0.3in][0.3in]{\parbox{0.6in}{\hfill\textsf{FRONT}\hfill\null\vspace{1ex}\\\null\hfill{\Huge$\uparrow$}\hfill\null}}}\end{turn}}}}

\newcommand{\squareFRRR}{\raisebox{-3.5pt}{\raisebox{-0.3in}{\begin{turn}{90}\framebox{\raisebox{0in}[0.3in][0.3in]{\parbox{0.6in}{\hfill\textsf{BACK}\hfill\null\vspace{1ex}\\\null\hfill{\Huge$\uparrow$}\hfill\null}}}\end{turn}}}}

\newcommand{\squareR}{\raisebox{3.5pt}{\raisebox{0.3in}{\begin{turn}{270}\framebox{\raisebox{0in}[0.3in][0.3in]{\parbox{0.6in}{\hfill\textsf{FRONT}\hfill\null\vspace{1ex}\\\null\hfill{\Huge$\uparrow$}\hfill\null}}}\end{turn}}}}

\newcommand{\squareFR}{\raisebox{3.5pt}{\raisebox{0.3in}{\begin{turn}{270}\framebox{\raisebox{0in}[0.3in][0.3in]{\parbox{0.6in}{\hfill\textsf{BACK}\hfill\null\vspace{1ex}\\\null\hfill{\Huge$\uparrow$}\hfill\null}}}\end{turn}}}}

\begin{document}

\maketitle

The footnotes in this document are only for flavor.

\subsection*{Left and right group actions}

One thing we have not really paid much attention to yet is that a
group acts from either the left, or from the right.
\begin{definition}
  A left action of a group $G$ on a set $X$ is a map $G \times X \to
  X$, which we write as $g \cdot x$, with the property that
$$
g \cdot (h \cdot x) = (gh) \cdot x.
$$
\end{definition}
\noindent
On the other hand (and I mean that literally),
\begin{definition}
  A right action of a group $G$ on a set $X$ is a map $X \times G \to
  X$, which we write as $x \cdot g$, with the property that
$$
(x \cdot g) \cdot h = x \cdot (gh).
$$
\end{definition}
These look similar, but they are not at all the same---in a left
action, the meaning of ``$gh$'' is ``do $h$, then do $g$.''  In a
right action, ``$gh$'' means ``do $g$, then do $h$.''  It is perhaps
significant to reflect on my use of the word ``means''---it is
precisely the group action that gives a meaning to a group element.

Philosophically, a \textbf{group is a symmetric thing without the
  thing}---only the symmetry\symbolfootnote[1]{that is, the beauty} remains.  A
group action is what glues this abstract symmetry back to something in
the real world.  The handedness, this choice of a left or a right
action, gives an interpretation of the group operation: does it mean
``first do this, then do that'' or does it mean ``first do that, then
do this''?  These are not the same.

Yet, they are not so different.  You can always turn a left action
into a right action (and vice versa) with a trick: if $G$ acts on the
left on a set $X$, then $G$ acts on the right after taking inverses.
Specifically, define the right action $x \cdot g$ to be $g^{-1} \cdot
x$.
\begin{exercise}
Check that this is a right action.
\end{exercise}
\noindent
Very concretely, what this says is this: if you want to read a
sequence of commands backwards, you should interpret each command as
an ``undo.''

\subsection*{Group action example}

All this theory is useless without something concrete, so here is an
example: a group\symbolfootnote[1]{A popular application of this group
  is to the ancient problem of \textit{square mattress flipping}.  One
  would like a single group element (e.g., $\AR \AX$) which, as it is
  applied repeatedly to the mattress, produces each of the 8 possible
  configurations of the square mattress.  This would be helpful
  because you could decide ``every month I will apply $\AR\AX$ to my
  square mattress'' and over the course of 8 months you would have
  rotated your mattress through all configurations.

  Tragically, it turns out that there is no such an element---or in
  more intimidating language ``the group of symmetries of the square
  is not cyclic.''  It is therefore a rather complicated matter to
  flip one's square mattress through all 8 configurations.  This
  remains problematic even in the smaller group of symmetries of
  rectangular mattresses.} acting on a square piece of paper with a
front and a back side.

The group has eight elements, which are
$$
\AN = \AX^2,\hspace{1em} \AR,\hspace{1em} \AR^2,\hspace{1em} \AR^3,\hspace{1em} \AX,\hspace{1em} \AX\AR = \AR^3\AX,\hspace{1em} \AX\AR^2 = \AR^2\AX,\hspace{1em} \AX\AR^3 = \AR\AX
$$
You can do calculations in this group, like
$$
\AR \AX \AR = \AR \AR^3 \AX = \AR^4 \AX = \AX.
$$
This group \textit{acts} on the set of configurations of a piece of
square paper.
\begin{align*}
\mbox{$\AR$ acts by rotating clockwise by $90^\circ$,} &\mbox{ turning \squareN\ into \squareR}. \\
\mbox{$\AX$ acts by flipping across the $x$-axis,} &\mbox{ turning \squareN\ into \squareFRR}.
%\item[Y-Flipping,] written $\AY$, flipping across the $y$-axis,\hfill turning \squareN\ into \squareF
\end{align*}
Once you decide whether to read the list of instructions from
left-to-right (or from right-to-left), you will have decided that this
is a right (or a left) action, respectively.

For English readers, you might scoff and say ``Why would I want to
read the instructions from right to left?''  Let me remind you that
you already know any example of such a situation!  The notation $f
\circ g$ means ``first apply $g$, then apply $f$'' which is arguably
backwards from what you might have wanted\symbolfootnote[4]{This is
  why I would be happier putting the arguments to functions on the
  left hand side---i.e., we should all be writing $(x)f$ instead of
  $f(x)$.}.

\subsection*{Actions on tensors}

The goal is to prove the following.
\begin{theorem}
  Let $S \in \Tensor{V}{k}$ and $T \in \Tensor{V}{\ell}$.  If $\Alt S
  = 0$, then $\Alt (S \otimes T) = 0$.
\end{theorem}
To prove this, we will define an additional concept (namely, an action
of the symmetric group $S_k$ on $\Tensor{V}{k}$) which should make it
clearer what exactly is happening.

Recall that an element $S \in \Tensor{V}{k}$ is a $k$-linear functional
$$
S : \overbrace{V \times \cdots \times V}^{\mbox{\scriptsize $k$ times}} \to \R
$$
A permutation $\sigma \in S_k$ will \textit{act} on $S \in
\Tensor{V}{k}$ to produce a new $k$-linear functional, $\sigma^\star S
\in \Tensor{V}{k}$.  But how?  How do we define this?
\begin{quotedefinition}
  Let $S \in \Tensor{V}{k}$ and $\sigma \in S_k$.  Then define
$$
\left(\sigma^\star S\right)\left(v_1,\ldots,v_k\right) =
S\left(v_{\sigma(1)}, v_{\sigma(2)}, \ldots, v_{\sigma(k)}\right).
$$
\end{quotedefinition}
I put this in quotation marks not because it isn't a definition (it
will be the definition we will use, as it agrees with what we have
been using thus far), but because it is more of a convention---there
ar other choices that are (arguably) better.

A good thing about the action of $S_k$ on $\Tensor{V}{k}$ is a
concise definition of our friend $\Alt$.
\begin{definition}
$\Alt S = \displaystyle\frac{1}{k!} \displaystyle\sum_{\sigma \in S_k} (-1)^{\sigma} \, \sigma^\star S$.
\end{definition}
Here I have used a convention that $(-1)^{\sigma}$ means the sign of
the permutation $\sigma$.  I think this makes the formulas nicer
looking (and, as good notation ought, suggests that this is a
homomorphism).

A bad thing about our ``definition'' is shown in the next pair of
exercises.
\begin{exercise}
Show that $\left(\sigma \tau\right)^\star S \neq \sigma^\star \tau^\star S$.
\end{exercise}
\begin{exercise}
Show that $\left(\sigma \tau\right)^\star S = \tau^\star \sigma^\star S$.
\end{exercise}
In fact, the whole reason I put the little stars as a superscript
instead of as a subscript is because of this fact: the superscript
star reminds me that something contravariant (``backwards'') is taking
place.  In other words, what we have built is actually a \textit{right
  action} of the symmetric group $S_k$ on tensors $\Tensor{V}{k}$.  If
we want a left action, we could take inverses---but in order to keep
consistent with the conventions we have already chosen, we will not do
this.

\begin{exercise}
The action of $S_k$ on $\Tensor{V}{k}$ is by linear maps, that is,
$$
\sigma^\star ( S + T ) = \sigma^\star S + \sigma^\star T
$$
for $\sigma \in S_k$ and $S, T \in \Tensor{V}{k}$.
\end{exercise}

I want to define another operation: a sort of ``tensor product'' on
the symmetric group.
\begin{definition}
The operation $\otimes : S_k \times S_\ell \to S_{k + \ell}$ is defined by\ldots
\end{definition}
\noindent
I will let you complete the definition appropriately; using the
$\otimes$ symbol is merely suggestive---of course this isn't a tensor
product at all.

We are now ready to ``prove'' the theorem.
\begin{align*}
(k + \ell)! \cdot \Alt(S \otimes T)
&= \sum_{\sigma \in S_{k + \ell}} (-1)^{\sigma} \, \sigma^\star (S \otimes T) \\
&= \sum_{\tau} \sum_{\sigma} (-1)^{\sigma \tau} \, (\sigma \tau)^\star  (S \otimes T) \\
&= \sum_{\tau} \sum_{\sigma} (-1)^{\sigma} \, (-1)^{\tau} \, \tau^\star \sigma^\star (S \otimes T) \\
&= \sum_{\tau} (-1)^{\tau} \, \tau^\star \left( \sum_{\sigma} (-1)^{\sigma} \, \sigma^\star (S \otimes T) \right) \\
&= \sum_{\tau} (-1)^{\tau} \, \tau^\star \left( \sum_{\sigma \in S_k} (-1)^{\sigma \otimes \id} \, (\sigma \otimes \id)^\star (S \otimes T) \right) \\
&= \sum_{\tau} (-1)^{\tau} \, \tau^\star \left( \sum_{\sigma \in S_k} (-1)^{\sigma} \, (\sigma^\star S) \otimes \id^\star T \right) \\
&= \sum_{\tau} (-1)^{\tau} \, \tau^\star \left( \sum_{\sigma \in S_k} (-1)^{\sigma} \, (\sigma^\star S) \otimes T \right) \\
&= \sum_{\tau} (-1)^{\tau} \, \tau^\star \left( \left( \sum_{\sigma \in S_k} (-1)^{\sigma} \, \sigma^\star S \right) \otimes T \right) \\
&= \sum_{\tau} (-1)^{\tau} \, \tau^\star \left( k! \cdot \Alt S \otimes T \right) \\
&= \sum_{\tau} (-1)^{\tau} \, \tau^\star \left( 0 \otimes T \right) = 0.
\end{align*}
I have supressed exactly what we are summing over (i.e., where do
$\tau$ and $\sigma$ take values?).  I hope you will take a moment to
reflect on this notation---its benefits, its drawbacks.  There is a
great discussion to be had about the purpose of
proof
%\symbolfootnote[1]{Do not ask ``how do I prove this?'' but ``for
%  whom do I prove this?''  Your audience matters.} 
and excessive machinery and such.

There is also a theme illustrated here: we have worked with the
tensors directly, without considering their inputs.  This
technique---reasoning about things which take parameters without
referring to the parameters---can be quite powerful.

\end{document}
