\documentclass[12pt]{article}
\usepackage{nopageno}
\usepackage{fullpage}

\begin{document}

\section*{Nonconstant, but having the same germ as the zero function}

Here is a nice way to get that result about $e^{-1/x^2}$.  We want to show that for small $x$ and all natural numbers $n$,
$$
0 \leq e^{-1/x^2} \leq x^{2n}.
$$
Equivalently, for small $x$ and all $n$,
$$
(1/x^2)^n \leq e^{1/x^2}
$$
Equivalently, for large $y$ and all $n$,
$$
y^n \leq e^{y}
$$
Now here we apply l'H\^opital's rule $n$ times to find $\lim_{y \to
  \infty} e^y/y^n > 1$, thereby deducing the above fact for large $y$.
But my disdain for l'H\^opital runs deep!  Let's say something else.

To show that $y^n \leq e^y$, use Taylor's theorem to say
$$
e^y \geq \frac{y^{n+1}}{(n+1)!} \geq y^n \cdot \frac{y}{(n+1)!} \geq y^n
$$
provided $y > (n+1)!$.

This is perhaps better, because it is also \textit{quantitative}---it says
that, yes, exponentials grow faster than polynomials, but only when
looking far out: $n^2$ is bigger than $1.0000001^n$ for
reasonably sized values of $n$.  Specifically,
$$
1.0000001^{395935000} < 395935000^2 \mbox{ but } 
1.0000001^{395936000} > 395936000^2.
$$

\end{document}