\documentclass[12pt,letterpaper]{article}

\title{Syllabus}
\author{Jim Fowler}

\usepackage{fullpage}
\pagestyle{empty}

%\usepackage[T1]{fontenc}
%\usepackage{textcomp}
%\usepackage{lmodern}
%\newcommand{\ditto}{\textquotedbl}

\setlength{\parindent}{0pt}

\newcommand{\peem}{\textsc{p.m.}}
\newcommand{\ayem}{\textsc{a.m.}}

\begin{document}

%%%%%%%%%%%%%%%%%%%%%%%%%%%%%%%%%%%%%%%%%%%%%%%%%%%%%%%%%%%%%%%%
{\Large\sf Syllabus\hfill
Math 131, Section 22\hfill
Autumn 2005} \\

Welcome to Math 131, the first course in a year-long Calculus
sequence.  This quarter, we will work through limits, derivatives, and
their applications.

\vfill
%%%%%%%%%%%%%%%%%%%%%%%%%%%%%%%%%%%%%%%%%%%%%%%%%%%%%%%%%%%%%%%%
\subsection*{Instructor}
Class meetings are Mondays, Wednesdays, and Fridays, 9:30--10:20\ayem\ in Ryerson 358. \\

\parbox{0.5\textwidth}
{\begin{tabular}{ll}
\textbf{Name:} & Jim Fowler \\
\textbf{Office:} & Math/Stat 102 \\
\textbf{Phone:} & 773--573--5659 \\
\textbf{Email:} & \texttt{jim@uchicago.edu}
\end{tabular}
}\parbox{0.5\textwidth}{
\textbf{Office Hours:} \parbox[t]{0.3\textwidth}{Monday 4:30--6:30\peem \\ Thursday 4:30--6:30\peem \\ or by appointment}} \\

\vspace{2ex} Please do come to office hours; they are a great time to work
through problems, review lectures, or just talk about mathematics.

\vfill
%%%%%%%%%%%%%%%%%%%%%%%%%%%%%%%%%%%%%%%%%%%%%%%%%%%%%%%%%%%%%%%%
\subsection*{Tutors}
Your tutor will hold a mandatory problem session Tuesdays and Thursdays, 9:00--10:20\ayem
\begin{itemize}
\setlength{\itemsep}{-1ex}
\item Brian Keto (\texttt{bketo@uchicago.edu}) in Eckhart 207
\item Danny Rosenthal (\texttt{danny@uchicago.edu}) in Eckhart 308 
\item Sara Rezvi (\texttt{arsinoe@uchicago.edu}) in Wieboldt 130
\end{itemize}
You will be assigned to a tutor on Wednesday, September 28, and the
first problem session will meet Thursday, September 29.  If for some
reason you want to switch to a different problem session, just talk to
Jim Fowler.

\vfill
%%%%%%%%%%%%%%%%%%%%%%%%%%%%%%%%%%%%%%%%%%%%%%%%%%%%%%%%%%%%%%%%
\subsection*{Text}

\textit{Calculus} by Verberg, Purcell, and Bigdon (8th Edition).  This course will cover most of chapters 1--3.

\vfill
%%%%%%%%%%%%%%%%%%%%%%%%%%%%%%%%%%%%%%%%%%%%%%%%%%%%%%%%%%%%%%%%
\subsection*{Course Website}

For the most up-to-date information, visit
\begin{center}
\verb+http://math.uchicago.edu/~fowler/math131/+
\end{center}
We will post announcements, homework assignments, and take-home
quizzes on the website.  A tentative schedule for lectures has already
been posted.

\vfill
%%%%%%%%%%%%%%%%%%%%%%%%%%%%%%%%%%%%%%%%%%%%%%%%%%%%%%%%%%%%%%%%
\subsection*{Course Requirements}

There are one thousand points possible in this course, broken down as follows:
\begin{description}
\item[Daily homework (100 points).]  Homework will usually be assigned
each class period, and due the following class period.  You should
work on the homework problems together, but you must write up your
solutions independently.  {\em Stay caught up on the daily
homeworks!}  It is tempting to fall behind, but difficult to catch up
again---especially in a mathematics course.
\item[2 midterms (201 points each).]  The midterms will be
in class.  The first will be on Friday, October 21 (at the end of week 4), and
the second will be on Friday, November 18 (at the end of week 8).
\item[8 quizzes (12 points each).]  Short take-home quizzes will be assigned most weeks.
\item[1 final exam (402 points).]  The final exam will be held
8:00--10:00\ayem\ on Wednesday, December 7, 2005.
\end{description}

\vfill
%%%%%%%%%%%%%%%%%%%%%%%%%%%%%%%%%%%%%%%%%%%%%%%%%%%%%%%%%%%%%%%%
\subsection*{Rescheduling a Midterm}

Contact Jim Fowler early in the quarter if you will not be able to
take a midterm on the scheduled day; we likely can accomodate you, but
we will need {\em plenty} of time to prepare.

%%%%%%%%%%%%%%%%%%%%%%%%%%%%%%%%%%%%%%%%%%%%%%%%%%%%%%%%%%%%%%%%
\subsection*{Department Policy on Final Exams}

\textit{It is the policy of the Department of Mathematics that the
following rules apply to final exams in all undergraduate mathematics
courses:}
\begin{enumerate}
\item \textit{The final exam must occur at the time and place designated on
the College Final Exam Schedule.}  In particular, \textit{no} final examinations
may be given during the tenth week of the quarter, except in the case
of graduating seniors.
\item Any student who wishes to depart from the scheduled final exam
time for the course must receive permission from Paul Sally (office is
Ryerson 350, phone is 2-7388, email is
\texttt{sally@math.uchicago.edu}).  Instructors are not permitted to
excuse students from the scheduled time of the final exam except in
the cases of an Incomplete.
\end{enumerate}

%%%%%%%%%%%%%%%%%%%%%%%%%%%%%%%%%%%%%%%%%%%%%%%%%%%%%%%%%%%%%%%%
\subsection*{Important Dates}

\begin{tabular}{ll@{ }r@{}l@{ }l}
\textbf{First Midterm:} & October  & 21&, & 9:30--10:20\ayem \\
\textbf{Second Midterm:} & November & 18&, & 9:30--10:20\ayem \\
\textbf{Thanksgiving:} & November & 24&\makebox[0pt][l]{--25} \\
\textbf{Review Session:} & December & 2&, & 9:30--10:20\ayem \\ 
\textbf{Final Exam:} & December & 7&,  & 8:00--10:00\ayem \\
\end{tabular}

%%%%%%%%%%%%%%%%%%%%%%%%%%%%%%%%%%%%%%%%%%%%%%%%%%%%%%%%%%%%%%%%
% \section*{Course Outline}

% \newcommand{\cls}[3]{#1 & #2 & (#3)}

% \begin{tabular}{rr|crlcrlcrl}
%      &    & \multicolumn{3}{c}{Monday} & \multicolumn{3}{c}{Wednesday} & \multicolumn{3}{c}{Friday} \\
% \hline
% Week & 1  & \cls{Sept}{26}{Overview}   & \cls{Sept}{28}{Inequalities} & \cls{Sept}{30}{Absolute Value} \\
% Week & 2  & \cls{Oct}{3}{Coordinates}     & \cls{Oct}{5}{Functions} & \cls{Oct}{7}{Limits} \\
% Week & 3  & \cls{\ditto}{10}{something} & \cls{\ditto}{12}{awesome} & \cls{\ditto}{14}{more awesome} \\
% Week & 4  & \cls{\ditto}{17}{something} & \cls{\ditto}{19}{awesome} & \cls{\ditto}{21}{\bf Midterm} \\
% Week & 5  & \cls{\ditto}{24}{something} & \cls{\ditto}{26}{awesome} & \cls{\ditto}{28}{more awesome} \\
% Week & 6  & \cls{\ditto}{31}{something} & \cls{Nov}{2}{awesome} & \cls{Nov}{4}{more awesome} \\
% Week & 7  & \cls{Nov}{7}{something} & \cls{\ditto}{9}{awesome} & \cls{\ditto}{11}{more awesome} \\
% Week & 8  & \cls{\ditto}{14}{something} & \cls{\ditto}{16}{awesome} & \cls{\ditto}{18}{more awesome} \\
% Week & 9  & \cls{\ditto}{21}{something} & \cls{\ditto}{23}{\bf Midterm} & \cls{\ditto}{25}{\bf Thanksgiving} \\
% Week & 10 & \cls{\ditto}{28}{something} & \cls{\ditto}{30}{awesome} & \cls{Dec}{2}{\bf Reading Period} \\
% \end{tabular}


% four introductory lectures

% quadratic inequalities
% PL defined functions (to prepare for limits)
% don't graph
% circle equations

% chapter 2 in 8-9 lectures (everything excpt 2.3 and 2.7)

% chpater 3: derivative in 12 lectures

% sections 3.1 to 3.7.
% no trig.
% do as much of 3.8 and 3.9 without trig 
% include section 4.1

\end{document}
