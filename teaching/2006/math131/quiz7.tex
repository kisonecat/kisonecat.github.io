\documentclass[12pt]{article}
\usepackage{fullpage}
\usepackage{nopageno}
\usepackage{ifthen}
\usepackage{amsmath}
\usepackage{amssymb}
\usepackage{graphicx} 
\usepackage{version}
\usepackage{amsthm}
\usepackage{multicol}
%\usepackage{add-copyright}

\excludeversion{solution}

\DeclareMathOperator{\ft}{ft}

\newcommand{\R}{\mathbb{R}}

\title{Take-Home Quiz 7}
\author{Math 131 Section 22}
\date{Due Monday, November 28, 2005}

\newcounter{problem}
\setcounter{problem}{1}

\newenvironment{problem}[1][]
{\begin{flushleft}\hangindent=1em\hangafter=1\noindent\textbf{Problem \arabic{problem}.}
\ifthenelse{\equal{#1}{}}{}{
\textbf{(#1 \ifthenelse{\equal{#1}{1}}{point}{points}).}}
}
{\addtocounter{problem}{1}\end{flushleft}}

\begin{document}
\maketitle

I wish you a wonderful Thanksgiving!
% Differentials are a good way to estimate things we would otherwise need a calculator to compute.

\begin{problem}[4]
Use differentials to estimate $(1.02)^{20}$.
\end{problem}

\begin{problem}[4]
I give you an orange with a diameter of 4 inches.  The rind is $0.2$
inches thick.  Using differentials, estimate the volume of the rind.
\end{problem}

\begin{problem}[4]
Recall that $\sqrt[4]{81} = 3$.  Use differentials to estimate
$\sqrt[4]{80}$.
\end{problem}

\end{document}
