\documentclass[12pt]{article}
\usepackage{fullpage}
\usepackage{nopageno}
\usepackage{ifthen}
\usepackage{amsmath}
\usepackage{amssymb}
\usepackage{graphicx} 
\usepackage{version}
\usepackage{amsthm}
\usepackage{multicol}
\usepackage{add-copyright}

\excludeversion{solution}

\DeclareMathOperator{\ft}{ft}

\newcommand{\R}{\mathbb{R}}

\title{Take-Home Quiz 2}
\author{Math 133 Section 22}
\date{Due Monday, April 17}

\newcounter{problem}
\setcounter{problem}{1}

\newenvironment{problem}[1][]
{\begin{flushleft}\hangindent=1em\hangafter=1\noindent\textbf{Problem \arabic{problem}.}
\ifthenelse{\equal{#1}{}}{}{
\textbf{(#1 \ifthenelse{\equal{#1}{1}}{point}{points}).}}
}
{\addtocounter{problem}{1}\end{flushleft}}

\begin{document}
\maketitle

\begin{problem}[3]
Compute the following:
$$
\int \frac{\ln x}{x} \, dx.
$$
\end{problem}

\begin{problem}[3]
Compute the following:
$$
\int (\ln x)^2 \, dx.
$$
\end{problem}

\begin{problem}[3]
Consider the integral $\displaystyle\int \displaystyle\frac{1}{x} \, dx$.  I will perform integration by parts, by setting
$$
\begin{array}{rclclcrcl}
\vspace{2ex}
u & = & & \displaystyle\frac{1}{x} & & \hspace{1em} & v & = & x \\
du & = & & \makebox[0in][r]{$-$} \displaystyle\frac{1}{x^2} \makebox[0in][l]{$\, dx$} & & \hspace{1em} & dv & = & dx
\end{array}
$$
This gives
$$
\int \frac{1}{x} \, dx = \frac{1}{x} \cdot x - \int x \cdot \frac{-1}{x^2} \, dx.
$$
Simplifying gives
$$
\int \frac{1}{x} \, dx = 1 + \int \frac{1}{x} \, dx.
$$
Canceling the integrals on both sides yields
$$
0 = 1,
$$
which is ridiculous!  Where did I make a mistake?
\end{problem}

\begin{problem}[3]
Evaluate the limit:
$$
\lim_{x \to 0^{+}} (\sin x)^x
$$
\end{problem}

\begin{problem}[3]
Evaluate the limit:
$$
\lim_{x \to 0} \left( \frac{1}{\sin x} - \frac{1}{x + x^2} \right).
$$
\end{problem}

\end{document}
