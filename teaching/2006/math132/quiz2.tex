\documentclass[12pt]{article}
\usepackage{fullpage}
\usepackage{nopageno}
\usepackage{ifthen}
\usepackage{amsmath}
\usepackage{amssymb}
\usepackage{graphicx} 
\usepackage{version}
\usepackage{amsthm}
\usepackage{multicol}
\usepackage{add-copyright}

\excludeversion{solution}

\DeclareMathOperator{\ft}{ft}

\newcommand{\R}{\mathbb{R}}

\title{Take-Home Quiz 2}
\author{Math 132 Section 22}
\date{Due Wednesday, January 18, 2006}

\newcounter{problem}
\setcounter{problem}{1}

\newenvironment{problem}[1][]
{\begin{flushleft}\hangindent=1em\hangafter=1\noindent\textbf{Problem \arabic{problem}.}
\ifthenelse{\equal{#1}{}}{}{
\textbf{(#1 \ifthenelse{\equal{#1}{1}}{point}{points}).}}
}
{\addtocounter{problem}{1}\end{flushleft}}

\begin{document}
\maketitle

Do not merely write down an answer---write also something to convince
the skeptical and foolish reader (namely, Jim) that your answer is
correct.  Mathematics is not a collection of disconnected truths; it
is a way to move from one truth to another.

%This quiz is worth 14 points---12 points are for ``correctness'' of
%answers, but an additional 2 points will be given out for
%``style''---awarded to answers which are clearly written, easy to
%follow, etc.

\begin{problem}[2]
Compute
$$
\frac{d}{dx} \left( \sin 3x - \cos 5x \right).
$$
\end{problem}

\begin{problem}[2]
Compute
$$
\frac{d}{dx} \left( \frac{\sin 2x}{\cos 3x} \right).
$$
\end{problem}

\begin{problem}[2]
Compute
$$
\frac{d}{dx} \left( \frac{ \sin^2 x }{\sin (x^2)} \right).
$$
\end{problem}

\begin{problem}[2]
Recall that $\cos x = \sin (\pi/2 - x)$.  Calculate the derivative of
$\cos x$ with respect to $x$ by calculating
$$
\frac{d}{dx} \sin \left( \frac{\pi}{2} - x \right).
$$
\end{problem}

\begin{problem}[3]
Recall the identity $\sin (2x) = 2 \sin x \cos x$.  Differentiate both
sides of this equation with respect to $x$ to get another identity.
\end{problem}

\begin{problem}[3]
When does
$$
\frac{d}{dx} \sin \left( \sin x \right)
$$
equal zero?
\end{problem}

\end{document}
