\documentclass[12pt]{article}
\usepackage{worksheet}
\title{Goldbach Conjecture}

\begin{document}

\Large

%%%%%%%%%%%%%%%%%%%%%%%%%%%%%%%%%%%%%%%%%%%%%%%%%%%%%%%%%%%%%%%%
\section*{The Goldbach Conjecture.}

\subsection*{Some background.}

I present to you the first twenty \textbf{prime numbers}
$$
2, 3, 5, 7, 11,
13, 17, 19, 23, 29,
31, 37, 41, 43, 47,
53, 59, 61, 67, 71, \ldots
$$

\subsection*{The Conjecture.}

Every even integer greater than two can be written as the sum of two primes.

\subsection*{Experimental evidence.}

$$
4 = 2 + 2, \hspace{1em}
6 = 3 + 3, \hspace{1em}
8 = 3 + 5, \hspace{1em}
10 = 3 + 7, \hspace{1em}
12 = 5 + 7.
$$

\subsection*{Can you keep going?}

\noindent $14 = \underline{\hspace{2em}} + \underline{\hspace{2em}}$
\hfill $16 = \underline{\hspace{2em}} + \underline{\hspace{2em}}$ \hfill\null
\vspace{3ex}

\noindent $18 = \underline{\hspace{2em}} + \underline{\hspace{2em}}$
\hfill $20 = \underline{\hspace{2em}} + \underline{\hspace{2em}}$ \hfill\null
\vspace{3ex}

\noindent $22 = \underline{\hspace{2em}} + \underline{\hspace{2em}}$
\hfill $24 = \underline{\hspace{2em}} + \underline{\hspace{2em}}$ \hfill\null
\vspace{3ex}

\noindent $26 = \underline{\hspace{2em}} + \underline{\hspace{2em}}$
\hfill $28 = \underline{\hspace{2em}} + \underline{\hspace{2em}}$ \hfill\null
\vspace{3ex}

\noindent $30 = \underline{\hspace{2em}} + \underline{\hspace{2em}}$
\hfill $32 = \underline{\hspace{2em}} + \underline{\hspace{2em}}$ \hfill\null
\vspace{3ex}

\noindent $34 = \underline{\hspace{2em}} + \underline{\hspace{2em}}$
\hfill $36 = \underline{\hspace{2em}} + \underline{\hspace{2em}}$ \hfill\null
\vspace{3ex}

\noindent $38 = \underline{\hspace{2em}} + \underline{\hspace{2em}}$
\hfill $40 = \underline{\hspace{2em}} + \underline{\hspace{2em}}$ \hfill\null
\vspace{3ex}

\noindent $42 = \underline{\hspace{2em}} + \underline{\hspace{2em}}$
\hfill $44 = \underline{\hspace{2em}} + \underline{\hspace{2em}}$ \hfill\null
\vspace{3ex}

\end{document}
