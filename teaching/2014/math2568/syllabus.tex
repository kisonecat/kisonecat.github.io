\documentclass[12pt]{handout}
%\usepackage{add-copyright}
\usepackage{geometry}
\geometry{margin=1in,top=0.4in,bottom=0.4in}

\title{Syllabus}
\author{Jim Fowler}
\course{Math 2568}
\date{Autumn 2013}

\usepackage[T1]{fontenc}
\usepackage{lmodern}
\usepackage{hyperref}
\usepackage{nopageno}

\newcommand{\peem}{\textsc{p.m.}}
\newcommand{\ayem}{\textsc{a.m.}}

\titlespacing*{\section}{0in}{*0}{*1}
\titlespacing*{\subsection}{0in}{*0}{*1}

\begin{document}
\maketitle

\noindent By using linear algebra, seeming different ``linear'' things
in algebra, geometry, and calculus---things like systems of certain
equations, rigid motions in geometry, certain differential
equations---can be placed in a common framework of vectors, matrices,
and linear transformations.  Viewing different things as somehow
analogous provides not only insight, but also a common toolkit of
surprisingly powerful algorithms.

%\section*{Resources}

%\noindent%
%We present five resources to help you to understand calculus.

\subsection*{Office hours}
If you have questions, want to work through linear algebra problems, or just talk
about mathematics, please attend office hours.

\vspace{1ex}%
\noindent\parbox{0.5\textwidth}{%
\noindent\begin{tabular}{@{}ll}
\textsf{Name:} & Jim Fowler \\
\textsf{Office:} & MW658 Mathematics Tower \\
\textsf{Phone:} & (773) 809--5659 \\
\textsf{Email:} & \href{mailto:fowler@math.osu.edu}{\texttt{fowler@math.osu.edu}} \\
\textsf{Website:} & \url{http://www.math.osu.edu/~fowler/}
\end{tabular}}
\noindent\parbox{0.5\textwidth}{%
\begin{tabular}{@{}ll}
\textsf{Office Hours:}
& Monday at 4\peem\ and \\
& Thursday at 2\peem \\
& and by appointment
\end{tabular}}

\vspace{1ex}\noindent
Please email me with any concerns you have; the success of this course
depends on open communication.

\subsection*{Textbook}
Our text is \href{http://books.google.com/books?id=rtKpQgAACAAJ&dq=isbn:0201658593}{the fifth edition of \textit{Introduction to Linear Algebra} by L.~W.~Johnson, R.~D.~Riess, and J.~T.~Arnold, published by Pearson.}  The textbook's ISBN is 0--201--65859--3.

\subsection*{Website}

I will post handouts on Carmen.
%\url{http://www.math.ohio-state.edu/~fowler/teaching/math758/}.

\subsection*{Lectures}

We meet weekdays from 9:10\ayem\ until 10:05\ayem\ in Lazenby Hall 0002 for an interactive lecture.

%\vfill
%\pagebreak
%%%%%%%%%%%%%%%%%%%%%%%%%%%%%%%%%%%%%%%%%%%%%%%%%%%%%%%%%%%%%%%%
\subsection*{Assessment}


There are 824 points possible in this course; earning an $\mathrm{A}$ or $\mathrm{B}$ or $\mathrm{C}$ or $\mathrm{D}$ requires earning $741$ or $659$ or $576$ or $494$ points, respectively.  The 824 points are broken down as follows.

\begin{description}
\item[\textsf{\textbf{12 quizzes (144 points; 12 points each).}}]
  An in-class quiz is scheduled most weeks on Friday, and returned the following Monday.

\item[\textsf{\textbf{2 midterms (280 points; 140 points each).}}]
The first midterm is Friday, September~20; the second midterm is Friday, November~1.

\item[\textsf{\textbf{1 final exam (400 points).}}]  The final examination will be held in our usual classroom at  8:00\ayem\ on Wednesday, December~11, 2013.  Students who earn 368 points on the final exam will earn an $\mathrm{A}$ for the course as a whole.

\end{description}

\vspace{1ex}

\subsection*{Extra credit}

\noindent Very occasionally, I may provide a worksheet you can complete for a few extra credit points.

\subsection*{You may use certain electronic calculators on exams}

A ``scientific calculator'' may be used on quizzes and exams, but it
certainly is not necessary.  Calculators with graphical capabilities
and computer algebra capabilities (such as the TI--84, TI--89, or
TI--92) will not be allowed.  Laptops, tablets, mobile phones, and the
like are also forbidden.

That said, it is my responsibility to produce exam questions which can
be answered without resorting to an electronic calculator.

\subsection*{Academic Misconduct}

It is the responsibility of the Committee on Academic Misconduct to
investigate or establish procedures for the investigation of all
reported cases of student academic misconduct. The term ``academic
misconduct'' includes all forms of student academic misconduct
wherever committed; illustrated by, but not limited to, cases of
plagiarism and dishonest practices in connection with
examinations. Instructors shall report all instances of alleged
academic misconduct to the committee. For additional information, see
the \href{http://studentaffairs.osu.edu/resource_csc.asp}{Code of Student
 Conduct} at \url{http://studentaffairs.osu.edu/resource_csc.asp}

\subsection*{Disabilities}

Students with disabilities that have been certified by the
\href{http://www.ods.ohio-state.edu}{Office for Disability Services}
will be appropriately accommodated, and should inform the instructor
as soon as possible of their needs. The Office for Disability Services
is located in 
\begin{quote}
\href{http://www.osu.edu/map/building.php?building=067}{150 Pomerene Hall}\\
1760 Neil Avenue\\
Columbus, OH\hspace{0.5em} 43210
\end{quote}
and can be reached by telephone at (614)~292--3307, by video relay
service at (614)~429--1334, and on the web at
\url{http://www.ods.ohio-state.edu/}.

\end{document}

