\documentclass[12pt]{handout}
%\usepackage{add-copyright}

\title{Homework 3}
\course{Math 2568}
\date{Attempt before Monday, September  9, 2013}
\author{Jim Fowler}

\usepackage[T1]{fontenc}
\usepackage{lmodern}
\usepackage{hyperref}

\newcommand{\peem}{\textsc{p.m.}}
\newcommand{\ayem}{\textsc{a.m.}}

\usepackage{nopageno}
\usepackage{multicol}
\geometry{margin=1cm}
%\geometry{landscape,margin=0.25in,bottom=0.25in,left=0.5in,right=0.5in}
\usepackage{tabularx}
\usepackage{rotating}

\setlength{\parindent}{0in}
\setlength{\parskip}{0in}
\usepackage{calc}

\begin{document}
\maketitle



As they say, \textit{mathematics is not a spectator sport.}  Stay involved by doing the homework.  Have fun trying to craft your own problems for fun!  If you encounter any troubles, please attend office hours.





\subsection*{Show your work}
When you ``solve'' a problem, the goal is not merely to give a correct answer!  Each problem has a story, and you should \textit{tell the story} by clearly explaining your argument.


\subsection*{The two boxed problems have solutions in the back}
Even if you don't do all the suggested problems (though you should!), the two boxed problems really should be done since you can check your answers quickly in the back of the textbook.

\section*{Suggested Problems for Practice}

From \textsection 1.5 (starting on page 46),
do problems \fbox{5}, 6, \fbox{31}, 32, 42, 44.
\vspace{1ex}


\end{document}
