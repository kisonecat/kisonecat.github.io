\documentclass[12pt]{handout}
%\usepackage{add-copyright}

\title{Homework 2}
\course{Math 2568}
\date{Attempt before Monday, September  2, 2013}
\author{Jim Fowler}

\usepackage[T1]{fontenc}
\usepackage{lmodern}
\usepackage{hyperref}

\newcommand{\peem}{\textsc{p.m.}}
\newcommand{\ayem}{\textsc{a.m.}}

\usepackage{nopageno}
\usepackage{multicol}
\geometry{margin=1cm}
%\geometry{landscape,margin=0.25in,bottom=0.25in,left=0.5in,right=0.5in}
\usepackage{tabularx}
\usepackage{rotating}

\setlength{\parindent}{0in}
\setlength{\parskip}{0in}
\usepackage{calc}

\begin{document}
\maketitle


I hope you've been enjoying
Math~2568 so far.  Ask yourself:
\textbf{are you doing the homework?}  If not, please try!  I can only
point you in the direction you should travel: you must make the
journey yourself.






\subsection*{Show your work}
When you ``solve'' a problem, the goal is not merely to give a correct answer!  Each problem has a story, and you should \textit{tell the story} by clearly explaining your argument.


\subsection*{The one boxed problems have solutions in the back}
Even if you don't do all the suggested problems (though you should!), the one boxed problems really should be done since you can check your answers quickly in the back of the textbook.

\section*{Suggested Problems for Practice}

From \textsection 1.2 (starting on page 14),
do problems 8, 18, 28, 50, 54.
\vspace{1ex}

From \textsection 1.3 (starting on page 28),
do problems 2, 4, 20, \fbox{27}, 28.
\vspace{1ex}


\end{document}
