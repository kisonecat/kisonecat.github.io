---
layout: default
title: Teaching Statement
---

\documentclass[12pt]{amsart}

\title{Teaching Statement}
\author{Jim Fowler}

\usepackage[left=1.5in,right=1.5in,top=1.25in,bottom=1.25in]{geometry}

\begin{document}
\maketitle

I love to teach.  As an instructor, I set high standards, and then
commit myself to serving my students with whatever means necessary to
ensure their success.  And we have a lot of fun of along the way.
Much of my enjoyment lies in a profound tension of mathematics: I must
guide my students towards rigor and discipline and deep
understanding---and yet, I must make that narrow path broadly
accessible and inviting to every student.  It is a wonderful
challenge.

\subsection*{High standards}

Many students begin my courses by equating mathematics with
\textit{computation}, but I want my students to understand mathematics
also as \textit{communication}.  Well-argued essays are expected in
English classes, and I likewise expect my students to write
convincing, clear mathematical arguments.  Inspiring my students to
``show their work'' shows me spots where they might be confused, which
transforms grading from mere assessment into a chance for additional,
targeted learning and feedback.  Emphasizing both careful thinking and
clear communiation makes my courses more relevant: students recognize
the real world value in these skills.

In addition to computing and communicating, I expect my students to
\textit{connect} the mathematics we study to real world scenarios.
Thus, I integrate applications and projects into the curriculum.  In
my experience, these are often the most difficult for my students, but
talent in translating between the abstract and concrete is critical in
many careers.  Mathematics is useful.

It is also beautiful.  Mathematics is a \textit{creative} and human
endeavor, and so I ask my students (on their homework) to invent their
own problems, or to explore a theorem by examining what happens when
they modify the hypotheses (much like we might discover the importance
of a car's engine by trying to drive without one).  So that my
students might see mathematics as a living endeavor, I point out the
boundary of human understanding by providing open problems related to
what we are studying, and I point out topics that they might study in
their future courses.

My goal through all this is to help my students to become
mathematicians, that they might be skilled at computing and
communicating complex concepts, and then to perceive how their skills
can shed light on the real world and on the beauty of abstraction
itself.

\subsection*{Making success achievable}

If I expect my students to put forth strong efforts in learning, I
must be a role model by putting forth my best effort in teaching.

Each student is helped best in a different way, so my first step is
information gathering; homework and tests are an obvious assessment
tools, but lecture can be, too.  I use lecture time to take instant
surveys of the class (i.e., ``close your eyes, raise your hand if you
have been following''), or to give instant quizzes (i.e., ``work on
the following and signal when you think you have it figured out;
discuss with your neighbor'').  Instead of watching mathematics, this
gets everyone doing mathematics and talking about mathematics; as a
side benefit, it gives me an idea of how the class is handling the
material, so I can adjust immediately.

I have been experimenting like this to make my lectures as engaging as
possible: I have made paper slide rules to let my students \textit{feel}
logarithms, used actual soap bubbles after solving the related
minimization problem, distributed paper models, played sounds
corresponding to functions and their derivatives, stacked a stable
tower of wooden blocks in the harmonic series, etc.  ``Lecture'' means
much more than me, talking.  I want to do everything I can so that my
students respond enthusiastically with hard work on the problem sets.

I do my best to create \textit{interesting} problem sets.  By
combining the textbook's problems with mini-projects, multi-part
problems, applications, proofs, conceptual questions, I hope to
provide questions to entice every student, to reward struggling
students with success and challenge successful students to work even
harder.  Weekly problem sessions and exam review sessions provide
organized time to discuss these problems and share ideas; my office
hours provide one-on-one time for instant feedback.

My goal is to provide a variety of resources with which to help my
students achieve high standards.  Every topic I teach has something
simply awesome about it, and I want to make sure every student has the
chance to experience it, whether in lecture, in office hours, or in working through problems.

%I often give a handout during lecture---some students prefer reading a
%summary, and distilling many ideas down to a page can be helpful.  I
%especially like handouts for providing a overview when working through
%the details of a difficult proof, or to provide extra reference
%material that is not included in the text.%

%
%I also invite my students to office hours; doing problems one-on-one
%with instant feedback is fantastic.  


%  Every topic has
%something awesome about it, and lecture provides a chance to 

%getting students to respond (patiently waiting for a response), to say more

%Write about the courses you would like to teach.?

\subsection*{Conclusion}

Teaching is a blessing, and I am excited to continue to improve my
talents that I might serve my students better.

\end{document}
