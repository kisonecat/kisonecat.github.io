\documentclass[12pt]{pset}

\course{Math 765}
\author{Jim Fowler}
\date{Winter 2011}

\newcommand{\CC}{\mathbb{C}}
\newcommand{\CP}{\mathbb{C}P}
\newcommand{\RP}{\mathbb{R}P}

\usepackage{enumerate}
\usepackage{nopageno}
\usepackage{hyperref}

\DeclareMathOperator{\Vect}{Vect}
\DeclareMathOperator{\GL}{GL}
\DeclareMathOperator{\codim}{codim}

\begin{document}
\maketitle

\noindent\textbf{This week, we study the smooth maps between manifolds.}
Please feel comfortable emailing me (\texttt{fowler@math.osu.edu}) if
you are having trouble with homework problems.  \textit{The exercises
  below should be handed in on Monday, January 24, 2011.}  These
exercises are designed to get you to think about
\textbf{transversality,} among the most powerful tools in the study of
smooth manifolds.

%%%%%%%%%%%%%%%%%%%%%%%%%%%%%%%%%%%%%%%%%%%%%%%%%%%%%%%%%%%%%%%%
\begin{problem}[Lee 8--2]

Let $F : \R^2 \to \R$ be defined by
$$
F(x,y) = x^3 + xy + y^3
$$
Which level sets of $F$ are embedded submanifolds of $\R^2$?

\end{problem}

\vfill

%%%%%%%%%%%%%%%%%%%%%%%%%%%%%%%%%%%%%%%%%%%%%%%%%%%%%%%%%%%%%%%%
\begin{problem}[Lee 8--14 and Lee 8--15]

  \begin{description}
    \item[(8--14)]
If $S \subset M$ is an embedded submanifold and $\gamma : J \to M$ is
a smooth curve whose image happens to lie in $S$, show that
$\gamma'(t)$ is in the subspace $T_{\gamma(t)} S$ of $T_{\gamma(t)} M$
for all $t \in J$.
    \item[(8--15)]
Give a counterexample if $S$ is immersed but not embedded.
  \end{description}

\end{problem}

\vfill

%%%%%%%%%%%%%%%%%%%%%%%%%%%%%%%%%%%%%%%%%%%%%%%%%%%%%%%%%%%%%%%%
\begin{problem}[Lee 8--16]

  Suppose $f : M \to N$ is a smooth map and $S \subset N$ is an
  embedded submanifold.  We say that $f$ is transverse to $S$ if, for
  every $p \in f^{-1}(S)$, the spaces $T_{f(p)} S$ and $f_\star T_p M$
  together span $T_{f(p)} N$.

  If $f$ is transverse to $S$, show that $f^{-1}(S)$ is an embedded
  submanifold of $M$ whose codimension\footnote{The codimension of a
    submanifold $N \subset M$ is $\dim M - \dim N$.} is equal to $\dim
  N - \dim S$.

\end{problem}

\vfill

\pagebreak

%%%%%%%%%%%%%%%%%%%%%%%%%%%%%%%%%%%%%%%%%%%%%%%%%%%%%%%%%%%%%%%%
\begin{problem}[Lee 8--17]

  Let $M$ be a smooth manifold. Two embedded submanifolds $S_1, S_2
  \subset M$ are said to be \textit{transverse} if, for each $p \in
  S_1 \cap S_2$, the tangent spaces $T_p S_1$ and $T_p S_2$ together
  span $T_p M$.
  \begin{itemize}
  \item 
  If $S_1$ and $S_2$ are transverse, show that $S_1 \cap S_2$ is an
  embedded submanifold of $M$ of dimension $\dim S_1 + \dim S_2 - \dim
  M$.  [It will be easier to remember this if you think of it as
  saying $\codim (S_1 \cap S_2) = \codim S_1 + \codim S_2$.]
  \item
  Give a counterexample when $S_1$ and $S_2$ are not transverse.
  \end{itemize}
  \textit{Hint:} You can invoke the previous problem, and an inclusion
  map, to make short work of this problem.

\end{problem}

\vspace{1in}

%%%%%%%%%%%%%%%%%%%%%%%%%%%%%%%%%%%%%%%%%%%%%%%%%%%%%%%%%%%%%%%%
\begin{harderproblem}

  Let $f : \R^2 \to \R$ be a polynomial so that its zero set
  $f^{-1}(0)$ is a smooth submanifold of $\R^2$.

  Recall that $\RP^1$ parametrizes lines through the origin in $\R^2$;
  show that for all but finitely many points of $\RP^1$, the
  corresponding line through the origin intersects $f^{-1}(0)$
  transversely.

  Is this still true if $f$ is not a polynomial, but merely smooth?

% for [x:y] \in \RP^1, consider f(lambda x, lambda y) = 0; transverse
% as long as there are no double roots;  p(lambda) = f(lambda,y) may
% fail to be transverse if the disc vanishes; the disc is itself a
% polynomial in y, so it has only finitely many zeros.

\end{problem}

\end{document}
