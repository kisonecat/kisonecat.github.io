\documentclass[12pt]{pset}

\course{Math 765}
\author{Jim Fowler}
\date{Winter 2011}

\newcommand{\CC}{\mathbb{C}}
\newcommand{\CP}{\mathbb{C}P}
\newcommand{\RP}{\mathbb{R}P}

\usepackage{enumerate}
\usepackage{nopageno}
\usepackage{hyperref}
\usepackage{tikz}

\DeclareMathOperator{\Vect}{Vect}
\DeclareMathOperator{\GL}{GL}
\DeclareMathOperator{\SO}{SO}
\DeclareMathOperator{\so}{\mathfrak{so}}
\DeclareMathOperator{\codim}{codim}
\DeclareMathOperator{\Mat}{Mat}
\DeclareMathOperator{\SymMat}{SymMat}
\DeclareMathOperator{\Id}{Id}
\DeclareMathOperator{\Sym}{Sym}
\newcommand{\Proj}{\mathbb{P}}
\newcommand{\LL}{\mathcal{L}}

\newcommand{\HH}{H}
\newcommand{\hh}{\mathfrak{h}}

\begin{document}
\maketitle

\noindent\textbf{Curves and flows.}  This week, we adopt a dynamic
perspective: our vector fields give rise to flows on a manifold.
Email me with questions at \texttt{fowler@math.osu.edu}.  \textit{The
  exercises below should be handed in on Monday, February 21, 2011.}

%%%%%%%%%%%%%%%%%%%%%%%%%%%%%%%%%%%%%%%%%%%%%%%%%%%%%%%%%%%%%%%%
\begin{problem}[Lee 17--4]

  Let $M$ be a connected smooth manifold.  Show that the group of
  diffeomorphisms of $M$ acts transitively on $M$.  More precisely,
  for any two points $p,q \in M$, show that there is a diffeomorphism
  $F : M \to M$ such that $F(p) = q$.

\end{problem}

\vfill

%%%%%%%%%%%%%%%%%%%%%%%%%%%%%%%%%%%%%%%%%%%%%%%%%%%%%%%%%%%%%%%%
\begin{problem}[Lee 17--13 (Collar neighborhoods exist)]

Let $M$ be a smooth manifold with boundary.  A subset $C \subset M$
containing $\partial M$ is called a \textit{collar} if $C$ is the
image of a smooth embedding $[0,1] \times \partial M \to M$ that
restricts to the obvious identification $\{ 0 \} \times \partial M
\to \partial M$.  This problem shows that every smooth manifold with
boundary has a collar.
\begin{description}
  \item[(a)] Show that here exists a vector field $N \in
    \mathcal{T}(M)$ whose restriction to $\partial M$ is
    inward-pointing.
    \item[(b)] For any positive real-valued function $\delta
      : \partial M \to \R$, define a subset $\mathcal{D}_\delta \subset
      [0,\infty) \times \partial M$ by
$$
\mathcal{D}_\delta = \{ (t,p) : p \in \partial M, 0 \leq t \leq
\delta(p) \}.
$$
Show that there are a smooth positive function $\delta : \partial M
\to \R$ and a smooth map $\theta : \mathcal{D}_\delta \to M$ such that
for each $p \in \partial M$, the map $t \mapsto \theta(t,p)$ is an
integral curve of $N$ starting at $p$.  \textit{Hint:} Use the ODE
theorem in local coordinates around each point of $\partial M$, and
define $\delta$ by means of a partition of unity.
\item[(c)] Show that $\theta$ is an embedding.
\item[(d)] Show that the image of $\theta$ is a collar.
\end{description}

\end{problem}

\vfill
\vfill

\pagebreak

%%%%%%%%%%%%%%%%%%%%%%%%%%%%%%%%%%%%%%%%%%%%%%%%%%%%%%%%%%%%%%%%
\begin{problem}[Lee Exercise 18.2]
  Prove these statements for smooth vector fields $V, W, X$ and a
  smooth function $f$.
  \begin{description}
  \item[(a)] $\LL_V W = -\LL_W V$.
  \item[(b)] $\LL_V[W,X] = [\LL_V W,X] + [W,\LL_V X]$.
  \item[(c)] $\LL_{[V,W]} X = \LL_V \LL_W X - \LL_W \LL_V X$.
  \item[(d)] $\LL_V(fW) = (Vf) W + f \LL_V W$.
  \item[(e)] If $F : M \to N$ is a diffeomorphism, then $F_\star(\LL_V
    W) = \LL_{F_\star V} F_\star W$.
  \end{description}

\end{problem}
\vfill
%%%%%%%%%%%%%%%%%%%%%%%%%%%%%%%%%%%%%%%%%%%%%%%%%%%%%%%%%%%%%%%%
\begin{problem}[Cartan's formula]

\newcommand{\InteriorMultiplication}{\lrcorner}
For a smooth vector field $X$ and a smooth form $\omega$, prove that
$$
\LL_X \omega = X \InteriorMultiplication (d\omega) + d(X
\InteriorMultiplication \omega).
$$  

\end{problem}

\vfill

%%%%%%%%%%%%%%%%%%%%%%%%%%%%%%%%%%%%%%%%%%%%%%%%%%%%%%%%%%%%%%%%
\begin{hardproblem}[Traveling along a vector field]

Does there exist a vector field on the plane $\R^2$ so that every two
points in the plane are connected by an integral curve?
  
\end{hardproblem}

\vfill

%%%%%%%%%%%%%%%%%%%%%%%%%%%%%%%%%%%%%%%%%%%%%%%%%%%%%%%%%%%%%%%%
\begin{problem}[Baker--Campbell--Hausdorff for the Heisenberg group]

  Let $\HH$ be the Lie group of $3 \times 3$ matrices of the form
  $$
  \begin{pmatrix}
    1 & a & b \\
    0 & 1 & c \\
    0 & 0 & 1
  \end{pmatrix}\hspace{1em}\mbox{for $a, b, c \in \R$,}
  $$
  and $\hh = T_e \HH$ be its tangent space at the identity.

  \vspace{1ex}\noindent%
  A vector $v \in \hh$ gives rise to the left-invariant vector
  field $X_v$ with $X_v(e) = v$; the function $\theta_{X_v} : \HH \times \R
  \to \HH$ is the flow along this vector field, and the exponential
  map $\exp : \hh \to \HH$ is defined by $\exp(v) =
  \theta_v(e,1)$.

  \vspace{1ex}\noindent%
  Given $a, b \in \hh$, find a formula for a vector $c \in \hh$
  in terms of $a$ and $b$ so that $$\exp c = (\exp a) \cdot (\exp b).$$
  Your formula will permit you to do calculations in the Lie group
  $\HH$ by doing calculations in the Lie algebra $\hh$.
  
\end{problem}

\vfill
\vfill

\end{document}
