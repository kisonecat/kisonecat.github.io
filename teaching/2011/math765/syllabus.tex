\documentclass[12pt]{handout}
%\usepackage{add-copyright}
\usepackage{geometry}
\geometry{margin=1in,top=0.45in,bottom=0.45in}

\title{Syllabus}
\course{Math 765}
\author{Jim Fowler}
\date{Winter 2011}

\usepackage[T1]{fontenc}
\usepackage{lmodern}
\usepackage{hyperref}
\usepackage{nopageno}

\newcommand{\peem}{\textsc{p.m.}}
\newcommand{\ayem}{\textsc{a.m.}}

\titlespacing*{\section}{0in}{*0}{*1}
\titlespacing*{\subsection}{0in}{*0}{*1}

\begin{document}
\maketitle

\noindent This course is the first in the differential branch of the
core topology and geometry curriculum of the Ph.D.~program and
provides a basic introduction to smooth manifolds.

% Some of the core concepts that will be discussed in the course
% include the definitions of differentiable manifolds, immersions and
% submersions, vector bundles, vector fields and flows on manifolds,
% differential operators, differential forms, Lie and exterior
% derivatives, Frobenius Theorem, some Lie theory, integration on
% manifolds, orientability, Stokes' theorem, and basic de Rham
% theory.

\section*{Resources}

\noindent%
We present five resources to help you to learn about manifolds.

\subsection*{Office hours}
If you have questions, want to work through problems, or just talk
about mathematics, please attend office hours.

\vspace{1ex}%
\noindent\parbox{0.5\textwidth}{%
\noindent\begin{tabular}{@{}ll}
\textsf{Name:} & Jim Fowler \\
\textsf{Office:} & MW658 Mathematics Tower \\
\textsf{Phone:} & (773) 809--5659 \\
\textsf{Email:} & \href{mailto:fowler@math.osu.edu}{\texttt{fowler@math.osu.edu}} \\
\textsf{Website:} & \url{http://www.math.osu.edu/~fowler/}
\end{tabular}}
\noindent\parbox{0.5\textwidth}{%
\begin{tabular}{@{}ll}
\textsf{Office Hours:}
& Monday Wednesday Friday \\
& 2:30--3:18\peem \\
& and by appointment
\end{tabular}}

\vspace{1ex}\noindent
Please email me with any concerns you have; the success of this course
depends on open communication.

\subsection*{Textbook}

Our text is John M. Lee's ``Introduction to Smooth Manifolds.''

\subsection*{Website}

I will post handouts at
\url{http://www.math.ohio-state.edu/~fowler/teaching/math765/}.

\subsection*{Lectures}

We meet Mondays, Wednesdays, and Fridays,
1:30--2:18\peem\ in Scott Lab E0245 for a lecture.

%\vfill
%\pagebreak
%%%%%%%%%%%%%%%%%%%%%%%%%%%%%%%%%%%%%%%%%%%%%%%%%%%%%%%%%%%%%%%%
\subsection*{Assessment}

There are 1000 points possible in this course, broken down as follows.
\begin{description}
\item[\textsf{\textbf{10 problem sets (600 points; 60 points each).}}]
  Homework is assigned each week, usually assigned on Monday, and
  collected the following Monday.\vspace{1ex}\\
  You should work on the homework problems together, but you must
  write up your solutions independently. \vspace{1ex}\\
  You must stay caught up with the homework. But I understand your
  schedules are very busy, so I will not penalize you for
  \textit{occasionally} turning in late homework.  Do not make a habit
  of it.

\item[\textsf{\textbf{10 quizzes (50 points; 5 points each).}}]  Each
  Friday, we will have a short quiz.

\item[\textsf{\textbf{1 final exam (350 points).}}]  The final
  examination will be a take-home exam, distributed at the last
  lecture and due the following Thursday.
\end{description}

\vspace{1ex}


\end{document}

