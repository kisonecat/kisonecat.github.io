\documentclass[12pt]{handout}

\title{Lecture 12: Divisibility}
\author{Jim Fowler}
\course{Math 345}
\date{Tuesday, October 12, 2010}
\usepackage{multicol}

\begin{document}
\maketitle

\section*{Textbook}

This lecture discusses section 4 of the textbook.

\section*{Homework} 

The homework is due Thursday, October 14, 2010.

From Section 4 of the textbook, do exercises 13, 14, and 15.

\section*{divisibility}

$d$ and $x$ integers.  $d$ divides $x$  means that there exists an integer $k$ so that $x = kd$.

\subsection*{what about zero?}

\section*{examples}

$a$ divides $a$.

If $a$ divides $b$ and $a$ divides $c$, then $a$ divides $b + c$.

If $a$ divides $b$ and $a$ divides $c$, then $a$ divides $b - c$.

If $a$ divides $b$ or $a$ divides $c$, then $a$ divides $bc$.

\section*{congruences}

$a, b, m$ integers.  $a \equiv b \pmod m$ means $m$ divides $a - b$.

prove it is an equivalence relation.

\section*{some amusing calculations}

$2^n \equiv 2 \pmod n$ for which values of $n$?

\begin{multicols}{3}
\begin{align*}
2^{3} &\equiv \mathbf{2} \pmod{3} \\
2^{4} &\equiv 0 \pmod{4} \\
2^{5} &\equiv \mathbf{2} \pmod{5} \\
2^{6} &\equiv 4 \pmod{6} \\
2^{7} &\equiv \mathbf{2} \pmod{7} \\
2^{8} &\equiv 0 \pmod{8} \\
2^{9} &\equiv 8 \pmod{9} \\
2^{10} &\equiv 4 \pmod{10} \\
2^{11} &\equiv \mathbf{2} \pmod{11} \\
2^{12} &\equiv 4 \pmod{12} \\
2^{13} &\equiv \mathbf{2} \pmod{13} \\
2^{14} &\equiv 4 \pmod{14} \\
2^{15} &\equiv 8 \pmod{15} \\
2^{16} &\equiv 0 \pmod{16} \\
2^{17} &\equiv \mathbf{2} \pmod{17} \\
2^{18} &\equiv 10 \pmod{18} \\
2^{19} &\equiv \mathbf{2} \pmod{19} \\
2^{20} &\equiv 16 \pmod{20} \\
2^{21} &\equiv 8 \pmod{21} \\
2^{22} &\equiv 4 \pmod{22} \\
2^{23} &\equiv \mathbf{2} \pmod{23} \\
2^{24} &\equiv 16 \pmod{24} \\
2^{25} &\equiv 7 \pmod{25} \\
2^{26} &\equiv 4 \pmod{26} \\
2^{27} &\equiv 26 \pmod{27} \\
2^{28} &\equiv 16 \pmod{28} \\
2^{29} &\equiv \mathbf{2} \pmod{29} \\
2^{30} &\equiv 4 \pmod{30} \\
2^{31} &\equiv \mathbf{2} \pmod{31} \\
2^{32} &\equiv 0 \pmod{32} \\
2^{33} &\equiv 8 \pmod{33} \\
2^{34} &\equiv 4 \pmod{34} \\
2^{35} &\equiv 18 \pmod{35} \\
2^{36} &\equiv 28 \pmod{36}
\end{align*}
\begin{align*}
2^{37} &\equiv \mathbf{2} \pmod{37} \\
2^{38} &\equiv 4 \pmod{38} \\
2^{39} &\equiv 8 \pmod{39} \\
2^{40} &\equiv 16 \pmod{40} \\
2^{41} &\equiv \mathbf{2} \pmod{41} \\
2^{42} &\equiv 22 \pmod{42} \\
2^{43} &\equiv \mathbf{2} \pmod{43} \\
2^{44} &\equiv 16 \pmod{44} \\
2^{45} &\equiv 17 \pmod{45} \\
2^{46} &\equiv 4 \pmod{46} \\
2^{47} &\equiv \mathbf{2} \pmod{47} \\
2^{48} &\equiv 16 \pmod{48} \\
2^{49} &\equiv 30 \pmod{49} \\
2^{50} &\equiv 24 \pmod{50} \\
2^{51} &\equiv 8 \pmod{51} \\
2^{52} &\equiv 16 \pmod{52} \\
2^{53} &\equiv \mathbf{2} \pmod{53} \\
2^{54} &\equiv 28 \pmod{54} \\
2^{55} &\equiv 43 \pmod{55} \\
2^{56} &\equiv 32 \pmod{56} \\
2^{57} &\equiv 8 \pmod{57} \\
2^{58} &\equiv 4 \pmod{58} \\
2^{59} &\equiv \mathbf{2} \pmod{59} \\
2^{60} &\equiv 16 \pmod{60} \\
2^{61} &\equiv \mathbf{2} \pmod{61} \\
2^{62} &\equiv 4 \pmod{62} \\
2^{63} &\equiv 8 \pmod{63} \\
2^{64} &\equiv 0 \pmod{64} \\
2^{65} &\equiv 32 \pmod{65} \\
2^{66} &\equiv 64 \pmod{66} \\
2^{67} &\equiv \mathbf{2} \pmod{67} \\
2^{68} &\equiv 16 \pmod{68} \\
2^{69} &\equiv 8 \pmod{69} \\
2^{70} &\equiv 44 \pmod{70} \\
2^{71} &\equiv \mathbf{2} \pmod{71}
\end{align*}
\begin{align*}
2^{72} &\equiv 64 \pmod{72} \\
2^{73} &\equiv \mathbf{2} \pmod{73} \\
2^{74} &\equiv 4 \pmod{74} \\
2^{75} &\equiv 68 \pmod{75} \\
2^{76} &\equiv 16 \pmod{76} \\
2^{77} &\equiv 18 \pmod{77} \\
2^{78} &\equiv 64 \pmod{78} \\
2^{79} &\equiv \mathbf{2} \pmod{79} \\
2^{80} &\equiv 16 \pmod{80} \\
2^{81} &\equiv 80 \pmod{81} \\
2^{82} &\equiv 4 \pmod{82} \\
2^{83} &\equiv \mathbf{2} \pmod{83} \\
2^{84} &\equiv 64 \pmod{84} \\
2^{85} &\equiv 32 \pmod{85} \\
2^{86} &\equiv 4 \pmod{86} \\
2^{87} &\equiv 8 \pmod{87} \\
2^{88} &\equiv 80 \pmod{88} \\
2^{89} &\equiv \mathbf{2} \pmod{89} \\
2^{90} &\equiv 64 \pmod{90} \\
2^{91} &\equiv 37 \pmod{91} \\
2^{92} &\equiv 16 \pmod{92} \\
2^{93} &\equiv 8 \pmod{93} \\
2^{94} &\equiv 4 \pmod{94} \\
2^{95} &\equiv 13 \pmod{95} \\
2^{96} &\equiv 64 \pmod{96} \\
2^{97} &\equiv \mathbf{2} \pmod{97} \\
2^{98} &\equiv 18 \pmod{98} \\
2^{99} &\equiv 17 \pmod{99} \\
2^{100} &\equiv 76 \pmod{100} \\
2^{101} &\equiv \mathbf{2} \pmod{101} \\
2^{102} &\equiv 64 \pmod{102} \\
2^{103} &\equiv \mathbf{2} \pmod{103} \\
2^{104} &\equiv 48 \pmod{104} \\
\end{align*}
\end{multicols}

\section*{pseudoprimes}

$2^{341} =
4479489484355608421114884561136888556243290994469299069799978201927583742360321890761754986543214231552$, which is congruent to $2$ modulo 341.  But 341 is not prime, being 11 times 31.

what is going on?

\begin{align*}
2^{341} &\equiv 2 \pmod{341} \\
2^{561} &\equiv 2 \pmod{561} \\
2^{645} &\equiv 2 \pmod{645} \\
2^{1105} &\equiv 2 \pmod{1105} \\
2^{1387} &\equiv 2 \pmod{1387} \\
2^{1729} &\equiv 2 \pmod{1729} \\
2^{1905} &\equiv 2 \pmod{1905} \\
2^{2047} &\equiv 2 \pmod{2047} \\
2^{2465} &\equiv 2 \pmod{2465} \\
2^{2701} &\equiv 2 \pmod{2701} \\
2^{2821} &\equiv 2 \pmod{2821} \\
2^{3277} &\equiv 2 \pmod{3277} \\
2^{4033} &\equiv 2 \pmod{4033} \\
2^{4369} &\equiv 2 \pmod{4369} \\
2^{4371} &\equiv 2 \pmod{4371} \\
2^{4681} &\equiv 2 \pmod{4681} \\
\end{align*}

\section*{necklace proof}

Theorem: If $p$ is prime, then $2^p \equiv 2 \pmod p$.

proof: want to show $2^p - 2$ is divisible by $p$.

$2^p - 2$ = number of strings of two symbols, where both symbols appear.

cyclic shifts split the remaining necklaces into groups of $p$ (because $p$ is prime).

% http://en.wikipedia.org/wiki/Proofs_of_Fermat's_little_theorem


\end{document}
