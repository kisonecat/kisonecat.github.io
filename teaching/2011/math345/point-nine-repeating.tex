\documentclass[11pt]{handout}
\usepackage{geometry}
\geometry{margin=1in,top=0.5in,bottom=0.5in}

\title{Point nine repeating}
\author{Jim Fowler}
\course{Math 345}

\usepackage{nopageno}
\usepackage{add-copyright}

\begin{document}
\maketitle

\subsection*{No maximum element.}

Every nonempty bounded set has a least upper bound, but not every set
contains a ``maximum.''

Does the set $(0,1)$ contains a largest element?  \textbf{No!}  For
every number in $(0,1)$, I can find a larger one: namely, if you say
that $x \in (0,1)$ is the largest number in $(0,1)$, then I will tell
you that $(1 + x)/2$ is also in $(0,1)$, but it is bigger.

It helps to think of a concrete example: you might say that $0.963$ is
the largest number in $(0,1)$, but I will retort that
$$
0.963 < \frac{1 + 0.963}{2} = .9815 \in (0,1)
$$
and $0.9815$ is bigger than your number.

\subsection*{Repeating decimals.}

You might claim that $0.\overline{9}$ is the ``biggest'' number in
$(0,1)$.  But I will say that $0.\overline{9} = 1$, so $0.\overline{9}$
is not in the set $(0,1)$.

Why?  What might we mean by $0.\overline{9}$?  Take a look at the following:
\begin{eqnarray*}
0.9 &=& 9 \cdot 10^{-1} \\
0.99 &=& 9 \cdot 10^{-1} + 9 \cdot 10^{-2} \\
0.999 &=& 9 \cdot 10^{-1} + 9 \cdot 10^{-2} + 9 \cdot 10^{-3} \\
0.9999 &=& 9 \cdot 10^{-1} + 9 \cdot 10^{-2} + 9 \cdot 10^{-3} + 9 \cdot 10^{-4} \\
\vdots & & \vdots \\
0.\overline{9} &=& 9 \cdot 10^{-1} + 9 \cdot 10^{-2} + 9 \cdot 10^{-3} + 9 \cdot 10^{-4} + \cdots,
\end{eqnarray*}
or in fancier notation, $\displaystyle\sum_{n=1}^\infty 9 \cdot 10^{-n}$.

In any case, $10 \cdot 0.\overline{9} = 9.\overline{9}$, so
\begin{eqnarray*}
9 \cdot 0.\overline{9} &=& (10 - 1) \cdot 0.\overline{9} \\
&=& 10 \cdot 0.\overline{9} - 0.\overline{9} \\
&=& 9.\overline{9} - 0.\overline{9} = 9.
\end{eqnarray*}
Divide both sides by $9$ to see that $0.\overline{9}$ must be another
name for $1$.

\subsection*{A much shorter argument.}

You might already believe that $0.\overline{3} = 1/3$.  Multiply both
sides by three, to get $0.\overline{9} = 1$.

\subsection*{There are many ways to write a number.}

Just because $1$ looks different than $0.\overline{9}$ doesn't mean it
is different: \textbf{IV} is not $4$, which is not ``four,'' which is
not $\stackrel{\cdot \cdot}{\cdot \cdot}$, but all of these (might) mean the same
thing.  \textit{The signifier is not the signified.}

\end{document}