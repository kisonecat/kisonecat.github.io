\documentclass[12pt]{handout}

\title{Lecture 21: Fibonacci numbers}
\author{Jim Fowler}
\course{Math 345}
\date{Thursday, October 28, 2010}


\begin{document}
\maketitle

\section*{Textbook}

This lecture discusses section 5 of the textbook.

\section*{Homework} 

The homework is due Tuesday, November  2, 2010.
From Section 5 of the textbook, do exercise 22.

\section*{fibonacci numbers}

every third fibonnaci number is odd.

more generally, every $k$-th number is a multiple of $F_k$

periodic mod $n$.

every number can be written as a sum of fibonacci numbers, using each number at most once

$F_k$ is the number of sequences of 1s and 2s that add up to $k-1$.

$F_{2n} = {F_{n+1}}^2 - {F_{n-1}}^2$.

Fibonacci primes with thousands of digits have been found, but it is not known whether there are infinitely many.

144 is the only nontrivial square Fibonacci number.

\subsection*{connection to pascal's triangle?}

\section*{calculus of finite differences}

a dictionary for translating calculus to sums and back again.

e.g., 
how to compute $\sum_{k=0}^n k^2$ via the calculus of finite
differences.



\end{document}
