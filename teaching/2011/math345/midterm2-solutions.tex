\documentclass[12pt]{midterm}
\title{Midterm 2 Solution Set}
\course{Math 345}
\date{November 2010}

\newcommand{\abs}[1]{\left|#1\right|}
\DeclareMathOperator{\realpart}{Re}
\DeclareMathOperator{\imagpart}{Im}

\settowidth{\introductionwidth}{\widthof{not to make simple things complicated,}}
\introduction{%
The essence of mathematics is \\
not to make simple things complicated, \\
but to make complicated things simple.
\null\hfill---Stan Gudder
}

\instructions{%
\begin{enumerate}
\item Write your name above.
\item Calculators are forbidden (and useless, anyhow).
\item Do not look inside the exam until instructed to do so.
\item You have \textbf{48 minutes} for this exam.
\item Justify your answers for full credit.
\item Show your work for generous partial credit.
\item Write your answers on the included pages, or request additional paper.
\item Answer all questions asked.
\item To prevent fire, do not divide by zero.
\vfill
\end{enumerate}
}

\newtheorem*{claim}{Claim}

\usepackage{xcolor}

%\usepackage{graphicx}
%\usepackage{eso-pic}

%\newcommand\BackgroundPicture[1]{%
%  \setlength{\unitlength}{1pt}%
%  \put(0,792){%
%  \parbox[t][\paperheight]{\paperwidth}{%
%     \vfill
%     \centering\includegraphics{#1}
%     \vfill
%}}} %
%\AddToShipoutPicture{\BackgroundPicture{barcode.pdf}}

\begin{document}
\begin{exam}

%%%%%%%%%%%%%%%%%%%%%%%%%%%%%%%%%%%%%%%%%%%%%%%%%%%%%%%%%%%%%%%%
\begin{problem}[360]
  State the binomial theorem, and then apply it to prove that, for
  nonnegative integers $n$,
$$
\sum_{k=0}^n \binom{n}{k} = 2^n.
$$
Be sure to explain your argument carefully.
\end{problem}

\begin{solution}\begin{solutiontext}
First, I state the binomial theorem: \\
For $x, y \in \mathbb{R}$ and a nonnegative integer $n$,
$$(x+y)^n = \sum_{k=0}^n {n \choose k}\,x^{n-k} \,y^k.$$

\begin{claim}
$$
\sum_{k=0}^n \binom{n}{k} = 2^n.
$$
\end{claim}
\begin{proof}
Let $x = y = 1$; then the binomial theorem gives
$$(1+1)^n = \sum_{k=0}^n {n \choose k} 1^{n-k} 1^k.$$
But $1^{k} = 1$ and $1^{n-k} = 1$, so 
$$2^n = (1+1)^n = \sum_{k=0}^n {n \choose k},$$
which is what we wanted to prove.
\end{proof}

\color{magenta!50!black}
\vfill
\setlength{\leftskip}{0in}
\subsection*{Commentary}

Some people gave an inductive proof of this fact; this is not needed.  All you need to do is apply the binommial theorem.

\end{solutiontext}\end{solution}

%%%%%%%%%%%%%%%%%%%%%%%%%%%%%%%%%%%%%%%%%%%%%%%%%%%%%%%%%%%%%%%%
\begin{problem}[360]
  Prove, by complete induction, that every integer $x \geq 2$ is
  either prime, or is the product of primes.
\end{problem}

\begin{solution}\begin{solutiontext}

    Let $P(x)$ be the statement that $x$ is either prime, or a product
    of primes.  In other words, $P(x)$ is the statement that $x = p_1
    \cdots p_n$ for primes $p_i$.
\begin{claim}
For all integers $x \geq 2$, the statement $P(x)$ is true.
\end{claim}
\begin{proof}
We proceed by induction. \\

\textbf{Base case.}  Let $x = 2$.  Then two is prime, so $P(2)$ is true.

\textbf{Inductive step.}  We assume $P(k)$ holds for $2 \leq k \leq
x$.  We want to show $P(x+1)$.

If $x+1$ is prime, then $P(x+1)$ is true.

If $x+1$ is not prime, then $x+1 = ab$ for $a,b \in \mathbb{N}$ with
$a \neq 1$ and $b \neq 1$.  Then $2 \leq a < x+1$ and $2 \leq b <
x+1$, so $P(a)$ and $P(b)$ are true.  This means
\begin{align*}
a &= p_1 \ldots p_n \mbox{ and} \\
b &= q_1 \cdots q_m 
\end{align*}
for primes $p_i$ and $q_j$, so
$$
x + 1 = ab = (p_1 \ldots p_n) (q_1 \cdots q_m)
$$
so $P(x+1)$ is true.

Thus, by strong induction, the statement $P(x)$ holds for all integers
$x \geq 2$.
\end{proof}


\color{magenta!50!black}
\vfill
\setlength{\leftskip}{0in}
\subsection*{Commentary}

Some people did not use complete induction---you really need to use complete induction here.


\end{solutiontext}\end{solution}

%%%%%%%%%%%%%%%%%%%%%%%%%%%%%%%%%%%%%%%%%%%%%%%%%%%%%%%%%%%%%%%%
\begin{problem}[360]
  Prove that there are infinitely many prime numbers.
\end{problem}

\begin{solution}\begin{solutiontext}
    \begin{claim}
      There are infinitely primes numbers.
    \end{claim}
    \begin{proof}
      Suppose not---then there are finitely many prime numbers, say $p_1, \ldots, p_n$.

      Consider $x = p_1 \cdots p_n + 1$.  By the previous problem, $x$
      can be written as a product of primes, so in particular, there
      is a prime number $p_i$ which divides $x$.  But $p_i$ also
      divides the product of all the primes, $p_1 \cdots p_n$.
      Therefore, by the theorem that an integer dividing $a$ and $b$
      divides $a - b$, we conclude that $p_i$ divides $x - p_1 \cdots
      p_n$.  But then $p_i$ divides $1$, but $p_i > 0$, so $p_i \leq
      1$. This implies $p_i = 1$, which is a contradiction.
    \end{proof}

\color{magenta!50!black}
\vfill
\setlength{\leftskip}{0in}
\subsection*{Commentary}

Many people gave very nicely written solutions for this problem.

\end{solutiontext}\end{solution}

%%%%%%%%%%%%%%%%%%%%%%%%%%%%%%%%%%%%%%%%%%%%%%%%%%%%%%%%%%%%%%%%
\begin{problem}[360]
Let $A = \{ k \in \mathbb{Z}\, |\, k \geq 3 \}$.  Prove by induction for each $n \in A$ that
$$
\binom{n}{3} = \frac{n \left( n - 1 \right) \left( n - 2 \right)}{6}.
$$
You may use the recurrence relation for binomial coefficients, and the fact that
$$
\binom{n}{2} = \frac{n(n-1)}{2}.
$$
You will get more points if you do the algebra in the induction
efficiently than if you do it correctly but inefficiently---don't just
expand everything in sight.  Look for a way to apply the distributive law
to make your work easier to understand.
\end{problem}

\begin{solution}\begin{solutiontext}
Let $P(n)$ be the statement that
      $$\binom{n}{3} = \frac{n \left( n - 1 \right) \left( n - 2 \right)}{6}.$$
      I claim, for an integer $n$ with $n \geq 3$, the statement $P(n)$ holds.
    \begin{proof}
      We proceed by induction.

\textbf{Base case.}  Consider $n = 3$.  The statement $P(3)$ is
      $$\binom{3}{3} = \frac{3 \left( 3 - 1 \right) \left( 3 - 2 \right)}{6} = 1,$$
      which is true by inspecting Pascal's triangle.

\textbf{Inductive step.}  Assume $P(n)$; we will show $P(n+1)$.
By the recurrence relation for binomial coefficients,
\begin{align*}
\binom{n+1}{3} &= \binom{n}{3} + \binom{n}{2} \\
&= \binom{n}{3} + \frac{n(n-1)}{2} && \text{(as we are told we may assume)}\\
&= \frac{n \left( n - 1 \right) \left( n - 2 \right)}{6} + \frac{n(n-1)}{2} && \text{(by the inductive hypothesis)} \\
%&= (n)(n-1) \left( \frac{n - 2}{6} + \frac{1}{2} \right) %&& \text{(by the distributive law)} \\
%&= (n)(n-1) \left( \frac{n - 2}{6} + \frac{3}{6} \right) 
&= (n)(n-1) \left( \frac{n - 2}{6} + \frac{1}{2} \right) = (n)(n-1) \frac{n + 1}{6}% = \frac{\left( n+1 \right) \left( (n+1) - 1 \right) \left( (n+1) - 2 \right)}{6},
&& \text{(by the distributive law)}
\end{align*}
which is the statement $P(n+1)$.  Therefore, by induction, $P(n)$ holds for all $n \geq 3$.
    \end{proof}

\color{magenta!50!black}
\vfill
\setlength{\leftskip}{0in}
\subsection*{Commentary}

Many people used the fact that
$$
\binom{n}{k} = \frac{n!}{k! \, (n-k)!}
$$
to prove this; however, the problem requires that you do this by induction.

\end{solutiontext}\end{solution}

%%%%%%%%%%%%%%%%%%%%%%%%%%%%%%%%%%%%%%%%%%%%%%%%%%%%%%%%%%%%%%%%
\begin{problem}[360]
Give a proof by induction that, for all $n \in \mathbb{N}$,
$$
\sum_{k=1}^n k^3 = \frac{n^2 (n+1)^2}{4}.
$$
Then, explain why $1^3 + 2^3 + 3^3 + \cdots + (2^{10})^3 \equiv 1 \pmod 3$.
\end{problem}

\begin{solution}\begin{solutiontext}

Let $P(n)$ be the statement that
$$
\sum_{k=1}^n k^3 = \frac{n^2 (n+1)^2}{4}.
$$
I claim $P(n)$ holds for all $n \in \mathbb{N}$.
\begin{proof}
  We proceed by induction.  \textbf{Base case.}  Let $n = 1$.  Then $P(1)$ asserts
$$
\sum_{k=1}^1 k^3 = \frac{1^2 (1+1)^2}{4} = 1,
$$
which is true.  \textbf{Inductive step.}  Suppose $P(n)$; we prove $P(n+1)$.  Since $P(n)$, we know
$$
\sum_{k=1}^n k^3 = \frac{n^2 (n+1)^2}{4}.
$$
Adding $(n+1)^3$ to both sides yields
\begin{align*}
\sum_{k=1}^{n+1} k^3
&= \frac{n^2 (n+1)^2}{4} + (n+1)^3 \\
&= \left( n+1 \right)^2 \left( \frac{n^2}{4} + (n+1) \right) \\
&= \left( n+1 \right)^2 \left( \frac{n^2}{4} + \frac{4n+4}{4} \right) \\
&= \left( n+1 \right)^2 \left( \frac{n^2 + 4n + 4}{4} \right) \\
&= \left( n+1 \right)^2 \left( \frac{(n+2)^2}{4} \right),
\end{align*}
which is the statement $P(n+1)$, so, by induction, we may conclude $\forall n \in \mathbb{N}\, P(n)$.
\end{proof}

\pagebreak

\begin{claim}
  $1^3 + 2^3 + 3^3 + \cdots + (2^{10})^3 \equiv 1 \pmod 3$
\end{claim}

\begin{proof}
  By the preceeding claim, $1^3 + 2^3 + 3^3 + \cdots + (2^{10})^3 = \displaystyle\frac{(2^{10})^2 (2^{10}+1)^2}{4}$, and
\begin{align*}
\frac{(2^{10})^2 (2^{10}+1)^2}{4}
&= \frac{2^{20} (2^{20}+2 \cdot 2^{10} + 1)}{4} \\
&= 2^{18} (2^{20}+2 \cdot 2^{10} + 1) \\
&= 4^{9} (4^{10}+2 \cdot 4^{5} + 1) \\
&\equiv 1^{9} (1^{10}+2 \cdot 1^{5} + 1) \pmod 3\\
&\equiv 1 (1+2 + 1) \pmod 3\\
&\equiv 4 \equiv 1 \pmod 3
\end{align*}
\end{proof}

\end{solutiontext}\end{solution}


\end{exam}


\end{document}
