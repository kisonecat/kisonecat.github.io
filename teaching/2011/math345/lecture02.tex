\documentclass[12pt]{handout}

\title{Lecture 2: Conditional sentences}
\author{Jim Fowler}
\course{Math 345}
\date{Thursday, September 23, 2010}

\begin{document}
\maketitle

\section*{Textbook}

This lecture discusses section 2 of the textbook.

\section*{Homework} 

The homework is due Tuesday, September 28, 2010.

From Section 2 of the textbook, do exercises 7, 8, and 10.

\section*{Conditional sentences}

$P \Rightarrow Q$

write out truth table

\section*{Negation of conditional sentence}

$\neg (P \Rightarrow Q)$ is equivalent to $P \wedge \neg Q$

\section*{Converse of conditional sentence}

the converse of $P \Rightarrow Q$ is $Q \Rightarrow P$.

\section*{Biconditonal sentences}

Show that $P \Leftrightarrow Q$ is the same as $P \Rightarrow Q$ and $Q \Rightarrow P$.

\section*{Quick survey of understanding}

\section*{Examples with numbers}

e.g., $x^2 = 1$


\section*{Tautologies}

a tautology is a proposition which is true, no matter the truth values
of the propositions it is built from.

formally, a proposition $P$ is a tautology if it is logically
equivalent to ``true.''

example: $P \vee \neg P$.


example: $P \wedge Q \Rightarrow P$.

\section*{Conditional proof}

How do you prove a statement like $P \Rightarrow Q$?

\section*{Modus Ponens}

If $P$ holds and $P \Rightarrow Q$ holds, then $Q$ holds.


\end{document}
