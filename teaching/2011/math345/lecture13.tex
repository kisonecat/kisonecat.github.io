\documentclass[12pt]{handout}

\title{Lecture 13: Congruences}
\author{Jim Fowler}
\course{Math 345}
\date{Wednesday, October 13, 2010}

\begin{document}
\maketitle

\section*{Textbook}

This lecture discusses section 4 of the textbook.

\section*{Homework} 

The homework is due Wednesday, October 20, 2010.

From Section 4 of the textbook, do exercises 25 and 26.

\section*{equivalence relations}

\section*{well-defined}

\section*{hardy and the usefulness of number theory}

Hardy said: ``I have never done anything `useful'. No discovery of mine has made,
or is likely to make, directly or indirectly, for good or ill, the
least difference to the amenity of the world.''

This turns out to be false.

Number theory is incredibly important to internet commerce.

prime numbers are important.

factoring numbers is hard.

simple example: proving that I wrote a secret document.  include the
product of two large primes.  at a future date, i declare ``the huge
number on the secret document is the product of $p$ and $q$'' which
proves that I must have written the document.

public-key cryptosystems

\section*{341 is pseudoprime}

\subsection*{modulo 11}

$341 = 31 \times 11$.

$$2^{11} \equiv 2 \pmod{11}$$

$$2^{31} \equiv 2^{11} \cdot 2^{11} \cdot 2^{9} \equiv 2^{11} \equiv 2 \pmod{11}.$$

$$
2^{341} \equiv \left(2^{31}\right)^{11} \equiv 2^{11} \equiv 2 \pmod{11}.
$$

\subsection*{modulo 31}

$$2^{31} \equiv 2 \pmod{31}$$

$$2^{341} \equiv 2^{11} \pmod{31}$$

$$2^{11} \equiv 2^5 \cdot 2^5 \cdot 2 \equiv (-1) \cdot (-1) \cdot 2 \equiv 2 \pmod{31}$$

\subsection*{combining these facts (without yet knowing CRT)}

$2^{341} \equiv 2 \pmod{11}$ and $2^{341} \equiv 2 \pmod{31}$.

since $2^{341} = 2 + 31k$ possible residues mod 341: 2, 33, 64, 95, 126, 157, 188, 219, 250, 281, 312

since $2^{341} = 2 + 11k$ possible residues mod 341: 2, 13, 24, 35, 46, 57, 68, 79, 90, 101, 112, 123, 134, 145, 156, 167, 178, 189, 200, 211, 222, 233, 244, 255, 266, 277, 288, 299, 310, 321, 332

\section*{necklace proof of fermat's little theorem}

one of many proofs

this is the easiest to see, i think.

\section*{find last digit}

Find the last digit of $2^{1000}$.

$2^{1000} \equiv \left(2^5\right)^{200} \equiv 2^{200} \equiv 2^{40} \equiv 2^8 \equiv 1 \pmod 5$

$2^{1000} \equiv 0 \pmod 2$

so $2^{1000} \equiv 6 \pmod{10}$

\section*{solving linear equations?}

\section*{solving quadratic equations?}

what about square roots modulo $n$?

\section*{example from hensel's lemma}

find a number $x$ so that \ldots

$x^2 \equiv 2 \pmod{7}$.  say $x = 3$

$x^2 \equiv 2 \pmod{49}$.  say $x = 3 + 7k$, say, $x = 10$.

$x^2 \equiv 2 \pmod{343}$.  say $x = 10 + 49k$, say, $x = 108$.

$x^2 \equiv 2 \pmod{2401}$.  say $x = 108 + 343k$, say, $x = 794$.

\subsection*{better example}

find a number $x$ so that

$x^3 \equiv 7 \pmod{10}$.  say $x = 3$.

$x^3 \equiv 7 \pmod{100}$.  say $x = 3 + 10k = 43$.

$x^3 \equiv 7 \pmod{1000}$.  say $x = 43 + 100k = 543$.

$x^3 \equiv 7 \pmod{10000}$.  say $x = 43 + 100k = 543$.

\subsection*{best example}

find a number $x$ so that

$x^2 \equiv 3 \pmod{11}$.  say $x = 6$.

$x^2 \equiv 3 \pmod{121}$.  say $x = 6 + 11k = 6 + 11\cdot 8 = 94$.

$x^2 \equiv 3 \pmod{1331}$.  say $x = 94 + 121k = 94 + 121 \cdot 4 = 578$.

$x^2 \equiv 3 \pmod{14641}$.  say $x = 578 + 1331k = 578 + 1331 \cdot 2 = 3240$.



\end{document}
