\documentclass[12pt]{handout}

\title{Lecture 3: Tautologies}
\author{Jim Fowler}
\course{Math 345}
\date{Monday, September 27, 2010}

\begin{document}
\maketitle

\section*{Textbook}

This lecture discusses section 2 of the textbook.

\section*{Homework}

The homework is due Wednesday, September 29, 2010.

From Section 2 of the textbook, do exercises 14, 15, and 17.

\section*{Modus Ponens}

If $P$ holds and $P \Rightarrow Q$ holds, then $Q$ holds.

\section*{Examples}

$P \Rightarrow (P \vee Q)$

$(P \wedge Q) \Rightarrow P$

$(P \wedge Q) \Rightarrow Q$

$(P \wedge Q) \Rightarrow (P \vee Q)$

$((P \vee Q) \wedge ((P \Rightarrow R) \wedge (Q \rightarrow R))) \Rightarrow Q$.

$\left( (P \Rightarrow Q) \wedge (Q \Rightarrow R) \right) \Rightarrow (P \Rightarrow R)$

\section*{Reinforcement}

Converse

Contrapositive

\section*{Examples}

\section*{Distributive law homework}

Have someone come up and write out the proof of the distributive law
$$
P \wedge (Q \vee R) \equiv (P \wedge Q) \vee (P \wedge R)
$$


\end{document}
