\documentclass[12pt]{handout}
\usepackage{geometry}
\geometry{margin=0.75in}
\usepackage{nopageno}

\title{Short Paper Instructions}
\course{Math 345}
\author{Jim Fowler}

\usepackage[T1]{fontenc}
\usepackage{lmodern}
\usepackage{hyperref}
\usepackage{multicol}
\newcommand{\peem}{\textsc{p.m.}}
\newcommand{\ayem}{\textsc{a.m.}}

\begin{document}
\maketitle

Right before you take the final exam, you can turn in a short paper on
a mathematical topic.  This is worth 360 points.  The paper should be
at least two pages long.  As is usually the case, writing your paper
will require first reading other sources.  You should cite sources
that you use.

The paper should discuss a topic that we didn't cover in the course.
Since a couple pages might not be enough to prove a theorem, you could
instead describe a mathematical object, give some examples, and state
a theorem.  Here are some possible topics:
\setlength{\itemsep}{0ex}

\begin{multicols}{3}
  \begin{itemize}
  \item Apollonian gasket
  \item Arrow's impossibility theorem
  \item Banach-Tarski paradox
  \item Bernoulli numbers
  \item Bezout's theorem
  \item Braid group
  \item Buffon's needle problem
  \item Busy Beaver function
  \item Cantor set
  \item Cardinal arithmetic
  \item Catalan numbers
  \item Category theory
  \item Cayley graphs
  \item Chebyshev polynomials
  \item Computable functions
  \item Configuration spaces
  \item Curves on surfaces
  \item Desargues' theorem
  \item Elliptic curves
  \item Equidecomposability
  \item Error correcting codes
  \item Euler characteristic
  \item Fermat's little theorem
  \item Finite fields
  \item G\"odel's incompleteness theorem
  \item Game of Nim
  \item Game theory
  \item Generating functions
  \item Geometric group theory
  \item Graph colorings
  \item Graph theory
  \item Graphs on surfaces
  \item Hyperbolic plane
  \item ISBN codes
  \item Inversive geomtery
  \item Klein bottle 
  \item Knot theory
  \item Latin squares
  \item Linkages
  \item M\"obius strip
  \item Napoleon's theorem
  \item Non-Euclidean geometry
  \item Nonabelian groups
  \item Ordinals
  \item Origami
  \item Pappus' theorem
  \item Pascal's mystic hexagon
  \item Planar graphs
  \item Polyhedra
  \item Primality testing
  \item Prime number theorem
  \item Projective geometry
  \item Public-key cryptography
  \item Quadratic reciprocity
  \item Quaternions
  \item Ramsey theory
  \item Random walks
  \item Reflections in sides of triangle
  \item Sperner's lemma
  \item Stable marriage problem
  \item Sums of two squares
  \item Surfaces
  \item Symmetries of platonic solids
  \item Three utilities problem
  \item Tilings of the plane
  \item Voting theory
  \end{itemize}
\end{multicols}

\end{document}
