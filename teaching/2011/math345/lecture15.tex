\documentclass[12pt]{handout}

\title{Lecture 15: Polynomials}
\author{Jim Fowler}
\course{Math 345}
\date{Monday, October 18, 2010}

\begin{document}
\maketitle

\section*{Textbook}

This lecture discusses section 4 of the textbook.

\section*{Homework} 

The homework is due Thursday, October 21, 2010.
From Section 4 of the textbook, do exercises 17, 18.

\section*{A message from Professor Falkner}

Dear Math 345 Student,

Are you currently majoring in mathematics?  If not, are you
considering it?  If you answered yes to either question, then I would
like you to know about an opportunity to get acquainted with a faculty
member in the Department of Mathematics whom you might eventually
decide you would like to have as your advisor for your major program.
A number of mathematics professors are offering to meet this quarter
with up to five students each to lead a five-session series of
mathematics-related activities.  Each of the professors has proposed a
different series of activities that they hope will interest students.
I strongly encourage you to participate in one of these activities
that interests you if your schedule permits it.

\subsection*{an impassioned plea to major in mathematics}

\subsection*{humorous example}

$n^2 + n + 41$ is a prime for each $n \in \mathbb{N}$?

try $n = 41$.

\section*{there are infinitely many prime numbers}

also, there are infinitely many composite numbers!  :-)

\section*{Solving congruences?}

\section*{quadratic polynomials}

Let $a,b \in \mathbb{Z}$.

Suppose $x^2 + ax + b = 0$ and $x \in \mathbb{Q}$.

What can we say about $x$?

\section*{third degree polynomials}

Let $r,s,t \in \mathbb{Z}$.

Suppose $x^3 = rx^2 + sx + t$.

Show that if $x$ is rational, $x$ is an integer.

\end{document}
