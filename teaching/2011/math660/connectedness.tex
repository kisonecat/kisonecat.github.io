\documentclass[12pt]{handout}
\usepackage{nopageno}
\usepackage{geometry}\geometry{margin=0.5in}

\title{Metric spaces}
\course{Math 660}
\author{Jim Fowler}
\date{Summer 2011}

\newcounter{question}
\newenvironment{question}{\vspace{1ex}\addtocounter{question}{1}\noindent\textbf{Task
    \arabic{question}.}}{\vfill}

\DeclareMathOperator{\interior}{Int}
\DeclareMathOperator{\closure}{Cl}

\begin{document}
\maketitle

\begin{definition}
  A set $X$ with a function $d : X \times X \to \R^{\geq 0}$ is a
  \textbf{metric space} provided
  \begin{itemize}
  \item $d(x,y) = 0$ iff $x = y$,
  \item $d(y,x) = d(x,y)$, and
  \item $d(x,z) \leq d(x,y) + d(y,z)$.
  \end{itemize}
\end{definition}

\begin{question}
  Let $P_2$ consist of pairs of indistinguishable points on the circle
  $S^1$; describe a metric space structure for $P_2$.
\end{question}

\begin{question}
  Give examples of \textbf{anti-metric spaces}, which are just like
  metric spaces, except the triangle inequality goes the other way.
\end{question}

\begin{definition}
  For $\epsilon > 0$ and $c \in X$, the \textbf{open ball of radius
    $\mathbf{\epsilon}$ and center $\mathbf{c}$} is
  $$
  B_\epsilon(c) = \{ x \in X : d(x,c) < \epsilon \}.
  $$
\end{definition}

\begin{definition}
  A set $N \subset X$ is a \textbf{neighborhood} of $x \in X$ if, for
  some $\epsilon > 0$, the ball $B_\epsilon(x)$ is contained in $N$.
\end{definition}

\begin{definition}
  A set $U \subset X$ is \textbf{open} if it is a neighborhood of each
  $x \in U$.
\end{definition}

\begin{question}
  Show that $B_\epsilon(x)$ is an open set.
\end{question}

\begin{definition}
  A set $V$ is \textbf{closed} if its complement is open.
\end{definition}

\begin{question}
  Define the interior of $X$, written $\interior X$, to be the union
  of all open sets contained in $X$.  Show that $\interior X$ is open.
\end{question}

\begin{question}
  Define the closure of $X$, written $\closure X$, to be the
  intersection of all closed sets containing $X$.  Show that $\closure
  X$ is closed.
\end{question}

\begin{question}
  A subset $S$ of $X$ is \textbf{dense} if $\closure S = X$; a metric
  space $X$ is \textbf{separable} if there is a countable dense
  subset.  Show that $\C$ is separable.
\end{question}

\begin{question}
The boundary (sometimes called the ``frontier'') of $X$, written
$\partial X$, is $\closure X - \interior X$.  Find a subset $U\subset
\C$ with $\partial U = \C$.
\end{question}

\begin{definition}
  Consider a metric space $X$ and a subset $Y \subset X$; a subset $U
  \subset Y$ is open if there exists an subset $U' \subset X$ so that
  \begin{itemize}
  \item $U'$ is open in $X$, and 
  \item $U = U' \cap Y$.
  \end{itemize}
  We likewise define \textbf{closed} for subsets of $Y$.
\end{definition}

\begin{definition}
  A subset of $X$ is \textbf{connected} if there does not exist a
  disconnection of $X$; a \textbf{disconnection} of $X$ is a pair $U,
  V \subset X$ of disjoint nonempty open subsets of $X$ with $U \cup V
  = X$.
\end{definition}

\begin{question}
  A function $f : X \to Y$ is \textbf{continuous} provided the
  pre-image of an open set is open; show that the continuous image of
  a connected set is connected.
\end{question}

\begin{question}
  A \textbf{homeomorphism} is a continuous function with continuous
  inverse; show that there is no homeomorphism from an $X$-shape to a
  $Y$-shape.
\end{question}

\begin{question}
  Describe the connected subsets of $\R$.
\end{question}

\begin{question}
  A closed, bounded, nonempty subset of $\R$ contains its supremum and
  infinum.
\end{question}

\begin{question}
  Show that a convex subset of the plane is connected.
\end{question}

\begin{question}
  A nonempty open set in the plane is connected iff it is
  path-connected.  Such a subset is called a \textbf{region}.
\end{question}

\begin{question}
  Every open subset of $\C$ is a countable union of disjoint open
  sets.
\end{question}

\begin{question}
  A nonempty closed set in the plane is connected iff it is
  path-connected?
\end{question}

\begin{question}
  Consider $X = \R - \Q$.  Is $X$ connected?
\end{question}

\begin{question}
  Consider $X = \C - \{ a+bi \in C : a, b \in \Q \}$.  Is $X$
  connected?
\end{question}

\begin{question}
  For a metric space $X$ and a open subset $S \subset X$ with
  $\closure S$ compact, define $n(S,X)$ to be the number of noncompact
  connected components of $X - S$.  How large can $n(S,\R)$ be?  How
  large can $n(S,\C)$ be?
\end{question}

\begin{question}
  A \textbf{geodesic metric space} is a metric space $X$ so that, for
  any $x, y \in X$, there exists an isometry $f : [0,d(x,y)] \to X$ with
  $f(0) = x$ and $f(d(x,y)) = y$.  Is $\C$ a geodesic metric space?
  Is $\C - \{0\}$ a geodesic metric space?
\end{question}

% \begin{question}
%   Given a normed vector space, describe the metric.  Use the theorem
%   on equivalence of norms for finite dimension vector spaces to show
%   that there exists a constant $C$ so that for any polynomial $f(x)$ of
%   degree $2000$, 
%   $$
%   f(0) \leq \int_{-1}^{1} |f(x)|^2 \, dx.
%   $$
% \end{question}

\end{document}

%%% Local Variables: 
%%% mode: latex
%%% TeX-master: t
%%% End: 
