\documentclass[12pt]{article}

\usepackage{amsmath}
\usepackage{amsthm}
\usepackage{amssymb}

\usepackage{fullpage}

\usepackage{add-copyright}

\newcommand{\Q}{\mathbb{Q}}
\newcommand{\R}{\mathbb{R}}
\newcommand{\N}{\mathbb{N}}
\newcommand{\Z}{\mathbb{Z}}

\theoremstyle{plain}% default
\newtheorem{theorem}{Theorem}[section]
\newtheorem{lemma}[theorem]{Lemma}
\newtheorem{proposition}[theorem]{Proposition}
\newtheorem{corollary}{Corollary}[theorem]
\theoremstyle{definition}
\newtheorem{definition}{Definition}[theorem]
\newtheorem{conjecture}{Conjecture}[theorem]
\newtheorem{example}{Example}[theorem]
\newtheorem{exercise}{Exercise}[theorem]
\newtheorem{remark}{Remark}[theorem]
\newtheorem*{warning}{Warning}

\title{Examples of Metric Spaces}
\author{Jim Fowler\ldots and You!}
\date{}

\begin{document}

\maketitle

We should carry with ourselves many examples of metric spaces.  So
far, we have subsets of $\R$ and the discrete metric space.

\section{One quick example}

\begin{proposition}
Let $S^1 = [0,2\pi)$, with the metric
$$
d_{S^1}( x, y ) = \min \left\{ x - y, 2\pi - (x-y) \right\}
$$ 
Then $(S^1, d_{S^1})$ is a metric space.
\end{proposition}

\begin{remark}
Thinking geometrically, $S^1$ is just the
unit circle; the distance between points $x$ and $y$ is the length
of the shortest arc joining $x$ and $y$.
\end{remark}

\section{Building new spaces from old}

There are a few techniques we will introduce here:
\begin{itemize}
\item the product of two metric spaces,
\item subsets of metric spaces,
\item rescaling the metric
\item the union of two metric spaces
\end{itemize}

\begin{definition}
Let $(X, d_X)$ and $(Y, d_Y)$ be metric spaces.  The \textbf{product} of $X$ and $Y$ (written $X \times Y$) is
the set of pairs
$$
\{ (x,y) : x \in X, y \in Y \}
$$
with the metric
$$
d_{X \times Y}\left( (x_1,y_1), (x_2, y_2) \right) = \max \left\{ d_X(x_1,x_2), d_Y(y_1,y_2) \right\}
$$
\end{definition}

\begin{proposition}
The space $(X \times Y, d_{X \times Y})$ is a metric space.
\end{proposition}

\begin{example}
We normally write $\R \times \R$ as $\R^2$.  Note that $d_{\R \times \R}$ as defined above is \textit{not} the usual Euclidean metric.
\end{example}

\begin{exercise}
What ``shape'' does a ball of radius $R$ have in the metric space $(\R \times \R, d_{\R \times \R})$?
\end{exercise}

\begin{remark}
Recall that any subset of a metric space is still a metric space.
\end{remark}

\begin{proposition}
  Let $(X,d)$ be a metric space, and pick $\epsilon > 0$.  Then $(X,
  \epsilon \cdot d)$ is again a metric space (where $\epsilon \cdot d$
  means the metric multipled by $\epsilon$).
\end{proposition}

\begin{proposition}
The  metric spaces $(\R, d)$ and $(\R, 1000d)$ are isometric.
\end{proposition}

\begin{proposition}
The  metric spaces $(S^1, d_{S^1})$ and $(S^1, 2 d_{S^1})$ are not isometric, but they are homeomorphic.
\end{proposition}

\begin{definition}
  Suppose $(X_1, d_1)$ and $(X_2, d_2)$ are metric spaces.  The \textbf{disjoint union} $X_1 \sqcup X_2$ is the metric space
having points $X_1 \sqcup X_2$, and metric
\begin{align*}
d_{X_1 \sqcup X_2}(x_1,y_1) &= d_1(x_1,y_1) \mbox{ if $x_1, y_1 \in X_1$,} \\
d_{X_1 \sqcup X_2}(x_2,y_2) &= d_2(x_2,y_2) \mbox{ if $x_2, y_2 \in X_2$,} \\
d_{X_1 \sqcup X_2}(x_1,y_2) &= 1 \mbox{ if $x_1 \in X_1$ and $y_2 \in X_2$,} \\
d_{X_1 \sqcup X_2}(x_2,y_1) &= 1 \mbox{ if $x_2 \in X_2$ and $y_1 \in X_1$.}
\end{align*}
\end{definition}
We can take the disjoint union of an indexed family of metric spaces: say $(X_i,d_i)$ are metric spaces for $i \in I$.  Then
$$
\bigsqcup_{i \in I} (X_i,d_i)
$$
can be defined as the above. 
\begin{example}
Let $(\{\star\}, d)$ be \textit{the} metric space consisting of a single point.  Then
$$
\bigsqcup_{i \in I} (\{\star\}, d)
$$
is the set $I$ with the discrete metric.
\end{example}

\begin{definition}
Let $(X,d)$ be a metric space, and suppose $\sim$ is an equivalence relation on the points of $X$, such that
$$
x \sim x' \mbox{ and } y \sim y' \Rightarrow d(x,y) = d(x',y').
$$
In other words, the equivalence relation is compatible with the metric.  Then the \textbf{quotient} space $(X/\sim, d)$ consists of equivalence classes of points of $X$, and metric
$$
d([x], [y]) = d(x,y),
$$
where $[x]$ is the equivalence class containing $x$.
\end{definition}

\begin{definition}
  Suppose $(X_1, d_1)$ and $(X_2, d_2)$ are metric spaces, and $f_i :
  (Y,d) \to (X_i,d)$ is an isometry.  Then the \textbf{union} of $X_1$
  and $X_2$ along $Y$, written $X_1 \cup_Y X_2$ is
$$
X_1 \sqcup X_2 / \sim
$$
where $x_1 \sim x_2$ if there exists $y \in Y$ with $f_i(y) = x_i$.
\end{definition}
This definition makes precise the notion of gluing together two metric spaces.

\section{The 2-adic numbers}

\begin{definition}
For $p/q \in \Q$, find integers $a, b, n$ so that
$$
\frac{p}{q} = 2^n \frac{a}{b}
$$
so that neither $a$ nor $b$ are divisible by two.  Then the \textbf{2-adic valuation} of $p/q$, written
$$
|p/q|_2
$$
is defined to be $2^{-n}$.  By convention, $|0|_2 = 0$.
\end{definition}
We think of the 2-adic valuation as measuring how many twos there are
in $p/q$.  Numbers containing a lot of twos (like 16) are small.

\begin{definition}
For $x, y \in \Q$, define the $2$-adic distance between them to be
$$
d_2(x,y) = |x-y|_2
$$
\end{definition}

\begin{definition}
The \textbf{2-adics} $(\Q_2,d_2)$ are the completion of $\Q$ with respect to the 2-adic metric $d_2$.
\end{definition}

By construction, $\Q_2$ is a complete metric space.

\begin{exercise}
Find an element of $\Q_2$ which is not in $\Q$.
\end{exercise}

\begin{exercise}
Is $\Q_2$ connected?
\end{exercise}

\begin{remark}
There is nothing special about $2$---we can similarly define $\Q_p$ for any prime $p$.
\end{remark}

\end{document}
