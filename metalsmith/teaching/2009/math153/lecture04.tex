\documentclass[12pt]{article}
\usepackage{fullpage}
\usepackage{amsthm}
\usepackage{amsmath}

\usepackage{amssymb}

\newcommand{\N}{\mathbb{N}}
\newcommand{\Q}{\mathbb{Q}}
\newcommand{\Z}{\mathbb{Z}}
\newcommand{\R}{\mathbb{R}}

\newtheorem*{example}{Example}
\newtheorem*{thm}{Theorem}

\title{Lecture 4: More about limits}
\author{Math 153 Section 57}
\date{Monday October  6, 2008}

\begin{document}
\maketitle

Continuing with chapter 11.3.

\section{Limits}

Repeat the definition: $\lim_{n \to \infty} a_n =
L$ means for every $\epsilon > 0$, there exists $K$ such that $n \geq
K$ implies $|a_n - L| < \epsilon$.

This \textbf{will} be on the first test.

Whether or not a sequence converges will be a big deal to us.  Why?
The answer to many of life's problems are be constructed as the limit
of a sequence.

Give some intuition for what the definition means.

\section{Examples}

The sequence $a_n = (-1)^n$ diverges.

\section{Useful theorems}

Limits are unique: if $\lim_{n \to \infty} a_n = L$ and also $\lim_{n
  \to \infty} a_n = M$, then $L = M$.

Limits only depend on their tails: if $\lim_{n \to \infty} a_n = L$
and $m \in \N$, then $\lim_{n \to \infty} a_{n+m} = L$.

\subsection{Sometimes we get convergence for free.}

Theorem: Nondecreasing bounded above sequences converge.

Theorem: Nonincreasing bounded below sequences converge.

Proofs are tricky: require the ``least upper bound'' axiom.  We will
assume that they hold.

Example?

Let $a_1 = 2$.  Let $a_{n+1} = a_n/2 + 1/a_n$.  The sequence is
decreasing and bounded below, and $\lim a_n = \sqrt{2}$.

Easy examples: $a_1 \in (0,1)$ and $a_{n+1} = a_n (1-a_n)$.  Very
easy dynamics---$\lim a_n = 0$.

Weird examples: $a_1 \in (0,1)$ and $a_{n+1} = 3.5 a_n (1-a_n)$.  Very
complicated dynamics!

\subsection{Continuity of addition, etc.}

If $\lim a_n = L$ and $\lim b_n = M$, then $\lim (a_n + b_n) = L + M$.

If $\lim a_n = L$ $\lim (c \cdot a_n) = c \cdot L$.

If $\lim a_n = L$ and $\lim b_n = M$, then $\lim (a_n \cdot b_n) = L \cdot M$.

Example: rational functions.

Some poetry: chiastic rules: the limit of the sum is the sum of the
limits.

\subsection{Squeezing theorem}

If there exists a $K$ so that $n \geq K$ implies $a_n \leq b_n \leq
c_n$, and if $\lim a_n = \lim c_n = L$, then $\lim b_n = L$.

Proof.

$\lim_{n \to \infty} 1/n^2 = 0$.

\end{document}
