\documentclass[12pt]{article}
\usepackage{fullpage}
\usepackage{amsthm}
\usepackage{amsmath}

\newtheorem*{example}{Example}
\newtheorem*{thm}{Theorem}

\title{Lecture 15: Kinds of convergence}
\author{Math 153 Section 57}
\date{Friday October 31, 2008}

\begin{document}
\maketitle

Following chapter 12.5.

\subsection{Ratio test}

sometimes ratio test fails because limit does not exist.

\subsection{Recall from last time}

If $\sum |a_n|$ conmverges, then $\sum a_n$ converges.

\subsection{Definitions}

We say $\sum a_n$ is ``absolutely convergent'' if $\sum |a_n|$
converges.

Absolute convergence implies convergence.

\subsection{Converse is false}

There are series which converge, but not absolutely.

\subsection{Alternating series}

Definition (signs alternate).

General form: $\sum (-1)^n a_n$ for $a_n > 0$.

Very easy to determine convergence: $\sum (-1)^n a_n$ converges
provided $\lim a_n = 0$ and $a_n$ are decreasing.

Proof:  one direction is obviousl.  The other direction: first look at partial sums $s_{2k}$.  Then
$$
s_{2k+2} = s_{2k} - (a_{2k+1} - a_{2k+2})
$$
so $s_{2k+2} < s_{2k}$.  So decreasing, and bounded below by $0$, so
converges to some number, say, $L$.

Also, $s_{2k+1} = s_{2k} + a_{2k+1}$, so taking limits, the odd terms also converge to $L$.  But if both even and odd subsequences converge to $L$, then the sequence converges to $L$.

\subsection{Examples}

$$
\sum_{n=0}^\infty \frac{(-1)^n}{n+1} = \log 2
$$

Proof?
$$
e^{\sum_{n=0}^k \frac{(-1)^n}{n+1}} = \frac{e^{1/1} e^{1/3} \cdots}{e^{1/2} e^{1/4} \cdots}
$$

$$
\sum \frac{(-1)^k}{\sqrt{k}}
$$

\subsection{Estimates}

If $\sum (-1)^n a_n = L$, then $|s_k - L| < a_{k+1}$.

$\log 2 \approx 0.6931471805599453094172321214581765680755001343$

$$
\sum_{n=0}^5 \frac{(-1)^n}{n+1} = \frac{1}{1} - \frac{1}{2} + \frac{1}{3} - \frac{1}{4} + \frac{1}{5} - \frac{1}{6}
$$
and $1/1 + 1/3 + 1/5 = 23/15$ and $1/2 + 1/4 + 1/6 = 11/12$, so the
answer is $23/15 - 11/12 = (92 - 55)/60 = 37/60$, which is $0.61\overline{6}$.

The true answer is off by no more than $1/7 \approx 0.14$, so we know $\log 2$ is between $0.47$ and $0.76$.

\subsection{Another estimate}

$\sum (-1)^n / n! = 1/e$.

$$
1/1 - 1/1 + 1/2 - 1/6 + 1/24 = 3/8 = 0.375
$$
The next term is $1/120$, so we know that $e$ is between $3/8 - 1/120
= 22/60 = 0.3\overline{6}$, and $3/8 + 1/120 = 23/60$.

And indeed, $e \approx 22.0727/60$.

\subsection{Yet another estimates}

We have
$$
\sum (-1)^k / (2k)! = \cos 1 \approx 0.54030230586813971
$$
But
$$
1/1 - 1/2 + 1/24 - 1/720 = 389/720
$$
So $\cos 1$ is between $389/720 - 1/40320 \approx 0.54025$ and $389/720 + 1/40320 \approx 0.54031$.

$$
1/1 - 1/2 + 1/24 = 13/24
$$
So $\cos 1$ is between $13/24 - 1/720 = 389/720 \approx 0.54027$ and $13/24 + 1/720 = 391/720 \approx 0.543057$.

In fact, $\cos 1 \approx 389.018 / 720$.

\subsection{Rearranging conditionally convergent series}

The order matters.

Example: rearrange the terms of $(-1)^n/n$.

\end{document}
