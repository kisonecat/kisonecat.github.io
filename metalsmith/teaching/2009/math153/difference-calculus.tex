\documentclass[12pt]{article}

\usepackage{fullpage}
\usepackage{nopageno}
\usepackage{amsmath}
\usepackage{amssymb}
\usepackage{add-copyright}

\usepackage{amsthm}
\theoremstyle{definition}
\newtheorem{condition}{Condition}
\newtheorem{definition}{Definition}
\newtheorem{example}{Example}
\newtheorem{problem}{Problem}
\newtheorem{exercise}{Exercise}

\title{Calculus of Finite Differences}
\author{Jim Fowler}

\begin{document}

\section*{Calculus of Finite Differences}

Mathematics is not just about solving problems: it is about analogies
(and analogies between analogies).  If two things are perfectly
analogous, they are exactly the same, so analogies are most
interesting when they are imperfect.

In this spirit, I present a ``dictionary,'' the term mathematicians
often use when speaking of an analogy between two fields of study.  In
this case, the analogy is between functions and sequences, between
calculus and ``the calculus of finite differences.''

\subsection*{Dictionary}
\begin{center}
\begin{tabular}{rcl}
function $f$ & $\leadsto$ & sequence $\mathbf{a}$ \\
$f(x)$ & $\leadsto$ & $\mathbf{a}_n$ \\
derivative $\left(\frac{d}{dx}f\right)(x)$ & $\leadsto$ & difference $(D\mathbf{a})_n = \mathbf{a}_{n+1} - \mathbf{a}_n$ \\
integral $F(x) = \displaystyle\int_0^x f(x) \, dx$ & $\leadsto$ & sum $\mathbf{b}_n = (I\mathbf{a})_n = \displaystyle\sum_{k=0}^n \mathbf{a}_k$ \\
fundamental theorem of Calculus &$\leadsto$& telescoping sum \\
differential equation &$\leadsto$& difference equation \\
\end{tabular}
\end{center}
A true statement about functions and their derivatives can be
``translated'' into a possibly true statement about sequences and
their differences.

\subsection*{Notation}

Sequences have bold names, e.g., $\mathbf{a}, \mathbf{b}, \mathbf{c}$.
The terms of a sequence are indexed by a subscript, e.g.,
$\mathbf{a}_n$; the index may be any integer, e.g.,
$\mathbf{a}_{-23}$.  Operators, which transform one sequence into
another, are denoted by a capital letter.  Some operators include:
\begin{center}
\begin{tabular}{ccc}
\textbf{Difference} & \textbf{Sum} & \textbf{Shift} \\
$\left(D\mathbf{a}\right)_n = \mathbf{a}_{n+1} - \mathbf{a}_n$ &
$\left(\Sigma \mathbf{a}\right)_n = \displaystyle\sum_{i=0}^n \mathbf{a}_i$ &
$(S\mathbf{a})_n = \mathbf{a}_{n+1}$.
\end{tabular}
\end{center}

\subsection*{The sum rule}
The ``derivative'' of a sum is the sum of the ``derivatives,'' as
follows:
\begin{eqnarray*}
\left( D\left(\mathbf{a} + \mathbf{b}\right) \right)_n 
&=& \left(\mathbf{a} + \mathbf{b} \right)_{n+1} - \left(\mathbf{a} + \mathbf{b} \right)_{n} \\
&=& \mathbf{a}_{n+1} + \mathbf{b}_{n+1} - \mathbf{a}_n - \mathbf{b}_n \\
&=& \left( \mathbf{a}_{n+1} - \mathbf{a}_n \right) + \left( \mathbf{b}_{n+1} - \mathbf{b}_n \right) \\
&=& \left(D\mathbf{a}\right)_n + \left(D\mathbf{b}\right)_n.
\end{eqnarray*}
Formally, this resembles $(f + g)'(x) = f'(x) + g'(x)$.

\subsection*{The fundamental theorem of Calculus}

Suppose $\mathbf{a} = D\mathbf{b}$, so $\mathbf{b}$ is an
``antiderivative'' for $\mathbf{a}$.  Then,
\begin{eqnarray*}
\left(\Sigma \mathbf{a}\right)_n &=& \left(\Sigma D\mathbf{b}\right)_n \\
&=& \sum_{i=0}^n \left( \mathbf{b}_{i+1} - \mathbf{b}_i \right) \\
&=& \left( \mathbf{b}_{n+1} - \mathbf{b}_n \right) + 
 \left( \mathbf{b}_{n} - \mathbf{b}_{n-1} \right) + 
 \left( \mathbf{b}_{n-1} - \mathbf{b}_{n-2} \right) + \cdots +
 \left( \mathbf{b}_{1} - \mathbf{b}_{0} \right) \\
&=& \mathbf{b}_{n+1} - \mathbf{b}_0.
\end{eqnarray*}
Formally, this resembles the fact that if $f(x) = F'(x)$, then
$\displaystyle\int_0^x f(x)\, dx = F(x) - F(0)$.

\subsection*{The product rule}
We can take successive differences of the product of two sequences as
follows:
\begin{eqnarray*}
D\left(\mathbf{a} \cdot \mathbf{b}\right)_n
&=& \left( \mathbf{a} \cdot \mathbf{b} \right)_{n+1}  - 
\left( \mathbf{a} \cdot \mathbf{b} \right)_n  \\
&=& \mathbf{a}_{n+1} \cdot \mathbf{b}_{n+1}  - 
\mathbf{a}_n \cdot \mathbf{b}_n  \\
&=& \mathbf{a}_{n+1} \cdot \mathbf{b}_{n+1} - 
\mathbf{a}_{n} \cdot \mathbf{b}_{n+1} + 
\mathbf{a}_{n} \cdot \mathbf{b}_{n+1} - 
\mathbf{a}_n \cdot \mathbf{b}_n  \\
&=& \left(\mathbf{a}_{n+1} - \mathbf{a}_n\right) \cdot \mathbf{b}_{n+1} +
\mathbf{a}_{n} \cdot \left(\mathbf{b}_{n+1} - \mathbf{b}_n\right) \\
&=& \left( D\mathbf{a}\right)_n \cdot \mathbf{b}_{n+1} + \mathbf{a}_{n} \cdot \left(D\mathbf{b} \right)_n.
\end{eqnarray*}
Formally, this resembles $(f \cdot g)'(x) = f'(x) \cdot g(x) + f(x) \cdot g'(x)$.

\subsection*{Exercises}

\begin{exercise}
Let $\mathbf{a}_n = n$.  What is $D\mathbf{a}$?
\end{exercise}

\begin{exercise}
Let $\mathbf{a}_n = n \cdot n$.  What is $D\mathbf{a}$?
\end{exercise}

\begin{exercise}
  Find a sequence $\mathbf{a}$ solving the difference equation
  $D\mathbf{a} = 2$.
\end{exercise}

\begin{exercise}
  Find a sequence $\mathbf{a}$ solving the difference equation $D^2
  \mathbf{a} = 2$.  Can you find another sequence solving the equation
  (besides $\mathbf{a} + C$)?  Can you find a third solution?
\end{exercise}

\begin{exercise}
  Find a nonzero sequence $\mathbf{a}$ solving the difference equation
  $D\mathbf{a} = \mathbf{a}$.
\end{exercise}

\begin{exercise}
  Find a nonzero sequence $\mathbf{a}$ solving the difference equation
  $D\mathbf{a} = c \cdot \mathbf{a}$.
\end{exercise}

\begin{exercise}
  Find a nonzero sequence $\mathbf{a}$ solving the difference equation
  $D\mathbf{a} = c \cdot \mathbf{a}$.  Is this always possible?
\end{exercise}

\begin{exercise}
  Find a nonzero sequence $\mathbf{a}$ solving the difference equation
  $D^2 \mathbf{a} = - \mathbf{a}$.  This is quite interesting; a sequence which begins
$$
2, 2, 0, -4,  -8, -8, 0, 16, 32, 32, 0, -64, -128, -128, 0, 256, 512, \ldots
$$
is a solution---but can you find the general pattern?
\end{exercise}

\subsection*{Shift equations}

Another equation you might enjoy solving is a ``shift equation''
(these are usually called \textbf{recurrence relations}): can you find
a sequence $\mathbf{a}_n$ which solves the equation
$$
S\mathbf{a} = 4 \cdot \mathbf{a}?
$$
The Fibonacci numbers $\mathbf{f}_n$ are a solution to a shift equation:
$$
S^2 \mathbf{f} = S \mathbf{f} + \mathbf{f}.
$$

\subsection*{Solving a shift equation}

As a toy problem to consider, find a sequence $\mathbf{a}_n$ which
solves
$$
S^2 \mathbf{a} = 5 \cdot S\mathbf{a} - 6 \cdot \mathbf{a}
$$
by the following method.  Rearrange this equation to have the form
$$
S^2 \mathbf{a} - 5 \cdot S\mathbf{a} + 6 \cdot \mathbf{a} = 0.
$$
``Factor'' the operators to get
$$
(S-3)\,(S-2)\,\mathbf{a} = 0.
$$
But $(S-2)$ kills the sequence $\mathbf{a}_n = 2^n$, and 
$(S-3)$ kills the sequence $\mathbf{a}_n = 3^n$.  Is it the case that
$$
\mathbf{a}_n = \alpha \cdot 2^n + \beta \cdot 3^n
$$
is a solution to $S^2 \mathbf{a} = 5 \cdot S\mathbf{a} - 6 \cdot
\mathbf{a}$, for any constants $\alpha$ and $\beta$?  What happens if
you use this same trick on the Fibonacci sequence?

\end{document}