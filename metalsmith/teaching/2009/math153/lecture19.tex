\documentclass[12pt]{article}
\usepackage{fullpage}
\usepackage{amsthm}
\usepackage{amsmath}

\newtheorem*{example}{Example}
\newtheorem*{thm}{Theorem}

\title{Lecture 19: Power series}
\author{Math 153 Section 57}
\date{Monday November 10, 2008}

\begin{document}
\maketitle

Following chapter 12.8.

\subsection*{Some loose ends}

using Taylor series to instantly solve differential equations

dividing series

\subsection*{Power series}

definition of power series

point out that discussing $\sum a_n x^n$ suffices, since $\sum a_n
(x-a)^n$ is just a slight modification.

\subsection*{Convergence}

If $\sum a_n x^n$ converges at $c$, then it converges absolutely for all $x$ with $|x| < |c|$.

Proof: use limit comparison test.  $|a_n x^n| / |a_n c^n|$ converges to zero, and since $\sum |a_n c^n|$ converges, so does $\sum |a_n x^n|$.

Ask about the set of convergence.  The possibilities are
\begin{itemize}
\item convergence only at zero (radius $0$)
\item converges everywhere (radius $\infty$)
\item converges for $(-r,r)$ and diverges for $(-\infty,-r) \cup (r,\infty)$.
\end{itemize}

interval of convergence

On the endpoints of the interval, we cannot say anything (examples: $x^n$, $(-1)^n x^n / n$, $x^n/n$, $x^n / n^2$)

\end{document}
