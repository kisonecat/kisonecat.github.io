\documentclass[12pt]{article}

\usepackage{hyperref}
\usepackage{fullpage}
\usepackage{nopageno}
\usepackage{amsthm}
\usepackage{amsmath}
\usepackage{amssymb}
\newcommand{\R}{\mathbb{R}}
\newcommand{\N}{\mathbb{N}}
\usepackage[margin=1in]{geometry}
\usepackage{wasysym}
\usepackage{add-copyright}

\usepackage{graphicx}

\title{Homework 11}
\date{Due Monday, November 10, 2008}

\long\def\symbolfootnote[#1]#2{\begingroup%
\def\thefootnote{\fnsymbol{footnote}}\footnote[#1]{#2}\endgroup}

\begin{document}
\maketitle

\begin{description}

\item[(a)] Without using a calculator, estimate $\sin 2$ to within 0.05.

\vfill

\item[(b)] Without using a calculator, estimate $\cos 2$ to within 0.05.

\vfill

\item[(c)] Find a series representation for $\sin^2 x$.  Hint: use a double-angle formula.

\vfill

\item[(d)]  The Taylor series for $\arctan$ is given by
$$
\arctan x = \sum_{n=0}^\infty \frac{(-1)^n x^{2n+1}}{2n + 1}
$$
and converges when $|x| \leq 1$.  Using the fact that $\arctan 1 = \pi/4$, approximate $\pi$ to within $0.5$.

\vfill

\item[(e)] Use Lagrange's theorem to prove that
$$
\cos x = \sum_{n=0}^\infty \frac{(-1)^n \, x^{2n}}{(2n)!}
$$
for all real $x$.

\vfill

\end{description}


\end{document}
