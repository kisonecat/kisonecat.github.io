\documentclass[12pt]{article}
\usepackage{fullpage}
\usepackage{amsthm}
\usepackage{amsmath}

\newtheorem*{example}{Example}
\newtheorem*{thm}{Theorem}

\title{Lecture 20: Differentiating and integrating series}
\author{Math 153 Section 57}
\date{Wednesday November 12, 2008}

\begin{document}
\maketitle

Following chapter 12.9.

\section{Differentiating term-by-term}

Define $f : (-r,r) \to R$ by $f(x) = \sum_{n=0}^\infty a_n x^n$.

If $\sum_{n=0}^\infty a_n x^n$ converges on $(-r,r)$, then
$$
\sum_{n=1}^\infty a_n \cdot n \cdot x^{n-1}
$$
the term-wise derivative, also converges on $(-r,r)$.

Moreover, $f$ is differentiable on $(-r,r)$, and
$$
f'(x) = \sum_{n=0}^\infty \frac{d}{dx} \left( a_n x^n \right) = \sum_{n=1}^\infty a_n \cdot n \cdot x^{n-1}
$$

Consequently: if you plug a power series into the machine that finds a
Taylor series expansion, you get out the original power series.  This
is a key point, because if we can produce a power series by some other
method, then we have found a Taylor series.

\section{Radius of convergence, not same interval}

$\sum_{n=1}^\infty x^n/n^2$ converges on $[-1,1]$, but the dreivative only converges on $[-1,1)$.

\section{Examples}

$\frac{d}{dx} e^x = e^x$

$\frac{d}{dx} \sin x = \cos x$

\section{Integrating term-by-term}

Define $f : (-r,r) \to R$ by a series
$$
f(x) = \sum_{n=0}^\infty a_n x^n
$$
which converges for $x \in (-r,r)$

Define
$$
F(x) = \sum_{n=0}^\infty \frac{a_n \, x^n}{n+1}
$$
which also converges for $x \in (-r,r)$, and 
$$
\int f(x) \, dx = F(x) + C
$$

Remember the C.

\section{Logs}

Since $1/(1+x) = \sum_{n=0}^\infty (-1)^n x^n$, we have that
$$
\log (1+x) = \sum_{n=0}^\infty \frac{(-1)^n x^{n+1}}{n+1} + C
$$
But $C = 0$.

\section{Endpoints}

Abel's theorem: if $\sum_{n=0}^\infty a_n x^n$ converges on $(-r,r)$, and $f(x)$ equals the series there.

If $f$ is left continuous at $r$ and the series converges, then $f(r) = \sum_{n=0}^\infty a_n r^n$.

If $f$ is right continuous at $-r$ and the series converges, then $f(r) = \sum_{n=0}^\infty a_n (-r)^n$.

So endpoints do get filled in correctly by series.

\section{Not a correct argument whatsoever}

$$
\sin x = \left(1 - x/\pi\right) \left(1 + x/\pi\right) \left(1 - \frac{x}{2\pi}\right) \left( 1 + \frac{x}{2\pi} \right) \cdots
$$
Multiply pairs of positive and negative roots, to get
$$
\frac{\sin x}{x} = \left(1 - \frac{x^2}{\pi^2}\right) \left(1 - \frac{x^2}{4\pi^2} \right) \cdots
$$
Multiply it out?
$$
1 - \left(\frac{1}{\pi^2} + \frac{1}{4\pi^2} + \frac{1}{16\pi^2} + \cdots\right) x^2 + \cdots
$$
But this should be the same as the Taylor series for $\frac{\sin x}{x}$.  So, maybe
$$
\sum_{n=1}^\infty \frac{1}{n^2\,\pi^2} = \frac{1}{3!}
$$
So maybe $\sum_{n=1}^\infty 1/n^2 = 1/6$.

Well, no.  We don't have any justification for making these arguments---why should that infinite product be equal to $(\sin x)/x$?

\end{document}
