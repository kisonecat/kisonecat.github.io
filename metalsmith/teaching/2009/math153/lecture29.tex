\documentclass[12pt]{article}
\usepackage{fullpage}
\usepackage{amsthm}
\usepackage{amsmath}

\newtheorem*{example}{Example}
\newtheorem*{thm}{Theorem}

\title{Lecture 29: More differential equations}
\author{Math 153 Section 57}
\date{Wednesday December  3, 2008}

\begin{document}
\maketitle

Following chapter 9.2.

It is the end.

Gottfried Leibniz and Isaac Newton's epic battle---commemorated in cookies.

Review sheet for differential equations.

% http://en.wikipedia.org/wiki/Examples_of_differential_equations

Solution curves for first order equations

Symmetries in solving differential equations:

\subsection*{Factoring the operator}

What about something like
$$
f''(x) - 2 f'(x) + f(x) = 0
$$
This is $(\frac{d}{dx} - 1) \, f(x) = 0$, so $e^x$ is a solution.  But
is there another?

Yes!  $x \, e^x$.

In general, an $n$-th order linear homogeneous equation has $n$
``distinct'' solutions.

Factor the operator---do not fear this.  This is just taking a
complicated procedure, and breaking it up into pieces, each of which
can be done in order (and, in fact, in any order!).

Factors of the form $( \frac{d}{dx} - r )$ contribute a $e^{rx}$.

Factors of the form $( \frac{d}{dx} - r )^n$ contribute a $e^{rx}$, and $x e^{rx}$, $x^2 e^{rx}$, through $x^{n-1} e^{rx}$.

Knowing that this is everything requires some fancier stuff
(``determinants, Wronksians, linear algebra'').

\subsection*{Encoding Fibonacci numbers}

Shift operator
$$
(S^2 - S - 1) f_n = 0
$$
But $x^2 - x - 1$ has solutions $x = (1 \pm \sqrt{5})/2$.  Call
these $A \approx -0.62$ and $B \approx 1.62$.

So we can write this as
$$
(S - A)(S - B) f_n = 0
$$
What sorts of sequences have $(S - A) f_n = 0$?  Have $(S-B) f_n = 0$?  

So $C A^n + D B^n = f_n$.

If $f_0 = 1$, then we want $C + D = 1$.

If $f_1 = 1$, then we want $CA + DB = 1$.  But $A = 1 - B$ and $C = 1 - D$, so
$$
(1-D)(1-B) + DB = 1
$$
so
$$
D = \frac{1}{\sqrt{5}} \rho
$$
and so
$$
f_n = (1 - \sqrt{5} \rho) (1-\rho)^n - \frac{1}{\sqrt{5}} \rho \rho^n
$$

\subsection*{Qualitative questions about solutions}

Suppose $f''(x) = -f(x)$.

Can you tell that $f$ is periodic?  That $f$ is bounded?

Find a conserved quantity!

$f'(x)^2 + f(x)^2$ is constant.  How do we know?  Differentiate!
$$
\frac{d}{dx} \left( f'(x)^2 + f(x)^2 \right) = 2 f''(x) f'(x) + 2 f'(x) f(x) = 0
$$
More than studying solutions: we know that $f$ must be bounded.  We
have discovered conservation of energy---as a consequence of the
equation.

More importantly, we have discovered that sine and cosine are related
by the Pythagorean identity!  At last, the ``circle'' is closed.

\subsection*{More complicated example}

% taylor series as a solution

% bibliography for other books you might want to look at
% Fourier Series by Korner
% Rudin's analysis
% GEneratingfunctionology

\subsection*{What next?}

Hand out.

\subsection*{Course evaluations}

\subsection*{Farewell}

\end{document}
