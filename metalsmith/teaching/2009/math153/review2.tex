\documentclass[12pt]{article}
\usepackage[margin=1in]{geometry}
\usepackage{nopageno}
\usepackage{multicol}
\usepackage{add-copyright}

\title{Midterm 2 Review Sheet}

\usepackage{amsthm}
\theoremstyle{definition}
\newtheorem*{example*}{Example}
\newcommand{\limn}{\displaystyle\lim_{n \to \infty}}

\usepackage{amssymb}
\newcommand{\R}{\mathbb{R}}
\newcommand{\N}{\mathbb{N}}

\begin{document}

\section*{The Second Midterm}

\begin{itemize}
\item There will be \textbf{12 questions} on the exam and it will last
  \textbf{50 minutes}.

\item The first question will ask you to explain what
  $\displaystyle\sum_{n=1}^\infty a_n = L$ means (i.e., give a
  definition of the limit of an infinite series, by invoking partial
  sums).

\item The last question will be an extra credit problem, with some
  true/false questions.

\item I was very impressed with your performance on the first midterm,
  and will be expecting greatness.  Do not be lulled into a false
  sense of security.

\end{itemize}

\section*{Topics covered on the exam}

\begin{multicols}{2}
\begin{itemize}
\item Define $\displaystyle\sum_{n=1}^\infty a_n = L$.
\item Determine whether a series converges by using
\begin{itemize}
\item the $n^{\mbox{th}}$ term test,
\item the integral test,
\item comparison test,
\item limit comparison test,
\item $p$-series test,
\item geometric series test,
\item harmonic series test,
\item the root test,
\item the ratio test,
\item alternating series test.
\end{itemize}
A wise student may infer that I am not likely to ask about the integral test, or the root test.
\item Provide statements of the preceeding tests.
\item Determine whether a series converges absolutely; determine whether a series conditionally.
\item Given a function, write down the first few terms of its Taylor series (around $0$ or around another point $a$)
\item Given a function, write down its Taylor series.
\item Approximate an alternating series and estimate the error in your approximation.
\item Approximate a Taylor series and estimate the error in your approximation.
\item State Taylor's theorem and Lagrange's theorem.
\item Determine the interval on which a power series converges.
\item Differentiate and integrate a power series term-by-term.
\end{itemize}
\end{multicols}

\pagebreak

\section*{What should I write for an answer?}

You should not merely give the answer: you should give an explanation.
For full credit, you must \textbf{justify you argument}.  In
particular, if you are claiming that a series converges, you must
state which test you are applying.

\section*{Memorize the following series}

\begin{eqnarray*}
\frac{1}{1-x} &=& \sum^{\infty}_{n=0} x^n = 1 + x + x^2 + x^3 + \cdots \hspace{1em}(\mbox{when $-1 < x < 1$)} \\
e^{x} &=& \sum^{\infty}_{n=0} \frac{x^n}{n!} = 1 + x + \frac{x^2}{2!} + \frac{x^3}{3!} + \cdots \\
\sin x &=& \sum^{\infty}_{n=0} \frac{(-1)^n}{(2n+1)!} x^{2n+1}\quad =  x - \frac{x^3}{3!} + \frac{x^5}{5!} - \cdots \\
\cos x &=& \sum^{\infty}_{n=0} \frac{(-1)^n}{(2n)!} x^{2n}\quad =  1 - \frac{x^2}{2!} + \frac{x^4}{4!} - \cdots \\
\log x &=& \sum_{n=1}^\infty \frac{(-1)^{n+1} \, (x-1)^n}{n} = (x-1) - \frac{(x-1)^2}{2} + \frac{(x-1)^3}{3} - \cdots \hspace{1em}\mbox{(when $0 < x \leq 2$)}
\end{eqnarray*}


\end{document}