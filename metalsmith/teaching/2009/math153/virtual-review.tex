\documentclass{article}
\usepackage[pdftex,dvipsnames]{color}
\usepackage[paperwidth=4in,paperheight=3in,margin=0.25cm]{geometry}
\usepackage{nopageno}
\usepackage{add-copyright}

\usepackage{amsthm}
\newtheorem*{theorem}{Theorem}
\newtheorem*{corollary}{Corollary}
\newtheorem*{lemmma}{Lemma}
\newtheorem*{proposition}{Proposition}

\theoremstyle{definition}
\newtheorem*{remark}{Remark}
\newtheorem*{example}{Example}
\newtheorem*{definition}{Definition}

\usepackage{amssymb}
\newcommand{\R}{\mathbb{R}}
\newcommand{\N}{\mathbb{N}}

\newcommand{\limn}{\displaystyle\lim_{n \to \infty}}

\title{Virtual Review Session for Midterm 2}
\author{Jim Fowler}

\begin{document}

\setlength{\parindent}{0in}
\pagecolor{black}
\color{white}

%We'll be gathering here at 6pm Central Time for a virtual review session.

The future is commonly understood to contain all events that have yet to occur.
Welcome to the future of education.

For the Alternating Series Test, if you have already proved a series not to absolutely convergent and the series fails one of the alternating series conditions (decreasing or positive...) does the series automatically converge even if the limit is 0?

It turns out that if an alternating series fails to converge absolutely, and fails the alternating series test, there isn't much we can say.

An example of a problematic series:
$a_n = 2/n$ if $n$ is odd, and $a_n = -1/(n-1)$ is $n$ is even.

If the sum of two converging series converges, if a sum is made up of a converging and diverging sequence does it diverge? I am guessing not, but I am not sure!

The sum of a divergin series and a diverging series?  May or may not converge.

$$
\sum n + \sum -n = \sum (n-n) = \sum 0
$$

In the taylor series thereom remainder, there is a $t$. What is that $t$?

$$
\frac{1}{n!} \int_0^x f^{(n+1)} (t)\, (x - t)^n \, dt
$$

When figuring out if series converge, can you use composition of
functions? For example, if you have the sine of a converging function
does sine converge?

$$
\sum \cos\left(\frac{1}{n^2}\right) \mbox{diverge}
$$

Please give a canon definition for the sum of a sub n equals L (the
first bullet point on the review sheet)

$$
\sum_{n=0}^\infty a_n = L
$$
means
$$
\lim_{k \to \infty} \sum_{n=0}^k a_n = L
$$

Define the sequence of partial sums
$$
s_k = \sum_{n=0}^k a_n
$$
A series converges to $L$ if the sequence of partial sums converges to $L$
$$
\lim_{k \to \infty} s_k = L
$$

\begin{theorem}[Taylor's theorem]
$f : \R \to \R$ is a smooth function, and
$$ f(x) = f(a) + \frac{f'(a)}{1!}(x - a) + \cdots + \frac{f^{(n)}(a)}{n!}(x - a)^n + R_n(x). $$
 $$ R_n(x) = \int_a^x \frac{f^{(n+1)} (t)}{n!} (x - t)^n \, dt$$
\end{theorem}

\begin{theorem}[Lagrange]
$$  R_n(x) = \frac{f^{(n+1)}(c)}{(n+1)!} x^{n+1}.$$
for some $c \in (0,x)$.
\end{theorem}

How did you apply the limit comparison test on that $sin(1/n^2)$
summation?

$$
\sum_{n=1}^\infty \frac{1}{n^2} = \sum a_n
$$

$$
\sum_{n=1}^\infty \sin\left(\frac{1}{n^2}\right) = \sum b_n
$$

$$
\lim_{n \to \infty} b_n/a_n = \lim_{n \to \infty} \frac{\sin(1/n^2)}{1/n^2} = 1
$$

The limit comparison test aplies, and $\sum a_n$ converges iff $\sum
b_n$ converges.

For the Limit Comparison test: if we are just making sure that lim
(original/compared series) L, and L must be positive. Then the original
converges only if the compared one converges. When will L not be positive?

$$
\sum 1/n = \sum a_n \hspace{1em} \sum 1/n^2 = \sum b_n
$$

$$
\lim_{n \to \infty} b_n/a_n = \lim_{n \to \infty} \frac{1/n^2}{1/n} = \lim 1/n = 0
$$

If $\sum a_n$ converge, then $\sum b_n$ converge.

$$
\sum_{n=1}^\infty \frac{n!}{(n+1)!} = \sum_{n=1}^\infty \frac{1}{n+1}
$$
which diverges by harmonic series test.

Could you perform the Taylor expansion for arctan(x)?
$$
\arctan x = \sum^{\infty}_{n=0} \frac{(-1)^n}{2n+1} x^{2n+1}
$$

$$
\frac{d}{dx} \arctan x = \frac{1}{1+x^2} = \sum_{n=0}^\infty (-1)^n x^{2n}
$$

$$
\int \frac{d}{dx} \arctan x \, dx = \int \frac{1}{1+x^2} \, dx = \int \sum_{n=0}^\infty (-1)^n x^{2n} \, dx
$$

$$
\arctan x + C = \sum_{n=0}^\infty \int (-1)^n x^{2n} \, dx = \sum_{n=0}^\infty (-1)^n \frac{x^{2n+1}}{2n+1} 
$$

2. Problem 25 on page 636 in the Salas book.
$$
f(x) = x \cdot \log \frac{1+x^2}{1-x^2}
$$

$$
\log \frac{a}{b} = \log a - \log b
$$

$$
\log(1+x) =  \sum^{\infty}_{n=1} (-1)^{n+1}\frac{x^n}n
$$

$$
\log(1+x^2) =  \sum^{\infty}_{n=1} (-1)^{n+1}\frac{x^{2n}}n
$$

$$
\log(1-x^2) = \sum^{\infty}_{n=1} (-1)^{2n+1} \frac{x^{2n}}n
$$

\begin{eqnarray*}
\log \frac{1+x^2}{1-x^2} &=& \log (1+x^2) - \log (1-x^2) \\
&=& \sum^{\infty}_{n=1} (-1)^{n+1}\frac{x^{2n}}n - \sum^{\infty}_{n=1} (-1)^{2n+1} \frac{x^{2n}}n \\
&=& \sum^{\infty}_{n=1} \left( (-1)^{n+1} - 1 \right) \frac{x^{2n}}n \\
\end{eqnarray*}

\begin{eqnarray*}
x \cdot \log \frac{1+x^2}{1-x^2} &=& x \cdot \left(\log (1+x^2) - \log (1-x^2)\right) \\
&=& x \cdot \left( \sum^{\infty}_{n=1} (-1)^{n+1}\frac{x^{2n}}n - \sum^{\infty}_{n=1} (-1)^{2n+1} \frac{x^{2n}}n \right) \\
&=& x \cdot \sum^{\infty}_{n=1} \left( (-1)^{n+1} - 1 \right) \frac{x^{2n}}n \\
&=& \sum^{\infty}_{n=1} \left( (-1)^{n+1} - 1 \right) \frac{x^{2n + 1}}n \\
\end{eqnarray*}

Is there any way to use Legrange Remainder method on a
function that is not a trig function, so that you cannot say
the nth derivative around zero is between -1 and 1?

\begin{theorem}[Lagrange]
$$  R_n(x) = \frac{f^{(n+1)}(c)}{(n+1)!} x^{n+1}.$$
for some $c \in (0,x)$.
\end{theorem}

\textbf{Problem:} We do not know what the point $c$ is.

\textbf{Solution:} We need to bound the $(n+1)$st derivative of $f$.

Examples of functions for which we can control the size of $n$th
derivaive: trig functions.  sums and products (by constants) of trig
functions.

Solutions to differential equations.  $f'''(x) = -f(x)$?

A priori information given to you: $|f^{(n)}(x)| < 17$

Secrets of the Exam:

Know Taylor series for sin and cosine and geometric series.

$$
R_k(x)
$$

Differentiate a Taylor series.

Conditional and absolute convergence.

Don't be afraid of large numbers.



\end{document}