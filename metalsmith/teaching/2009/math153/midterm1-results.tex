\documentclass[12pt]{article}
\usepackage{fullpage}
\usepackage{nopageno}
\usepackage{add-copyright}
\title{Midterm 1 Results}

\usepackage{amssymb}
\newcommand{\R}{\mathbb{R}}
\newcommand{\N}{\mathbb{N}}

\begin{document}

\large

I have finally finished grading the exams.  And I must admit: the exam
was \textbf{very long}---so long that my compatriots said it could not
be done!  But we proved them wrong!

You did very well.  To quantify:
\begin{center}
25\% of you scored above 220. \\
50\% of you scored above 210. \\
75\% of you scored above 200.
\end{center}
The median was 210.  The mean was 206.  The standard deviation was 21.
This means that many of you scored over 90\%, which is very good.

Here is the breakdown per problem:
\begin{center}
\begin{tabular}{ll}
\textbf{Problem} & \textbf{Average Score} \\
\hline
Problem 10	& 79\% \\
Problem 9	& 80\% \\
Problem 6	& 81\% \\
Problem 11	& 83\% \\ 
Problem 4	& 86\% \\
Problem 1	& 90\% \\
Problem 8	& 90\% \\
Problem 2	& 92\% \\
Problem 5	& 92\% \\
Problem 3	& 93\% \\
Problem 12	& 96\% \\
Problem 7	& 99\%
\end{tabular}
\end{center}

\subsection*{Problem 1}

Quanitifiers (those ``for all'' and ``there exists'') are very
important, and their order matters.  ``For every cat $C$, there is a
bowl $B$, so that $C$ eats from $B$'' is not the same as ``There is a
bowl of food $B$, such that for every cat $C$, $C$ eats from $B$'' In
the first world, every cat has a bowl of food.  In the other, there is
one bowl of food that all the cats must eat from.

In the same way, ``for every $\epsilon > 0$, there is a $K$\ldots'' is not
at all the same thing as ``there is a $K$, for every $\epsilon > 0$.''

\subsection*{Problem 2}

Your proofs should begin ``Let $\epsilon > 0$.''  This problem proved tricky because the limit was not equal to one.

\subsection*{Problem 3}

The definition should include for all $n \in \N$.

\subsection*{Problem 4}

I found it difficult to grade this problem, as very few people gave
convincing arguments for $c_n$ being unbounded above: I was fairly
generous.

In any case, be careful not to use circular logic (i.e., do not claim
that $c_n$ diverges because $c_n$ is unbounded, and then claim that
$c_n$ is unbounded because it diverges).

\subsection*{Problem 5}

You should have shown $d_n > d_{n+1}$ algebraically, or at least
claimed that $n^2$ is increasing, so $-n^2$ is decreasing, so $20 -
n^2$ is decreasing.

\subsection*{Problem 6}

Many people failed to notice $\sin(\pi n) = 0$ when $n \in \N$
(admittedly, it might feel like me tricking you, but this question
hits at the difference between limits of sequences and of functions).

Many people tried to use l'H\^opital's rule on this sequence---but
l'H\^opital is only for functions (unless you do something clever to
explain why the limit of a sequence is equal to the limit of some
similarly defined function).

\subsection*{Problem 7}

People did very well on this problem.  Be careful to mention that cosine is continuous.

\subsection*{Problem 8}

Some people found fancier things to squeeze between, but
$\displaystyle\frac{n^4 \pm 1}{n^4}$ works---you don't need to make
things harder than they already are.

\subsection*{Problem 9}

Many people forgot that additional assumption: that $g'(x) \neq 0$ for
$x$ near $a$.  The theorem is not true without this assumption.

\subsection*{Problem 10}

This problem turned out to be harder than I expected.  I think many
people ran out of time; other people seemed to have trouble applying
the chain rule (though, again, this problem was designed to test that:
applying the chain rule to $\sin^2 (2x)$ can be tricky).

\subsection*{Problem 11}

One person did a very nice thing by pointing out
$$
\lim_{x \to 0^{+}} x^{2x} = \left(\lim_{x \to 0^{+}} x^{x}\right)^2 = 1^2 = 1.
$$
Most people just used l'H\^opital to evaluate this limit, but many of you forgot to undo the logarithm at the end:
$$
\log \lim_{x \to 0^{+}} x^{2x} = 0 \mbox{ implies } \lim_{x \to 0^{+}} x^{2x} = e^0 = 1.
$$
Of course, the problem is designed to lead you into trap (as one of
you pointed out explicitly!), since the other limit equals $0$.

\subsection*{Problem 12}

People mostly did well on this problem, though I think some of you ran out of time.

\Large

\section*{Overall impressions.}

Your arguments in problems~4 and~5 should have been better.

Many of you can do a better job presenting your answers.  I'm very
sloppy human being, but mathematics (which comes from the Greek word
for ``discipline'' or ``disciple'') does demand a certain amount of
care, lest you make mistakes along the way.  The chain rule, for
instance, is a powerful tool, and we must be careful not to hurt
ourselves when wielding such tools.

Additionally, your mathematics should more closely resemble billiards
than, say, tennis: your goal is to announce your next move clearly,
and then perform it properly---not to surprise me with fast-moving
shots precisely aimed to the place I am not looking.

\end{document}
