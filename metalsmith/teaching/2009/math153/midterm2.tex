\documentclass[12pt]{article}

\usepackage{fullpage}
\usepackage{fancyhdr}
\usepackage{nopageno}
\usepackage{ifthen}
\usepackage{amsmath}
\usepackage{amssymb}
\usepackage{amsthm}
\usepackage{multicol}
\usepackage{version}
\usepackage{graphicx}
\usepackage{psfrag}
\usepackage{amssymb}
\newcommand{\R}{\mathbb{R}}
\newcommand{\N}{\mathbb{N}}

\usepackage[margin=0.75in]{geometry}
\usepackage{nopageno}
\newcommand{\HasSolutions}{true}
\excludeversion{titlepage}
\excludeversion{filler}

%\newcommand{\HasSolutions}{false}
%\includeversion{titlepage}
%\includeversion{filler}

\ifthenelse{\equal{\HasSolutions}{true}}{\includeversion{solutiontext}}{\excludeversion{solutiontext}}

\newenvironment{solution}{\subsubsection*{Solution.}}{}

\title{Midterm 2}
\author{Math 153 Section 57}
\date{November 17, 2008}

\newcounter{problem}
\setcounter{problem}{1}

\newcounter{totalpoints}
\setcounter{totalpoints}{0}

\makeatletter
\def\inputtoc{%
    \makeatletter
    \@input{\jobname.toc}%
    \if@filesw
      \expandafter\newwrite\csname tf@toc\endcsname
      \immediate\openout \csname tf@toc\endcsname \jobname.toc\relax
    \fi
    \@nobreakfalse
    }
\makeatother

\newenvironment{problem}[1][]%
{\ifthenelse{\equal{\HasSolutions}{true}}{}{\pagebreak}%
\addtocounter{totalpoints}{#1}%
%\ifthenelse{\equal{\HasSolutions}{true}}{}{\rfoot{\fbox{\Huge\hspace{2em}/#1}}}
\ifthenelse{\equal{\HasSolutions}{true}}{\vspace{3ex}\noindent\makebox[0in]{\vspace{5ex}\hspace{-1em}\rule{1ex}{3ex}\raisebox{2ex}{\rule{2ex}{1ex}}}\vspace{-6ex}}{}%
\subsection*{Problem \arabic{problem}.%
\addtocontents{toc}{Problem \arabic{problem} \protect& \hspace{2em} \protect& /#1 \protect\\ \protect\hline}%
\ifthenelse{\equal{\HasSolutions}{true}}{}{\hfill\fbox{\Huge\hspace{2em}/#1}}
}}%
{%
\addtocounter{problem}{1}%
%\ifthenelse{\equal{\HasSolutions}{true}}{}{\vfill\fbox{\Huge\hspace{2em}/#1}}%
}%

\renewcommand{\headrulewidth}{0in}

\cfoot{}

\newcommand{\answerbox}[1]{\ifthenelse{\equal{\HasSolutions}{true}}{\fbox{#1}}{#1}}

\newcommand{\TrueChoice}{\answerbox{True} False}
\newcommand{\FalseChoice}{True \answerbox{False}}

\begin{document}

\setlength{\headheight}{30pt}

%%%%%%%%%%%%%%%%%%%%%%%%%%%%%%%%%%%%%%%%%%%%%%%%%%%%%%%%%%%%%%%%
% Cover Page
%\markboth{}{\hfill\textbf{Name:} \underline{Solution Set} \\}

\begin{titlepage}
\maketitle

\begin{center}
$$
\int \frac{1}{\mbox{cabin}} \, d\mbox{cabin} = \log \mbox{cabin} + \mbox{sea}.
$$
\end{center}
\vfill
\vfill

\noindent
\hspace{1in}
\textbf{Name: } \hrulefill
\hspace{1in}

\vfill

\begin{multicols}{2}
\begin{enumerate}
\item Write your name above.
\item Do not look inside the exam until instructed to do so.
\item You have \textbf{50 minutes} for this exam.
\item Justify your answers for full credit.
\item Show your work for generous partial credit.
\item Calculators are forbidden.
\item Write your answers on the included pages, or request additional paper.
\item Answer all questions asked.
\item \textbf{The final question is for extra credit.}
\vfill
\end{enumerate}

\begin{center}
\large
\begin{tabular}{|lrl|}
\hline
\inputtoc
\\ \hline
\end{tabular}
\end{center}

\end{multicols}

\vfill
\vfill
%\markboth{}{\hfill\textbf{Name:} \hrulefill \\}
\end{titlepage}

\pagebreak
\null

%%%%%%%%%%%%%%%%%%%%%%%%%%%%%%%%%%%%%%%%%%%%%%%%%%%%%%%%%%%%%%%%
% Problems

%%%%%%%%%%%%%%%%%%%%%%%%%%%%%%%%%%%%%%%%%%%%%%%%%%%%%%%%%%%%%%%%
% \item Define $\displaystyle\sum_{n=1}^\infty a_n = L$.
% timeguess 2
\begin{problem}[15]
Define $\displaystyle\sum_{n=1}^{\infty} a_n = L$.
\end{problem}

\begin{solution}\begin{solutiontext}
    The series $\displaystyle\sum_{n=1}^{\infty} a_n$ converges to $L$
    if the sequence of partial sums
$$
s_k := \sum_{n=1}^k a_n
$$
converges to $L$, that is, $\lim_{k \to \infty} a_k = L$.

A common problem here was that people mixed up $k$'s and $n$'s, or
forgot to use a limit.

This definition is conceptually very important (a triumph of
mathematics!).  Series are \textit{not} piles of numbers to add up
simultaneously---they are lists to be added up \textbf{in order}---and
this affects convergence!
\end{solutiontext}\end{solution}

%%%%%%%%%%%%%%%%%%%%%%%%%%%%%%%%%%%%%%%%%%%%%%%%%%%%%%%%%%%%%%%%
% \item Determine whether a series converges by using...
% limit comparison test, or comparison test
% timeguess 5
\begin{problem}[25]
\begin{description}
\item[(a)] Does the series $\displaystyle\sum_{n=1}^\infty \displaystyle\frac{\sqrt{n}}{1 + \sqrt{n}}$ converge?
\item [(b)] Does the series $\displaystyle\sum_{n=3}^\infty \displaystyle\frac{n^2}{n^4 - 7}$ converge?
\end{description}
Justify your answers.
\end{problem}

\begin{solution}\begin{solutiontext}
\begin{description}
\item[(a)] No, by the $n$-th term test:
$$
\lim_{n \to \infty} \frac{\sqrt{n}}{1 + \sqrt{n}} = \lim_{n \to \infty} \frac{1}{1 + 1/\sqrt{n}} = 1 \neq 0
$$
so the series must diverge.  Some people tried to use the limit
comparison test (comparing with $b_n = 1$), but this is equivalent to
the $n$-th term test, and makes everything look unnecessarily
complicated.
\item[(b)] Yes, by the limit comparison test (this is the sum of a
  ratio of polynomials, so the limit comparison test is guaranteed to
  work).  Set $a_n = (n^2)/(n^4 - 7)$ and $b_n = 1/n^2$.  Then
$$
\lim_{n \to \infty} a_n/b_n = \lim_{n \to \infty} \frac{n^4}{n^4 - 7} = 1.
$$
Since this limit is nonzero, $\sum a_n$ converges if and only if $\sum
b_n$ converges.  But $\sum b_n$ converges by the $p$-series test, so
the given series $\sum a_n$ converges.

Many people fell into a trap that I placed in this problem: you cannot
use the more ``obvious'' comparison test because
$$
\frac{n^2}{n^4 - 7} \leq \frac{1}{n^2}
$$
is not true (try $n = 3$).  The limit comparison test comes in very
handy for this.
\end{description}
\end{solutiontext}\end{solution}

%%%%%%%%%%%%%%%%%%%%%%%%%%%%%%%%%%%%%%%%%%%%%%%%%%%%%%%%%%%%%%%%
% \item Provide statements of the preceeding tests.
% timeguess 3
\begin{problem}[15]
Provide a precise statement of the \textbf{ratio test} for determining whether the series $\displaystyle\sum_{n=1}^\infty a_n$ converges.
\end{problem}

\begin{solution}\begin{solutiontext}
Suppose $a_n > 0$ for all $n$, and that $\lim_{n\to\infty} a_{n+1}/a_n = L$.  Then,
\begin{itemize}
\item If $L > 1$, then $\sum a_n$ diverges.
\item If $L < 1$, then $\sum a_n$ converges.
\item If $L = 1$, then the test is inconclusive.
\end{itemize}
Some people neglected to write down $\lim_{n \to \infty}$.

\end{solutiontext}\end{solution}

%%%%%%%%%%%%%%%%%%%%%%%%%%%%%%%%%%%%%%%%%%%%%%%%%%%%%%%%%%%%%%%%
% \item Determine whether a series converges absolutely; determine whether a series conditionally.
% timeguess 6
\begin{problem}[20]
For which real numbers $x > 0$ does the series
$$
\sum_{n=1}^\infty \frac{(-1)^n}{n^x}
$$
converge absolutely?  For which $x > 0$ does it converge conditionally?  Justify your answer.
\end{problem}

\begin{solution}\begin{solutiontext}
    Andy Chiang pointed out that I should have had $n = 1$ instead of
    $n = 0$; I have corrected that here.  This problem is a basically
    $p$-series in disguise, so the convergence test will be easy---the
    hard part is the difference between absolute and conditional
    convergence.

    To check for absolute convergence, note that
$$
\sum_{n=1}^\infty \left| \frac{(-1)^n}{n^x} \right| = 
\sum_{n=1}^\infty \frac{1}{n^x} 
$$
But this is a $p$-series, so it converges exactly when $x > 1$.  Thus, the
given series \textbf{converges absolutely} provided $x > 1$.

Let $a_n = n^x$.  Then the given series is an alternating series
$\sum_{n=1}^\infty (-1)^n a_n$.  For any $x$, all of the $a_n$ are
positive, and $a_n$ is a decreasing sequence, and $\lim_{n \to \infty}
a_n = 0$, so by the alternating series test, the given series
converges for all $x > 0$.  Therefore, it converges conditionally
provided $0 < x \leq 1$.

(Some people said that it converges conditionally for all $x > 0$;
yes, the series \textbf{converges} for all $x > 0$, but to converge
\textbf{conditionally}, the series must converge while the series of
the absolute values diverges---this only happens when $0 < x \leq 1$).

\end{solutiontext}\end{solution}

%%%%%%%%%%%%%%%%%%%%%%%%%%%%%%%%%%%%%%%%%%%%%%%%%%%%%%%%%%%%%%%%
% \item Given a function, write down the first few terms of its Taylor series (around $0$ or around another point $a$)
% timeguess 8
\begin{problem}[25]
Define $f : \R \to \R$ by
$$
f(x) = \cos \left( \sin x \right)
$$
Find the first four terms (i.e., up to and including $x^3$) of a Taylor series expansion for $f$ around $x = 0$.
\end{problem}

\begin{solution}\begin{solutiontext}

    You can do this problem by differentiating; since $\cos \sin x$ is
    an \textit{even} function, the Taylor series will only involve
    even degree terms, so you do not need to compute the coefficient
    on $x^1$ and $x^3$---they are both automatically zero.  This means
    you only need to compute $f^{(2)}(0)$ which is not so bad.

    Nevertheless, you can also solve this problem by substituting the
    series for $\sin x$ into the series for $\cos x$, as follows:
\begin{eqnarray*}
\cos \left( \sin x \right) &=&
\sum_{n=0}^\infty \frac{(-1)^n}{(2n)!} \left(
1 - \frac{x^3}{3!} + \frac{x^5}{5!} - \cdots \right)^{2n}\\
&=& 1 - \frac{\left( x - \frac{x^3}{3!} + \frac{x^5}{5!} - \cdots \right)^2}{2} + \frac{\left( x - \cdots \right)^4}{4!} + \cdots \\
&=& 1 - \frac{x^2}{2} + \mbox{terms of degree four or more}
\end{eqnarray*}

\end{solutiontext}\end{solution}

%%%%%%%%%%%%%%%%%%%%%%%%%%%%%%%%%%%%%%%%%%%%%%%%%%%%%%%%%%%%%%%%
% \item Given a function, write down its Taylor series.
% timeguess 5
\begin{problem}[30]
Consider the function
$$
f(x) = \frac{e^x - e^{-x}}{2}.
$$
Let $\displaystyle\sum_{n=0}^\infty a_n x^n$ be a power series for $f(x)$.
Compute both $a_n$ and $f^{(42189)}(0)$.
\end{problem}

\begin{solution}\begin{solutiontext}
    This function is ``hyperbolic sine'' so there is some reason to
    care about it; you'll notice in this problem that its Taylor
    series looks just like the Taylor series for sine, except that
    there is no $(-1)^n$ term.  Additionally, since $\sinh$ is an odd
    function, only odd degree terms will appear in its Taylor series.

The Taylor series for $e^x$ is $\sum_{n=0}^\infty \frac{x^n}{n!}$ so
\begin{eqnarray*}
\frac{e^x - e^{-x}}{2}
&=& \frac{1}{2} \left( \sum_{n=0}^\infty \frac{x^n}{n!} - \sum_{n=0}^\infty \frac{(-x)^n}{n!} \right) \\
&=& \frac{1}{2} \left( \sum_{n=0}^\infty \frac{1 - (-1)^n}{n!} x^n \right)
\end{eqnarray*}
Therefore the coefficient on $x^n$ is
$$
a_n = \frac{1 - (-1)^n}{n! \cdot 2}.
$$
Note that this vainshes when $n$ is even, and is equal to $1/n!$ when
$n$ is odd.

Since $f^{(n)}(0) = n!\, a_n$, we conclude
$$
f^{(42189)}(0) = 42189! \cdot \frac{1 - (-1)^{42189}}{42189! \cdot 2} = \left( 1 - (-1)^{42189} \right) / 2 = 2 / 2 = 1.
$$

Many of you neglected to compute $a_n$---you lost
points for this; I asked for a computation of $a_n$ to force you to write down the answer in a particular way (I feared that some would suggestively write down the first few terms series, some would not simplify it sufficiently, etc.---to be fair to those who might do more work, I asked for the most specific thing I could ask for).  Additionally, many people wrote down
$$
f(x) = \sum_{n=0}^\infty \frac{x^{2n+1}}{(2n+1)!}
$$
which is true, but this expresion makes it very complicated to see
what the coefficient on the $x^n$ term should be.  One way to express
it is like this:
\begin{eqnarray*}
a_{2n+1} &=& \frac{1}{(2n+1)!} \\
a_{2n} &=& 0 
\end{eqnarray*}
Or you could say,
$$
a_n = \begin{cases}
1/n! & \mbox{if $n$ is odd}, \\
0 & \mbox{if $n$ is even.}
\end{cases}
$$
\end{solutiontext}\end{solution}

%%%%%%%%%%%%%%%%%%%%%%%%%%%%%%%%%%%%%%%%%%%%%%%%%%%%%%%%%%%%%%%%
% \item State Taylor's theorem and Lagrange's theorem.
% timeguess 6
\begin{problem}[15]
Suppose $f : \R \to \R$ is a smooth function.  Define
$$
g(x) = f(x) - f(0) - f'(0) x - \frac{f''(0) \, x^2}{2}
$$
Recall that \textbf{Taylor's theorem} provides an integral equal to
$g(x)$.  What is that integral?  (Hint: if you have forgetten it or
want to check your answer, you might try integration by parts).
\end{problem}

\begin{solution}\begin{solutiontext}


By Taylor's theorem,
$$
f(x) = f(0) + f'(0) \, x + \frac{f''(0) \, x^2}{2} + R_2(x).
$$
Therefore,
$$
g(x) = f(x) - f(0) - f'(0) x - \frac{f''(0) \, x^2}{2} = R_2(x).
$$
But,
$$
g(x) = R_2(x) = \frac{1}{2} \int_0^x f^{(3)}(t) \, (x-t)^2 \, dt
$$
This problem was designed to test your knowledge of Taylor's theorem
without exactly asking you to just give a statement---you got credit
if you wrote down the $n = 2$ case.

\end{solutiontext}\end{solution}

%%%%%%%%%%%%%%%%%%%%%%%%%%%%%%%%%%%%%%%%%%%%%%%%%%%%%%%%%%%%%%%%
% \item Approximate a Taylor series and estimate the error in your approximation.
% timeguess 5
\begin{problem}[20]
  Let $p : \R \to \R$ be a smooth function, such that
\begin{itemize}
\item $p(0) = 0$ (in other words, my initial position is at the origin),
\item $p'(0) = 2000$ (in other words, my initial velocity is $2000\,\mbox{m}/\mbox{s}$), and
\item $|p''(t)| < 500$ for all $t \in \R$ (in other words,  I will avoid accelerating more than $500\,\mbox{m}/\mbox{s}^2$)
\end{itemize}
Use \textbf{Lagrange's theorem} to bound $p(2)$, my position after two seconds.

\end{problem}

\begin{solution}\begin{solutiontext}
By Lagrange's theorem
$$
p(x) = p(0) + p'(0) \, x + R_1(x)
$$
and $R_1(x) = f''(c) \, x^2/2!$ for some $c \in (0,x)$.  

In our case, $x = 2$, so
$$
p(2) = p(0) + p'(0) \, 2 + R_1(2) = 2000 \cdot 2 + R_1(2)
$$
So $p(2) - 4000 = R_1(2)$.

Since $|p''(t)| < 500$, we know
$$
|R_1(2)| = |f''(c) \, 2^2/2| \leq 500 \cdot 4 / 2 = 1000
$$
Consequently, $|p(2) - 4000| \leq 1000$, and therefore,
$$
3000 \leq p(2) \leq 5000.
$$
Take a moment to reflect on the reasonableness of this: if you are
travelling at a particular speed, and you refuse to accelerate or
decelerate too much, then you have to cover a certain distance!  This
is a physical consequence of Lagrange's theorem.  In problems like
this, physical intuition can reassure you that you've written down the
correct answer.

\end{solutiontext}\end{solution}

%%%%%%%%%%%%%%%%%%%%%%%%%%%%%%%%%%%%%%%%%%%%%%%%%%%%%%%%%%%%%%%%
% \item Determine the interval on which a power series converges
% timeguess 
\begin{problem}[20]
  For which real numbers $x$ does the series
  $\displaystyle\sum_{n=1}^\infty \displaystyle\frac{x^n}{2^n \, n^2}$
  converge?  Remember to show your work.

\end{problem}

\begin{solution}\begin{solutiontext}
    We use the ratio test to check for absolute convergence (this
    suffices for power series, save possibly at the endpoints)
$$
\lim_{n \to \infty} \left|\frac{a_{n+1}}{a_n}\right| = 
\lim_{n \to \infty} \left|\frac{x^{n+1} / (2^{n+1} (n+1)^2)}{x^n / (2^n n^2)}\right| = \lim_{n \to \infty} \left|\frac{x \, n^2}{2 \,(n+1)^2} \right|
$$
But $\lim_{n \to \infty} (n/n+1)^2 = 1$, so the above limit is equal to $|x|/2$  By the ratio test, the series converges absolutely provided $|x|/2 < 1$, i.e., provided $-2 < x < 2$.

We need to check endpoints.  At $x = 2$, the series becomes
$$
\sum_{n=1}^\infty \frac{2^n}{2^n \, n^2} = \sum_{n=1}^\infty \frac{1}{n^2}
$$
which is a convergent $p$-series.

At $x = -2$, the series also converges---even absolutely, since
$$
\sum_{n=1}^\infty \left| \frac{(-2)^n}{2^n \, n^2} \right| = \sum_{n=1}^\infty \frac{1}{n^2}
$$
which is a convergent $p$-series again.

Therefore, the series converges exactly when $x \in [-2,2]$.

\end{solutiontext}\end{solution}

%%%%%%%%%%%%%%%%%%%%%%%%%%%%%%%%%%%%%%%%%%%%%%%%%%%%%%%%%%%%%%%%
% \item Determine the interval on which a power series converges.
% timeguess 7
\begin{problem}[20]
  For which real numbers $x$ does $\displaystyle\sum_{n=0}^\infty
  \displaystyle\frac{n! \, x^n}{n!!}$ converge?  Note: $n!!$ means the
  factorial of the factorial of $n$.  Justify your answer.
\end{problem}

\begin{solution}\begin{solutiontext}
    Yes, it is a power series, so you might be tempted to apply the
    ratio test; but we've had problems about the ratio test
    already---and more importantly, the ratio test is helpful
    precisely when taking a ratio will cancel some terms---but here,
    taking the ratio of $n!!$ and $(n+1)!!$ doesn't seem helpful.

    What we do know is that $n!!$ grows ridiculuosly quickly, which
    should suggest that a comparison test will be helpful.  So let's
    apply the comparison test.  First, observe that, provided $n \geq
    2$
$$
n! = 1 \cdot 2 \cdots (n-1) \cdot (n) > (n-1) \cdot (n) > (n-1)^2 \geq n^2/4
$$
As a result, $n!! > (n!)^2 / 4$.

Therefore,
$$
\frac{n! \, x^n}{n!!} \leq \frac{n! \, x^n}{(n!)^2 / 4} = \frac{1}{4} \cdot \frac{x^n}{n!}.
$$
But
$$
\sum_{n=1}^\infty \frac{1}{4} \frac{x^n}{n!} = \frac{e^x}{4}
$$
converges for all $x \in \R$, so by the comparison test, the given series also converges for all $x \in \R$.

This problem was designed to test your use of the comparison test
(since the second problem was more naturally solved using the limit
comparison test), but there are surely other ways to solve it.  Many
of you had the right intuition (and you got points for that), saying
that the $n!!$ will grow so much more quickly, etc.  This is great!  I
hope that when you see series like this, you'll be able to guess the
truth right away; but to justify your intuition of ``grows very fast''
you should use a comparison test.
\end{solutiontext}\end{solution}

%%%%%%%%%%%%%%%%%%%%%%%%%%%%%%%%%%%%%%%%%%%%%%%%%%%%%%%%%%%%%%%%
% \item Differentiate and integrate a power series term-by-term.
% timeguess 3
\begin{problem}[20]
Use the power series for $\displaystyle\frac{1}{1-x}$ to find a power series $\displaystyle\sum a_n x^n$ for $f(x) = \displaystyle\frac{1}{(1-x)^3}$.

\noindent
For which $x \in \R$ does the power series converge to $f(x)$?
\end{problem}

\begin{solution}\begin{solutiontext}

Note that
$$
\frac{d^2}{dx^2} \frac{1}{1-x} = \frac{2}{(1-x)^3}
$$
Additionally,
$$
\frac{d^2}{dx^2} \sum_{n=0}^\infty x^n = 
\sum_{n=0}^\infty \frac{d^2}{dx^2} x^n = 
\sum_{n=2}^\infty (n)(n-1) x^{n-2}.
$$
Putting these facts together gives
$$
\frac{1}{(1-x)^3} = \sum_{n=2}^\infty \frac{n \, (n-1)}{2} x^{n-2} = \sum_{n=2}^\infty \binom{n}{2} x^{n-2}
$$
Since $\sum_{n=0}^\infty x^n$ converges to $1/(1-x)$ for all $x \in
(-1,1)$, the same is true of the second derivatives (by the thereom on
term-by-term differentiation).

There were many popular mistakes: the most frequent was to argue that
$$
\frac{1}{1-x^3} = \sum_{n=0}^\infty (x^3)^n = \sum_{n=0}^\infty x^{3n}
$$
which is true.  The trouble is that $1/(1-x)^3$ is not $1/(1-x^3)$.

    Many people included an argument that the series they found
    converged; regardless of whether or not the series you found
    converges, that does not address the question of whether the
    series converges to $f(x)$---for that, you need to either estimate
    the remainder, or apply a theorem about term-by-term
    differentiation.

\end{solutiontext}\end{solution}

%%%%%%%%%%%%%%%%%%%%%%%%%%%%%%%%%%%%%%%%%%%%%%%%%%%%%%%%%%%%%%%%
% bonus true/false questions
\begin{problem}[0]
  The following questions are for extra credit.  \textbf{Circle your
    answer.}

\begin{description}
%\vfill
\item[\FalseChoice] If $\displaystyle\sum_{n=1}^\infty a_n$ converges, then $\displaystyle\sum_{n=1}^\infty |a_n|$ converges.
Counterexample: $a_n = (-1)^n/n$.
%\vfill
\item[\TrueChoice] If $\displaystyle\sum_{n=1}^\infty a_n$ diverges, then $\displaystyle\sum_{n=1}^\infty |a_n|$ diverges.
This is the contrapositive of the theorem that says absolutely convergent series also converge conditionally.
%\vfill
\item[\TrueChoice] If $\displaystyle\sum_{n=1}^\infty |a_n|$ converges, then $\displaystyle\sum_{n=1}^\infty a_n$ converges.
This is the theorem that says absolutely convergent series also converge conditionally.
%\vfill
\item[\FalseChoice] If $\displaystyle\sum_{n=1}^\infty |a_n|$ diverges, then $\displaystyle\sum_{n=1}^\infty a_n$ diverges.
Counterexample: $a_n = (-1)^n/n$.
%\vfill
\item[\FalseChoice] If $\displaystyle\sum_{n=1}^\infty a_n$ diverges and $\displaystyle\sum_{n=1}^\infty b_n$ diverges, then $\displaystyle\sum_{n=1}^\infty (2 \cdot a_n + 3 \cdot b_n)$ diverges.
Counterexample: $a_n = 3n$ and $b_n =  - 2n$.
%\vfill
\item[\TrueChoice] If $\displaystyle\sum_{n=1}^\infty a_n$ converges and $\displaystyle\sum_{n=1}^\infty b_n$ converges, then $\displaystyle\sum_{n=1}^\infty (2 \cdot a_n + 3 \cdot b_n)$ converges.
Sums of convergent series converge.
%\vfill
\item[\FalseChoice] If $\displaystyle\sum_{n=1}^\infty a_n$ diverges and $\displaystyle\sum_{n=1}^\infty b_n$ diverges, then $\displaystyle\sum_{n=1}^\infty (a_n \cdot b_n)$ diverges.
Counterexample: $a_n = b_n = 1/n$.
%\vfill
\item[\FalseChoice] If $\displaystyle\sum_{n=1}^\infty a_n$ converges and $\displaystyle\sum_{n=1}^\infty b_n$ converges, then $\displaystyle\sum_{n=1}^\infty (a_n \cdot b_n)$ converges.
Counterexample: $a_n = b_n = (-1)^n / \sqrt{n}$.
%\vfill
\item[\TrueChoice] If $\displaystyle\sum_{n=1}^\infty a_n$ converges, then $\displaystyle\sum_{n=100}^\infty a_n$ converges.
Convergence only depends on the tail of the series.
%\vfill
\item[\TrueChoice] If $\displaystyle\sum_{n=1}^\infty a_n$ diverges, then $\displaystyle\sum_{n=100}^\infty a_n$ diverges.
Divergence only depends on the tail of the series.
%\vfill
\item[\TrueChoice] If $\displaystyle\sum_{n=1}^\infty a_n$ converges, then $\displaystyle\sum_{n=2}^\infty \left(a_n - a_{n-1}\right)$ converges.
A shifted convergent series still converges, so $\displaystyle\sum_{n=2}^\infty \left(a_n - a_{n-1}\right)$ is the difference of two convergent series.
%\vfill
\null
\end{description}

\end{problem}

\ifthenelse{\equal{\HasSolutions}{true}}{}{%
\pagebreak%
\section*{Extra Paper}%
\pagebreak%
\null%
\pagebreak%
\null%
\pagebreak%
\null%
\pagebreak%
\null%
}

%%%%%%%%%%%%%%%%%%%%%%%%%%%%%%%%%%%%%%%%%%%%%%%%%%%%%%%%%%%%%%%%
\addtocontents{toc}{\textbf{Total} \protect& \protect& /\arabic{totalpoints}}

\end{document}
