\documentclass[12pt]{article}

\usepackage{fullpage}
\usepackage{nopageno}
\usepackage{amsthm}
\usepackage{amsmath}
\usepackage{amssymb}
\newcommand{\R}{\mathbb{R}}
\usepackage{add-copyright}
\usepackage{geometry}
\geometry{top=0.25in,bottom=0.50in}

\title{Homework and Quiz 5}
\date{Due Wednesday, July 1, 2009}

\begin{document}
\maketitle

\subsection*{Ungraded homework}

Remember, the midterm will be on Monday, July 6, 2009.  Make sure you
are well-prepared to perform all the tasks on the review sheet.

\vfill

\subsection*{Graded Quiz}

\begin{description}
\item[(a)] Suppose $P_1$ and $P_2$ are both planes through the origin,
  with normal vectors $n_1$ and $n_2$, respectively.  Find a point
  besides the origin which is contained in both $P_1$ and $P_2$.
  \textit{Hint:} find the point in terms of $n_1$ and $n_2$.
\vfill
\item[(b)] Let $v = (1,2,3)$ and $w = (4,5,6)$.  Can you write
  $(7,8,9)$ as
$$
\alpha \cdot v + \beta \cdot w,
$$
and if you can, do so.  Can you do it in more than one way?
\vfill
\item[(c)] Let $u,v,w$ be vectors in $\R^3$, and $t \in \R$.  Find the
  first and second derivative of the vector-valued function
$$
f(t) = u \times (v + t w).
$$
\item[(d)] Consider the vector-valued functions
\vfill
\begin{eqnarray*}
f(t) &=& (t,t^2,t^3) \mbox{ and } \\
g(t) &=& (t,t,t). \\
\end{eqnarray*}
The graphs of these functions intersect at $(1,1,1)$; find the cosine
of the angle at which they intersect.
\vfill
\item[(e)] Consider three planes $P_1, P_2, P_3$ which pairwise intersect.  Name the three lines of intersection as follows:
$$
L_{1,2} = P_1 \cap P_2, \hspace{1em}
L_{2,3} = P_2 \cap P_3, \hspace{1em}
L_{1,3} = P_1 \cap P_3.
$$
Can you choose the three planes so that $L_{1,2}$ and $L_{2,3}$ are skew?  Can you choose the three planes so that $L_{1,2}$, $L_{2,3}$, and $L_{1,3}$ all intersect in a common point?

\end{description}

\end{document}
