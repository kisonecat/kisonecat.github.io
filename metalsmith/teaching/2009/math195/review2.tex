\documentclass[12pt]{article}
\usepackage{fullpage}
\usepackage{nopageno}
\usepackage{amsmath}
\usepackage{amssymb}
\usepackage{amsthm}
\usepackage{multicol} 
\newcommand{\R}{\mathbb{R}}

\usepackage{add-copyright}

\newtheorem*{example}{Example}
\newtheorem*{thm}{Theorem}

\title{Midterm 2: Frequently Asked Questions}
\author{Math 195 Section 91}
\date{Wednesday July 15, 2009}

\begin{document}
\maketitle

\section*{What will the exam be like?}

The oral exam will last between thirty and \textbf{sixty minutes}, and
again, will be worth 225 points.  If you have not yet emailed me your preferences, please do so today.

During the exam, you will stand at the chalkboard; I will invite you
to demonstrate various pieces of mathematics (e.g., ``Write down a
polynomial; find the partial derivative of it with respect to $x$'').
As you do so, I will ask additional questions.  Ideally, we will
complete around six problems together.

\section*{Can I use my calculator?}

Calculators are \textbf{forbidden} but feel free to ask me questions
during the oral exam if you get stuck.

\section*{What might I have to do on this exam?}

I may be asking you to:
\begin{multicols}{2}
\begin{itemize}
\item Differentiate a vector-valued function
\item Graph a vector-valued function
\item Discuss functions of several variables
\item Compute a limit of a function of several variables
\item Convert a limit in cartesian coordinates to polar coordinates
\item Define continuity for functions of several variables
\item Give an example of a limit that does not exist
\item Compute partial derivatives
\item Compute higher partial derivatives
\item Give an example in which mixed partials commute
\item Describe what a partial derivative is measuring
\item Write down a linear approximation to a function
\item Find the tangent plane to a function at a point
\item \textbf{Compute partial derivatives with the chain rule}
\item Compute the gradient of a function
\item Interpret the gradient as ``steepest ascent''
\item Compute directional derivatives
\item Define global maximum and global minimum
\item Define local maximum and local minimum
\item Find critical points
\item Find maximum and minimum values over a given region
\item Apply the second derivative test
\item Optimize a function given a constraint with Lagrange multipliers
\end{itemize}
\end{multicols}


\end{document}