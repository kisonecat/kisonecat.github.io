\documentclass[12pt]{handout}
%\usepackage{add-copyright}

\title{Homework 6}
\course{Math 2568}
\date{Attempt before Monday, September 30, 2013}
\author{Jim Fowler}

\usepackage[T1]{fontenc}
\usepackage{lmodern}
\usepackage{hyperref}

\newcommand{\peem}{\textsc{p.m.}}
\newcommand{\ayem}{\textsc{a.m.}}

\usepackage{nopageno}
\usepackage{multicol}
\geometry{margin=1cm}
%\geometry{landscape,margin=0.25in,bottom=0.25in,left=0.5in,right=0.5in}
\usepackage{tabularx}
\usepackage{rotating}

\setlength{\parindent}{0in}
\setlength{\parskip}{0in}
\usepackage{calc}

\begin{document}
\maketitle





It is a good idea to do more than just the assigned problems from the
book: try your hand at inventing your own problems!  See what you can do with the
mathematical machinery.





\subsection*{The three boxed problems have solutions in the back}
Even if you don't do all the suggested problems (though you should!), the three boxed problems really should be done since you can check your answers quickly in the back of the textbook.

\section*{Suggested Problems for Practice}

From \textsection 2.2 (starting on page 128),
do problems 18, 20, 28, 32, 34.
\vspace{1ex}

From \textsection 2.3 (starting on page 135),
do problems 6, 10, 26, 32, 36, 42.
\vspace{1ex}

From \textsection 3.1 (starting on page 164),
do problems 6, 8, \fbox{21}.
\vspace{1ex}

From \textsection 3.2 (starting on page 167),
do problems \fbox{3}, \fbox{9}, 22, 24.
\vspace{1ex}


\end{document}
