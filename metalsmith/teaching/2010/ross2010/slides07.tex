\documentclass[14pt]{beamer}

\setbeamertemplate{navigation symbols}{}

\usepackage{pgf}
\usepackage{amsmath}
\usepackage{amsthm}
%\newtheorem*{problem}{Problem}
\usepackage{amssymb}
\newcommand{\N}{\mathbb{N}}
\newcommand{\R}{\mathbb{R}}

\usepackage{stmaryrd}
\newcommand{\boundary}{\partial}
\newcommand{\collapses}{\searrow}
\newcommand{\expands}{\nearrow}
\newcommand{\she}{\ssearrow\nnearrow}
\newcommand{\join}{\ast}
\newcommand{\subdivided}{\triangleleft}
\newcommand{\fullsubcomplex}{\Subset}
\newcommand{\BallThm}[1]{\mbox{$\textcolor{green!50!black}{\mbox{\textbf{BallThm}}_{#1}}$}}
\newcommand{\SphereThm}[1]{\mbox{$\textcolor{green!50!black}{\mbox{\textbf{SphereThm}}_{#1}}$}}

\DeclareMathOperator{\st}{st}
\DeclareMathOperator{\vertices}{vert}
\DeclareMathOperator{\lk}{lk}
\DeclareMathOperator{\cl}{cl}
\DeclareMathOperator{\cone}{cone}
\DeclareMathOperator{\interior}{int}

\title{Topology of  \\ Piecewise-Linear Manifolds}
\author{Jim Fowler}
\date{Lecture 7 \\ Summer 2010}

\newcommand{\setbackgroundpicture}[1]{%
\usebackgroundtemplate{
\begin{pgfpicture}{0in}{0in}{\paperwidth}{\paperheight}
\pgfputat{\pgfxy(0,0)}{\includegraphics[width=\paperwidth]{#1}}
\color{white}
\pgfsetfillopacity{0.8}
\pgfrect[fill]{\pgfxy(0,0)}{\pgfpoint{\paperwidth}{\paperheight}}
\end{pgfpicture}
}
}
\newcommand{\clearbackgroundpicture}{\usebackgroundtemplate{}}

\begin{document}

\begin{frame}
\maketitle
\end{frame}

\begin{frame}
  \frametitle{Regular neighborhoods}
  
  \large
  Suppose $X \subset M$.
  \pause
  The polyhedron $X$ \\
  might have many neighborhoods in $M$. \\
  \vspace{1ex}\pause
  But there is an ``essentially unique'' \\
  regular neighborhood.
\end{frame}

%\setbackgroundpicture{zeeman.pdf}
\begin{frame}
 \vfill
  \begin{center}
   \Huge E. C. Zeeman \\
   \pause\vfill
   {\normalsize \textbf{Goal:} \textit{``Unknotting combinatorial balls''}
     \\\pause
     a paper in the \textbf{Annals}
     }
  \end{center}
  \vfill
\end{frame}
%\clearbackgroundpicture

\setbackgroundpicture{Nf_knots.png}
\begin{frame}
  \frametitle{Main result}

\Large
  \begin{theorem}
    Any embedding $S^p \subset S^q$ \\is unknotted, if $q - p \geq 3$.
  \end{theorem}
\pause
  \begin{corollary}
        Any embedding $S^1 \subset S^{4}$ \\is unknotted.
  \end{corollary}

\end{frame}
\clearbackgroundpicture

\begin{frame}
  \frametitle{Some background}

  Can $S^1$ be knotted in $S^3$?

  \pause\vfill
  What does \textit{knotted} even mean?
\end{frame}

\begin{frame}
  \frametitle{One point compactification}

  \vfill
  $$\mathbb{R}^1 \cong S^1 - \mbox{point}$$
  \vfill\pause
  $$\mathbb{R}^2 \cong S^2 - \mbox{point}$$
  \vfill\pause
  $$\mathbb{R}^3 \cong S^3 - \mbox{point}$$
  \vfill\pause

  So we can draw $S^1 \subset S^3$ as if the circle were in
  $\mathbb{R}^3$.
\end{frame}



\begin{frame}
 \begin{center}
 \includegraphics[height=\textheight]{Blue_Trefoil_Knot.png}
 \end{center}
\end{frame}

\begin{frame}
\begin{center}
\includegraphics[height=\textheight]{TrefoilKnot_01.pdf}
\end{center}
\end{frame}

\begin{frame}
\begin{center}
\includegraphics[width=\textwidth]{Knot_table.pdf}
\end{center}
\end{frame}

\begin{frame}
 \frametitle{Reidemeister Move---Type I}
 \begin{center}
   \includegraphics[height=0.8\textheight]{reidemeister-1.pdf}
 \end{center}
\end{frame}

\begin{frame}
 \frametitle{Reidemeister Move---Type II}
 \begin{center}
   \includegraphics[height=0.8\textheight]{reidemeister-2.pdf}
 \end{center}
\end{frame}

\begin{frame}
 \frametitle{Reidemeister Move---Type III}
 \begin{center}
   \includegraphics[width=\textwidth]{reidemeister-3.pdf}
 \end{center}
\end{frame}

\begin{frame}
 \frametitle{Reidemeister Moves}
 \begin{center}
   \includegraphics[width=\textwidth]{reidemeister.pdf}
 \end{center}
\end{frame}

\begin{frame}
  \frametitle{Knots}

  \large
  \begin{theorem}
    Two knots are the same \\
    if a diagram for the one \\
    can be transformed into the other \\
    via Reidemeister moves.
  \end{theorem}

\end{frame}

\setbackgroundpicture{reidemeister.pdf}
\begin{frame}
  \frametitle{High dimensional knots?}

  \Large
  \begin{center}
    Are there ``Reidemeister moves'' for $S^2$'s in $S^4$?
  \end{center}

\end{frame}
\clearbackgroundpicture

% \begin{frame}
%    \includegraphics[height=0.95\textheight]{roseman.pdf}\raisebox{0.10\textheight}{{\parbox{0.35\textwidth}{\noindent
%          Roseman
%          moves \\\scriptsize
%          From \textit{A Combinatorial Description of Knotted Surfaces
%    and Their Isotopies} by Carter, Rieger, and Saito.}}}

% \end{frame}

\setbackgroundpicture{two-giraffes-871277133292IxUu.jpg}
\begin{frame}
  \frametitle{Pairs}

  \begin{definition}
    ``$(p,q)$-sphere pair'' means $S^q \subset S^p$. \\
    We sometimes write $(S^p,S^q)$.

    \vspace{1ex}
    ``$(p,q)$-ball pair'' means $B^q \subset B^p$, \\
    with $B^q$ properly embedded in $B^p$, \\
    meaning $\partial B^q \subset \partial B^p$ 
    and $\interior B^q \subset \interior B^p$. \\
    We sometimes write $(B^p,B^q)$.
  \end{definition}

  \vfill
  \pause
  ``Pair'' means either a ball-pair or sphere-pair.

  \vfill
  \pause
   $\partial (B^p,B^q) = (S^{p-1},S^{q-1})$.

\end{frame}
\clearbackgroundpicture

\setbackgroundpicture{wedding.jpg}
\begin{frame}
  \frametitle{Joins of pairs}

  $(S^p,S^q) \join S^k$ is a \pause sphere pair.\pause

  \vfill

  $(S^p,S^q) \join B^k$ is a \pause ball pair.\pause

  \vfill

  $(B^p,B^q) \join S^k$ is a \pause ball pair.\pause

  \vfill

  $(B^p,B^q) \join B^k$ is a \pause ball pair.\pause

  \vfill
  \pause

  We'll call the join of a pair and a point a \textbf{cone pair}.

\end{frame}
\clearbackgroundpicture

\setbackgroundpicture{ALVIN_submersible.jpg}
\begin{frame}
  \frametitle{Subpairs}

  $X = (X^p,X^q)$ and $Y = (Y^r,Y^s)$ are pairs \\
  we say $Y$ is a subpair of $X$ \\
  \quad (written $Y \subset X$ or $X \supset Y$) \\
  if $Y^r \subset X^p$ and $T^s = X^q \cap Y^r$.

 \vfill\pause

  If $P = (S^p,S^q) \supset Q = (B^p,B^q)$, then
  $$ P - \interior Q = (S^p - \interior B^p, S^q - \interior B^q)$$
  is a ball pair (via regular neighborhood machinery).

\end{frame}

\clearbackgroundpicture

\setbackgroundpicture{Mona_Lisa_headcrop.jpg}
\begin{frame}
  \frametitle{Faces of pairs}

  If $Q' = (B^{p-1},B^{q-1})$ is\\
  contained in the boundary of $Q = (B^p,B^q)$,\\
  we call $Q'$ a \textbf{face} of $Q$.

 \vfill\pause

  \begin{theorem}
   If ball pairs intersect in their common boundary, \\
   their union is a sphere pair.
 \end{theorem}

 \vfill\pause

  \begin{theorem}
   If ball pairs intersect in a face, \\
   their union is a ball pair.
 \end{theorem}

\end{frame}
\clearbackgroundpicture

\setbackgroundpicture{two-giraffes-871277133292IxUu.jpg}
\begin{frame}
  \frametitle{Standard pairs}

  $\Gamma^{p,q} = (S^{p-q} \Delta^q, \Delta^q)$ is the standard
  $(p,q)$-ball pair. \\
  $\partial \Gamma^{p+1,q+1}$ is the standard $(p,q)$-sphere pair.

  \vfill\pause
  A pair is \textbf{unknotted} \\
  if it is homeomorphic to a standard pair.

\end{frame}
\clearbackgroundpicture

\setbackgroundpicture{marble-ball.jpg}
\begin{frame}

  \begin{theorem}[\BallThm{p,q}]
    If $p - q \geq 3$, then any $(p,q)$-ball pair \\
    is unknotted.
  \end{theorem}

  \vfill

  \begin{theorem}[\SphereThm{p,q}]
    If $p - q \geq 3$, then any $(p,q)$-sphere pair \\
    is unknotted.
  \end{theorem}

  \vfill\pause
  \begin{proof}
    By induction.
  \end{proof}
\end{frame}
\clearbackgroundpicture

\setbackgroundpicture{Fermilab_-_400_MeV_Lambertson_Magnet.jpg}
\begin{frame}<1-3>[label=overview]
  \frametitle{Overview of the induction}

  \vfill

  \uncover<4->{\checkmark\hspace{0.25em} }Prove $\BallThm{p,0}$ by hand.\\
  \vspace{1ex}
  \uncover<5->{\checkmark\hspace{0.25em} }\uncover<2->{$\BallThm{p,q}$
    implies $\SphereThm{p,q}$.}\\
  \vspace{1ex}
  \uncover<6->{\checkmark\hspace{0.25em} }\uncover<3->{$\BallThm{p-1,q-1}$ and $\SphereThm{p-1,q-1}$ \\
    \quad\quad\quad together imply $\BallThm{p,q}$.}

  \vfill

\end{frame}
\clearbackgroundpicture

\begin{frame}
  \frametitle{Base case}

  \begin{lemma}
    $\BallThm{p,0}$ is true.
  \end{lemma}
  \pause
  \begin{proof}
    A ball $B^p$ with a marked point $B^0$ \\
    is homeomorphic \\
    to any other such \\
    (via regular neighborhood theory).
  \end{proof}

\end{frame}

\setbackgroundpicture{Fermilab_-_400_MeV_Lambertson_Magnet.jpg}
\againframe<3-4>{overview}
\clearbackgroundpicture

\begin{frame}
  \begin{lemma}
    $\BallThm{p,q} \Rightarrow \SphereThm{p,q}$
  \end{lemma}
  \pause
  \begin{proof}
    $P = (S^p,S^q)$.  Choose vertex $x \in S^q$. \\\pause
    $P = Q \cup \{x\} \join \partial Q$, \\\pause
    where $Q = (S^p - \st(x,S^p), S^q - \st(x,S^q))$.\\
    \vspace{1ex}\pause
    Extend $Q \cong \Gamma^{p,q}$ to \\
    a homeomorphism $P \cong \partial \Gamma^{p+1,q+1}$.
  \end{proof}
\end{frame}

\setbackgroundpicture{Fermilab_-_400_MeV_Lambertson_Magnet.jpg}
\againframe<4-5>{overview}
\begin{frame}
\frametitle{The last step}

  \begin{center}
 $\BallThm{p-1,q-1} \mbox{ and } \SphereThm{p-1,q-1}$ \\
 $\Downarrow$ \\
 $\BallThm{p,q}$
\end{center}

  \vfill
  \begin{center}
    This will require more machinery, \\
    building on simplicial collapse \\
    and regular neighborhoods.
  \end{center}
  \vfill

\end{frame}
\clearbackgroundpicture

\begin{frame}
  \begin{center}
 $\BallThm{p-1,q-1} \mbox{ and } \SphereThm{p-1,q-1}$ \\
 $\Downarrow$ \\
 $\BallThm{p,q}$
\end{center}
\vfill\pause
\begin{lemma}
 Assuming $\BallThm{p-1,q-1}$ and $\SphereThm{p-1,q-1}$, \\
 $(B^p,B^q)$ with $p-q \geq 3$ is unknotted \\
 provided $B^p \collapses B^q$.
\end{lemma}
\vfill\pause
\begin{lemma}
 If $p - q \geq 3$ and $(B^p,B^q)$ is any ball pair, \\
 then $B^p \collapses B^q$.
\end{lemma}

\end{frame}

\begin{frame}
  \frametitle{But first\ldots}

  \vfill\large
  Before we can proceed, \\
  we will prove a couple of helpful lemmas.
  \vfill

\end{frame}

\begin{frame}
  \begin{lemma}
    $Q_1$ and $Q_2$ are unknotted $(p,q)$-ball pairs. \\
    Any homeomorphism $h : \partial Q_1 \stackrel{\cong}{\longrightarrow} \partial Q_2$ \\
    extends to a homeomorphism $h' : Q_1 \stackrel{\cong}{\longrightarrow} Q_2$.
  \end{lemma}
  \pause
  \begin{proof}[Proof (via \textit{Alexander trick}).]
    $y = $ point in interior of $\Delta^q$; \\
    then since $Q_i$ is unknotted, we have maps \\
    $f_i : Q_i \stackrel{\cong}{\longrightarrow} \{y\} \join \partial
    \Gamma^{p,q}$ \\  \pause
    \vspace{1ex}$g : \partial \Gamma^{p,q} \stackrel{{f_1}^{-1}}{\longrightarrow} \partial Q_1 \stackrel{h}{\longrightarrow} \partial Q_2
    \stackrel{f_2}{\longrightarrow} \partial \Gamma^{p,q}$ \\  \pause
    \vspace{1ex}$h' : Q_1 \stackrel{f_1}{\longrightarrow} \{y\} \join \partial
    \Gamma^{p,q} \stackrel{\cone g}{\longrightarrow} \{y\} \join \partial
    \Gamma^{p,q} \stackrel{{f_2}^{-1}}{\longrightarrow} Q_2$
\end{proof}
In short, \textbf{radial extension}.
\end{frame}

\begin{frame}
  \begin{lemma}[assume $\BallThm{p-1,q-1}$] 
    If $Q_1, Q_2$ are unknotted $(p,q)$-ball pairs, \\
    and $Q_3 = Q_1 \cap Q_2$ is a face, \\
    then $Q_1 \cup Q_2$ is unknotted.
 \end{lemma}
\pause
 \begin{proof}
%   $Q_4 = \partial Q_1 - \interior Q_3$ \hfill $Q_5 = \partial Q_2 -
%   \interior Q_3$ \\
%   $Q_3$, $Q_4$, $Q_5$ are unknotted \\
%   $\partial Q_3 = \partial Q_4 = \partial Q_5$ \\
   Choose $Q_3 \stackrel{\cong}{\longrightarrow} \Gamma^{p-1,q-1}$,
   extend over $\partial Q_1 - \interior Q_3$ to $h : \partial Q_1
   \stackrel{\cong}{\longrightarrow} \Gamma^{p-1,q-1} \cup
   C\partial \Gamma^{p-1,q-1}$, \\\pause
   \vspace{1ex}$h$ extends to $Q_1 \to C\Gamma^{p-1,q-1}$ \\\pause
   Similarly produce $Q_2 \to C\Gamma^{p-1,q-1}$ \\\pause
   \vspace{1ex}Glue together $Q_1 \cup Q_2 \cong S\Gamma^{p-1,q-1}$.
\end{proof}
\end{frame}

\setbackgroundpicture{Bangsar.JPG}
\begin{frame}
  \frametitle{Regular neighborhoods}

   $M$, an $n$-manifold, $X \subset M$ a polyhedron \\
    a \textbf{regular neighborhood} of $X$ in $M$ \\
    is a subpolyhedron $N \subset M$ such that
    \begin{itemize}
      \item $N$ is a closed neighborhood of $X$ 
      \item $N$ is an $n$-manifold 
      \item $N \collapses X$.
    \end{itemize}

  \begin{theorem}
    If $N_1$ and $N_2$ are regular neighborhoods of $X \subset M$, \\
    there's a homeomorphism $N_1 \to N_2$ keeping $X$ fixed.
  \end{theorem}
\end{frame}
\clearbackgroundpicture

\begin{frame}
\begin{lemma}
 Assuming $\BallThm{p-1,q-1}$ and $\SphereThm{p-1,q-1}$, \\
 $(B^p,B^q)$ with $p-q \geq 3$ is unknotted \\
 provided $B^p \collapses B^q$.
\end{lemma}

\vfill
\pause

\textcolor{red!50!black}{\textbf{Warning:}} The lemma is false if $p-q = 2$.\\
If $(B^4,B^2) = \cone (S^3,S^1)$, \\
then $B^4 \collapses B^2$ because cones collapse to a subcone \\

\end{frame}

\begin{frame}
  \begin{lemma}
  $B^p \collapses B^q \Rightarrow (B^p,B^q) \mbox{ unknotted}$ \\    
  \end{lemma}
 \pause
   Pick regular neighborhood $N$ of $B^q$; \\\pause
   then $(B^p,B^q) \cong (N,B^q)$.\\\pause
   $B^q = K_k \collapses K_{k-1} \collapses \cdots \collapses K_0 =
   \{x\}$\\\pause
   $Q_i = $ simplicial neighborhood of $K_i$ in $(B^p,B^q)$. \\\pause
   \vfill
   Proceed by induction.\\\pause
   \vfill
   $Q_0 = \{x\} \join L$ where $L = (\lk(x,B^p),\lk(x,B^q))$, \\\pause
   and $L$ is unknotted by either $\BallThm{p-1,q-1}$ or $\SphereThm{p-1,q-1}$.

\end{frame}

\begin{frame}
  Suppose $Q_{i-1}$ is unknotted. \\\pause
  \vfill
  $K_i - K_{i-1}$ consists of a principal simplex $A$ \\with a free face
  $C$. \\\pause
  Pick $a \in \interior A$ and $c \in \interior C$. \\\pause
  \vfill
  $Q_a = \{a\} \join (\lk(a,B^p),\lk(a,B^q))$ is unknotted. \\\pause
  $Q_c = \{c\} \join (\lk(c,B^p),\lk(c,B^q))$ is unknotted. \\\pause
  \vfill
  $Q_i = Q_{i-1} \cup Q_a \cup Q_c$ is union of unknotted ball pairs
  along faces; therefore, $Q_i$ is unknotted. \pause\hfill\qed
\end{frame}

\begin{frame}
  \frametitle{All that remains}
  
  \textcolor{red!50!black}{\textbf{If we could prove}}
  \begin{lemma}
    If $p - q \geq 3$ and $(B^p,B^q)$ is any ball pair, \\
    then $B^p \collapses B^q$.
  \end{lemma}
  \textcolor{red!50!black}{\textbf{we would finish the argument.}}

  \pause\vfill
  Hitherto, no use of the codimension assumption.
  \pause\vfill
  \begin{proof}[Proof Technique.]
    Sunny collapse.
  \end{proof}
\end{frame}

% [[skim:///Users/jim/Papers/MR160218.pdf?page=10][page 10 of MR160218.pdf]]

\setbackgroundpicture{stop-sign.jpg}
\begin{frame}
\end{frame}
\clearbackgroundpicture

\end{document}



