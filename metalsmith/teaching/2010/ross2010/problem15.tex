\documentclass[12pt]{pset}
%\usepackage{add-copyright}

\geometry{margin=0.5in}
\title{Problem Set 15}
\course{Piecewise-Linear Topology}
\author{Jim Fowler}
\date{Summer 2010}

\usepackage{tikz}
\usetikzlibrary{matrix}
\usepackage{enumerate}

\newcommand{\defnword}[1]{\textbf{#1}}

\usepackage{stmaryrd}
\newcommand{\boundary}{\partial}
\newcommand{\collapses}{\searrow}
\newcommand{\expands}{\nearrow}
\newcommand{\she}{\ssearrow\nnearrow}
\newcommand{\join}{\ast}
\newcommand{\subdivided}{\triangleleft}
\geometry{margin=1in,top=0.5in,bottom=0.5in}
\DeclareMathOperator{\st}{st}
\DeclareMathOperator{\trace}{trace}
\DeclareMathOperator{\vertices}{vert}
\DeclareMathOperator{\lk}{lk}
\DeclareMathOperator{\cl}{cl}
\DeclareMathOperator{\interior}{int}
\DeclareMathOperator{\id}{id}
\newcommand{\fullsubcomplex}{\Subset}
\newcommand{\cupp}{\smallsmile}
\newcommand{\Ct}{C_{\mbox{\scriptsize ord}}}
\newcommand{\Chaint}{C^{\mbox{\scriptsize ord}}}

\usepackage{hyperref}

\begin{document}
\maketitle

% [[skim:///Users/jim/Papers/MR755006.pdf?page=2][page 2 of MR755006.pdf]]

\subsubsection*{Cup product.} We discuss cup products from a somewhat
different perspective.

%Why does the sequence split?  last time we used the fact that $X
%\times I$ is the same as $X$ to compute that $H_n(X \times S^1) =
%H_n(X) \oplus H_{n-1}(X)$.

%\parskip 0.45\baselineskip
%\parindent 0pt

\begin{definition*}
  Define $\Chaint_n(X)$ to be formal sums of ordered simplexes (an
  ordered simplex being a possibly degenerate simplex, with a chosen
  order on the vertices---i.e., the vertices might be repeated!).  It
  is very easy to see that a simplicial map $f : X \to Y$ induces a
  chain map between the chain complexes $\Chaint_\bullet(X) \to
  \Chaint_\bullet(Y)$, even if $f$ crushes some simplexes.

  It is the case that $\Chaint_\bullet(X)$ and $C_\bullet(X)$ are
  chain homotopy equivalent, and therefore the homology of these
  complexes is the same.

  Let $\Ct^n(X)$ be the corresponding cochains, i.e., homomorphisms
  from $\Chaint_n(X)$ to the integers.
  Given $\alpha \in \Ct^n(X)$ and $\beta \in \Ct^m(X)$, we define $\alpha \cupp \beta$ by
  $$
  (\alpha \cupp \beta) [v_0,\ldots,v_{n+m+1}] = \alpha [v_0,\ldots,v_n] \cdot \beta [v_n,\ldots,v_{n+m+1}].
  $$
  This product on $\Ct^\bullet(X)$ descends to a product on $H^\bullet(X)$.
\end{definition*}

\begin{problem}
  Compute $H^\bullet(T^3)$ as a ring. Recall $T^3 = S^1 \times T^2 =
  S^1 \times S^1 \times S^1$.
\end{problem}

\begin{problem}
  Let $X = S^2 \vee S^1 \vee S^1$.  Recall that $A \vee B$ means that
  $A$ attached to $B$ along a single point.  Show that $H_n(X) =
  H_n(T^2)$ and $H^n(X) = H^n(T^2)$, but that $H^\bullet(T^2)$ and
  $H^\bullet(X)$ differ as rings, and therefore $T^2 \not\simeq X$.
\end{problem}

\subsection*{Commutativity of cup product}

\begin{problem}
  Let $\alpha$ and $\beta$ be generators of $H^1(T^2)$; compare
  $\alpha \cupp \beta$ and $\beta \cupp \alpha$.
\end{problem}

\begin{problem}
  Define $\epsilon_n = \left(-1\right)^{n(n+1)/2}$, and a homomorphism $\rho :
  \Chaint_n(X) \to \Chaint_n(X)$ by
  $$
  \rho( [v_0,\ldots,v_n] ) = \epsilon_n [v_n, \cdots, v_0].
  $$
  Show that the map $\rho$ is a chain map, and find a chain homotopy
  between $\rho$ and the identity.
\end{problem}

% [[skim:///Users/jim/Teaching/ross2010/problem15.pdf?page=1][page 1 of problem15.pdf]]

\begin{problem}
Let $\rho : \Chaint_n(X) \to \Chaint_n(X)$ be as above, and let $\rho^\star :
\Ct^n(X) \to \Ct^n(X)$ be the induced map on cochains. Then verify the following:
  \begin{enumerate}[\bfseries (1)]
  \item $\epsilon_{n+m} = (-1)^{nm} \epsilon_n \epsilon_m$.
  \item $(\rho^\star \varphi \cupp \rho^\star \psi) [v_0,\ldots,v_{n+m}] =
    \varphi( \epsilon_n [v_n,\ldots,v_0] ) \cdot \psi( \epsilon_m[v_{n+m},\ldots,v_n] )$.
  \item $\rho^\star (\psi \cupp \varphi) [v_0,\ldots,v_{n+m}] =
    \epsilon_{n+m} \varphi( [v_n,\ldots,v_0] ) \cdot \psi( [v_{n+m},\ldots,v_n] )$.
  \item $\rho^\star (\varphi \cupp \psi) = (-1)^{nm} \left( \rho^\star \varphi \cupp \rho^\star \psi \right)$.
  \end{enumerate}
  Since $\rho$ is chain homotopic to the identity, we may conclude
  that $\varphi \cupp \psi = (-1)^{nm} \left( \psi \cupp \varphi
  \right)$ in cohomology.  You should be thrilled to see that the cup
  product is (graded) commutative once we pass to cohomology, but far
  from commutative on the level of cochains.
\end{problem}

\end{document}

