\documentclass[12pt]{pset}
%\usepackage{add-copyright}

\title{Problem Set 1}
\course{Piecewise-Linear Topology}
\author{Jim Fowler}
\date{Summer 2010}

\renewcommand{\labelitemi}{$\circ$}

\begin{document}
\maketitle

\noindent Problem Set 1 is \textit{intentionally} vague: I want you to think a
bit about how you would make these notions precise.  Problems
 marked with a $\bullet$ should be written up.

\vfill\begin{problem}
  Build a circle $S^1$ by gluing some line segments together
  along their vertices.  How many different ways are there of doing this?
\end{problem}

\vfill\begin{problem}
 Build a sphere $S^2$ by gluing triangles together along their
 boundaries.  How are the various ways of doing this related to each
 other?
\end{problem}

\vfill\begin{definition*}
An object built by gluing together triangles is called a
\textit{simplicial complex}.
\end{definition*}

\vfill\begin{problem}
Build a torus $T^2$ by gluing triangles together.\hfill%
\raisebox{2ex-\height}[0in][0in]{\parbox{0.25\textwidth}{\includegraphics[width=0.25\textwidth]{torus.pdf} \\ \null\hfill \raisebox{1ex}{a torus}\hfill\null}}
\end{problem}

\vfill\begin{definition*}
 A function $f : K \to L$ sending
\begin{itemize}
\item vertices to vertices,
\item edges to either edges or vertices, and
\item \begin{tabular}[t]{@{}l@{ }l@{ }l}
    triangles & to & triangles, or \\
    & to & edges, or \\
    & to & vertices
    \end{tabular}
\end{itemize}
is called a \textit{simplicial map}.
\end{definition*}

\vfill\begin{requiredproblem}
  Find a simplicial map $f : T^2 \to S^1$ so that, for every vertex $x \in
  S^1$, the preimage $f^{-1}(x)$ is a circle.
\end{requiredproblem}

\vfill\begin{requiredproblem} Let $k \in \N$.  Find a simplicial map
  $f : T^2 \to T^2$ so that, for every point $x \in T^2$, the preimage
  $f^{-1}(x)$ consists of $k$ points?  And consider: what does this
  even mean when $x$ is not a vertex?
\end{requiredproblem}

\vfill\begin{remark*}
You might be worried that, because there are different ways of
building $T^2$ out of triangles, the notion of ``simplicial map'' is
ill-defined.  You're right: we'll have to fix that.  Nevertheless\ldots
\end{remark*}

\vfill\begin{requiredproblem}
  Find a simplicial map $f : T^2 \to S^2$ which doesn't crush any edges (i.e., edges are sent to edges, not to vertices).
\end{requiredproblem}

\vfill\begin{requiredproblem}
  Find a simplicial map $f : S^2 \to T^2$ which doesn't crush any edges.
\end{requiredproblem}

\vfill\begin{problem}
  Suppose $f : S^1 \to S^2$ is an injective simplicial map (i.e.,
  distinct simplexes are sent to distinct simplexes).  Does the image of
  $f$ necessarily separate $S^2$ into two pieces?
\end{problem}

\vfill\begin{problem}
Suppose $f : S^1 \to T^2$ is an injective simplicial map.  Into how many pieces can the image of $f$ separate $T^2$?
\end{problem}

\vfill\begin{problem} Do there exist simplicial maps $f : T^2 \to S^2$
  and $g : S^2 \to T^2$ which are inverses of each other?  If not, why
  not?
\end{problem}

% \begin{problem}
% Build a 3-sphere $S^3$ by gluing tetrahedra along their boundaries.
% \end{problem}

% %\begin{problem}
% %Build a 3-torus $T^3$ by gluing together tetrahedra; 
% %\end{problem}

% \begin{problem}
% Does there exist a map $f : S^3 \to S^2$ so that, for every $x \in S^2$, the preimage $f^{-1}(x)$ is an $S^1$?
% \end{problem}

\end{document}

