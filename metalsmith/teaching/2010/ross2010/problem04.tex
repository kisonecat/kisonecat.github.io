\documentclass[12pt]{pset}
%\usepackage{add-copyright}

\title{Problem Set 4}
\course{Piecewise-Linear Topology}
\author{Jim Fowler}
\date{Summer 2010}

\newcommand{\defnword}[1]{\textbf{#1}}

\usepackage{stmaryrd}
\newcommand{\boundary}{\partial}
\newcommand{\collapses}{\searrow}
\newcommand{\expands}{\nearrow}
\newcommand{\she}{\ssearrow\nnearrow}
\newcommand{\join}{\ast}
\newcommand{\subdivided}{\triangleleft}
\geometry{margin=1in}%,top=0.5in,bottom=0.5in}
\DeclareMathOperator{\st}{st}
\DeclareMathOperator{\vertices}{vert}
\DeclareMathOperator{\lk}{lk}
\DeclareMathOperator{\cl}{cl}
\DeclareMathOperator{\interior}{int}

\begin{document}
\maketitle

\subsubsection*{A new equivalence.} Problem Set 4 introduces
\textit{simple homotopy equivalence}, an equivalence relation between
simplicial complexes which is coarser than simplicial isomorphism or PL
homeomorphism; Please write down answers to problems labeled
  with $\bullet$.

%\parskip 0.5\baselineskip
%\parindent 0pt

 \begin{definition*}
   Let $K$ be a complex, and $\sigma \in K$ a simplex. Call $\sigma$ a
   \textbf{principal simplex} if the only simplex containing $\sigma$
   is $\sigma$ itself (i.e., it isn't contained in a larger simplex).

   For a simplex $\sigma \in K$, call $\tau < \sigma$ a \textbf{free
     face} of $\sigma$ if the only simplexes containing $\tau$ are
   $\tau$ and $\sigma$.

   If $L$ and $K = L \sqcup \{ \sigma, \tau \}$ are simplicial
   complexes, and $\sigma$ is a principal simplex of $K$, and $\tau$
   is a free face of $\sigma$, then $L$ is an \textbf{elementary
     simplicial collapse} of $K$.  If $K_1, K_2, \ldots, K_n$ are
   complexes, with $K_{i+1}$ an elementary simplicial collapse of
   $K_i$, then $K_n$ is a \textbf{simplicial collapse} of $K_1$,
   denoted $K_1 \collapses K_n$; conversely, $K_1$ a
   \textbf{simplicial expansion} of $K_n$, denoted $K_n \expands K_1$.

   Finally, $K$ is \textbf{simple homotopy equivalent} to $L$ if you can reach transform $K$ into $L$ via a sequence of
    \begin{itemize}
    \item PL homeomorphisms,
    \item simplicial collapses,
    \item simplicial expansions.
    \end{itemize}
    In this case, we write $K \she L$.
\end{definition*}

\begin{requiredproblem}
  Find two manifolds with boundary which are simple homotopy
  equivalent, but not PL homeomorphic.
\end{requiredproblem}

\begin{definition*}
  Let $K$ be a complex.  If $K \collapses \Delta^0$, i.e., it
  collapses to a point, then we call the complex \textbf{collapsible}.
\end{definition*}

\begin{requiredproblem}
 Prove that the complex $\Delta^n$ is collapsible.
\end{requiredproblem}

\begin{problem}
  Construct infinitely many collapsible complexes, no two of which are
  PL homeomorphic.
\end{problem}

\noindent\parbox{0.75\textwidth}{%
\begin{problem}
Show that \textit{Bing's House with Two Rooms,} as shown on the right,
is simple homotopy equivalent to a point, but is not collapsible.
\end{problem}}%
\hspace{5pt}%
\parbox{0.25\textwidth}{%
\includegraphics[width=0.25\textwidth-5pt]{bings-house.pdf}%
}

\begin{problem}
  Show that the \textit{dunce cap} is not collapsible,
  but is simple homotopy equivalent to a point; the dunce cap is
  constructed by starting with triangle $ABC$, and identifying edge
  $AB$ with edge $AC$, and also with edge $CB$.
\end{problem}

\begin{requiredproblem}
  Is $\chi(K) = \chi(L)$ if $K$ and $L$ are simple homotopy equivalent?
\end{requiredproblem}

\begin{definition*}
  The \textbf{cone} of a complex $K$ (written $CK$) is $K \join
  \Delta^0$, i.e., the join of $K$ with a point.
\end{definition*}

\begin{requiredproblem}
  Show that the cone of any complex is collapsible.
\end{requiredproblem}

\begin{definition*}
  The \textbf{suspension} of a complex $K$ (written $SK$) is $K \join
  S^0$, where $S^0 = \boundary \Delta^1$, that is, two disjoint
  points.  In this notation, $SS^n \cong S^{n+1}$.
\end{definition*}

\begin{requiredproblem}
  Find a complex $K$ so that $SK \she K$.
\end{requiredproblem}

\begin{problem}
  Is $K$ (non-)collapsible precisely when $SK$ is (non-)collapsible?
\end{problem}

\begin{problem}
 Is the join of collapsible complexes necessarily collapsible?
\end{problem}

\begin{problem}
  Is join well-defined on simple homotopy types?  That is, if $K \she
  K'$ and $L \she L'$, is it then the case that $K \join L \she K'
  \join L'$?
\end{problem}

\subsection*{3-manifolds}

\begin{problem}
  Prove that there exist two distinct 3-manifolds.
\end{problem}

\begin{problem}
  Find a copy of $T^2 \# T^2 \# T^2$ inside $T^3$, which divides $T^3$
  into two components, homeomorphic to each other.
\end{problem}

\begin{problem}
 Let $f : T^2 \to T^2$ be a homeomorphism, and use $f$ to glue
 together two copies of $S^1 \times \Delta^2$.  Describe 
 two distinct $3$-manifolds that can result from this procedure
 (though it might be hard to prove that they are distinct, given what
 we know thus far!).
\end{problem}


\subsection*{Connected sum}

\begin{problem}
  Given $n$-manifolds $M$ and $N$, compute $\chi(M \# N)$ in terms of
  $\chi(M)$ and $\chi(N)$.
\end{problem}

\begin{requiredproblem}
Show that $T^2$ and $T^2 \# T^2$ are not PL homeomorphic.
\end{requiredproblem}

\begin{problem}
 Let $K$ be the Klein bottle; determine whether $K \# K$ and $T^2 \#
 K$ are PL homeomorphic.
\end{problem}

\begin{problem}[Hard]
  Suppose that $M \# N = M$.  Does it follow that $N$ is a sphere?
\end{problem}


\subsection*{Orientability}

\begin{requiredproblem}
  Construct an orientation on $S^n$.
\end{requiredproblem}

\begin{problem}
  Does there exist a connected 2-manifold $M^2$ with
  $$\chi(M) > 0 \mbox{ and } w_1(M) = 1. $$
\end{problem}

\begin{problem}
  Give an example of a non-orientable $3$-manifold $M^3$.
\end{problem}

\begin{problem}
  Let $K$ be the Klein bottle; is there an orientation on the
  4-manifold $K \times K$?
\end{problem}

\begin{problem}
  Consider how $w_1(M \times N)$ might relate to $w_1(M)$ and
  $w_1(N)$.
\end{problem}


\subsection*{Products}

\begin{problem}
  Show that if $M$ is a manifold, then $M \times S^1$ is a manifold.  What does $M \times S^1$ even mean?
\end{problem}

\begin{problem}
  Compute $\chi(K \times L)$ in terms of $\chi(K)$ and $\chi(L)$.
\end{problem}

\begin{problem}
  Show that there exist infinitely many distinct 4-manifolds.
\end{problem}

\end{document}

