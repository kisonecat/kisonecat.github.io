\documentclass[12pt]{pset}
%\usepackage{add-copyright}

\geometry{margin=0.5in}
\title{Problem Set 12}
\course{Piecewise-Linear Topology}
\author{Jim Fowler}
\date{Summer 2010}

\usepackage{tikz}
\usetikzlibrary{matrix}

\newcommand{\defnword}[1]{\textbf{#1}}

\usepackage{stmaryrd}
\newcommand{\boundary}{\partial}
\newcommand{\collapses}{\searrow}
\newcommand{\expands}{\nearrow}
\newcommand{\she}{\ssearrow\nnearrow}
\newcommand{\join}{\ast}
\newcommand{\subdivided}{\triangleleft}
\geometry{margin=1in,top=0.5in,bottom=0.5in}
\DeclareMathOperator{\st}{st}
\DeclareMathOperator{\trace}{trace}
\DeclareMathOperator{\vertices}{vert}
\DeclareMathOperator{\lk}{lk}
\DeclareMathOperator{\cl}{cl}
\DeclareMathOperator{\interior}{int}
\DeclareMathOperator{\id}{id}
\newcommand{\fullsubcomplex}{\Subset}

\usepackage{hyperref}

\begin{document}
\maketitle

% [[skim:///Users/jim/Papers/MR755006.pdf?page=2][page 2 of MR755006.pdf]]

\subsubsection*{Homeomorphism invariance.} Today, 
we will see why homology is invariant under subdivision.

%Why does the sequence split?  last time we used the fact that $X
%\times I$ is the same as $X$ to compute that $H_n(X \times S^1) =
%H_n(X) \oplus H_{n-1}(X)$.

%\parskip 0.45\baselineskip
%\parindent 0pt

\begin{definition*}
  The \textbf{barycentric subdivision} of a complex $K$ is the
  complex $K'$ where
  \begin{itemize}
  \item the vertices of $K'$ are the simplexes of $K$, and
  \item the $n$-simplexes of $K'$ are chains $\sigma_0 \subsetneq \sigma_1 \subsetneq \cdots \subsetneq \sigma_n$ of simplexes $\sigma_i$ of $K$.
  \end{itemize}

  \noindent Given a PL map $f : K \to L$, a simplicial map $g : K \to L$ is a
  \textbf{simplicial approximation} to $f$ provided
  $$
  f(\st v) \subset \st g(v) \mbox{ for all vertices $v \in K$.}
  $$
  Here, $\st$ denotes the star of a vertex.
\end{definition*}

% [[skim:///Users/jim/Papers/MR1867354.pdf?page=187][page 187 of
% MR1867354.pdf]]

\begin{problem}
  Show that two simplicial approximations $g_1, g_2 : K \to L$ induce
  isomorphic maps on homology.
\end{problem}

\begin{theorem*}[The simplicial approximation theorem]
  If $K$ and $L$ are simplicial complexes, and $f : K \to L$ is a PL
  map, then there is a simplicial map $g : K^{(n)} \to L$ which is a
  simplicial approximation to $f$.  Here, $L^{(n)}$ denotes the barycentric subdivision of
  $L$, iterated $k$ times.
\end{theorem*}

\begin{problem}
  Let $f$ be a PL map; show that a simplicial approximation $g$ is
  homotopic to $f$.
\end{problem}

\begin{problem}[from lecture]
  Use the simplicial approximation theorem to prove
  the Lefschetz fixed point theorem.
\end{problem}

% [[skim:///Users/jim/Papers/MR755006.pdf?page=54][page 54 of MR755006.pdf]]

%\vfill

\begin{definition*}
  A chain complex $C_\bullet$ is \textbf{acyclic} if it has trivial
  homology (to think about ``trivial homology,'' we might prefer to
  think of $C_\bullet$ as an \textbf{augmented} complex).

  An \textbf{acyclic carrier} from a simplicial complex $K$ to a
  simplicial complex $L$ is an assignment $\Phi$, associating to each
  simplex $\sigma \in K$ a subcomplex $\Phi(\sigma)$ of $L$ so that
  \begin{itemize}
  \item $\Phi(\sigma) \neq \varnothing$,
  \item $\Phi(\sigma)$ is acyclic, and
  \item if $\tau < \sigma$, then $\Phi(\tau) \subset \Phi(\sigma)$.
  \end{itemize}
  A chain map $f : C_\bullet(K) \to C_\bullet(L)$ is \textbf{carried} by
  $\Phi$ if each $f(\sigma)$ is carried by $\Phi(\sigma)$.
\end{definition*}

\begin{theorem*}[Acyclic carrier theorem]
  Let $\Phi$ be an acyclic carrier from a simplicial complex $K$ to a
  simplicial complex $L$.  Then there exists an
  augmentation-preserving chain map $f : C_\bullet(K) \to C_\bullet(L)$,
  carried by $\Phi$ and unique up to chain homotopy.
\end{theorem*}

\begin{problem}[from lecture]
  Use the acyclic carrier theorem to show that homology is invariant under PL homeomorphism.
\end{problem}


\end{document}

