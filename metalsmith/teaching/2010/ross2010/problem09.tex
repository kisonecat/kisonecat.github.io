\documentclass[12pt]{pset}
%\usepackage{add-copyright}

\title{Problem Set 9}
\course{Piecewise-Linear Topology}
\author{Jim Fowler}
\date{Summer 2010}

\newcommand{\defnword}[1]{\textbf{#1}}

\usepackage{stmaryrd}
\newcommand{\boundary}{\partial}
\newcommand{\collapses}{\searrow}
\newcommand{\expands}{\nearrow}
\newcommand{\she}{\ssearrow\nnearrow}
\newcommand{\join}{\ast}
\newcommand{\subdivided}{\triangleleft}
\geometry{margin=1in}%,top=0.5in,bottom=0.5in}
\DeclareMathOperator{\st}{st}
\DeclareMathOperator{\vertices}{vert}
\DeclareMathOperator{\lk}{lk}
\DeclareMathOperator{\cl}{cl}
\DeclareMathOperator{\interior}{int}
\newcommand{\fullsubcomplex}{\Subset}

\begin{document}
\maketitle

\subsubsection*{Homology.} Today we begin studying the ``homology'' of
simplicial complexes, a sort of theory that revolves around a basic
question: if something looks locally like it bounds something, is it
actually the boundary of something?

\parskip 0.25\baselineskip
\parindent 0pt

\begin{definition*}
  An oriented $n$-simplex is an $n$-simplex with an ordering on the
  vertices, where two oriented simplexes are the same if the order
  differs by an even permutation.  Our convention will be that
  oppositely oriented simplexes are inverses of each other, i.e.,
  if $\sigma$ is an odd permutation, then
  $$
  [v_0,\ldots,v_n] = - [v_{\sigma(0)},\ldots,v_{\sigma(n)}].
  $$

  An $n$-\textbf{chain} is a formal sum of oriented $n$-simplexes,
  e.g.,
  $$
  \alpha = 17 [v_0,\ldots,v_n] - 13 [w_0,\ldots,w_n], \mbox{ where
    $v_i$ and $w_i$ are vertices in the complex.}
  $$
 The collection of all $n$-chains in a complex $X$ is written
  $C_n(X)$.  This is an \textbf{abelian group}.  

  The boundary $\partial$ of an oriented $n$-simplex
  $[v_0,\ldots,v_n]$ is the $(n-1)$-chain
  $$
  \sum_{i=0}^n (-1)^i [v_0,\ldots,\hat{v}_i,\ldots,v_n],
  $$
  where the hat $\hat{v}_i$ denotes a missing term.  The boundary of a
  chain is the boundary of each of its terms (i.e., defined by
  linearity), so we have a function $\partial : C_n(X) \to C_{n-1}(X)$.
\end{definition*}

\begin{requiredproblem}
  Show that $\partial \circ \partial = 0$, that is, show that the
  boundary of the boundary is empty.
\end{requiredproblem}

\begin{problem}
  Let $\sigma$ be an oriented $n$-simplex $[v_0,\ldots,\v_n]$, and
  $\tau = [v_1,v_0,v_2,v_3,\ldots,v_n]$; how do $\partial(\sigma)$
  and $\partial(\tau)$ relate?
\end{problem}

\begin{definition*}
 An $n$-chain $\alpha$ is an $n$-\textbf{cycle} if $\partial \alpha =
 0$.  The collection of all $n$-cycles in a complex $X$ is written $Z_n(X)$.

 An $n$-chain $\alpha$ is an $n$-\textbf{boundary} if there is some
 $(n+1)$-chain $\beta$ with $\partial \beta = \alpha$.  The collection
 of all $n$-boundaries in a complex $X$ is written $B_n(X)$.  By the
 previous exercise, \textit{every boundary is a cycle}.
 
 Define the \textbf{homology} of $x$ by $H_n(X) = Z_n(X)/B_n(X)$,
 meaning that $H_n(X)$ consists of cycles, where two cycles are
 identified if they differ by a boundary.
\end{definition*}

\begin{requiredproblem}
  Show that $H_n(X)$ is an abelian group.  For example, is the sum of
  two cycles a cycle?  Is addition well-defined after identifying
  cycles which differ by a boundary?
\end{requiredproblem}

\begin{requiredproblem}
Let $X$ be the boundary of the triangle (i.e., the simplicial
complex with three vertices and three edges, in other words, the
circle).  Compute $H_0(X)$ and $H_1(X)$.
\end{requiredproblem}

\begin{requiredproblem}
 Let $X$ be the triangle (i.e., the simplex $\Delta^2$ regarded as a
 simplicial complex).  Compute $H_0(X)$ and $H_1(X)$.
\end{requiredproblem}

\begin{requiredproblem}
 Let $X$ be a triangulation of the sphere.  Compute $H_n(X)$.
\end{requiredproblem}

\begin{problem}
Let $X$ be the boundary of the triangle, and $Y$ be the boundary of
the square.  Compute $Z_n(X)$, $Z_n(Y)$, $B_n(X)$, $B_n(Y)$, $H_n(X)$
and $H_n(Y)$ for $n = 0, 1$.  Compare them.
\end{problem}

\end{document}

