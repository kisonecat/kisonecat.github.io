\documentclass[12pt]{pset}
%\usepackage{add-copyright}

\geometry{margin=0.5in}
\title{Problem Set 11}
\course{Piecewise-Linear Topology}
\author{Jim Fowler}
\date{Summer 2010}

\usepackage{tikz}
\usetikzlibrary{matrix}

\newcommand{\defnword}[1]{\textbf{#1}}

\usepackage{stmaryrd}
\newcommand{\boundary}{\partial}
\newcommand{\collapses}{\searrow}
\newcommand{\expands}{\nearrow}
\newcommand{\she}{\ssearrow\nnearrow}
\newcommand{\join}{\ast}
\newcommand{\subdivided}{\triangleleft}
\geometry{margin=1in}%,top=0.5in,bottom=0.5in}
\DeclareMathOperator{\st}{st}
\DeclareMathOperator{\trace}{trace}
\DeclareMathOperator{\vertices}{vert}
\DeclareMathOperator{\lk}{lk}
\DeclareMathOperator{\cl}{cl}
\DeclareMathOperator{\interior}{int}
\DeclareMathOperator{\id}{id}
\newcommand{\fullsubcomplex}{\Subset}

\usepackage{hyperref}

\begin{document}
\maketitle

% [[skim:///Users/jim/Papers/MR755006.pdf?page=2][page 2 of MR755006.pdf]]

\subsubsection*{Homotopy.} Today, 
%we will see why homology is invariant under subdivision, and 
we will introduce
\textit{homotopy}, a relationship between maps $X \to Y$.
We will assume that homology is invariant under subdivision.

%Why does the sequence split?  last time we used the fact that $X
%\times I$ is the same as $X$ to compute that $H_n(X \times S^1) =
%H_n(X) \oplus H_{n-1}(X)$.

\parskip 0.45\baselineskip
\parindent 0pt

I had originally intended today to discuss why homology is invariant
under PL homeomorphism, but I think it is wiser to delay that by one
day---so for today, we'll assume that homology is invariant under subdivision.

%\begin{problem}
%  Thus far, we have treated homology as an invariant of simplicial
%  complexes.  Show that it is actually an invariant of polyhedra, and
%  functorial with respect to PL maps.
%\end{problem}

\subsection*{Homotopy}

\begin{definition*}
  We can describe $X \times I$ as a simplicial complex; in particular, for
  $\Delta^n \times I$,
  \begin{align*}
    [v_0,\ldots,v_n] &= \Delta^n \times \{0\} \\
    [w_0,\ldots,w_n] &= \Delta^n \times \{1\} \\
    \Delta^n \times I &= \bigcup [v_0,\ldots,v_i,w_i,\ldots,w_n].
    \end{align*}

  Maps $f, g : X \to Y$ are \textbf{homotopic} if there is a map $F :
  X \times I \to Y$ with $F |_{X \times \{0\}} = f$ and $F |_{X \times
    \{1\}} = g$.  We write $f \simeq_F g$ if $f$ and $g$ are homotopic
  via $F$, and call $F$ a \textbf{homotopy}.
\end{definition*}

\begin{problem}
  Why is homotopy an equivalence relation between maps?% (i.e., reflexive, symmetric, and transitive).
\end{problem}

\begin{definition*}
  Given $f, g : X \times I$ with $f \simeq_F g$, define the \textbf{prism
    operator} $P : C_n(X) \to C_{n+1}(Y)$ via
  $$
  P(\sigma) = \sum_i (-1)^i F \circ (\sigma \times I)
  |_{[v_0,\ldots,v_i,w_i,\ldots, w_n]}.
  $$
\end{definition*}

\begin{problem}
  Show that $\partial \circ P + P \circ \partial = g_\star - f_\star$.
  When this is the case for some $P$ which raises degree, we say that
  the chain maps $g_\star$ and $f_\star$ are \textbf{chain homotopic}
  and call $P$ a \textbf{chain homotopy}.
\end{problem}

\begin{requiredproblem}
  Show that homotopic maps induce the same map on homology (or more
  generally, show that chain homotopic maps induce the same map on
  homology).
\end{requiredproblem}

\begin{definition*}
  Two spaces $X$ and $Y$ are \textbf{homotopy equivalent} if there
  exist maps $f : X \to Y$ and $g : Y \to X$ so that $g \circ f \simeq
  \id_X$ and $f \circ g \simeq \id_Y$.  In this case, we will write $X
  \simeq Y$.
\end{definition*}

\begin{problem}
  Show that homotopy equivalence is an equivalence relation between spaces.
\end{problem}



\begin{requiredproblem}
  Find spaces $X$ and $Y$ so that $X \simeq Y$ but $X \not\cong Y$.
\end{requiredproblem}

\begin{problem}
  Show that if $X \she Y$, then $X \simeq Y$.  (Interestingly, the
  converse is quite false, but producing a counterexample would take us
  beyond this course; Marshall Cohen's beautiful book \textit{A Course
    in Simple-homotopy Theory} discusses such issues at length).
\end{problem}

\begin{problem}
  Show that if $X \simeq Y$, then $H_n(X) \cong H_n(Y)$.  The converse
  is false, but that would be difficult to show with the machinery we
  have developed thus far.
\end{problem}

\begin{requiredproblem}
 Show that $CX \simeq \mbox{point}$.
\end{requiredproblem}

\begin{problem}
  If $X \simeq Y$, is it the case that $SX \simeq SY$?
\end{problem}

% [[skim:///Users/jim/Downloads/AT.pdf?page=121][page 121 of AT.pdf]]

\subsection*{Fixed point theorems}

Throughout mathematics, one finds theorems of the following shape: if
such-and-such $f : X \to X$ such-and-such, then there exists $x \in X$
with $f(x) = x$.  These theorems turn out to be quite useful.

\begin{problem}[No retraction theorem] Show that there is no map $f :
  D^n \to S^{n-1}$ so that $f |_{\partial D^n} =
    \id_{S^{n-1}}$.  Such a map is called a \textbf{retraction}.
\end{problem}

\begin{problem}[Brouwer fixed point theorem] Show that every PL map
  $f : D^n \to D^n$ has a fixed point.  \textit{Hint:} Suppose $f :
  D^n \to D^n$ does not have a fixed point; intersect the line through
  $x$ and $f(x)$ with $\partial D^n = S^{n-1}$ to build a retraction.
  But is this a PL map?
\end{problem}

\subsection*{Coefficients}

\begin{definition*}[Homology with coefficients]
  Thus far, our chains have had integer coefficients; we can define
  $C_n(X;R)$ to be formal sums of oriented $n$-simplexes in $X$ with
  coefficients in the ring $R$.  We write $H_n(X;R)$ for the
  corresponding homology.

  The $n^{\mbox{\scriptsize th}}$ Betti number of $X$ is $b_n(X) := \dim H_n(X;\Q)$.  The
  Euler characteristic of $X$ is
  $$
  \chi(X) = \sum_n (-1)^n \dim C_n(X;\Q) = \sum_n (-1)^n \dim H_n(X;\Q).
  $$
  The first equality is the definition; the second is a theorem.
\end{definition*}

\begin{problem}[Lefschetz fixed point theorem]
  Given a map $f : X \to X$, define the \textbf{Lefschetz number} of
  $f$ to be
  $$
  \Lambda(f) = \sum_n (-1)^n \trace \left(f_\star : H_n(X;\Q) \to H_n(X;\Q) \right).
  $$
  Show that if $\Lambda(f) \neq 0$, then $f$ has a fixed point.
\end{problem}

\begin{problem}
  Show that the Lefschetz fixed point theorem implies Brouwer fixed
  point theorem.
\end{problem}

\begin{problem}
  Suppose $\chi(X) \neq 0$.  Show that any map $f : X \to X$ homotopic
  to $\id_X$ has a fixed point.
\end{problem}

\begin{problem}
  Find a space $X$ and a fixed-point-free map $f : X \to X$ with $f \simeq \id_X$.
\end{problem}

\end{document}

