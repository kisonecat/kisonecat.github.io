\documentclass[14pt]{beamer}

\setbeamertemplate{navigation symbols}{}

\usepackage{pgf}
\usepackage{amsmath}
\usepackage{amsthm}
%\newtheorem*{problem}{Problem}
\usepackage{amssymb}
\newcommand{\N}{\mathbb{N}}
\newcommand{\R}{\mathbb{R}}

\usepackage{wasysym}
\usepackage{stmaryrd}
\newcommand{\boundary}{\partial}
\newcommand{\collapses}{\searrow}
\newcommand{\sunnycollapses}{\searrow\hspace{-12pt}\mbox{\raisebox{5pt}{\sun}}\hspace{2pt}}
\newcommand{\expands}{\nearrow}
\newcommand{\she}{\ssearrow\nnearrow}
\newcommand{\join}{\ast}
\newcommand{\subdivided}{\triangleleft}
\newcommand{\fullsubcomplex}{\Subset}
\newcommand{\BallThm}[1]{\mbox{$\textcolor{green!50!black}{\mbox{\textbf{BallThm}}_{#1}}$}}
\newcommand{\SphereThm}[1]{\mbox{$\textcolor{green!50!black}{\mbox{\textbf{SphereThm}}_{#1}}$}}

\DeclareMathOperator{\st}{st}
\DeclareMathOperator{\vertices}{vert}
\DeclareMathOperator{\lk}{lk}
\DeclareMathOperator{\cl}{cl}
\DeclareMathOperator{\cone}{cone}
\DeclareMathOperator{\interior}{int}

\title{Topology of  \\ Piecewise-Linear Manifolds}
\author{Jim Fowler}
\date{Lecture 9 \\ Summer 2010}

\newcommand{\setbackgroundpicture}[1]{%
\usebackgroundtemplate{
\begin{pgfpicture}{0in}{0in}{\paperwidth}{\paperheight}
\pgfputat{\pgfxy(0,0)}{\includegraphics[width=\paperwidth]{#1}}
\color{white}
\pgfsetfillopacity{0.8}
\pgfrect[fill]{\pgfxy(0,0)}{\pgfpoint{\paperwidth}{\paperheight}}
\end{pgfpicture}
}
}
\newcommand{\clearbackgroundpicture}{\usebackgroundtemplate{}}

\begin{document}

\begin{frame}
\maketitle
\end{frame}

\begin{frame}
\frametitle{Elementary simplicial collapse}

\begin{definition}
  Let $L$ and $K = L \cup \cl \{ \sigma, \tau \}$ be complexes \\

  \vspace{1ex}If $\sigma$ is a principal simplex of $K$, and \\
  $\tau$ is a free face of $\sigma$, then \\
  $L$ is an \textbf{elementary simplicial collapse} of $K$.
\end{definition}

\end{frame}

\begin{frame}
\frametitle{Simplicial collapse}

\begin{definition}
  Let $K_1, K_2, \ldots, K_n$ be complexes, with \\
  $K_{i+1}$ an elementary simplicial collapse of $K_i$.

  \vspace{1ex}Call $K_n$ a \textbf{simplicial collapse} of $K_1$, and \\
  write $K_1 \collapses K_n$.

  \vspace{1ex}Call $K_1$ a \textbf{simplicial expansion} of $K_n$, and \\
  write $K_n \expands K_1$.
\end{definition}

\end{frame}

\setbackgroundpicture{problem07.pdf}
\begin{frame}
  \frametitle{Collapse onto a graph}

  \begin{problem}
    Show that $T^2 - D^2$ and $S^3 - 3D^2$ \\
    collapse onto $S^1 \vee S^1$, \\
    the union of two circles along one point.
\end{problem}
  \pause
  Start with an explicit triangulation. \\\pause
  Incidentally, what is $\chi$?
\end{frame}
\clearbackgroundpicture

\setbackgroundpicture{Bangsar.JPG}
\begin{frame}
  \frametitle{Regular neighborhoods}

   $M$, an $n$-manifold, $X \subset M$ a polyhedron \\
    a \textbf{regular neighborhood} of $X$ in $M$ \\
    is a subpolyhedron $N \subset M$ such that
    \begin{itemize}
      \item $N$ is a closed neighborhood of $X$ 
      \item $N$ is an $n$-manifold 
      \item $N \collapses X$.
    \end{itemize}

  \begin{theorem}
    If $N_1$ and $N_2$ are regular neighborhoods of $X \subset M$, \\
    there's a homeomorphism $N_1 \to N_2$ keeping $X$ fixed.
  \end{theorem}
\end{frame}
\clearbackgroundpicture

\setbackgroundpicture{problem07.pdf}
\begin{frame}
  \frametitle{Possible neighborhoods}
  \vfill
  \begin{problem}
    What are the possible regular neighborhoods of $S^1$ inside a
    2-manifold?  \\
    Which surfaces collapse to a circle?
 \end{problem}
 \vfill
\end{frame}
\clearbackgroundpicture

%%%%%%%%%%%%%%%%%%%%%%%%%%%%%%%%%%%%%%%%%%%%%%%%%%%%%%%%%%%%%%%%

\setbackgroundpicture{zeeman.pdf}
\begin{frame}
 \vfill
 \begin{center}
   Today is the last lecture covering the work of  \\
   \Huge E. C. Zeeman \\
   \vfill
   {\normalsize \textbf{Goal:} \textit{``Unknotting combinatorial balls''}
     \\
     a paper in the \textbf{Annals}
     }
  \end{center}
  \vfill
  \pause
  \begin{center}
  {\textbf{Wednesday:} (co)homology}
  \end{center}
\end{frame}
\clearbackgroundpicture

\setbackgroundpicture{Nf_knots.png}
\begin{frame}
  \frametitle{Main result}

\vfill

\Large
  \begin{theorem}
    Any embedding $S^p \subset S^q$ \\is unknotted, if $q - p \geq 3$.
  \end{theorem}

\vfill

\end{frame}
\clearbackgroundpicture

\setbackgroundpicture{two-giraffes-871277133292IxUu.jpg}
\begin{frame}
  \frametitle{Pairs}

  \begin{definition}
    ``$(p,q)$-sphere pair'' means $S^q \subset S^p$. \\
    We sometimes write $(S^p,S^q)$.

    \vspace{1ex}
    ``$(p,q)$-ball pair'' means $B^q \subset B^p$, \\
    with $B^q$ properly embedded in $B^p$, \\
    meaning $\partial B^q \subset \partial B^p$ 
    and $\interior B^q \subset \interior B^p$. \\
    We sometimes write $(B^p,B^q)$.
  \end{definition}

  \vfill
 ``Pair'' means either a ball-pair or sphere-pair.

  \vfill
  $\partial (B^p,B^q) = (S^{p-1},S^{q-1})$.

\end{frame}
\clearbackgroundpicture

\setbackgroundpicture{Mona_Lisa_headcrop.jpg}
\begin{frame}
  \frametitle{Faces of pairs}

  If $Q' = (B^{p-1},B^{q-1})$ is\\
  contained in the boundary of $Q = (B^p,B^q)$,\\
  we call $Q'$ a \textbf{face} of $Q$.

 \vfill

  \begin{theorem}
   If ball pairs intersect in their common boundary, \\
   their union is a sphere pair.
 \end{theorem}

 \vfill\pause

  \begin{theorem}
   If ball pairs intersect in a face, \\
   their union is a ball pair.
 \end{theorem}

\end{frame}
\clearbackgroundpicture

\setbackgroundpicture{two-giraffes-871277133292IxUu.jpg}
\begin{frame}
  \frametitle{Standard pairs}

  $\Gamma^{p,q} = (S^{p-q} \Delta^q, \Delta^q)$ is the standard
  $(p,q)$-ball pair. \\
  $\partial \Gamma^{p+1,q+1}$ is the standard $(p,q)$-sphere pair.

  \vfill
  A pair is \textbf{unknotted} \\
  if it is homeomorphic to a standard pair.

\end{frame}
\clearbackgroundpicture

\setbackgroundpicture{marble-ball.jpg}
\begin{frame}

  \begin{theorem}[\BallThm{p,q}]
    If $p - q \geq 3$, then any $(p,q)$-ball pair \\
    is unknotted.
  \end{theorem}

  \vfill

  \begin{theorem}[\SphereThm{p,q}]
    If $p - q \geq 3$, then any $(p,q)$-sphere pair \\
    is unknotted.
  \end{theorem}

  \vfill\pause
  \begin{proof}
    By induction.
  \end{proof}
\end{frame}
\clearbackgroundpicture

\setbackgroundpicture{Fermilab_-_400_MeV_Lambertson_Magnet.jpg}
\begin{frame}<4-6>[label=overview]
  \frametitle{Overview of the induction}

  \vfill

  \uncover<4->{\checkmark\hspace{0.25em} }Prove $\BallThm{p,0}$ by hand.\\
  \vspace{1ex}
  \uncover<5->{\checkmark\hspace{0.25em} }\uncover<2->{$\BallThm{p,q}$
    implies $\SphereThm{p,q}$.}\\
  \vspace{1ex}
  \uncover<6->{\checkmark\hspace{0.25em} }\uncover<3->{$\BallThm{p-1,q-1}$ and $\SphereThm{p-1,q-1}$ \\
    \quad\quad\quad together imply $\BallThm{p,q}$.}

  \vfill

\end{frame}
\clearbackgroundpicture

\begin{frame}
  \begin{center}
 $\BallThm{p-1,q-1} \mbox{ and } \SphereThm{p-1,q-1}$ \\
 $\Downarrow$ \\
 $\BallThm{p,q}$
\end{center}
\vfill\pause
\begin{lemma}
 Assuming $\BallThm{p-1,q-1}$ and $\SphereThm{p-1,q-1}$, \\
 $(B^p,B^q)$ with $p-q \geq 3$ is unknotted \\
 provided $B^p \collapses B^q$.
\end{lemma}
\vfill\pause
\begin{lemma}
 If $p - q \geq 3$ and $(B^p,B^q)$ is any ball pair, \\
 then $B^p \collapses B^q$.
\end{lemma}

\end{frame}

\begin{frame}
  \begin{lemma}[assume $\BallThm{p-1,q-1}$] 
    If $Q_1, Q_2$ are unknotted $(p,q)$-ball pairs, \\
    and $Q_3 = Q_1 \cap Q_2$ is a face, \\
    then $Q_1 \cup Q_2$ is unknotted.
 \end{lemma}
 \vfill
 We proved this last time.
 \vfill
\end{frame}

\begin{frame}
\begin{lemma}
 Assuming $\BallThm{p-1,q-1}$ and $\SphereThm{p-1,q-1}$, \\
 $(B^p,B^q)$ with $p-q \geq 3$ is unknotted \\
 provided $B^p \collapses B^q$.
\end{lemma}

\end{frame}

\begin{frame}
  \begin{lemma}
  $B^p \collapses B^q \Rightarrow (B^p,B^q) \mbox{ unknotted}$ \\    
  \end{lemma}
 \pause
   Pick regular neighborhood $N$ of $B^q$; \\\pause
   then $(B^p,B^q) \cong (N,B^q)$.\\\pause
   $B^q = K_k \collapses K_{k-1} \collapses \cdots \collapses K_0 =
   \{x\}$\\\pause
   $Q_i = $ simplicial neighborhood of $K_i$ in $(B^p,B^q)$. \\\pause
   \vfill
   $Q_0 = \{x\} \join L$ where $L = (\lk(x,B^p),\lk(x,B^q))$, \\\pause
   and $L$ is unknotted by either $\BallThm{p-1,q-1}$ or $\SphereThm{p-1,q-1}$.

\end{frame}

\begin{frame}
  Suppose $Q_{i-1}$ is unknotted. \\\pause
  \vfill
  $K_i - K_{i-1}$ consists of a principal simplex $A$ \\with a free face
  $C$. \\\pause
  Pick $a \in \interior A$ and $c \in \interior C$. \\\pause
  \vfill
  $Q_a = \{a\} \join (\lk(a,B^p),\lk(a,B^q))$ is unknotted. \\\pause
  $Q_c = \{c\} \join (\lk(c,B^p),\lk(c,B^q))$ is unknotted. \\\pause
  \vfill
  $Q_i = Q_{i-1} \cup Q_a \cup Q_c$ is union of unknotted ball pairs
  along faces; therefore, $Q_i$ is unknotted. \pause\hfill\qed
\end{frame}


\begin{frame}
  \frametitle{All that remains}
  
  \textcolor{red!50!black}{\textbf{If we could prove}}
  \begin{lemma}
    If $p - q \geq 3$ and $(B^p,B^q)$ is any ball pair, \\
    then $B^p \collapses B^q$.
  \end{lemma}
  \textcolor{red!50!black}{\textbf{we would finish the argument.}}

  \pause\vfill
  \textbf{Note:} we haven't used the fact $p - q \geq 3$ yet.
  \pause\vfill
  \begin{proof}[Proof Technique.]
    Sunny collapse.
  \end{proof}
  \vfill\pause
\end{frame}

% [[skim:///Users/jim/Papers/MR160218.pdf?page=10][page 10 of MR160218.pdf]]

\setbackgroundpicture{stop-sign.jpg}
\begin{frame}
\pause
\vfill
\begin{center}
  Reminder about sunny collapse.
\end{center}
\end{frame}
\clearbackgroundpicture

\setbackgroundpicture{trees-with-shadows.jpg}
\begin{frame}
  \frametitle{Shadows}

  $I^p = I^{p-1} \times I$, \\
  thinking of $I^{p-1}$ as horizontal \\
  and $I$ as vertical.

  \vfill
  $X$ a polyhedron, $X \subset I^p$. \\
  a point $y \in I^p$ lies in the \textbf{shadow} of $X$, if $y$ lies
  below a point of $X$.
\end{frame}

\begin{frame}
  \begin{definition}
    $X$ a polyhedron, $X \subset I^p$. \\
    $X^\ast =$ closure of points in $X$ \\
    which lie on the same vertical line as some other point of $X$.
  \end{definition}
  \vfill\pause
  ($X^\star$ are points that are in shadow or \\
  \quad doing the overshadowing)
  \vfill\pause
  \begin{lemma}
    $X^\star$ is a subpolyhedron of $X$.
  \end{lemma}
\end{frame}

\begin{frame}
  \begin{columns}
    \begin{column}{0.5\textwidth}
  \begin{lemma}
    If $X$ is a polyhedron, $X \subset I^p$, \\
    $\dim X = q \leq p$, and \\
    $\dim (X \cap \partial I^p) < p - 1$, \\
    then there exists
    $h : I^p \stackrel{\cong}{\longrightarrow} I^p$, \\
    $h(X) = X$, \\
    $h(X) \cap \left( I^{p-1} \times \{1\} \right) = \varnothing $, \\
    $h(X) \cap \left( I^{p-1} \times \{0\} \right) = \varnothing $, \\
    $h(X) \cap \mbox{vertical line} = \mbox{finite set}$, and \\
    $\dim h(X)^\star \leq 2q - p + 1$.
  \end{lemma}
  \end{column}
  \begin{column}{0.5\textwidth}
    \pause
    \begin{proof}
      Since $\partial I^p \not\subset X$, \\
      pick vertical
      top-dim face $I^{p-2} \times I$ of $I^p$. \\
      Throw $X
      \cap \partial I^p$ into the interior of this face.

\pause
      \vspace{1ex} Shift vertices into general position.\pause
    \end{proof}
  \end{column}
  \end{columns}
\end{frame}
\clearbackgroundpicture

\setbackgroundpicture{sunny-street.jpg}
\begin{frame}
  \frametitle{Sunny collapse}

  $Y \subset X \subset I^p$. \\
  An elementary collapse $X \collapses Y$ is \textbf{sunny} \\
  if no point of $X - Y$ is in the shadow of $X$.

  \vfill A \textbf{sunny collapse} is a sequence of sunny elementary
  collapses.  \pause Write $X \sunnycollapses Y$.

  \vfill If $X \sunnycollapses \mbox{pt}$, call $X$ sunny collapsible.
\end{frame}
\clearbackgroundpicture

\begin{frame}
\large
\vfill
 \begin{center}
   Sunny collapse in codimension two \\
   can be obstructed
 \end{center}
 \vfill
\end{frame}

% warn them about polyhedral collapse?

\begin{frame}
  \frametitle{All that remains}
  
  \textcolor{red!50!black}{\textbf{If we could prove}}
  \begin{lemma}
    If $p - q \geq 3$ and $(B^p,B^q)$ is any ball pair, \\
    then $B^p \collapses B^q$.
  \end{lemma}
  \textcolor{red!50!black}{\textbf{we would finish the argument.}}

  \pause
  \begin{lemma}
    Suppose $(I^p,X)$ is a $(p,q)$-ball pair,\\
    $p - q \geq 3$ \\
    $X \cap \left( I^{p-1} \times \{1\} \right) = \varnothing $, \\
    $X \cap \left( I^{p-1} \times \{0\} \right) = \varnothing $, \\
    $X \cap \mbox{vertical line} = \mbox{finite set}$, and \\
    $\dim X^\star \leq 2q - p + 1$. \\
    \textbf{Then} $X \sunnycollapses \mbox{point}$.
  \end{lemma}
\end{frame}

\begin{frame}
  \frametitle{Producing a sunny collapse}

  $Y_0 \subset \cdots \subset Y_q$ \\
  $Z_0 \subset \cdots \subset Z_q$ \\
  and $f_i : CY_i \to Z_i$ \\
  so that
  \begin{itemize}
  \item $Y_i$ is $(q-i-1)$-dimensional 
  \item $\dim {Z_i}^\star \leq q-i-2$ \\
  \item ${f_i}^{-1}({Z_i}^\star)$ doesn't meet cone point, \\
    \quad meets cone pieces finitely
  \item ${Z_i} \sunnycollapses {Z_{i-1}}$
  \end{itemize}

\end{frame}

\begin{frame}
  \frametitle{Start the induction}
  
  $Y_0 = \partial \Delta^q$ \hfill $Z_0 = X$

  \vfill

  $\dim X^\star = \dim {Z_0}^\star \leq 2q - p + 1 \leq q-2$.

  \vfill

  Choose homeo $\Delta^q \to X$, \\
  subdivide $\Delta^q$ at a vertex in general position with respect to
  $f^{-1}(X^\star)$ \\
  get $f_0 : C\partial \Delta^q \to Z_0$
\end{frame}

\begin{frame}
  \frametitle{From $i-1$ to $i$}

  $v := \mbox{vertex of $CY_{i-1}$}$.
  $W_{i-1} := {f_{i-1}}^{-1}({Z_{i-1}}^\star)$.

  $p : W_{i-1} \to Y_{i-1}$ projection, not PL, but, after triangulating,
  can be made to send simplexes in $W$ to simplexes in $Y_{i-1}$ of
  the same dimension.

  $Y_i = $ the $(q-i-1)$ simplexes in such a triangulation of $Y_{i-1}$.
\end{frame}

\begin{frame}
  \frametitle{Trouble!}
  
  $CY_i$ is a subcone of $CY_{i-1}$ \\
  but we can't define $f_i$ by restriction \\
  because then $Z_i = f(C_i) \supset f(W_{i-1}) = {Z_{i-1}}^\star$ is the
  wrong dimension.

  \vfill

  Subdivide $W_{i-1}$, and order the simplexes $A_j$ from the ground to the
  sky, i.e., \\
  the points that overshadow $f(\interior A_k)$ are in $\bigcup_{j <
    k} f(\interior A_j)$
\end{frame}

\begin{frame}
  \frametitle{Blisters}

  Add blisters $J_j$ to each $A_j$.

  Define $f_i : CY_i \to Z_{i-1}$, and $Z_i = f_i(CY_i)$, 
  after pushing up the blisters.

  The point is $\dim {Z_i}^\star \leq q-i-2$.
\end{frame}

\begin{frame}
  \frametitle{The sunny collapse}

  $J = \bigcup \mbox{blisters}$
  \vfill

  $C_{i-1} \collapses CY_i \cup J$, which under $f$\\
  $Z_{i-1} \collapses Z_i \cup fJ$ is sunny.

  \vfill
  But we have ordered these!
\end{frame}

\begin{frame}
  \frametitle{Producing a sunny collapse}

  $Y_0 \subset \cdots \subset Y_q$ \\
  $Z_0 \subset \cdots \subset Z_q$ \\
  and $f_i : CY_i \to Z_i$ \\
  so that
  \begin{itemize}
  \item $Y_i$ is $(q-i-1)$-dimensional 
  \item $\dim {Z_i}^\star \leq q-i-2$ \\
  \item ${f_i}^{-1}({Z_i}^\star)$ doesn't meet cone point, \\
    \quad meets cone pieces finitely
  \item ${Z_i} \sunnycollapses {Z_{i-1}}$
  \end{itemize}

\end{frame}


%%%%%%%%%%%%%%%%%%%%%%%%%%%%%%%%%%%%%%%%%%%%%%%%%%%%%%%%%%%%%%%%

\setbackgroundpicture{puzzle.jpg}
\begin{frame}
\large
\vfill
\begin{center}
  Putting together all the pieces again.
\end{center}
\vfill  
\end{frame}
\clearbackgroundpicture

\begin{frame}
  If $X \subset I^p$, define $X^\sharp$ to be all points in $I^p$
  lying in shadow of $X$.

  \vfill
  \pause
  $M = \left(I^{p-1} \times \{0\}\right) \cup X^\sharp$.

\vfill
\pause
First show $I^p \collapses M$.
\end{frame}

\begin{frame}
 \includegraphics[width=\textwidth]{sunny-step-one-bad.pdf}
\end{frame}

\begin{frame}
  If $X \subset I^p$ and $X \sunnycollapses \mbox{point}$, then
$$
X = X_0 \sunnycollapses X_1 \sunnycollapses \cdots \sunnycollapses K_n
= \mbox{point}.
$$\pause
Define $M_i = \left(I^{p-1} \times \{0\}\right) \cup X \cup
{X_i}^\sharp$.

\vfill
\pause
\textbf{Goal: } $I^p \collapses M_0 \collapses \cdots \collapses M_n
\collapses X$.
\end{frame}

\begin{frame}
 \includegraphics[width=\textwidth]{sunny-step-two-bad.pdf}
\end{frame}


\end{document}



