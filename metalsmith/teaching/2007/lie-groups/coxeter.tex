\documentclass[12pt]{article}
\usepackage{fullpage}
\usepackage{amsmath}
\usepackage{amsthm}
\usepackage{add-copyright}

%%%%%%%%%%%%%%%%%%%%%%%%%%%%%%%%%%%%%%%%%%%%%%%%%%%%%%%%%%%%%%%%
% theorem styles
\newtheorem{thm}{Theorem}[section]
\newtheorem{lemma}[thm]{Lemma}
\newtheorem{proposition}[thm]{Proposition}
\newtheorem{corollary}[thm]{Corollary}

\theoremstyle{remark}
\newtheorem{remark}[thm]{Remark}

\theoremstyle{definition}
\newtheorem{defn}[thm]{Definition}
\newtheorem{exercise}[thm]{Exercise}

\newcommand{\defnword}[1]{\textbf{#1}}

%%%%%%%%%%%%%%%%%%%%%%%%%%%%%%%%%%%%%%%%%%%%%%%%%%%%%%%%%%%%%%%%
% number systems
\usepackage{bbm}
\newcommand{\C}{\mathbbm{C}}
\newcommand{\R}{\mathbbm{R}}
\newcommand{\Q}{\mathbbm{Q}}
\newcommand{\Z}{\mathbbm{Z}}

%%%%%%%%%%%%%%%%%%%%%%%%%%%%%%%%%%%%%%%%%%%%%%%%%%%%%%%%%%%%%%%%
% title
\title{Root Systems and Coxeter Groups}
\author{Undergraduate Lie Theory Seminar}
\date{}

%%%%%%%%%%%%%%%%%%%%%%%%%%%%%%%%%%%%%%%%%%%%%%%%%%%%%%%%%%%%%%%%
\begin{document}

\maketitle

%%%%%%%%%%%%%%%%%%%%%%%%%%%%%%%%%%%%%%%%%%%%%%%%%%%%%%%%%%%%%%%%
% introductory quotation
\begin{quote}
The belief that all simple (having no continuous moduli) objects in 
the nature are controlled by the Coxeter groups is a kind of religion. \\
\null\hfill---V.I. Arnold, \textit{First steps of local contact algebra.}
\end{quote}

%I obviously make no claim whatsoever to the originality of this document.

%%%%%%%%%%%%%%%%%%%%%%%%%%%%%%%%%%%%%%%%%%%%%%%%%%%%%%%%%%%%%%%%
\section{Root systems}

Unfortunately, I again define root system without much motivation.
Nonetheless, I hope you will be surprised to discover how restrictive
the definition of root system is---there are four infinite families,
and five ``exceptional'' root systems that just don't belong.

\begin{defn}[Hyperplanes and reflections]
Let $V$ be a vector space with inner product, and $v \in V$ a vector.
Then the \defnword{hyperplane} normal to $v$ is
$$
H_v := \{ w \in V : \langle w, v \rangle = 0 \}.
$$
The \defnword{reflection} through the hyperplane $H_v$ is the linear map $s_v : V \to V$ defined by
$$
s_v(w) = w - 2 \frac{\langle v, w \rangle}{\langle v, v \rangle} v.
$$
\end{defn}

\begin{exercise}
Check that ${s_v}^2$ is the identity.
\end{exercise}

\begin{defn}[Root system.]
Let $V$ be a vector space.  A \defnword{root system} is subset $\Phi
\subset V$ satisfying:
\begin{itemize}
\item The set $\Phi$ is finite.
\item The span of $\Phi$ is $V$.
\item If $v \in \Omega$, then $s_v(\Phi) = \Phi$.
\item If $v \in \Phi$, then $\Phi \cap \R v = \{ v, -v \}$.
\end{itemize}
The elements of a root system are called \defnword{roots}.
\end{defn}

\begin{defn}[Isomorphism]
Two root
systems $\Phi \subset V$ and $\Phi' \subset V'$ are
\defnword{isomoprhic} if there is an isometry $f : V \to V'$ with
$f(\Phi) = \Phi'$.
\end{defn}

\begin{defn}[Direct sums]
Let $\Phi$ and $\Phi'$ be root systems in $V$ and $V'$, respectively.
Then $\Phi \oplus \Phi'$ is a root system in the vector space $V
\oplus V'$.  A root system $\Phi$ is \defnword{reducible} if it can be
written as such an (orthogonal) direct sum, and \defnword{irreducible}
otherwise.
\end{defn}

\begin{exercise}
For a vector space $V$ with orthogonal basis $\{ e_i \}$, the set $\{
\pm e_i \}$ is a root system.  This is a reducible root system.
\end{exercise}

\begin{defn}[Crystallographic root system.]
A root system $\Phi$ is \defnword{crystallographic} if, for all $v,
w \in \Phi$, we have
$$
2 \frac{\langle v, w \rangle}{\langle v, v \rangle} \in \Z.
$$
\end{defn}

\begin{exercise}
Play around with this crystallographic condition; what are the
possible angles between $v, w \in \Phi$ when $\Phi$ is
crystallographic?
\end{exercise}

\begin{remark}
A Coxeter group does not in general have a crystallographic root
system; but the groups that we will be working with in the
classification of Lie groups are all crystallographic, so we might as
well simplify our work by considering only the special sort of Coxeter
group arising here.
\end{remark}

\begin{exercise}
Find four non-isomorphic crystallographic root systems in $\R^2$.
\end{exercise}

%%%%%%%%%%%%%%%%%%%%%%%%%%%%%%%%%%%%%%%%%%%%%%%%%%%%%%%%%%%%%%%%
\section{Weyl Groups}

\begin{defn}[Weyl group.]
Let $\Phi \subset V$ be a root system.  Then the \defnword{Weyl
group} of $\Phi$ is the group generated by $s_v : V \to V$ for all
$v \in \Phi$.
\end{defn}

\begin{exercise}
Is the Weyl group well-defined (i.e., do isomorphic root systems give
isomorphic Weyl groups?).
\end{exercise}

\begin{exercise}
Is the Weyl group finite?
\end{exercise}

\begin{exercise}
What are the Weyl groups of the four crystallographic root systems in $\R^2$?
\end{exercise}

\begin{exercise}
Find a root system having the symmetric group on four letters, $S_4$,
as its Weyl group.
\end{exercise}

%%%%%%%%%%%%%%%%%%%%%%%%%%%%%%%%%%%%%%%%%%%%%%%%%%%%%%%%%%%%%%%%
\section{Positive systems}

\begin{defn}[Positive system]
Let $\Phi \subset V$ be a root system, and choose $w \in V$.  Define
$$
\Phi^{+}(w) := \{ v \in \Phi : \langle v, w \rangle > 0 \}.
$$
We say that $w$ is \defnword{regular} if $\Phi = \pm \Phi^{+}(w)$,
and \defnword{singular} otherwise.  If $w$ is regular, we call
$\Phi^{+}(w)$ a \defnword{positive system} for $\Phi$.
\end{defn}
This is only one way to define a positive system (it makes clearer the
relationship between positive systems and Weyl chambers, defined
later).  Alternatively, we can say that a positive system is any
subset $\Phi^{+} \subset \Phi$ so that $\Phi$ is the disjoint union of
$\Phi^{+}$ and $-\Phi^{+}$, and so that $\Phi^{+}$ is closed under
sums.

\begin{exercise}
Give examples of positive systems for some  root systems in $\R^2$.
\end{exercise}

\begin{defn}[Decomposing roots]
Let $\Phi^{+}$ be a positive system for a root system $\Phi$.  Then $v \in \Phi^{+}$
is \defnword{decomposable} if it is the sum of two other roots in
$\Phi^{+}$, and \defnword{indecomposable} otherwise.
\end{defn}

%%%%%%%%%%%%%%%%%%%%%%%%%%%%%%%%%%%%%%%%%%%%%%%%%%%%%%%%%%%%%%%%
\section{Simple systems}

\begin{defn}[Positive cone]
Given vectors $v_1, \ldots, v_n \in V$, the \defnword{positive cone}
of $\{ v_1, \ldots, v_n \}$ are the nonnegative linear combinations of the $v_i$.
\end{defn}

\begin{defn}[Simple systems]
Let $\Phi^{+}$ be a positive system for a root system $\Phi$.  A
\defnword{simple system} $\Pi$ for $\Phi^{+}$ is a minimal subset of
$\Phi^{+}$ generating the same positive cone as $\Phi^{+}$.  The
elements of a simple system are called \defnword{simple roots}.
\end{defn}
This definition of simple system depends on a choice of positive
system, but we can also define simple systems intrinsically---a simple
system $\Pi$ is a subset of $\Phi$ spanning $V$, so that every root is
a non-positive or a non-negative linear combination of roots in $\Pi$.

\begin{exercise}
Let $\Phi^{+}$ be a positive system for a root system $\Phi$.  Show
that the indecomposable roots form a simple system.
\end{exercise}

\begin{exercise}
Consider a simple system $\Pi$ for a root system $\Phi \subset V$.
Then the span of $\Pi$ is $V$, so maps between simple systems
determine maps between vector spaces (and it will therefore be enough
to classify simple systems).
\end{exercise}

\begin{defn}[Weyl chambers]
Let $\Phi \subset V$ be a root system.  The connected components of $V
- \bigcup_{v \in \Phi} H_v$ are the \defnword{Weyl chambers} of
$\Phi$.
\end{defn}

\begin{exercise}
There is a bijection between Weyl chambers of $\Phi$ and positive
systems, and between positive systems and simple systems.  In fact,
the Weyl group acts transitively on positive systems and simple
systems.
\end{exercise}

There's a lot more to say about the action of the Weyl group on the
positive systems, but that is for another day.

%%%%%%%%%%%%%%%%%%%%%%%%%%%%%%%%%%%%%%%%%%%%%%%%%%%%%%%%%%%%%%%%
\section{Reducibility}



\begin{exercise}
Give examples of reducible and irreducible root systems.
\end{exercise}

\begin{exercise}
A simple system $\Pi$ for a root system $\Phi$ is reducible if and
only if $\Phi$ is reducible.
\end{exercise}

%%%%%%%%%%%%%%%%%%%%%%%%%%%%%%%%%%%%%%%%%%%%%%%%%%%%%%%%%%%%%%%%
\section{Cartan matrix}

\begin{defn}[Cartan matrix]
Given a simple system $\Pi = \{a_i\}$ for $\Phi$, we define
the \defnword{Cartan matrix} $A = (A_{ij})$ by
$$
A_{ij} = 2 \frac{\langle a_i, a_j \rangle}{\langle a_i, a_i \rangle}.
$$
\end{defn}

\begin{exercise}
What are the four possible values of $A_{ij}$ in the special case of a
crystallographic root system?
\end{exercise}

\begin{exercise}
What are the Cartan matrices for the root systems in $\R^2$?
\end{exercise}

\begin{exercise}
The Cartan matrix determines $\Phi$ up to isomorphism.
\end{exercise}

\begin{exercise}
What does the Cartan matrix for a reducible root system look like?
\end{exercise}


%%%%%%%%%%%%%%%%%%%%%%%%%%%%%%%%%%%%%%%%%%%%%%%%%%%%%%%%%%%%%%%%
\section{Coxeter graph}

\begin{defn}[Coxeter graph]
Given a Cartan matrix $A$ for a simple system $\Pi = \{ a_i \}$, its
associated \defnword{Coxeter graph} has a vertex for each simple root,
and a $A_{ij}A_{ji}$ edges between roots $a_i$ and $a_j$.
\end{defn}

\begin{exercise}
Draw the Coxeter graphs for your favorite root systems.  Does the
Coxeter graph determine the root system?
\end{exercise}

\begin{defn}[Dynkin Diagram]
Given a Cartan matrix $A$ for a simple system $\Pi = \{ a_i \}$, its
associated \defnword{Dynkin diagram} is its Coxeter graph, decorated
with an arrow on double and triple edges, drawn from $a_i$ to $a_j$
whenever $a_i$ is longer than $a_j$.
\end{defn}

\begin{exercise}
What does the Coxeter graph for a reducible root system look like?
Does the Dynkin diagram determine the root system?  \textit{Hint:}
does the Dynkin diagram determine the Cartan matrix?
\end{exercise}

%%%%%%%%%%%%%%%%%%%%%%%%%%%%%%%%%%%%%%%%%%%%%%%%%%%%%%%%%%%%%%%%
\section{Summarizing the situation}

From root systems, we produced positive systems, which gave rise to
simple systems, which allowed us to define the Cartan matrix, and
finally, the Coxeter graph---which doesn't quite determine the root
system---and the Dynkin diagram---which does determine the root
system.

Consider a finite undirected graph $\Gamma = (V,E)$.  Let $V = \{ v_1,
\ldots, v_n \}$, and define $c_{ij}$ to be the number of edges
connecting $v_i$ and $v_j$.  On the real vector space spanned by the
vertices of $V$, define a bilinear form (which we call the
\defnword{adjacency form}) by $b(v_i,v_j) = -\sqrt{c_{ij}}$.

\begin{defn}[Positive graphs]
A graph $\Gamma = (V,E)$ is \defnword{positive} if the adjacency form $b$ is positive definite.
\end{defn}

\begin{exercise}
A Coxeter graph is a positive graph.
\end{exercise}

%%%%%%%%%%%%%%%%%%%%%%%%%%%%%%%%%%%%%%%%%%%%%%%%%%%%%%%%%%%%%%%%
\section{Sketch: Classifying positive graphs}

\begin{defn}[Subgraph]
Given a graph $\Gamma = (V,E)$, a graph $\Gamma' = (V',E')$ is a
\defnword{subgraph} of $\Gamma$, written $\Gamma' \subset \Gamma$, if
$V' \subset V$ and $E' \subset E$.
\end{defn}

\begin{exercise}
A subgraph of a positive graph is positive.
\end{exercise}

\begin{exercise}
Check that the Coxeter graphs $A_i$, $B_i$, $D_i$, $E_6$, $E_7$,
$E_8$, $F_4$, and $G_2$ are all positive graphs.
\end{exercise}

\begin{exercise}
Find a family $P$ of ``poisonous'' graphs so that
\begin{itemize}
\item No $\Gamma \in P$ is positive.
\item If $\Gamma$ is any non-positive connected graph, then there
exists $\Gamma' \subset \Gamma$ with $\Gamma' \in P$.
\end{itemize}
\end{exercise}

Having finished this exercise, you will have shown that every positive
graph is one of the Coxeter graphs we know about.

\begin{exercise}
Decorate your Coxeter graphs to find the possible Dynkin diagrams.
\end{exercise}

\begin{exercise}
Find root systems having each of the possible Dynkin diagrams.
\end{exercise}

\end{document}
