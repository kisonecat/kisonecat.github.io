\documentclass[12pt]{article}
\usepackage{rotating}
\usepackage{worksheet}
\usepackage{xypic}
\title{Symmetries of a square}

\newcommand{\AN}{\mathbf{N}}
\newcommand{\AX}{\mathbf{X}}
\newcommand{\AY}{\mathbf{Y}}
\newcommand{\AR}{\mathbf{R}}

\newcommand{\squareN}{\framebox{\raisebox{0in}[0.3in][0.3in]{\parbox{0.6in}{\hfill\textsf{FRONT}\hfill\null\vspace{1ex}\\\null\hfill{\Huge$\uparrow$}\hfill\null}}}}

\newcommand{\squareF}{\framebox{\raisebox{0in}[0.3in][0.3in]{\parbox{0.6in}{\hfill\textsf{BACK}\hfill\null\vspace{1ex}\\\null\hfill{\Huge$\uparrow$}\hfill\null}}}}

\newcommand{\squareRR}{\begin{turn}{180}\framebox{\raisebox{0in}[0.3in][0.3in]{\parbox{0.6in}{\hfill\textsf{FRONT}\hfill\null\vspace{1ex}\\\null\hfill{\Huge$\uparrow$}\hfill\null}}}\end{turn}}

\newcommand{\squareFRR}{\begin{turn}{180}\framebox{\raisebox{0in}[0.3in][0.3in]{\parbox{0.6in}{\hfill\textsf{BACK}\hfill\null\vspace{1ex}\\\null\hfill{\Huge$\uparrow$}\hfill\null}}}\end{turn}}

\newcommand{\squareRRR}{\raisebox{-3.5pt}{\raisebox{-0.3in}{\begin{turn}{90}\framebox{\raisebox{0in}[0.3in][0.3in]{\parbox{0.6in}{\hfill\textsf{FRONT}\hfill\null\vspace{1ex}\\\null\hfill{\Huge$\uparrow$}\hfill\null}}}\end{turn}}}}

\newcommand{\squareFRRR}{\raisebox{-3.5pt}{\raisebox{-0.3in}{\begin{turn}{90}\framebox{\raisebox{0in}[0.3in][0.3in]{\parbox{0.6in}{\hfill\textsf{BACK}\hfill\null\vspace{1ex}\\\null\hfill{\Huge$\uparrow$}\hfill\null}}}\end{turn}}}}

\newcommand{\squareR}{\raisebox{3.5pt}{\raisebox{0.3in}{\begin{turn}{270}\framebox{\raisebox{0in}[0.3in][0.3in]{\parbox{0.6in}{\hfill\textsf{FRONT}\hfill\null\vspace{1ex}\\\null\hfill{\Huge$\uparrow$}\hfill\null}}}\end{turn}}}}

\newcommand{\squareFR}{\raisebox{3.5pt}{\raisebox{0.3in}{\begin{turn}{270}\framebox{\raisebox{0in}[0.3in][0.3in]{\parbox{0.6in}{\hfill\textsf{BACK}\hfill\null\vspace{1ex}\\\null\hfill{\Huge$\uparrow$}\hfill\null}}}\end{turn}}}}


\begin{document}

%%%%%%%%%%%%%%%%%%%%%%%%%%%%%%%%%%%%%%%%%%%%%%%%%%%%%%%%%%%%%%%%
\section*{Symmetries of a square.}

\section*{Basic operations.}
\begin{description}
\item[Rotation,] written $\AR$, which rotates clockwise by $90^\circ$,\hfill turning \squareN\ into \squareR
\item[X-Flipping,] written $\AX$, which flips across a horizontal line,\hfill turning \squareN\ into \squareFRR
\item[Y-Flipping,] written $\AY$, which flips across a vertical line,\hfill turning \squareN\ into \squareF
\item[Nothing,] written $\AN$, which does nothing,\hfill turning \squareN\ into \squareN
\end{description}

\vfill

\section*{Useful facts.}
\begin{description}
\item[Four rotations] \hfill $\squareN \stackrel{\AR}{\mapsto} \squareR \stackrel{\AR}{\mapsto} \squareRR \stackrel{\AR}{\mapsto} \squareRRR \stackrel{\AR}{\mapsto} \squareN$ \hfill $\AR\AR\AR\AR = \AN$
\item[Two X-flips] \hfill $\squareN \stackrel{\AX}{\mapsto} \squareFRR \stackrel{\AX}{\mapsto} \squareN$ \hfill $\AX\AX = \AN$
\item [Two Y-flips] \hfill $\squareN \stackrel{\AY}{\mapsto} \squareF \stackrel{\AY}{\mapsto} \squareN$ \hfill $\AY\AY = \AN$
\item [X's and Y's] $\squareN \stackrel{\AX}{\mapsto} \squareFRR \stackrel{\AY}{\mapsto} \squareRR \stackrel{\AX}{\mapsto} \squareF \stackrel{\AY}{\mapsto} \squareN$ \hfill $\AX\AY\AX\AY = \AN$
%\item [X's and R's] $\squareN \stackrel{\AX}{\mapsto} \squareFRR \stackrel{\AR}{\mapsto} \squareFRRR \stackrel{\AX}{\mapsto} \squareRRR \stackrel{\AR}{\mapsto} \squareN$ \hfill $YRYR = N$
\item [Y's and R's] $\squareN \stackrel{\AY}{\mapsto} \squareF \stackrel{\AR}{\mapsto} \squareFR \stackrel{\AY}{\mapsto} \squareRRR \stackrel{\AR}{\mapsto} \squareN$ \hfill $\AY\AR\AY\AR = \AN$

\end{description}

\vfill

\end{document}
