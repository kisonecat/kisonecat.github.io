\documentclass[12pt]{article}
\usepackage{worksheet}
\title{Sums of Squares}

\begin{document}

\section*{Sums of squares.}

\subsection*{The Problem.}

A \textbf{square number} is a whole number multiplied by itself.  For
instance, nine is square because $9 = 3 \times 3$.  Here is a list of
some square numbers:
$$
0,\hspace{1em} 1,\hspace{1em} 4 ,\hspace{1em}9,\hspace{1em} 16,\hspace{1em} 25,\hspace{1em} 36,\hspace{1em} 49,\hspace{1em} \ldots
$$
It is a surprising fact that \textbf{every whole number is the sum of
  four squares.}  For example:
\begin{eqnarray*}
10 &=& 4 + 4 + 1 + 1 \\
11 &=& 9 + 1 + 1 + 0 \\
12 &=& 9 + 1 + 1 + 1 \\
13 &=& 9 + 4 + 0 + 0 \\
14 &=& 9 + 4 + 1 + 0
\end{eqnarray*}
Write the numbers 15, 16, 17, 18, 19, 20, 21, and 22 as the sum of
four squares.  Can every number be written as the sum of \textbf{three}
squares?

\subsection*{Your solution.}

\end{document}
