\documentclass[12pt]{article}
\usepackage{fullpage}
\usepackage{nopageno}
\usepackage{ifthen}
\usepackage{amsmath}
\usepackage{amssymb}
\usepackage{graphicx} 
\usepackage{version}
\usepackage{amsthm}
\usepackage{multicol}
\usepackage{add-copyright}

\excludeversion{solution}

\DeclareMathOperator{\ft}{ft}

\newcommand{\R}{\mathbb{R}}

\title{Take-Home Quiz 6}
\author{Math 133 Section 22}
\date{Due Wednesday, May 31}

\newcounter{problem}
\setcounter{problem}{1}

\newenvironment{problem}[1][]
{\begin{flushleft}\hangindent=1em\hangafter=1\noindent\textbf{Problem \arabic{problem}.}
\ifthenelse{\equal{#1}{}}{}{
\textbf{(#1 \ifthenelse{\equal{#1}{1}}{point}{points}).}}
}
{\addtocounter{problem}{1}\end{flushleft}}

\begin{document}
\maketitle

\begin{problem}[5]
  In this problem, we will develop an explicit formula for the sequence
$$
a_0 = 1, \hspace{1em} a_1 = 1, \hspace{1em} a_{n+2} = a_{n+1} + a_{n},
$$
namely, the \textbf{Fibonacci sequence}.  We work in a few steps:
\begin{description}
\item[Step 1.]  Suppose $f(x) = \displaystyle\sum_{n=0}^\infty a_n x^n$.  Explain why
$f(x) = \displaystyle\frac{1}{1 - x - x^2}$.
\item[Step 2.]  Define
$\phi_1 = \displaystyle\frac{1 + \sqrt{5}}{2}, \displaystyle\phi_2 = \frac{1 - \sqrt{5}}{2}$.
The number $\phi_1$ is the \textbf{golden ratio}.  Prove
$$
\frac{1}{1 - x - x^2} = \frac{1}{x \sqrt{5}} \cdot \left( \frac{x \phi_1}{1 - x \phi_1} - \frac{x \phi_2}{1 - x \phi_2} \right).
$$
\item[Step 3.] Use the fact that $\displaystyle\frac{x \phi_1}{1 - x \phi_i} = \displaystyle\sum_{n=1}^\infty {\phi_i}^n x^n$ to show
$$
\frac{1}{1 - x - x^2} = \frac{1}{x \sqrt{5}} \cdot \left( \sum_{n=1}^\infty {\phi_1}^n x^n - \sum_{n=1}^\infty {\phi_2}^n x^n \right).
$$
Rearrange to prove
$$
\sum_{n=0}^\infty a_n x^n = \frac{1}{1 - x - x^2} = \frac{1}{\sqrt{5}} \sum_{n=0}^\infty \left( {\phi_1}^{n+1} - {\phi_2}^{n+1} \right) x^n.
$$
Equate corresponding coefficients to conclude
$$
a_n = \frac{1}{\sqrt{5}} \left( {\phi_1}^{n+1} - {\phi_2}^{n+1} \right).
$$
\item[Step 4.] Use the formula and a calculator to compute $a_{15}$.
\item[Step 5.] For three extra credit points, compute $\displaystyle\lim_{n \to \infty} \displaystyle\frac{a_{n+1}}{a_n}$ using the formula.
\end{description}
\end{problem}

\begin{problem}[3]
  Find a Taylor series expansion for $f(x) = x^2 + 3x + 1$ around the
  point $a = 2$.
\end{problem}

\begin{problem}[3]
  Find the first three terms of the Taylor series expansion for $f(x)
  = \sqrt[3]{x}$ around the point $a = 1$.  Use this to approximate
  $\sqrt[3]{1.1}$.
\end{problem}

\begin{problem}[2]
  Use the first three nonzero terms of the Maclaurin series for $\sin x$
  to approximate $\sin 1$.
\end{problem}

\begin{problem}[2]
  Use the Maclaurin series expansion for $f(x) = e^{x^2}$ to compute
  $f^{(100)}(0)$.
\end{problem}



\end{document}
