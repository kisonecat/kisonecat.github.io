\documentclass[12pt,letterpaper]{article}

\title{Syllabus}
\author{Jim Fowler}

\usepackage{multicol}
\usepackage{fullpage}
\pagestyle{empty}

\usepackage{html}

%\usepackage[T1]{fontenc}
%\usepackage{textcomp}
%\usepackage{lmodern}
%\newcommand{\ditto}{\textquotedbl}

%\setlength{\parindent}{0pt}

\newcommand{\peem}{\textsc{p.m.}}
\newcommand{\ayem}{\textsc{a.m.}}

\begin{document}

%%%%%%%%%%%%%%%%%%%%%%%%%%%%%%%%%%%%%%%%%%%%%%%%%%%%%%%%%%%%%%%%
\section*{\Large\sf Syllabus\hfill
Math 133, Section 22\hfill
Spring 2006}

Welcome to Math 133, the third quarter of a year-long Calculus
sequence.  This course emphasizes three topics: transcendental
functions, integration techniques, and infinite series.

\begin{htmlonly}
You can also download this syllabus as a \htmladdnormallink{PDF
file}{http://math.uchicago.edu/~fowler/math133/syllabus.pdf}.
\end{htmlonly}

%%%%%%%%%%%%%%%%%%%%%%%%%%%%%%%%%%%%%%%%%%%%%%%%%%%%%%%%%%%%%%%%
\section*{Resources}

We present five resources to help you to learn Calculus.

\subsection*{Office Hours}
If you have questions, want to work through problems, or just talk
about mathematics, I invite you to come to my office hours.
\begin{multicols}{2}
\begin{tabular}{ll}
\textbf{Name:} & Jim Fowler \\
\textbf{Office:} & Math/Stat 102 \\
\textbf{Phone:} & 773--573--5659 \\
\textbf{Email:} & \texttt{jim@uchicago.edu}
\end{tabular}

\begin{tabular}{ll}
\textbf{Office Hours:} & Tuesday 4:00--6:00\peem \\
& Thursday 5:00--7:00\peem \\
& or by appointment
\end{tabular}
\end{multicols}
\noindent
Please email me with any concerns you have; the success of this course
depends on open communication.

\subsection*{Textbook}
Our text is \textit{Calculus} by Verberg, Purcell, and Bigdon (8th
Edition).

\subsection*{Website}
The website is on chalk; we will post homework assignments and take-home quizzes.

\subsection*{Lectures}
We meet Mondays, Wednesdays, and Fridays, 9:30--10:20\ayem\ in Ryerson
358 for an interactive lecture.

\subsection*{Problem Session}
You meet with your tutor on Tuesdays and Thursdays, 9:00--10:20\ayem\
for an interactive problem session.  The tutors, in
alphabetical order, are:
\begin{itemize}
\setlength{\itemsep}{-1ex}
\item Danny Rosenthal (\texttt{danny@uchicago.edu}) in Wiebolt 111,
\item Sara Rezvi (\texttt{arsinoe@uchicago.edu}) in Harper 125,
\item Steve Balady (\texttt{sbalady@uchicago.edu}) in Wiebolt 102.
\end{itemize}
If for any reason you want to switch to a different problem session,
contact Jim Fowler.
\pagebreak

%%%%%%%%%%%%%%%%%%%%%%%%%%%%%%%%%%%%%%%%%%%%%%%%%%%%%%%%%%%%%%%%
\section*{Requirements}

%% 26  27 28 29 30 31t  1  week 1
%%  2  3  4  5  6  7t  8  week 2
%%  9 10 11 12 13 14t 15  week 3
%% 16 17 18 19 20 21m 22  week 4
%% 23 24 25 26 27 28t 29  week 5
%% 30  1  2  3  4  5t  6  week 6
%%  7  8  9 10 11 12t 13  week 7
%% 14 15 16 17 18 19m 20  week 8
%% 21 22 23 24 25 26t 27  week 9
%% 28 *  30 31  *  *   3  week 10
%%  4  5  6  7  8  9  10  finals

There are one thousand points possible in this course, broken down as follows:
\begin{description}
\item[Daily homework (95 points).]  Homework is assigned Mondays and
Wednesdays, and due the following class period.  You should work on
the homework problems together, but you must write up your solutions
independently.
\item[2 midterms (200 points each).]  The midterms are in class.  The
first midterm is on Friday, April 21, at the end of week 4; the second
midterm is on Friday, May 19, at the end of week 8.
\item[7 quizzes (15 points each).]  Short take-home quizzes are
assigned Fridays without an in-class exam, and due the following
Monday.
\item[1 final exam (400 points).]  The final exam will be held
8:00--10:00\ayem\ on Monday, June 5, 2006.
\end{description}

%%%%%%%%%%%%%%%%%%%%%%%%%%%%%%%%%%%%%%%%%%%%%%%%%%%%%%%%%%%%%%%%
\subsection*{Department Policy on Final Exams}

\textit{It is the policy of the Department of Mathematics that the
following rules apply to final exams in all undergraduate mathematics
courses:}
\begin{enumerate}
\item \textit{The final exam must occur at the time and place designated on
the College Final Exam Schedule.}  In particular, \textit{no} final examinations
may be given during the tenth week of the quarter, except in the case
of graduating seniors.
\item Any student who wishes to depart from the scheduled final exam
time for the course must receive permission from Paul Sally (office is
Ryerson 350, phone is 2-7388, email is
\texttt{sally@math.uchicago.edu}).  Instructors are not permitted to
excuse students from the scheduled time of the final exam except in
the cases of an Incomplete.
\end{enumerate}

%%%%%%%%%%%%%%%%%%%%%%%%%%%%%%%%%%%%%%%%%%%%%%%%%%%%%%%%%%%%%%%%
\subsection*{Rescheduling a Midterm}

Contact Jim Fowler early in the quarter if you will not be able to
take a midterm on the scheduled day; we likely can accomodate you, but
we will need time to prepare.

%%%%%%%%%%%%%%%%%%%%%%%%%%%%%%%%%%%%%%%%%%%%%%%%%%%%%%%%%%%%%%%%
\subsection*{Late Quizzes and Homework}

You must stay caught up.  It is tempting to fall behind, but difficult
to catch up again---this is true of all courses, but especially true
of a course in mathematics.  That said, I understand your schedules
are busy, so I will not penalize you for \textit{infrequently} turning
in work \textit{a day or two late.}  Do not make a habit of it!

\end{document}
