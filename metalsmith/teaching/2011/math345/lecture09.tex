\documentclass[12pt]{handout}

\title{Lecture 9: Number theory}
\author{Jim Fowler}
\course{Math 345}
\date{Wednesday, October  6, 2010}

\usepackage{hyperref}

\begin{document}
\maketitle

\section*{Textbook}

This lecture discusses section 4 of the textbook.

\section*{Homework} 

The homework is due Monday, October 11, 2010.

From Section 4 of the textbook, do exercises 1 and 2.

\section*{number theory}

number theory: study the properties of whole numbers

e.g., prime numbers

\section*{easy-to-state questions remain open}

\subsection*{perfect numbers}

$6 = 1 + 2 + 3$

$28 = 1 + 2 + 4 + 7 + 14$

$496 = 1+ 2+ 4+ 8+ 16+ 31+ 62+ 124+ 248$

$8128$ is perfect, too.  47 such numbers are known today.

infinitely many perfect numbers?  unknown.

any odd perfect numbers?  unknown.  it is known that such a number
must have more than 300 digits.

Descartes writes in 1638 \footnote{\url{http://www-history.mcs.st-andrews.ac.uk/HistTopics/Perfect_numbers.html}}
\begin{quote}
I think I am able to prove that there are no even numbers which are perfect apart from those of Euclid; and that there are no odd perfect numbers, unless they are composed of a single prime number, multiplied by a square whose root is composed of several other prime number. But I can see nothing which would prevent one from finding numbers of this sort.
\end{quote}

\subsection*{collatz conjecture}

unknown

\section*{definitions for today}

$x$ is even if there exists $k \in \mathbb{Z}$ so that $x = 2k$.

$x$ is odd if there exists $k \in \mathbb{Z}$ so that $x = 2k+1$.

\section*{example: odd plus odd is even}

what do we need to use?  distributivity.

\section*{properties you are allowed to use}

two operations ($+$ and $\cdot$) both associative, commutative, with
identities.  distributivity.

\section*{can you prove that an integer is either even or odd?}

no?  why not?

a deeper reason: there are mathematical systems which satisfy all the
properties we can use, but for which ``even'' and ``odd'' don't make
sense.


\end{document}
