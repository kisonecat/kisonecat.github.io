\documentclass[12pt]{midterm}

\title{Midterm 1 Solution Set}
\course{Math 345}
\date{October 2010}

\newcommand{\abs}[1]{\left|#1\right|}
\DeclareMathOperator{\realpart}{Re}
\DeclareMathOperator{\imagpart}{Im}

\settowidth{\introductionwidth}{\widthof{We often hear that mathematics consists mainly of ``proving theorems.''}}
\introduction{%
We often hear that mathematics consists mainly of ``proving theorems.'' \\
Is a writer's job mainly that of ``writing sentences?'' \\
\null\hfill---Gian-Carlo Rota
}

\instructions{%
\begin{enumerate}
\item Write your name above.
\item Calculators are forbidden (and useless, anyhow).
\item Do not look inside the exam until instructed to do so.
\item You have \textbf{48 minutes} for this exam.
\item Justify your answers for full credit.
\item Show your work for generous partial credit.
\item Write your answers on the included pages, or request additional paper.
\item Answer all questions asked.
\item To prevent fire, do not divide by zero.
\vfill
\end{enumerate}
}

\newtheorem*{claim}{Claim}

\usepackage{xcolor}

%\usepackage{graphicx}
%\usepackage{eso-pic}

%\newcommand\BackgroundPicture[1]{%
%  \setlength{\unitlength}{1pt}%
%  \put(0,792){%
%  \parbox[t][\paperheight]{\paperwidth}{%
%     \vfill
%     \centering\includegraphics{#1}
%     \vfill
%}}} %
%\AddToShipoutPicture{\BackgroundPicture{barcode.pdf}}

\begin{document}
\begin{exam}

%%%%%%%%%%%%%%%%%%%%%%%%%%%%%%%%%%%%%%%%%%%%%%%%%%%%%%%%%%%%%%%%
% this comes straight from the fake exam
\begin{problem}[360]
  Consider the proposition:
  $$
  \left(Q \Rightarrow R\right) \Rightarrow \left( (P \Rightarrow Q) \Rightarrow (P \Rightarrow R) \right)
  $$
  Is the proposition a tautology?  If not, explain why not.  If it is
  a tautology, use the method of conditional proof to prove that it is
  a tautology.  Do not use cases, and be careful not to skip any
  steps.
\end{problem}

\begin{solution}\begin{solutiontext}

\begin{claim}
  The given proposition is a tautology.
\end{claim}
\begin{proof}

Assume $Q \Rightarrow R$.\hfill (A1) \\
We want to prove that $(P \Rightarrow Q) \Rightarrow (P \Rightarrow R)$.

\vspace{1ex}%
\noindent%
\setlength{\leftskip}{0.5in}%
To prove this latter statement, we will assume $P \Rightarrow Q$.\hfill (A2) \\
we want to prove $P \Rightarrow R$.

\vspace{1ex}%
\noindent%
\setlength{\leftskip}{1in}%
To prove this final statement, we will assume $P$.\hfill (A3) \\
We want to prove $R$.

\vspace{1ex}%
\noindent%
Assumptions (A3) and (A2) are that $P$ is true and that $P \Rightarrow Q$\\
so by modus ponens, $Q$ is true.

\vspace{1ex}%
\noindent%
But assumption (A1) is that $Q \Rightarrow R$, \\
so by modus ponens, $R$ is true.

\vspace{1ex}%
\noindent%
We have proved $R$ is true, which is what we wanted to prove.
\end{proof}

\color{magenta!50!black}
\vfill
\setlength{\leftskip}{0in}
\subsection*{Commentary}

Many people submitted solutions for this problem that were much more
complicated than the proof I offer above; the problem text asks you
not to split into cases.  The structure of the proof is modeled on the
structure of the proposition: the proof begins by considering the
outermost implication arrow, which involves proving another
implication, which itself involves proving an implication.

\end{solutiontext}\end{solution}

%%%%%%%%%%%%%%%%%%%%%%%%%%%%%%%%%%%%%%%%%%%%%%%%%%%%%%%%%%%%%%%%
\begin{problem}[360]
  Let $x$ be an integer. %You may assume for this problem that if an integer is not even, it is odd, and if it is not odd, it is even.

  \vspace{1ex}

  \noindent Write down the \textbf{contrapositive} of the conditional sentence
  \begin{quote}
    If $x$ is even, then $x^2$ is even.
  \end{quote}
  Is this a true statement?  Is the contrapositive a true statement? \\
  If yes, prove it.  If not, find a counterexample.
\end{problem}

\begin{solution}\begin{solutiontext}
    Let $P$ be the proposition ``$x$ is even'' and $Q$ be the
    proposition ``$x^2$ is even.''  The contrapositive of $P
    \Rightarrow Q$ is $(\neg Q) \Rightarrow (\neg P)$, so the
    contrapositive of the given statement is ``If $x^2$ is not even, then $x$ is not even.''

    \vspace{1ex}\noindent%
    The original statement is true.
    \begin{claim}
      If $x$ is even, then $x^2$ is even.
    \end{claim}
    \begin{proof}
      Assume $x$ is even. \\
      Then there exists an integer $k$ so that $x = 2k$, \\
      and so $x^2 = (2k)^2 = 4k^2$.

      \vspace{1ex}\noindent%
      But $4k^2 = 2(2k^2)$, \\
      so $x^2$ is also twice an integer, and therefore even.
    \end{proof}

    \vspace{1ex}\noindent%
    The contrapositive is also true \\
    (because an implication holds if and only if the contrapositive of the implication holds).

\color{magenta!50!black}
\vfill
\setlength{\leftskip}{0in}
\subsection*{Commentary}

Many people submitted separate proofs for the original statement and
the contrapositive: this is unnecessary.  It is easier to prove the
original statement than the contrapositive in my opinion.

\end{solutiontext}\end{solution}

%%%%%%%%%%%%%%%%%%%%%%%%%%%%%%%%%%%%%%%%%%%%%%%%%%%%%%%%%%%%%%%%
\begin{problem}[360]
  Consider the four propositions
  \begin{eqnarray}
    & & P \vee (Q \wedge R), \label{prop1} \\
    & & (P \wedge Q) \Rightarrow R, \label{prop2} \\
    & & (\neg P) \vee (\neg Q) \vee R, \mbox{ and } \label{prop3} \\
    & & (P \vee Q) \wedge (P \vee R). \label{prop4}
  \end{eqnarray}
  Exactly which of these propositions are logically equivalent to
  which other propositions?
  \vspace{1ex}

  \noindent Provide justification for all the claims
  you make; in particular, if you claim that two propositions are
  logically equivalent, you must prove this, and if you claim that
  they are \textit{not} equivalent, you must explain why not.
  \vspace{1ex}

  \noindent I prefer arguments that don't involve cases.
\end{problem}

\begin{solution}\begin{solutiontext}
I claim that (\ref{prop1}) and (\ref{prop4}) are equivalent, and (\ref{prop2}) and (\ref{prop3}) are equivalent, but 
neither (\ref{prop1}) nor (\ref{prop4}) are equivalent to (\ref{prop2}) or (\ref{prop3}).

\begin{claim}
  $\mbox{(\ref{prop1})} \equiv \mbox{(\ref{prop4})}$.
\end{claim}
\begin{proof}
This is precisely the distributive law.
\end{proof}

\begin{claim}
  $\mbox{(\ref{prop2})} \equiv \mbox{(\ref{prop3})}$.
\end{claim}
\begin{proof}
  \begin{align*}
    \mbox{(\ref{prop2})}
    &\equiv (P \wedge Q) \Rightarrow R \\
    &\equiv \left( \neg \left(P \wedge Q\right) \right) \vee R && \text{(definition of $\Rightarrow$)} \\
    &\equiv \left( \left( \neg P \right) \vee \left( \neg Q\right) \right) \vee R && \text{(de Morgan's law)} \\
    &\equiv \mbox{(\ref{prop3})}. \\
  \end{align*}
\end{proof}

\begin{claim}
  $\mbox{(\ref{prop1})} \not\equiv \mbox{(\ref{prop3})}$.
\end{claim}
\begin{proof}
  If $P$ is true, $Q$ is true, and $R$ is false, then (\ref{prop3}) is
  false, but (\ref{prop1}) is true.  So $\mbox{(\ref{prop1})}
  \not\equiv \mbox{(\ref{prop3})}$.  Providing a specific example is the
  quickest way to verify this.
\end{proof}

\color{magenta!50!black}
\vfill
\setlength{\leftskip}{0in}
\subsection*{Commentary}

A complete solution to this problem requires proving three statements
(namely that $\mbox{(\ref{prop1})} \equiv \mbox{(\ref{prop4})}$, that
$\mbox{(\ref{prop2})} \equiv \mbox{(\ref{prop3})}$, and that
$\mbox{(\ref{prop1})} \not\equiv \mbox{(\ref{prop3})}$).  You received
120 points for each claim you proved.  Many people failed to discuss
why $\mbox{(\ref{prop1})} \not\equiv \mbox{(\ref{prop3})}$ and lost
points because of this; the problem text asks ``exactly which''
propositions are equivalent, so you must discuss both which are
equivalent and which are inequivalent.

\end{solutiontext}\end{solution}


%%%%%%%%%%%%%%%%%%%%%%%%%%%%%%%%%%%%%%%%%%%%%%%%%%%%%%%%%%%%%%%%
\begin{problem}[360]
  Let $x$ and $y$ be real numbers, and consider the following
  proposition:
  \begin{quote}
    If $x$ is rational and $y$ is irrational, then $x + 2y$ is irrational.
  \end{quote}
  If the proposition is true, prove it; if not, give a counterexample.

\end{problem}

\begin{solution}\begin{solutiontext}
This is a true proposition.

\begin{claim}
    If $x$ is rational and $y$ is irrational, then $x + 2y$ is irrational.
\end{claim}
\begin{proof}
  Assume that $x$ is rational, $y$ is irrational.  For a contradiction,
  we assume that $x+2y$ is rational.

  Since the difference of rational numbers is rational, $(x+2y) - x$ is
  rational, so $2y$ is rational.  But $1/2$ is rational, and since the
  product of rational numbers is rational, $(1/2) \cdot 2y = y$ is
  rational.  But this is a contradiction---$y$ is irrational.
\end{proof}

\vspace{1ex}\noindent%
For completeness, I include proofs of two results that I used.
\begin{claim}
The difference of rational numbers is rational.
\end{claim}
\begin{proof}
Assume $x, y \in \mathbb{Q}$.  Then there exist integers $a,b,c,d \in \mathbb{Z}$ with $b, d \neq 0$ so that
$$
x = \frac{a}{b} \mbox{ and } y = \frac{c}{d}
$$
and, combining denominators,
$$
x - y = \frac{ad - bc}{bd}
$$
is a rational number, since $ad - bc \in \mathbb{Z}$, $bd \in \mathbb{Z}$, and $bd \neq 0$.
\end{proof}

\begin{claim}
The product of rational numbers is rational.
\end{claim}
\begin{proof}
Assume $x, y \in \mathbb{Q}$.  Then there exist integers $a,b,c,d \in \mathbb{Z}$ with $b, d \neq 0$ so that
$$
x = \frac{a}{b} \mbox{ and } y = \frac{c}{d}
$$
and, multiplying,
$$
xy = \frac{ac}{bd}
$$
is a rational number, since $ac, bd \in \mathbb{Z}$ and $bd \neq 0$.
\end{proof}

\color{magenta!50!black}
\vfill
\setlength{\leftskip}{0in}
\subsection*{Commentary}

A number of people stated that $2y$ is irrational by claiming that a
rational number times an irrational number is irrational; this is not
a true statement (consider when the rational number is zero).

\end{solutiontext}\end{solution}


%%%%%%%%%%%%%%%%%%%%%%%%%%%%%%%%%%%%%%%%%%%%%%%%%%%%%%%%%%%%%%%%
% something with quantifiers
\begin{problem}[360]
  Is the statement
  $$
  \forall x \in \mathbb{R} \, \left( \exists y \in \mathbb{R}\, \left( y^2 < x \right) \right)
  $$
  true or false?  For full credit, justify your answer.
\end{problem}

\begin{solution}\begin{solutiontext}
    The proposition is false. \\
    I claim that the negation of $\forall x \in \mathbb{R} \, \left( \exists y \in
      \mathbb{R}\, \left( y^2 < x \right) \right)$ is true, namely
    \begin{claim}
      $\exists x \in \mathbb{R} \, \left( \forall y \in
      \mathbb{R}\, \left( y^2 \geq x \right) \right)$
    \end{claim}
    \begin{proof}
      Set $x = -1$. \\
      Let $y$ be a real number. \\
      Then $y^2 \geq 0$, so $y^2 \geq 0 > -1 = x$, which is what I wanted to prove.
    \end{proof}

\color{magenta!50!black}
\vfill
\setlength{\leftskip}{0in}
\subsection*{Commentary}

Many students gave a specific choice of $x$ and $y$ for which $y^2
\not< x$; this is not enough; you need to explain why there is a
value of $x$ for which no value of $y$ will satisfy $y^2 < x$.

\end{solutiontext}\end{solution}


\end{exam}

\pagebreak

\vfill
\null

\end{document}
