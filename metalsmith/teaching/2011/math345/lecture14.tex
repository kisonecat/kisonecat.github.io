\documentclass[12pt]{handout}

\title{Lecture 14: Prime numbers}
\author{Jim Fowler}
\course{Math 345}
\date{Thursday, October 14, 2010}



\begin{document}
\maketitle

\section*{Textbook}

This lecture discusses section 4 of the textbook.

\section*{Homework} 

The homework is due Wednesday, October 20, 2010.
From Section 4 of the textbook, do exercise 27.

Suppose that for all $x$, $y\in \mathbb{Z}$, if $xy\equiv 0 \mod p$,
then $x \equiv 0 \pmod p$ or $y\equiv 0 \pmod p$.  Show that $p$ is
prime.

\section*{A message from Professor Falkner}

Dear Math 345 Student,

Are you currently majoring in mathematics?  If not, are you
considering it?  If you answered yes to either question, then I would
like you to know about an opportunity to get acquainted with a faculty
member in the Department of Mathematics whom you might eventually
decide you would like to have as your advisor for your major program.
A number of mathematics professors are offering to meet this quarter
with up to five students each to lead a five-session series of
mathematics-related activities.  Each of the professors has proposed a
different series of activities that they hope will interest students.
I strongly encourage you to participate in one of these activities
that interests you if your schedule permits it.

\subsection*{an impassioned plea to major in mathematics}

\section*{definition of prime numbers}

A positive integer $p$ is prime means that $p \neq 1$ and that for all $a, b \in \mathbb{N}$, if $p = ab$ then $a=1$ or $b = 1$.

\subsection*{humorous example}

$n^2 + n + 41$ is a prime for each $n \in \mathbb{N}$?

\section*{there are infinitely many prime numbers}

also, there are infinitely many composite numbers!  :-)

\section*{divisibility}

Let $p$ be an integer, $p \geq 2$.

Suppose that for all $x,y \in \mathbb{Z}$, if $p$ divides $xy$, then $p$ divides $x$ or $p$ divides $y$.

Show that $p$ is prime.  (the other direction, ``if $p$ is prime, then\ldots'' requires induction).

\section*{inverses modulo $p$}



\end{document}
