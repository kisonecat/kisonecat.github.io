\documentclass[12pt]{handout}

\title{Lecture 36: Diagonalization argument}
\author{Jim Fowler}
\course{Math 345}
\date{Tuesday, November 30, 2010}



\begin{document}
\maketitle

\section*{Textbook}

This lecture discusses section 15 of the textbook.

\section*{Homework} 

The homework is due Friday, December  3, 2010.

From Section 15 of the textbook, do exercise 9.

\section*{Comparing cardinality}

To say that the cardinality of A is less than or equal to the cardinality of B
means that A is equinumerous to a subset of B.

\section*{Diagonalization generally}

Suppose $f : A \to P(A)$.  Then consider
$$
S = \{ x \in A : x \not\in f(x) \}.
$$
If $S$ is in the range of $f$, then there is an $a \in A$ so that $S =
f(a)$.  Is $a \in S$?  If $a \in S$, then $a \not\in f(a) = S$, and
vice versa!  So $S$ cannot be in the range of $f$.



\end{document}
