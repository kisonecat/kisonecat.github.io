\documentclass[12pt]{handout}

\title{Lecture 32: Counting}
\author{Jim Fowler}
\course{Math 345}
\date{Monday, November 22, 2010}



\begin{document}
\maketitle

\section*{Textbook}

This lecture discusses section 13 of the textbook.

\section*{Homework} 

The homework is due Wednesday, November 24, 2010.

From Section 13 of the textbook, do exercises 1, 6, and 8.

\section*{topics}

 Let A and B be sets. To say that A is equinumerous to B means that
 there exists a bijection from A to B.

this is an equivalence relation---prove it.

define ``finite.''

define ``infinite.''

possible for an infinite set to equinumerous to a proper subset of
itself

A finite set cannot be equinumerous to a proper
subset of itself.

An injection from a finite set to itself is a surjection.

A subset of a finite set is finite.







\end{document}
