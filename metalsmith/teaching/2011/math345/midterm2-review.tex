\documentclass[12pt]{handout}
\usepackage{geometry}
\geometry{margin=0.75in}
\usepackage{nopageno}

\title{Midterm 2 Review}
\course{Math 345}
\author{Jim Fowler}

\usepackage[T1]{fontenc}
\usepackage{lmodern}
\usepackage{hyperref}
\usepackage{multicol}
\newcommand{\peem}{\textsc{p.m.}}
\newcommand{\ayem}{\textsc{a.m.}}

\begin{document}
\maketitle

\noindent This sheet summarizes what sort of content will appear on
the second midterm.  \textbf{Calculators are not permitted} and would be
useless anyhow.

\section*{Possible topics}

The second midterm may include material through lecture 13
(``congruences'') through lecture 23 (``strong induction'').  This
means that you should work through section four, five, six, and seven of the textbook.

\begin{multicols}{2}
\begin{itemize}
\item Induction
\item Complete induction
\item Least elements
\item Divisibility
\item Prime numbers
\item Proof that there are infinitely many prime numbers
\item Sums of $n^{\mbox{\scriptsize th}}$ powers
\item Telescoping sums
\item Differences between successive terms in a sequence
\item Fibonacci numbers
\item Pascal's triangle
\item Patterns in Pascal's triangle
\item Binomial theorem
\item Proof of Binomial theorem
\item $\displaystyle\binom{n+1}{k} = \displaystyle\binom{n}{k-1} + \displaystyle\binom{n}{k}$
\item Congruences (``mod'' arithmetic)
\end{itemize}
\end{multicols}

\subsection*{Proof that there are infinitely many primes}

The exam will definitely ask you to provide a proof that there are
infinitely many prime numbers, so you should be prepared to write up a
nice proof of this fact for the midterm.

\section*{Problems}

There will again be five problems on the midterm, each worth
\textbf{360 points}, for a total of \textbf{1800 points}.  The
\textbf{fake midterm} will give you an idea of the sorts of problems
you might expect, but the fake midterm is longer than the real
midterm.

\end{document}
