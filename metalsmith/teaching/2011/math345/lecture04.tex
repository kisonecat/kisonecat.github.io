\documentclass[12pt]{handout}

\title{Lecture 4: Solving equations}
\author{Jim Fowler}
\course{Math 345}
\date{Tuesday, September 28, 2010}



\begin{document}
\maketitle

\section*{Textbook}

This lecture discusses section 2 of the textbook.

\section*{Homework} 

The homework is due Thursday, September 30, 2010.

From Section 2 of the textbook, do exercises 19 and 20.

We did this example last time in class, so it shouldn't be too hard.

\section*{contradiction}

what is a contradiction?  $Q \wedge \neg Q$

\section*{Why does this work?}

look at a truth table

\section*{how to prove a negative statement}

proof by contradiction

example: $((P \Rightarrow Q) \wedge \neg Q) \Rightarrow \neg P$.

\section*{Versus contrapositive}

Want to prove $P \Rightarrow Q$?

Method of Contradiction: Assume P and Not Q and prove some sort of contradiction.

Method of Contrapositive: Assume Not Q and prove Not P.

\section*{example from textbook}

if $x^2$ is even, then $x$ is even.

\section*{example: diophantine equation}

There are no positive integer solutions to the diophantine equation $x^2 - y^2 = 1$

proof: factoring

\section*{example: largest number}

theorem: there is no largest integer

\section*{last digit of squares}

If $n$ is a perfect square, then $n$ ends with a $0, 1, 4, 5, 6$ or $9$.

\section*{Ramsey theory}

There are six people at a party.  At least 3 know each other, or 3 don't.

\end{document}
