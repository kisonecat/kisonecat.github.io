\documentclass[14pt]{chalkfjord}

\title{Lecture 28: Homology}
\author{Math 660---Jim Fowler}
\date{Friday, July 29, 2011}

\begin{document}

\begin{frame}\titlepage\end{frame}

\begin{frame}
  \Huge
  \textbf{Maximum Principle!}
\end{frame}

\begin{frame}
  \begin{theorem}
    If $f : \Omega \to \C$ is analytic, nonconstant, \\
    \quad $|f(z)|$ has no maximum in $\Omega$.
  \end{theorem}

  \begin{corollary}
    If $f : K \to \C$ is continuous for $K$ compact, \\
    \quad and $f |_{\mbox{interior}}$ is analytic, \\
    \quad then the maximum of $|f(z)|$ is on the boundary of $K$.
  \end{corollary}
\end{frame}

\begin{frame}
  Proof 1: local mapping.

  Proof 2: use Cauchy's integral formula on circles
\end{frame}

\begin{frame}
  \begin{lemma}[Schwarz]
    $f(z)$ analytic on $D_1(0)$ and $|f(z)| \leq 1$, $f(0) = 0$, then
    \\
    \quad $|f(z)| \leq |z|$ and $f'(0) \leq 1$.
    If these inequalities are realized in the disk, then \\
    \quad $f(z) = cz$ where $|c| = 1$.
  \end{lemma}
  \pause
  \begin{proof}
    Apply maximum principle to $f(z)/z$.
  \end{proof}
\end{frame}

\begin{frame}
  \Huge
  \textbf{Homology!}
\end{frame}

\begin{frame}
  A chain is a formal sum of arcs.

  A cycle is a chain ``equivalent to'' a sum of closed curves.

\end{frame}

\begin{frame}
  \frametitle{Simply-connected regions}

  $\Omega \subset \C$ is \textbf{simply connected} if $\C - \Omega$ is
  connected.

  \begin{theorem}
    $\Omega$ is simply connected iff \\
    \quad for all $a \in \C - \Omega$, \\
    \quad\quad and all cycles $\gamma$,
    \quad $\eta(\gamma,a) = 0$.
  \end{theorem}
\end{frame}

\begin{frame}
  \frametitle{Proof of theorem}

  $\C - \Omega = A \cup B$ for closed, disjoint $A$, $B$.

  \pause

  Suppose $A$ bounded; pick $a \in A$.

  \pause

  Cover $\C$ by small squares \\
  \quad (so small that each square may touch \\
  \quad $A$ or $B$ but not both)

  \pause

  $$\gamma = \sum_{\mbox{$S$ touching $A$}} \partial S$$

  \pause

  $\gamma \subset \Omega$ and $\eta(\gamma,a) = 1$.  \pause\hfill $\Rightarrow\Leftarrow$
\end{frame}

\begin{frame}
  $\gamma \subset \Omega$

  Define $\gamma \sim 0$ \\
  \quad if $\eta(\gamma,a) = 0$ \\
  \quad for all $a \in \C - \Omega$.

  \pause

  Define $\gamma_1 \sim \gamma_2$ \\  
  \quad if $\gamma_1 - \gamma_2 \sim 0$.

\end{frame}

% This lecture covers section 4.4.1 and 4.4.2 and 4.4.3 of the textbook.

\end{document}
