\documentclass[12pt]{handout}
%\usepackage{add-copyright}
\usepackage{geometry}
\geometry{margin=1in}
\usepackage{nopageno}

\title{Syllabus}
\course{Math 660}
\author{Jim Fowler}

\usepackage[T1]{fontenc}
\usepackage{lmodern}
\usepackage{hyperref}

\newcommand{\peem}{\textsc{p.m.}}
\newcommand{\ayem}{\textsc{a.m.}}

\begin{document}
\maketitle

\noindent This is a beginning graduate course in complex analysis; as such, we study \textit{analysis} (e.g., the rigorous foundations of calculus) as it applies to \textit{functions of a complex variable}.  The resulting theory is strikingly beautiful.



\section*{Resources}

\noindent%
We present 5 resources to help you to learn complex analysis.

\subsection*{Professor's office hours}
If you have questions, want to work through problems, or just talk
about mathematics, please attend office hours.

\vspace{1ex}%
\noindent\parbox{0.5\textwidth}{%
\noindent\begin{tabular}{@{}ll}
\textsf{Name:} & Jim Fowler \\
\textsf{Office:} & MW658 Mathematics Tower \\
\textsf{Phone:} & (773) 809--5659 \\
\textsf{Email:} & \href{mailto:fowler@math.osu.edu}{\texttt{fowler@math.osu.edu}} \\
\textsf{Website:} & \url{http://www.math.osu.edu/~fowler/}
\end{tabular}}
\noindent\parbox{0.5\textwidth}{%
\begin{tabular}{@{}ll}
\textsf{Office Hours:}
& Monday 3:30--5:00\peem \\
& Wednesday 3:30--5:00\peem \\
& and by appointment
\end{tabular}}

\vspace{1ex}\noindent
Please email me with any concerns you have; the success of this course
depends on open communication.

\subsection*{Textbook}
Our text is \textit{Complex Analysis}, 3rd edition, by
Lars Ahlfors (ISBN 0070006571); this course covers chapters 1--5.

\subsection*{Website}
The course website is available at \url{http://www.math.osu.edu/~fowler/teaching/math660/}.

\subsection*{Lectures}
We meet Mondays, Tuesdays, Wednesdays, Thursdays, and Fridays,
12:30--1:18\peem\ in Caldwell Lab 171 for an interactive lecture.

%%%%%%%%%%%%%%%%%%%%%%%%%%%%%%%%%%%%%%%%%%%%%%%%%%%%%%%%%%%%%%%%
\subsection*{Assessment}

There are 1000 points possible in this course, broken down as
follows:
\begin{description}
\item[\textsf{\textbf{9 problem sets (360 points; 40 points each).}}]  Homework is due each Monday.  You should work on the homework
  problems together, but you must write up your solutions
  independently. To permit electronic grading, I ask that you turn in your homework on the provided paper.\vspace{1ex}\\
You must stay caught up with the homework.  It is tempting to fall behind, but difficult
to catch up again---this is true of all courses, but especially true
of a course in mathematics.  That said, I understand your schedules
are very busy, so I will not penalize you for \textit{infrequently}
turning in work \textit{a day or two late.}  Do not make a habit of
it!




\item[\textsf{\textbf{Participation (100 points).}}]  Each time you present something on the blackboard, you will receive 15 points, and each in-class quiz is worth 5 points; there is a maximum of 100 participation points.

\item[\textsf{\textbf{1 midterm (240 points).}}]  The midterm is in-class.
The midterm is Friday, July 22.

\item[\textsf{\textbf{1 final exam (300 points).}}]  The cumulative final exam will be held in our usual classroom at
11:30\ayem--1:18\peem\ on Wednesday, August 24, 2010.
\end{description}

\noindent%
Calculators are \textbf{not permitted} during any exams.
\vspace{1ex}

\end{document}

