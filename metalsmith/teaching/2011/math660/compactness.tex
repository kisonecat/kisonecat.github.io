\documentclass[12pt]{handout}
\usepackage{nopageno}
\usepackage{geometry}\geometry{margin=0.5in}

\title{Compactness}
\course{Math 660}
\author{Jim Fowler}
\date{Summer 2011}

\newcounter{question}
\newenvironment{question}{\vspace{1ex}\addtocounter{question}{1}\noindent\textbf{Task
    \arabic{question}.}}{\vfill}

\DeclareMathOperator{\interior}{Int}
\DeclareMathOperator{\closure}{Cl}

\begin{document}
\maketitle

\begin{question}
  A \textbf{geodesic metric space} is a metric space $X$ so that, for
  any $x, y \in X$, there exists an isometry $f : [0,d(x,y)] \to X$ with
  $f(0) = x$ and $f(d(x,y)) = y$.  Is $\C$ a geodesic metric space?
  Is $\C - \{0\}$ a geodesic metric space?
\end{question}

\begin{definition}
A metric space is \textbf{complete} if every Cauchy sequence converges.
\end{definition}

\begin{definition}
  A metric space $X$ is \textbf{totally bounded} if, for every
  $\epsilon > 0$, the space $X$ can be covered by finitely many
  $\epsilon$-balls.
\end{definition}

\begin{question}
  Suppose $X$ is a finite set; show that
  \begin{itemize}
  \item a function $f : X \to \R$ is bounded,
  \item a function $f : X \to \C$ attains a maximum,
  \item any cover of $X$ admits a finite subcover, and
  \item any sequence in $X$ has a convergent subsequence.
  \end{itemize}
\end{question}

\begin{definition}
  A space $X$ is \textbf{compact} if every open covering of $X$ admits
  a finite subcovering.
\end{definition}

\begin{question}
  A subset $K$ is compact iff it is complete and totally bounded.
\end{question}

\begin{question}
  A subset $K \subset \C$ is compact iff $K$ is closed and bounded.
\end{question}

\begin{question}
  The continuous image of a closed set is closed?
\end{question}

\begin{question}
  The continuous image of a bounded set is bounded?
\end{question}

\begin{question}
  The continuous image of a compact space is compact?
\end{question}

\begin{question}
  The pre-image of a compact space under a continuous function is
  compact?
\end{question}

\begin{question}
  A closed subset of a compact space is compact?
\end{question}

\begin{question}
  For any open cover $\mathcal{U}$ of a compact metric space $X$,
  there exists $\delta > 0$ so that for every $x \in X$, the open ball
  $B_\delta(x)$ is contained in some $U \in \mathcal{U}$.
\end{question}

\begin{question}
  Let $K$ be compact; any continuous function $f : K \to Y$ is
  uniformly continuous.  (In other words, if the domain is compact,
  the local condition of continuity gives rise to a global condition
  of \textit{uniform} continuity.)
\end{question}

\begin{question}
  For a metric space $X$ and a open subset $S \subset X$ with
  $\closure S$ compact, define $n(S,X)$ to be the number of noncompact
  connected components of $X - S$.  How large can $n(S,\R)$ be?  How
  large can $n(S,\C)$ be?
\end{question}

\pagebreak
\null

\begin{question}
  How many different functions can you form by composing $f(z) = 1-z$
  and $g(z) = 1/z$?
\end{question}

\begin{question}
  How many different functions can you form by composing $f(z) = 1+z$
  and $g(z) = -1/z$?
\end{question}

\begin{question}
  Find an analytic function sending $B_1(0) \subset \C$ to the upper
  half-plane.
\end{question}

\begin{question}
  Suppose $f(z)$ is a rational function of order $n$; what can you say
  about the order of $f \circ f$?
\end{question}

\vfill

\end{document}

%%% Local Variables: 
%%% mode: latex
%%% TeX-master: t
%%% End: 
