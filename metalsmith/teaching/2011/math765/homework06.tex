\documentclass[12pt]{pset}

\course{Math 765}
\author{Jim Fowler}
\date{Winter 2011}

\newcommand{\CC}{\mathbb{C}}
\newcommand{\CP}{\mathbb{C}P}
\newcommand{\RP}{\mathbb{R}P}

\usepackage{enumerate}
\usepackage{nopageno}
\usepackage{hyperref}
\usepackage{tikz}

\DeclareMathOperator{\Vect}{Vect}
\DeclareMathOperator{\GL}{GL}
\DeclareMathOperator{\SO}{SO}
\DeclareMathOperator{\codim}{codim}
\DeclareMathOperator{\Mat}{Mat}
\DeclareMathOperator{\SymMat}{SymMat}
\DeclareMathOperator{\Id}{Id}
\DeclareMathOperator{\Sym}{Sym}
\newcommand{\Proj}{\mathbb{P}}

\begin{document}
\maketitle

\noindent\textbf{Stokes' theorem.}  At the end of this week, we'll
have Stokes' theorem, a beautiful generalization of the fundamental
theorem of Calculus.  As usual, email me with questions at
\texttt{fowler@math.osu.edu}.  \textit{The exercises below should be
  handed in on Monday, February 14, 2011.}

%%%%%%%%%%%%%%%%%%%%%%%%%%%%%%%%%%%%%%%%%%%%%%%%%%%%%%%%%%%%%%%%
\begin{problem}[Lee 13--3]

  Suppose $M$ and $N$ are oriented smooth manifolds and $F : M \to N$
  is a local diffeomorphism.  If $M$ is connected, show that $F$ is
  either orientation-preserving or orientation-reversing.

\end{problem}
\vfill

%%%%%%%%%%%%%%%%%%%%%%%%%%%%%%%%%%%%%%%%%%%%%%%%%%%%%%%%%%%%%%%%
\begin{problem}[Lee 14--21]

  Suppose $M$ and $N$ are compact, connected, oriented smooth
  manifolds and $F, G : M \to N$ are diffeomorphisms.  If $F$ and $G$
  are homotopic, show that they are either both orientation-preserving
  or both orientation-reversing.

  \vspace{1ex}\noindent\textit{Hint:} Use Whitney
  approximation theorem (page 252) and Stokes' theorem on $M
  \times I$.

\end{problem}
\vfill

%%%%%%%%%%%%%%%%%%%%%%%%%%%%%%%%%%%%%%%%%%%%%%%%%%%%%%%%%%%%%%%%
\begin{problem}[Lee 13--6]

  Show that $\RP^n$ is orientable iff $n$ is odd.

\end{problem}
\vfill

%%%%%%%%%%%%%%%%%%%%%%%%%%%%%%%%%%%%%%%%%%%%%%%%%%%%%%%%%%%%%%%%
\begin{problem}[Lee 14--5a]
  
  Suppose $\tilde{M}$ and $M$ are smooth $n$-manifolds, and $\pi :
  \tilde{M} \to M$ is a smooth $k$-sheeted covering map.  If
  $\tilde{M}$ and $M$ are oriented and $\pi$ is
  orientation-preserving, show that $\int_{\tilde{M}} \pi^\star \omega
  = k \int_M \omega$ for any compactly supported $n$-form $\omega$ on
  $M$.

%  \vspace{2ex}\noindent Does $\int_{\tilde{M}} \pi^\star \omega = k \int_M \omega$ hold for
%  all compactly supported $k$-forms $\omega$ on $M$ when $k < n$?
  
\end{problem}
\vfill
\vfill

\pagebreak

%%%%%%%%%%%%%%%%%%%%%%%%%%%%%%%%%%%%%%%%%%%%%%%%%%%%%%%%%%%%%%%%
\begin{problem}[Bicycle chains]

% otherwise known as the Whitney–Graustein theorem
% 

  \begin{description}
  \item[(a)] Let $i : S^1 \to \R^2$ be the standard embedding of the
    circle in the plane, $i(\theta) = \left(\cos \theta, \sin
      \theta\right)$, and let $j : S^1 \to \R^2$ be the figure eight
    immersion $$j(\theta) = \left(\cos \theta, \sin (2\theta)\right).$$
    Can you connect $j$ and $i$ by a smooth family of immersions?  In
    other words, is there a smooth map $F : S^1 \times I \to \R^2$ so
    that each $f_t : S^1 \to \R^2$ given by $f_t(\theta) =
    F(\theta,t)$ is an immersion and $f_0 \equiv i$ and $f_1 \equiv
    j$?

  \item[(b)] Let $i : S^1 \to \R^2$ be the standard embedding of the
    circle in the plane, and let $j : S^1 \to \R^2$ be the immersion
    \begin{center}
      \begin{tikzpicture}
        \draw[line width=2pt]
        (0,0) .. controls ++(-1,0) and ++(0,0) ..
        (-1,1) .. controls ++(0,1) and ++(0,0) .. 
        (0,2) .. controls ++(1,0) and ++(0,1) .. 
        (1,0) .. controls ++(0,-2) and ++(-2,0) ..
        (3,-3) .. controls ++(2,0) and ++(0,0) .. 
        (5,0) .. controls ++(0,1) and ++(1,0) .. 
        (3,2) .. controls ++(-1,0) and ++(0,0) .. 
        (2,0) .. controls ++(0,-1) and ++(-1,0) .. 
        (3,-2) .. controls ++(1,0) and ++(1,0) .. 
        (3,0) .. controls ++(0,0) and ++(0,0) .. 
        (0,0);
      \end{tikzpicture}
    \end{center}
  Can you connect $j$ and $i$ by a smooth family of immersions?
  \end{description}


\end{problem}

\vfill

%%%%%%%%%%%%%%%%%%%%%%%%%%%%%%%%%%%%%%%%%%%%%%%%%%%%%%%%%%%%%%%%
\begin{problem}[Lee 14--6]

If $M$ is a compact, smooth, oriented manifold with boundary, show
that there does not exist a smooth retraction of $M$ onto its boundary.  

\vspace{1ex}\noindent\textit{Hint:} Consider an orientation form on
$\partial M$.

\end{problem}

\vfill
\vfill

\end{document}
