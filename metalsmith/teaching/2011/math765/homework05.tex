\documentclass[12pt]{pset}

\course{Math 765}
\author{Jim Fowler}
\date{Winter 2011}

\newcommand{\CC}{\mathbb{C}}
\newcommand{\CP}{\mathbb{C}P}
\newcommand{\RP}{\mathbb{R}P}

\usepackage{enumerate}
\usepackage{nopageno}
\usepackage{hyperref}

\DeclareMathOperator{\Vect}{Vect}
\DeclareMathOperator{\GL}{GL}
\DeclareMathOperator{\SO}{SO}
\DeclareMathOperator{\codim}{codim}
\DeclareMathOperator{\Mat}{Mat}
\DeclareMathOperator{\SymMat}{SymMat}
\DeclareMathOperator{\Id}{Id}
\DeclareMathOperator{\Sym}{Sym}
\newcommand{\Proj}{\mathbb{P}}

\begin{document}
\maketitle

\noindent\textbf{Tensors and Forms.}  This week, the goal is
forms---and along the way, we'll have our first glimpse of Riemannian
metrics and the exterior derivative.  This problem set is pretty heavy
on the algebra of tensors; please email me with questions at
\texttt{fowler@math.osu.edu}.  \textit{The exercises below should be
  handed in on Monday, February 7, 2011.}

%%%%%%%%%%%%%%%%%%%%%%%%%%%%%%%%%%%%%%%%%%%%%%%%%%%%%%%%%%%%%%%%
\begin{problem}[Veronese surface]

  Let $V = \R^3$, and consider the map $s : V \to \Sym^2 V$ which
  sends $s(v)$ to $v \cdot v$.
  \begin{description}
  \item[(a)] Show that, although $s$ is not a linear map, $s$
    nonetheless induces a map $\Proj(s) : \Proj(V) \to \Proj(\Sym^2
    V)$, i.e., $s$ sends lines to lines.
  \item[(b)] Is $\Proj(s)$ a smooth map?  Compute its derivative
    $\Proj(s)_\star : T\RP^2 \to T\RP^5$.  You might want to think
    about a nice way to describe a vector in $T\RP^n$.
  \item[(c)] Is $\Proj(s)$ a smooth embedding?
  \end{description}


\end{problem}

%%%%%%%%%%%%%%%%%%%%%%%%%%%%%%%%%%%%%%%%%%%%%%%%%%%%%%%%%%%%%%%%
\begin{problem}[Lee Exercise 12.3]

Show that the wedge product is the unique associative, bilinear, and
anticommutative map $\Lambda^k(V) \times \Lambda^\ell (V) \to
\Lambda^{k + \ell}(V)$ satisfying
$$
(\omega^1 \wedge \cdots \wedge \omega^k)(X_1,\ldots,X_k) = \det\left(
\omega^j(X_i) \right)
$$  
for covectors $\omega^1, \ldots, \omega^k$ and vectors $X_1, \ldots, X_k$.

\end{problem}

%%%%%%%%%%%%%%%%%%%%%%%%%%%%%%%%%%%%%%%%%%%%%%%%%%%%%%%%%%%%%%%%
\begin{problem}[Lee 12--6]

  Define a $2$-form $\Omega$ on $\R^3$ by
$$
\Omega = x\, dy \wedge dz + y\, dz \wedge dx + z\, dx \wedge dy
$$
\begin{description}
\item[(a)] Compute $\Omega$ in spherical coordinates
  $(\rho,\varphi,\theta)$ defined by
$$(x,y,z) = (\rho sin \varphi \cos \theta, \rho \sin \varphi \sin \theta, \rho \cos \varphi).$$
\item[(b)] Compute $d\Omega$ in both Cartesian and spherical
  coordinates and verify that both expressions represent the same 3-form. 
\item[(c)] Compute the restirction $\Omega|_{S^2} = \iota^\star
  \Omega$ using coordinates $\varphi,\theta)$ on the open subset where
  these coordinates are defined.
\item[(d)] Show that $\Omega|_{S^2}$ is nowhere zero.
\end{description}

\end{problem}

%%%%%%%%%%%%%%%%%%%%%%%%%%%%%%%%%%%%%%%%%%%%%%%%%%%%%%%%%%%%%%%%
\begin{problem}[Lee 12--5]

A $k$-covector $\eta$ on a finite-dimensional vector space $V$ is said
to be \textit{decomposable} if it can be written
$$
\eta = \omega^1 \wedge \cdots \wedge \omega^k
$$
where $\omega^1, \ldots, \omega^k$ are covectors.  For what values of
$n$ is it true that every $2$-covector on $\R^n$ is decomposable?

\end{problem}

%%%%%%%%%%%%%%%%%%%%%%%%%%%%%%%%%%%%%%%%%%%%%%%%%%%%%%%%%%%%%%%%
\begin{problem}[Lee 12--17, also known as \textsc{Cartan's Lemma}]

  Let $M$ be a smooth $n$-manifold, and let $\omega^1, \ldots,
  \omega^k$ be independent smooth $1$-forms on an open subset $U
  \subset M$.  If $\alpha^1, \ldots, \alpha^k$ are $1$-forms on $U$
  such that
$$
\sum_{i=1}^k \alpha^i \wedge \omega^i = 0
$$
show that there are smooth functions $f_{ij}$ so that
$$
\alpha^i = \sum_{j=1}^k f_{ij} \omega^j
$$

\end{problem}

%%%%%%%%%%%%%%%%%%%%%%%%%%%%%%%%%%%%%%%%%%%%%%%%%%%%%%%%%%%%%%%%
\begin{problem}

  There are one line answers to these two questions.
  \begin{description}
  \item[(a)] Does every closed manifold $M^n$ admit Riemannian metric?
  \item[(b)] Does every closed manifold $M^n$ admit a symplectic form?
  \end{description}
  If you answer in the affirmative, prove it; if not, give a
  counterexample.

\end{problem}

\end{document}
