\documentclass[12pt]{amsart}

\title{Research Statement}
\author{Jim Fowler}

\usepackage[left=1.5in,right=1.5in,top=0.5in,bottom=1in]{geometry}

\usepackage{hyperref}

\newcommand{\mr}[1]{\cite{MR#1}}
\newcommand{\nomr}[1]{\nocite{MR#1}}
\newcommand{\arxiv}[1]{\cite{arxiv#1}}

\usepackage{amsmath}
\usepackage{amsthm}
\newtheorem{theorem}{Theorem}
\newtheorem{corollary}[theorem]{Corollary}
\newtheorem*{corollary*}{Corollary}
\newtheorem*{theorem*}{Theorem}
\newtheorem{lemma}[theorem]{Lemma}
\newtheorem*{lemma*}{Lemma}

\theoremstyle{definition}
\newtheorem{remark}[theorem]{Remark}
\newtheorem*{remark*}{Remark}
\newtheorem{example}[theorem]{Example}
\newtheorem*{example*}{Example}
\newtheorem{definition}[theorem]{Definition}
\newtheorem*{definition*}{Definition}
\newtheorem{conjecture}[theorem]{Conjecture}
\newtheorem*{conjecture*}{Conjecture}
\newtheorem{question}[theorem]{Question}

\DeclareMathOperator{\Sp}{Sp}

\usepackage{amssymb}
\newcommand{\R}{\mathbb{R}}
\newcommand{\Q}{\mathbb{Q}}
\newcommand{\N}{\mathbb{N}}
\newcommand{\Z}{\mathbb{Z}}

\newcommand{\B}{\mathcal{B}}

\usepackage[style=numeric]{biblatex}
\addbibresource{references.bib}

\begin{document}
\maketitle

I am a geometer, and I am interested in the geometric topology of
manifolds and in geometric group theory.  Manifolds and their
symmetries are ubiquitous throughout in mathematics and its
applications.  Surfaces were studied already in the nineteenth century
by Riemann, and higher dimensional manifolds became central of objects
in the twentieth beginning with the work of Poincar\'e, and continuing
with the work of Thom, Milnor, Wall, Smale, Novikov---to name just a
few.

Broadly speaking, my work focuses on aspherical manifolds, Poincar\'e
duality groups, and computations for high-dimensional manifolds.  My
work makes use of the techniques and ideas from algebraic and
combinatorial topology, surgery theory, number theory, and complexity
theory.  Specifically, a major theme of much of my work is the extent
to which a geometric object with some singularities or deficiencies
can be ``resolved'' or ``improved'' to a nicer object.  For instance,

\begin{itemize}
\item a \href{http://arxiv.org/abs/1304.3730}{recent paper of mine}
  shows that there are aspherical topological $n$-manifolds (for $n
  \geq 6$) that cannot be ``resolved'' by a simplicial complex, i.e.,
  they are not triangulable;
\item there is a PL $\mathbb{Q}$-homology $4n$-manifold $M$ having
  cohomology ring $H^\star(M;\Q) = \Q[x]/(x^3)$, which I have shown
  cannot be improved to a smooth manifold unless $n = 2^a + 2^b$;
\item my thesis addressed the question of whether the classifying
  space for a $\Q$-PD group can be ``resolved'' by an aspherical
  closed $\mathbb{Q}$-homology manifold; and finally
\item I have studied the extent to which a cochain can be replaced by
  a polynomially-bounded cochain, culminating in my developing
  polynomially-bounded variant of algebraic $K$-theory (see
  \href{http://www.ams.org/mathscinet-getitem?mr=2962981}{MR2962981}).
\end{itemize}

To make progress on questions like these, I make significant use of
computational techniques, and often use the computer algebra system
\href{http://sagemath.org/}{\texttt{Sage}} to run experiments.  

In addition to my geometric projects, I also work on adaptive
learning.  As of November 2013, there have been 147k student
enrollments in my \href{http://kisonecat.com/teaching/2014/calculus-two/}{online courses}, which has lead to millions of correct (and incorrect)
answers being submitted.  The resulting data is now being used to
train a hidden Markov model that I designed.  The Transforming
Undergraduate Education in \href{http://www.nsf.gov/funding/pgm_summ.jsp?pims_id=5741}{STEM (TUES) Type~1} award for my proposal
DUE--1245433 will contribute to this project.

\end{document}

%%% Local Variables: 
%%% mode: latex
%%% TeX-master: t
%%% End: 
