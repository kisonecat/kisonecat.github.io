---
layout: post
title: Building aspherical manifolds.
tags: talks mathematics
date: 2007-01-25 19:17:59 +0000
---

I gave a Farb student seminar talk on a lovely paper \cite{MR690848}.

I also used some of the material in \cite{MR1937019}
which summarizes other the many applications of the "reflection group trick," and works through some examples with cubical complexes.

The main result is

\begin{theorem}
Suppose $B\pi = K(\pi,1)$ is a finite complex.  Then there is a closed aspherical manifold $M^n$ and a retraction $\pi_1(M) \to \pi$.
\end{theorem}

This manifold $M$ can be explictly constructed by gluing together copies of the regular neighorhood of $B\pi$ embedded in some Euclidean space.  The application of this theorem is to "promote" a finite complex to a closed aspherical manifold.  For instance, we have a finite complex with non-residually-finite fundamental group: define the group $\pi = \langle a, b : a b^2 a^{-1} = b^3 \rangle$, which is not residually finite, and observe that the presentation 2-complex is aspherical, so we have a finite $B\pi$.  Then using the theorem to "promote" this to a closed aspherical manifold, we get a manifold $M^n$ with fundamental group retracting onto $\pi$.  But a group retracting onto a non-residually-finite group is also non-residually finite, so we have found a closed aspherical manifold $M^n$ with non-residually-finite fundamental group.

Just to whet your appetite, let me introduce a few of the main players, so as to give a sense of how to glue together copies of the regular neighborhood of $B\pi$.

Let $L$ be a simplicial complex, and $V = L^{(0)}$, the vertices of $L$.  

From $L$ we construct two things: some complexes to glue together, and some groups with which to do the gluing.  First, we construct the groups.  Define $J$ to be the group $(\Z/2\Z)^V$, i.e., the abelian group generated by $v \in V$ with $v^2 = 1$.  Next define $W_L$ to be the right-angled Coxeter group having $L^{(1)}$ as its Coxeter diagram; specifically, $W_L$ is the group with generators $v \in V$ and relations $v^2 = 1$ for $v \in V$ and also the relations $v_i v_j = v_j v_i$ if the edge $(v_i,v_j)$ is in $L$.  Note that $J$ is the abelianization of $W_L$.

Next we will build the complexes to be glued together with the above groups.  Let $K$ be the cone on the barycentric subdivision of $L$, and define closed subspaces $\{ K_v \}_{v \in V}$ by setting $K_v$ to be the closed star of the vertex $v$ in the subdivision of $L$.  Note that $K_v$ are subcomplexes of the boundary of $K$, and that a picture would be worth a thousand words right now.

Having the complexes and the groups, we will glue together copies of $K$ along the $K_v$'s, thinking of the latter as the mirrors.  Specifically, define $P_L = (J \times K)/\sim$ with $(g,x) \sim (h,y)$ provided that $x = y$ and $g^{-1} h \in J_{\sigma(x)}$, where $\sigma(x) = \{ v \in V : x \in K_v \}$, and $J_{\sigma(x)}$ is the subgroup of $J$ generated by $\sigma(x)$.  That is a mouthful, but it really is just carefully taking a copy $K$ for each group element of $J$ and gluing along the $K_v$'s in the appropriate manner.  The resulting compplex $P_L$ has a $J$ action with fundamental domain $K$.  Similarly, we use $W_L$ to define a complex $\Sigma_L = (W_L \times K)/\sim$.

The topology of $\Sigma_L$ is related to the complex $L$ that we started with.  For example, if $L$ is the triangulation of $S^{n-1}$, then $\Sigma_L$ is a manifold.  Similarly, if $L$ is a flag complex, then $\Sigma_L$ is contractible.

The idea, now, is to take some finite complex $B\pi$, embed it in $\R^N$, and take a regular neighborhood; the result is a manifold $X$ with boundary $\partial X$, and with $\pi_1 X = \pi$.  Triangulate $\partial X$ as a flag complex, and call the resulting  complex $L$.  Instead of gluing together copies of $K$, glue together copies of $X$ along the subdivision of $L$ to get $P_L(X) = (J \times X)/\sim$ and $\Sigma_L(X) = (W_L \times X)/\sim$.  With some work, we check that $\Sigma_L(X)$ is contractible because $L$ is flag, and that the contractible space $\Sigma_L(X)$ covers the closed manifold $P_L(X)$, which is therefore aspherical.  Since $P_L(X) \to X \to P_L(X)$ is a retraction of spaces, we have found our desired aspherical manifold $M = P_L(X)$ with a retraction of fundamental groups.

