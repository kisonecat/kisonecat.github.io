---
layout: post
title: Efficient construction of the reals.
tags: mathematics
date: 2006-10-20 04:48:38 +0000
---

Today in Geometry/Topology seminar, quasihomomorphisms $\Z \to \Z$ were discussed, i.e., the set of maps $f : \Z \to \Z$ such that $| f(a+b) - f(a) - f(b) |$ is uniformly bounded, modulo the relation of being a bounded distance apart.  These come up when defining rotation and translation numbers, for instance.

Anyway, Uri Bader mentioned that these quasihomomorphisms form a field, isomorphic to $\R$, under pointwise addition and composition.  I hadn't realized that this is a general construction.  Given a finitely generated group (with fixed generating set, so we have the word metric $d$ on the group), I can define a quasihomomorphism $f : G \to G$ by demanding $d(f(ab),f(a)f(b))$ be uniformly bounded, and where two quasihomomorphisms $f, g$ are equivalent if $d(f(a),g(a))$ is uniformly bounded.  Let's call the resulting object $\hat{G}$ for now.

What can be said about $\hat{G}$?  For instance, what is $\hat{F_2}$?

